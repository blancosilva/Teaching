\PassOptionsToPackage{table}{xcolor}
\documentclass[9pt,xcolor=x11names,compress]{beamer}

%% General document %%%%%%%%%%%%%%%%%%%%%%%%%%%%%%%%%%
\usepackage{graphicx}
\usepackage{tikz}
\usetikzlibrary{decorations.fractals,lindenmayersystems}
\usepackage{wasysym}
%%%%%%%%%%%%%%%%%%%%%%%%%%%%%%%%%%%%%%%%%%%%%%%%%%%%%%


%% Beamer Layout %%%%%%%%%%%%%%%%%%%%%%%%%%%%%%%%%%
\useoutertheme[subsection=false,shadow]{miniframes}
\useinnertheme{default}
\usefonttheme{serif}
\usepackage{palatino}

\setbeamerfont{title like}{shape=\scshape}
\setbeamerfont{frametitle}{shape=\scshape}

\setbeamercolor*{lower separation line head}{bg=DeepSkyBlue4} 
\setbeamercolor*{normal text}{fg=black,bg=white} 
\setbeamercolor*{alerted text}{fg=DeepSkyBlue4} 
\setbeamercolor*{example text}{fg=black} 
\setbeamercolor*{structure}{fg=black} 
 
\setbeamercolor*{palette tertiary}{fg=black,bg=black!10} 
\setbeamercolor*{palette quaternary}{fg=black,bg=black!10} 

\setbeamertemplate{blocks}[rounded][shadow=true]
\setbeamercolor{block title}{bg=DeepSkyBlue4}
\setbeamercolor{block title example}{bg=DeepSkyBlue4}
\setbeamercolor{block body}{bg=black!15!white}
\setbeamercolor{block body example}{bg=black!15!white}

\setbeamertemplate{navigation symbols}{}
%%%%%%%%%%%%%%%%%%%%%%%%%%%%%%%%%%%%%%%%%%%%%%%%%%

\title{Lesson 2: Intercepts, Change and Average Rate of Change}
\author[Francisco Blanco-Silva]{Francisco Blanco-Silva}
\institute[USC]{University of South Carolina}
\date{
\pgfdeclarelindenmayersystem{plant}{
	\rule{X -> F-[[X]+x]+F[+FX]-X}
	\rule{F -> FF}
}
\begin{tikzpicture}[color=DeepSkyBlue4,rotate=74]
    \draw [l-system={plant, axiom=X, order=6, step=0.5pt, angle=15}]
    lindenmayer system; 
	\end{tikzpicture}
	\begin{tikzpicture}[color=DeepSkyBlue4,rotate=74]
    \draw [l-system={plant, axiom=X, order=6, step=0.5pt, angle=25}] lindenmayer system;
	\end{tikzpicture}
	\begin{tikzpicture}[color=DeepSkyBlue4,rotate=74]
    \draw [l-system={plant, axiom=X, order=6, step=0.5pt, angle=30}] lindenmayer system;
	\end{tikzpicture}
}

\begin{document}
\frame{\titlepage}

\section{What do we know?}
\subsection{}

\begin{frame}[t]
\frametitle{What do we know?}
\framesubtitle{The General Program}

\begin{itemize}
\item Background Material (Basic Algebra)
\item Functions
\begin{itemize}
	\item Definition
\end{itemize}
\end{itemize}
\end{frame}

\subsection{}

\begin{frame}[c]\frametitle{Warm-up}
    
\framesubtitle{A somewhat advanced Example}

\begin{example}
\alert{The solid waste $W$ generated each year in the cities of the US is increasing.  The solid waste generated (in millions of tons) was 238.3 in 2000, and 251.3 in 2006. 

Find a formula for this function by using the equation of the line through these two points.}

	\pause We always start indicating independent variable and dependent variable.

	\pause Independent variable: $t$, years after 2000 (let's keep it simple!)

	Dependent variable: $W$, in millions of tons of waste.

	\pause We are searching then for a function of the kind $W = f(t) = \underbrace{a + mt}_{\text{a line!}}$

	\pause 
	\begin{align*}
		a &= f(0) = 238.3 &
		m &= \frac{251.3 - 238.3}{6-0} = \frac{13}{6} \approx 2.167
	\end{align*}
	\pause Solution: 
	\begin{equation*}
		W = f(t) = 238.3 + \tfrac{13}{6}t
	\end{equation*}
\end{example}
\end{frame}

\begin{frame}[c]\frametitle{Warm-up}
    
\framesubtitle{Another somewhat advanced Example}

\begin{example}
\alert{A city's population was 30,700 in the year 2000, and is growing by 850 people a year.}
\begin{itemize}
	\item \alert{Give a formula for the city's population, $P$, as a function of the number of years, $t$, since 2000.}

	\uncover<2->{$P = f(t) = 30700 + 850t$}
	\item \alert{What is the population predicted to be in 2020?}
	\uncover<3->{$f(20) = 30700 + 850 \times 20 = 47700$ people}
	\item \alert{When is the population expected to reach 55,000?}
	\begin{align*}
	\uncover<4->{f(t) &= 55000 \\}
	\uncover<5->{30700 + 850t &= 55000 \\}
	\uncover<6->{850t &= 55000 - 30700 = 24300 \\}
	\uncover<7->{t &= \frac{24300}{850} \approx 28.59\, \text{years}}
	\end{align*}
	\uncover<8>{Solution: The population of this city will reach 55000 sometime during 2028.}
\end{itemize}
\end{example}
\end{frame}

\section{Extracting Basic Information from Functions}

\subsection{}

\begin{frame}[c]\frametitle{Extracting Basic Information from Functions}
    
\framesubtitle{Intercepts, change, average rate of change}

In the previous examples, we have learned of the significance of the intersection of a function $y = f(x)$ with the axes:

\begin{itemize}
	\item<2-> The $y$-intersect indicates the \alert{initial value} of the function.

	\begin{quotation}We obtain $y$-intersects by evaluating $f(0)$.\end{quotation}
	\item<3-> The $x$-intersects (there might be many!) indicate when the function is zero.

	\begin{quotation}We obtain $x$-intersects by solving the equation $f(x) = 0$\end{quotation}
\end{itemize}

\pause\pause\pause Other relevant information we can ask:

\begin{itemize}
	\item<4-> The \alert{change} of $f$ between $x=a$ and $x=b$
	\begin{equation*}
	\Delta y = f(b) - f(a)\, \text{(units = dep.~variable)}
	\end{equation*}
	\item<5-> The \alert{average rate of change} of $f$ between $x=a$ and $x=b$:
	\begin{equation*}
	\frac{\Delta y}{\Delta x} = \frac{f(b) - f(a)}{b-a}\, \text{(units = dep.~variable/ind.~variable)}
	\end{equation*}
\end{itemize}
\end{frame}

\subsection{}

\begin{frame}[c]\frametitle{Examples}

\begin{example}[Winning Height in Men's Olympic Pole Vault]

\begin{center}
\rowcolors{2}{white}{gray!50!white}
	\begin{tabular}{|c||c|c|c|c|c|c|c|}	
	\rowcolor{DeepSkyBlue4}	
	&&&&&&&\\
	Year & 1960 & 1964 & 1968 & \dots & 1992 & 1996 & 2000 \\
	Height (in) & 185 & 201 & 213 & \dots & 228 & 233 & 232
	\end{tabular}
\end{center}

\begin{itemize}
	\item  \alert{What was the change in height from 1960 to 1968?}

	\uncover<2->{$\Delta f = f(b) - f(a)$} \uncover<3->{$ = 213 - 185 = 28$ inches}
	\item \alert{What was the average rate of change in height from 1960 to 1968?}

	\uncover<4->{ARC $= \dfrac{f(b)-f(a)}{b-a}$}
	\uncover<5->{$=\dfrac{213 - 185}{1968-1960} = \dfrac{28}{8} = 3.5\, \text{inches}/\text{year}$}
	\item \alert{What was the change in height from 1992 to 2000?}

	\uncover<6->{$232 - 228 = 4$ inches}
	\item \alert{What was the average rate of change in height from 1992 to 2000?}

	\uncover<7->{$\dfrac{232 - 228}{2000 - 1992} = \dfrac{4}{8} = 0.5\, \text{inches}/\text{year}$}
\end{itemize}
\end{example}
\end{frame}

\begin{frame}[c]\frametitle{Examples}
    
\begin{example}
\alert{High levels of PCB in the environment affect pelicans' eggs.  As the concentration of PCB in environment increases, the thickness of eggshells decreases, making the eggs more likely to break.}

\begin{center}
\rowcolors{2}{white}{gray!50!white}
	\begin{tabular}{|c||c|c|c|c|c|c|}	
	\rowcolor{DeepSkyBlue4}	
	&&&&&&\\
	PCB (parts per million) & 87 & 147 & 204 & 289 & 356 & 452 \\
	thickness (mm) & 0.44 & 0.39 & 0.28 & 0.23 & 0.22 & 0.14
	\end{tabular}
\end{center}
\alert{What is the ARC of the thickness from 87 ppm to 452 ppm?  Explain with words what that means.}
	
\begin{equation*}
\uncover<2->{\text{ARC} = \frac{0.14 - 0.44}{452 - 87} = \frac{-0.3}{365} \approx -0.0008\, \text{mm}/\text{ppm}}
\end{equation*}

\uncover<3->{\only<4>{\textcolor{red}}{The thickness of pelican eggs} \only<5>{\textcolor{red}}{decreases by an average of} \only<6>{\textcolor{red}}{$0.0008$ mm} \only<7>{\textcolor{red}}{for every additional part per million of PCB in the environment.}}

\end{example}

\end{frame}

\begin{frame}[c]\frametitle{Examples}
    
\begin{example}[Tobacco production in the US]
\begin{center}
\begin{tikzpicture}
\draw (0,0) node [scale=0.899999]{
\rowcolors{2}{white}{gray!50!white}
	\begin{tabular}{|c||c|c|c|c|c|c|c|c|}	
	\rowcolor{DeepSkyBlue4}
	&&&&&&&&\\
	Year & 1996 & 1997 & 1998 & 1999 & 2000 & 2001 & 2002 & 2003 \\
	Production & 1517 & 1787 & 1480 & 1293 & 1053 & 991 & 879 & 831  \\
	\rowcolor{white}
	(million pounds) &&&&&&&&
	\end{tabular}};
\end{tikzpicture}
\end{center}
\begin{itemize}
	\item \alert{What is the average rate of change in tobacco production between 1996 and 2003?  Interpret your answer.}

	\uncover<2->{$\text{ARC} = \dfrac{831-1517}{2003-1996} = \dfrac{-686}{7} = -98$ millions of pounds per year}

	\uncover<3->{The production of tobacco in US decreased by an average of 98 millions of pounds every year between 1996 and 2003.}
	\item \alert{During this seven-year period, is there any interval during which the average rate of change was positive?  If so, when?}

	\uncover<4->{Only between 1996 and 1997.}
\end{itemize}
\end{example}

\end{frame}

\end{document}