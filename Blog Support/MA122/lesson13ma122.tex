\PassOptionsToPackage{table}{xcolor}
\documentclass[9pt,xcolor=x11names,compress]{beamer}

%% General document %%%%%%%%%%%%%%%%%%%%%%%%%%%%%%%%%%
\usepackage{graphicx}
\usepackage{tikz}
\usetikzlibrary{decorations.fractals,lindenmayersystems}
\usepackage{wasysym}
%%%%%%%%%%%%%%%%%%%%%%%%%%%%%%%%%%%%%%%%%%%%%%%%%%%%%%


%% Beamer Layout %%%%%%%%%%%%%%%%%%%%%%%%%%%%%%%%%%
\useoutertheme[subsection=false,shadow]{miniframes}
\useinnertheme{default}
\usefonttheme{serif}
\usepackage{palatino}

\setbeamerfont{title like}{shape=\scshape}
\setbeamerfont{frametitle}{shape=\scshape}

\setbeamercolor*{lower separation line head}{bg=DeepSkyBlue4} 
\setbeamercolor*{normal text}{fg=black,bg=white} 
\setbeamercolor*{alerted text}{fg=DeepSkyBlue4} 
\setbeamercolor*{example text}{fg=black} 
\setbeamercolor*{structure}{fg=black} 
 
\setbeamercolor*{palette tertiary}{fg=black,bg=black!10} 
\setbeamercolor*{palette quaternary}{fg=black,bg=black!10} 

\setbeamertemplate{blocks}[rounded][shadow=true]
\setbeamercolor{block title}{bg=DeepSkyBlue4}
\setbeamercolor{block title example}{bg=DeepSkyBlue4}
\setbeamercolor{block body}{bg=black!15!white}
\setbeamercolor{block body example}{bg=black!15!white}

\setbeamertemplate{navigation symbols}{}
%%%%%%%%%%%%%%%%%%%%%%%%%%%%%%%%%%%%%%%%%%%%%%%%%%

\title{Lesson 13: Using the Derivative I}
\author[Francisco Blanco-Silva]{Francisco Blanco-Silva}
\institute[USC]{University of South Carolina}
\date{
\pgfdeclarelindenmayersystem{testrhombus}{
	\rule{F -> F+FF++FF+F+++F++F+++F}
}
\begin{tikzpicture}[color=DeepSkyBlue4]
    \draw [l-system={testrhombus, axiom=F+FF++FF+F+++F++F+++F, order=3, step=1.5pt, angle=90}]
    lindenmayer system; 
	\end{tikzpicture}	
}

\begin{document}
\frame{\titlepage}

\section{What do we know?}
\subsection{}

\begin{frame}
\frametitle{What do we know?}
\framesubtitle{The General Program}
\begin{columns}[T]
\begin{column}{0.6\linewidth}
\begin{itemize}
\item Functions
\begin{itemize}
\item $x-$ and $y-$\alert{intercepts} ($f(x)=0$, $f(0)$)
\item \alert{Change} from $x=a$ to $x=b$ 
\begin{equation*}
	\Delta y = f(b)-f(a)
\end{equation*}
\item \alert{Average Rate of Change} from $x=a$ to $x=b$
\begin{equation*}
ARC=\frac{\Delta y}{\Delta x}=\frac{f(b)-f(a)}{b-a} 
\end{equation*}
\item \alert{Relative Change} from $x=a$ to $x=b$
\begin{equation*}
RC=\frac{\Delta y}{f(a)}=\frac{f(b)-f(a)}{f(a)}
\end{equation*}
\item \alert{Instantaneous Rate of Change} at $x=a$
\begin{equation*}
	f'(a)
\end{equation*}
\item \alert{Relative Rate of Change} at $x=a$
\begin{equation*}
	\frac{f'(a)}{f(a)}
\end{equation*}
\end{itemize}
\end{itemize}
\end{column}
\begin{column}{0.4\linewidth}
\begin{itemize}
	\item Linear Functions: $f(x)=b+mx$
	\item Exponential Functions $P_0 a^t = P_0 (1+r)^t = P_0 e^{kt}$
	\item Power Functions \newline \makebox[1cm]{} $kx^p$
	\item Polynomials $a_0+a_1x+a_2x^2+\dotsb+a_n x^n$
\end{itemize}
\end{column}
\end{columns}
\end{frame}

\section{Using the Derivative}
\subsection{Definitions}

\begin{frame}\frametitle{Using the Derivative}
    
    Taking derivatives of a derivative yields what we call a \alert{second derivative}.  We denote them by $f''(x)$.  \pause Let us compute some:

    \begin{block}
    	{Compute the second derivative of the following functions:}
    	\begin{align*}
    		f(x)&=\pi x^2 - 3ex +13 \uncover<3->{&f'(x)&= 2\pi x - 3e} \\
    		\uncover<4->{&&f''(x)&=2\pi} \\
    		g(x)&=\ln (3x^2-4) \uncover<5->{&g'(x)&=\frac{6x}{3x^2-4} \\}
    		\uncover<6->{&&g''(x)&=\frac{6(3x^2-4)-6x\cdot 6x}{(3x^2-4)^2}}\\
    		h(x)&=\big( x^5 - x^{1/3}\big)^2 \uncover<7->{&h'(x)&=2 \big( x^5 - x^{1/3} \big) \big( 5x^4 - \tfrac{1}{3}x^{-2/3} \big) \\}
    		\uncover<8->{&&&=\big( 2x^5 - 2x^{1/3}\big) \big( 5x^4 - \frac{1}{3}x^{-2/3} \big) \\}
    		\uncover<9->{&&&= 10x^{9} -\frac{32}{3}x^{13/3} + \frac{2}{3}x^{-1/3} \\}
    		\uncover<10->{&&h''(x)&= 90x^8 - \frac{416}{9}x^{10/3} - \frac{2}{9}x^{-4/3}}
    	\end{align*}
    \end{block}
\end{frame}

\begin{frame}\frametitle{Using the Derivative}
\framesubtitle{How does the sign of derivatives affect the shape of a function?}
Because the derivative is linked to the values of slopes of tangent lines, we have the following results:
\begin{center}
\rowcolors{2}{white}{gray!50!white}
	\begin{tabular}{c||c}	
		\rowcolor{DeepSkyBlue4}	
		If the derivative is\dots & then, the function is\dots \\
		$f'(x)>0$ & increasing at $x$ \\
		$f'(x)<0$ & decreasing at $x$ \\
		$f'(x)=0$ & \emph{stationary} at $x$
	\end{tabular}	
\end{center}
\pause We say that $f$ has a \alert{critical point} at $x$ if $f'(x)=0$, or $f'(x)$ is not defined.

\pause The sign of the second derivative also offers a clear interpretation in terms of graph shape.
\begin{center}
\rowcolors{2}{white}{gray!50!white}
	\begin{tabular}{c||c}	
		\rowcolor{DeepSkyBlue4}	
		If the second derivative is\dots & then, the function is\dots \\
		$f''(x)>0$ & concave upwards at $x$ ($\smile$) \\
		$f''(x)<0$ & concave downwards at $x$ ($\frown$) \\
		$f''(x)=0$ & the concavity might change at $x$
	\end{tabular}
\end{center}
\pause We say that $f$ has an \alert{inflection point} at $x$ if the concavity of $f$ changes at $x$.
\end{frame}

\begin{frame}\frametitle{Using the Derivative}
    
\framesubtitle{Maxima and Minima}
\begin{definition}
	We say that a critical point $x=a$ is
	\begin{itemize}
		\item a (local) \alert{minimum} of $f$, if $f(a)\leq f(x)$ for nearby values of $x$.
		\item a (local) \alert{maximum} of $f$, if $f(a) \geq f(x)$ for nearby values of $x$.
	\end{itemize}
\end{definition}
\pause We have two tests to verify whether a critical point is a maximum or a minimum
\pause \begin{block}
	{The First Derivative Test}
	\begin{itemize}
		\item If $f'(c)=0$, $f'(x)<0$ for $x<c$, and $f'(x)>0$ for $x>c$, then $x=c$ is a local minimum.
		\item If $f'(c)=0$, $f'(x)>0$ for $x<c$, and $f'(x)<0$ for $x>c$, then $x=c$ is a local maximum.
	\end{itemize}
\end{block}

\pause \begin{block}
	{The Second Derivative Test}
	\begin{itemize}
		\item If $f'(c)=0$, and $f''(c)>0$, then $x=c$ is a local minimum.
		\item If $f'(c)=0$, and $f''(c)<0$, then $x=c$ is a local maximum.
	\end{itemize}
\end{block}
\end{frame}

\subsection{Examples}

\begin{frame}\frametitle{Using the Derivative}
\framesubtitle{Examples}
\begin{block}
	{Find all critical points and inflection points of the following function}
	\begin{center}
		\begin{tikzpicture}[scale=0.8]
			\draw[thick] (0,0) -- (8,0) node[right]{$t$};
			\draw[thick] (0,0) -- (0,7);
			\draw[DeepSkyBlue4,very thick] (0,0) to [out=80, in=180] (1,3)
												 to [out=0, in=180] (3,2)
												 to [out=0, in=180] (5,4)
												 to [out=0, in=180] (6,3)
												 to [out=0, in=-90] (7,6);
			\draw (6,6) node{$f(t)$};
			\only<2->{
			\filldraw[red] (1,3) circle (2pt) node[above]{M};
			\filldraw[red] (3,2) circle (2pt) node[below]{m};
			\filldraw[red] (5,4) circle (2pt) node[above]{M};
			\filldraw[red] (6,3) circle (2pt) node[below]{m};
			}
			\only<3->{
			\filldraw[black] (2,2.5) circle (2pt);
			\filldraw[black] (4,3) circle (2pt);
			\filldraw[black] (5.5,3.5) circle (2pt);
			}
		\end{tikzpicture}
	\end{center}
\end{block}
\end{frame}


\begin{frame}\frametitle{Using the Derivative}
\framesubtitle{Examples}
\begin{example}
The figure below is a graph of $f'(x)$.  Find the $x$--values that are critical points of the function $f$ itself. Are they local maxima, local minima, or neither?
\begin{center}
	\begin{tikzpicture}
		\draw (-3,0) -- (3,0);
		\draw (0,-1.25) -- (0,2.25);
		\draw (0.125,-1) -- (-0.125,-1) node[left]{$-2$};
		\draw (0.125,1) -- (-0.125,1) node[left]{$2$};
		\draw (0.125,2) -- (-0.125,2) node[left]{$4$};
		\draw (-2,0.125) -- (-2,-0.125) node[below]{$-2$};
		\draw (-1,0.125) -- (-1,-0.125) node[below]{$-1$};
		\draw (1,0.125) -- (1,-0.125) node[below]{$1$};
		\draw (2,0.125) -- (2,-0.125) node[below]{$2$};
		\draw[DeepSkyBlue4,thick] (-1.6,-1.25) to [out=70,in=220] (-1,1)
											   to [out=0, in=180] (0,0.5)
											   to [out=0, in=180] (1,0)
											   to [out=20, in=-110] (1.6,2.25);
		\only<2->{
		\filldraw[red] (-1.45,0) circle (2pt);
		\filldraw[red] (1,0) circle (2pt);
		}
	\end{tikzpicture}
\end{center}
\end{example}
\pause We have two critical points ($f'(x)=0$):
\begin{itemize}
 	\item<3->{One at $c=-1.5$.  Note how $f'(x)<0$ for $x<-1.5$ and $f'(x)>0$ for $x>-1.5$.  It must be a minimum at $c=-1.5$.}
 	\item<4->{Another at $c=1$.  Note how $f'(x)>0$ both before and after $c=1$.  This point is neither maximum nor minimum.}
 \end{itemize} 
\end{frame}

\begin{frame}\frametitle{Using the Derivative}
\framesubtitle{Examples}
\begin{example}
	The function $f(x)=x^4-7x^3+17x$ has a critical point at $x=1$.  Use the second derivative test to identify it as a local maximum or local minimum.
\end{example}
\pause We need to evaluate $f''(1)$.
\begin{align*}
	\uncover<3->{f'(x)&=4x^3 -21 x^2+17 \\}
	\uncover<4->{f''(x)&=12x^2-42x \\}
	\uncover<5->{\alert{f''(1)}&\alert{= 12-42 =-30<0}}
\end{align*}
\uncover<6->{Therefore, $x=1$ is a local maximum of $f$.}

\vspace{2cm}
\end{frame}

\begin{frame}\frametitle{Using the Derivative}
\framesubtitle{Examples}
\begin{example}
	Use the first derivative to find all critical points, and use the second derivative to find all inflection points of the function 
	\begin{equation*}
	f(x)=2x^3+3x^2-180x+3.
	\end{equation*}  
	Identify each critical point as a local maximum, a local minimum, or neither.
\end{example}
\pause We have $f'(x)=6x^2+6x-180$; therefore, the critical points are
\begin{equation*}
	x=\frac{-6\pm\sqrt{36-4\cdot6\cdot(-180)}}{12}=\{-6,5\}
\end{equation*}
\pause To decide whether they are local max or min, we will make use of a sign chart:
\begin{center}
	\begin{tikzpicture}[scale=0.4]
		\uncover<3->{
		\draw[<->] (-10,0) -- (10,0);	
		\draw (-10.9,0) node{$-\infty$};
		\draw (10.7,0) node{$\infty$};
		\foreach \x in {-9,...,9}{\draw (\x,0.125) -- (\x,-0.125); }
		\draw (0,-0.25) node[below]{$0$};}
		\uncover<4->{
		\draw (-10,3) -- (10,3);
		\draw (-6,0) -- (-6,3);
		\draw (5,0) -- (5,3);
		\draw (-6,-0.25) node[below]{$-6$};
		\draw (5,-0.25) node[below]{$5$};
		}
		\uncover<5->{
		\draw (-10,2) node[DeepSkyBlue4,scale=0.6]{$6\cdot(-7)^2+6\cdot(-7)-180=72>0$};
		\draw[->,thick,DeepSkyBlue4] (-9,0.5) -- (-7,1.5);
		}
		\uncover<6->{
		\draw (0,2) node[DeepSkyBlue4,scale=0.6]{$6\cdot 0^2+6\cdot 0 -180 = -180<0$};
		\draw[->,thick,DeepSkyBlue4] (-1,1.5) -- (1,0.5);
		}
		\uncover<7->{
		\draw (8.5,2) node[DeepSkyBlue4,scale=0.6]{$6\cdot 6^2+6\cdot 6 -180 = 72>0$};
		\draw[->,thick,DeepSkyBlue4] (7,0.5) -- (9,1.5);
		}
		\uncover<8->{
		\draw (-4.5,-0.25) node[DeepSkyBlue4,below]{$\leftarrow M$};
		\draw (6.2,-0.25) node[DeepSkyBlue4,below]{$\leftarrow m$};
		}
	\end{tikzpicture}
\end{center}
\end{frame}

\begin{frame}\frametitle{Using the Derivative}
\framesubtitle{Examples}
\begin{example}
	Use the first derivative to find all critical points, and use the second derivative to find all inflection points of the function 
	\begin{equation*}
	f(x)=2x^3+3x^2-180x+3.
	\end{equation*}  
	Identify each critical point as a local maximum, a local minimum, or neither.
\end{example}
To look for inflection points, we need to compute the second derivative:
\begin{align*}
	f(x)&=2x^3+3x^2-180x+3, &f'(x)&=6x^2+6x-180\uncover<2->{, &f''(x)&=12x+6}
\end{align*}
\uncover<3->{The inflection points are among the zeros of the second derivative:}
\begin{align*}
	\uncover<4->{f''(x)&=0}
	\uncover<5->{, &12x+6&=0} 
	\uncover<6->{, &x&=\frac{-6}{12}=-0.5}
\end{align*}
\uncover<6->{To decide whether this is an actual inflection point, we need another sign chart!}
\begin{center}
	\begin{tikzpicture}[scale=0.4]
		\uncover<6->{
		\draw[<->] (-10,0) -- (10,0);	
		\draw (-10.9,0) node{$-\infty$};
		\draw (10.7,0) node{$\infty$};
		\foreach \x in {-9,...,9}{\draw (\x,0.125) -- (\x,-0.125); }
		\draw (-10,2) -- (10,2);
		\draw (-0.5,0) -- (-0.5,2);
		\draw (-0.5,-0.25) node[below]{$-0.5$};
		}
		\uncover<7->{
		\draw (-8,1.5) node[DeepSkyBlue4,scale=0.6]{$12\cdot(-1)+6=-6<0$};
		\draw[thick,DeepSkyBlue4] (-4,0.5) parabola bend (-3, 1.5) (-2,0.5);
		}
		\uncover<8->{
		\draw (7,1.5) node[DeepSkyBlue4,scale=0.6]{$12\cdot 0 +6=6>0$};
		\draw[thick,DeepSkyBlue4] (0.5,1.5) parabola bend (1.5,0.5) (2.5,1.5);
		\draw[DeepSkyBlue4] (4,-0.25) node[below]{$\leftarrow \text{inflection point}$};
		}
	\end{tikzpicture}
\end{center}
\end{frame}

\begin{frame}\frametitle{Using the Derivative}
\framesubtitle{Examples}
\begin{example}
	Find constant $a$ and $b$ so that the minimum of the parabola $f(x)=x^2+ax+b$ is at the point $(6,2)$.
\end{example}
\pause If the point $(6,2)$ is in the graph of $f$, it must be $f(6)=2$.  This gives us a first condition on the unknown parameters $a$ and $b$:
\begin{align*}
	\uncover<3->{f(6)&=2}
	\uncover<4->{, &6^2+6a+b&=2}
	\uncover<5->{, &\alert{6a+b}&\alert{=-34}}
\end{align*}
\uncover<6->{If the point $(6,2)$ is to be a minimum, in particular it must be a critical point: this gives a second condition on $a$ and $b$: $f'(6)=0$}
\begin{align*}
	\uncover<7->{f'(x)&=2x+a \\}
	\uncover<8->{f'(6)&=0}
	\uncover<8->{, &2\cdot 6 + a &=0}
	\uncover<9->{, &\alert{a} &\alert{= -12}}
\end{align*}
\uncover<10->{Two conditions are enough to find the value of two unknowns:}
\begin{equation*}
	\uncover<11->{
	\begin{cases}
		6a+b=-34 \\ a=-12
	\end{cases}\qquad}
	\uncover<12->{
	\begin{cases}
		a=-12\\ b=-34+6\cdot 12=38
	\end{cases}\qquad
	}
\end{equation*}
\end{frame}

\begin{frame}\frametitle{Using the Derivative}
\framesubtitle{Examples}
\begin{example}
	Find constant $a$ and $b$ so that the minimum of the parabola $f(x)=x^2+ax+b$ is at the point $(6,2)$.
\end{example}    
The equation of the parabola is then $f(x)=x^2-12x+38$.  It is easy to check that $(6,2)$ is indeed a minimum (just in case!)
\begin{align*}
	\uncover<2->{f(x)&=x^2-12x+38}
	\uncover<3->{, &f'(x)&=2x-12}
	\uncover<4->{, &f''(x)&=2}
\end{align*}
\uncover<5->{If $(6,2)$ is to be a minimum, it should be $f''(6)>0$.  This is obviously satisfied.  We are good.}

\vspace{3cm}
\end{frame}

\begin{frame}\frametitle{Using the Derivative}
\framesubtitle{Examples}
\begin{example}
	For what values of $a$ and $b$ does $f(x)=a(x-b\ln x)$ have a local extremum at the point $(5,8)$?
\end{example}
\pause  We need two conditions, because we have two unknowns.  The first comes from the fact that the point $(5,8)$ is in the graph of $f$:
\begin{align*}
	\uncover<3->{f(5)&=8}
	\uncover<4->{, &\alert{a(5-b\ln 5)}&\alert{=8}}
\end{align*}
\uncover<5->{The second comes from the fact that $(5,8)$ is a critical point: $f'(5)=0$}
\begin{align*}
	\uncover<6->{f'(x)&=a\Big( 1 - \frac{b}{x} \Big)}
	\uncover<7->{, &f'(5)&=0}
	\uncover<8->{, &\alert{a \Big( 1 - \frac{b}{5}\Big)} &\alert{= 0}}
\end{align*}
\uncover<9->{Let us solve for $a$ and $b$.  The second condition looks easier: it can only be $a=0$ or $b=5$}
\begin{itemize}
	\item<10-> If $a=0$, then the first condition is not satisfied, so we need to disregard this option.
	\item<11-> If \alert{$b=5$}, the first condition turns into
	\begin{equation*}
		a(5-5\ln 5)=8
		\uncover<12->{,\qquad \alert{a}=\frac{8}{5-5\ln 5}\approx \alert{-2.625369982}}
	\end{equation*}
\end{itemize}
\end{frame}
\end{document}