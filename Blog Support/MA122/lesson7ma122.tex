\PassOptionsToPackage{table}{xcolor}
\documentclass[9pt,xcolor=x11names,compress]{beamer}

%% General document %%%%%%%%%%%%%%%%%%%%%%%%%%%%%%%%%%
\usepackage{graphicx}
\usepackage{tikz}
\usetikzlibrary{decorations.fractals,lindenmayersystems}
%%%%%%%%%%%%%%%%%%%%%%%%%%%%%%%%%%%%%%%%%%%%%%%%%%%%%%


%% Beamer Layout %%%%%%%%%%%%%%%%%%%%%%%%%%%%%%%%%%
\useoutertheme[subsection=false,shadow]{miniframes}
\useinnertheme{default}
\usefonttheme{serif}
\usepackage{palatino}

\setbeamerfont{title like}{shape=\scshape}
\setbeamerfont{frametitle}{shape=\scshape}

\setbeamercolor*{lower separation line head}{bg=DeepSkyBlue4} 
\setbeamercolor*{normal text}{fg=black,bg=white} 
\setbeamercolor*{alerted text}{fg=DeepSkyBlue4} 
\setbeamercolor*{example text}{fg=black} 
\setbeamercolor*{structure}{fg=black} 
 
\setbeamercolor*{palette tertiary}{fg=black,bg=black!10} 
\setbeamercolor*{palette quaternary}{fg=black,bg=black!10} 

\setbeamertemplate{blocks}[rounded][shadow=true]
\setbeamercolor{block title}{bg=DeepSkyBlue4}
\setbeamercolor{block title example}{bg=DeepSkyBlue4}
\setbeamercolor{block body}{bg=black!15!white}
\setbeamercolor{block body example}{bg=black!15!white}

\setbeamertemplate{navigation symbols}{}
%%%%%%%%%%%%%%%%%%%%%%%%%%%%%%%%%%%%%%%%%%%%%%%%%%

\title{Lesson 7: Power Functions.  Basic Operations with Functions}
\author[Francisco Blanco-Silva]{Francisco Blanco-Silva}
\institute[USC]{University of South Carolina}
\date{
\pgfdeclarelindenmayersystem{Sierpinski triangle}{
\rule{F -> G-F-G}
\rule{G -> F+G+F}}
	\begin{tikzpicture} 
    \shadedraw [top color=DeepSkyBlue4, bottom color=black, draw=DeepSkyBlue4]
[l-system={Sierpinski triangle, step=1pt, angle=60, axiom=F, order=8}]
lindenmayer system -- cycle;
	\end{tikzpicture}
}

\begin{document}
\frame{\titlepage}

\section{What do we know?}
\begin{frame}
\frametitle{What do we know?}
\begin{columns}[T]
\begin{column}{0.6\linewidth}
\begin{itemize}
\item Functions
\begin{itemize}
\item $x-$ and $y-$intercepts ($f(x)=0$, $f(0)$)
\item Change from $x=a$ to $x=b$ 
\begin{equation*}
	\Delta y = f(b)-f(a)
\end{equation*}
\item Average Rate of Change from $x=a$ to $x=b$
\begin{equation*}
ARC=\frac{\Delta y}{\Delta x}=\frac{f(b)-f(a)}{b-a} 
\end{equation*}
\item Relative Change from $x=a$ to $x=b$
\begin{equation*}
RC=\frac{\Delta y}{f(a)}=\frac{f(b)-f(a)}{f(a)}
\end{equation*}
\end{itemize}
\end{itemize}
\end{column}
\begin{column}{0.4\linewidth}
\begin{itemize}
	\item Linear Functions: $f(x)=b+mx$
	\item Exponential Functions $P_0 a^t = P_0 (1+r)^t = P_0 e^{kt}$
\end{itemize}
\end{column}
\end{columns}
\end{frame}

\section{Power Functions}
\subsection{Definition}
\begin{frame}\frametitle{Power Functions}
\framesubtitle{Definitions}
We say that $Q(x)$ is a \alert{power function} of $x$ if $Q(x)$ is proportional to a constant power of $x$:
\begin{equation*}
	Q(x)=kx^p
\end{equation*}
The coefficient $k$ is called \emph{constant of proportionality}, and $p$ is the \emph{power}.
\pause
\begin{block}{Which of these functions are power functions?}
\begin{align*}
	y&=5 \sqrt{x}\uncover<3->{\alert{=5x^{1/2}}} \uncover<3->{& k&=5 &p&=\tfrac{1}{2} \\}
	y&=2x\uncover<4->{\alert{=2x^1}} \uncover<4->{& k&=2 &p&=1 \\}
	y&=\frac{3}{x}\uncover<5->{\alert{=3x^{-1}}} \uncover<5->{&k&=3 &p&=-1 \\}
	y &= \big( 3x^5 \big)^2\uncover<6->{\alert{=9x^{10}}} \uncover<6->{&k&=3^2=9 &p&=10 \\}
	y&= \frac{3}{8x}\uncover<7->{\alert{=\tfrac{3}{8}x^{-1}}} \uncover<7->{&k&=\frac{3}{8} &p&=-1 \\}
	y&= \frac{5}{2\sqrt{x}}\uncover<8->{\alert{=\tfrac{5}{2}x^{-1/2}}} \uncover<8->{&k&=\frac{5}{2} &p&=-\tfrac{1}{2} \\}
	y&=\pi\uncover<9->{\alert{=\pi x^0}} \uncover<9->{& k&=\pi &p&=0}
\end{align*}
\end{block}
\end{frame}

\begin{frame}\frametitle{Power Functions}
\framesubtitle{Definitions}
Sums of power functions with non-negative integer exponent are called \alert{polynomials}
\begin{equation*}
	y=p(x)=\alert{a_nx^n}+a_{n-1}x^{n-1}+\dotsb+a_1x+a_0
\end{equation*}
The highest of the powers, $n$, is called the \alert{degree} of the polynomial.  The power function with that highest power, $a_nx^n$, is called the \alert{leading term} of the polynomial, and the corresponding coefficient $a_n$ is called the \alert{leading coefficient}.
\pause
\begin{example}
	The following are polynomials.  What are their degrees and leading coefficients?
	\begin{align*}
		y&=\pi \uncover<3->{&n&=0 &a_0&=\pi\\}
		y&=1-2x+3x^4-5x^6 \uncover<4->{&n&=6 &a_6&=-5 \\}
		y&=\tfrac{1}{2}x^3 - \tfrac{3}{4}x^5 + \tfrac{5}{6}x^7 \uncover<5->{&n&=7 &a_7&=\tfrac{5}{6} }
	\end{align*}
\end{example}
\end{frame}

\section{Operations with Functions}
\subsection{Tables}
\begin{frame}\frametitle{Operations with functions}
\framesubtitle{Vertical Operations}
\begin{center}
\rowcolors{1}{gray!25}{gray!25}
   \begin{tabular}{c||l} 
   \rowcolor{DeepSkyBlue4} &\\
   \rowcolor{DeepSkyBlue4}
   \textbf{Operation on }$f(x)$ & \textbf{Effect on the graph} \\
   \rowcolor{DeepSkyBlue4} & \pause \\ 
   & \\ $f(x)+C\qquad (C>0)$ & Vertical shift (up) by $C$ units \\
   & \pause \\ 
   \rowcolor{white} &\\ \rowcolor{white}
   $f(x)-C\qquad (C>0)$ & Vertical shift (down) by $C$ units \\
   \rowcolor{white} & \pause \\ 
   &\\ $a f(x)\qquad (a>1)$ & Vertical stretching by a factor of $a$ \\
   & \pause \\ 
   \rowcolor{white} &\\
   \rowcolor{white}
   $a f(x)\qquad (0<a<1)$ & Vertical compression by a factor of $1/a$ \\
   \rowcolor{white} & \pause \\ 
   &\\ $-f(x)$ & Reflection with respect to the $x$--axis  \\ & \\
   \end{tabular}
   \end{center}
\end{frame}

\subsection{Examples}

\begin{frame}\frametitle{Operations with functions}
\framesubtitle{Vertical Operations}
\begin{example}
Given the graph of the function $f(x)$ below, sketch the graph of the function $2f(x)-3$
\end{example}
\begin{center}
\begin{tikzpicture}
\draw[step=0.25, help lines] (-2,-2) grid (2,2);
\draw(0,-2)--(0,2);
\draw(-2,0)--(2,0);
\foreach \x in {-2,-1.75,-1.5,...,2}{
\draw (-0.1,\x)--(0.1,\x);
\draw (\x, -0.1)-- (\x, 0.1);
}
\draw[DeepSkyBlue4,ultra thick](-2,0.5)--(-1,1)--(0,0)--(1,1)--(2,1);
\only<2->{
	\draw (2.5,1) node[DeepSkyBlue4
]{$f(x)$};
	\draw(3,-2.5) node[DeepSkyBlue4]{$f(x) \to 2f(x) \to 2f(x)-3$};}
\only<1>{
	\draw(-2,0.3) node[scale=0.8,DeepSkyBlue4
]{(-8,2)};
	\draw(-1,1.2) node[scale=0.8,DeepSkyBlue4
]{(-4,4)};
	\draw(0,-0.2) node[scale=0.8,DeepSkyBlue4
]{(0,0)};
	\draw(1,1.2) node[scale=0.8,DeepSkyBlue4
]{(4,4)};
	\draw(2,1.2) node[scale=0.8,DeepSkyBlue4
]{(8,4)};
}
\begin{scope}[xshift=6cm]
\draw[step=0.25, help lines] (-2,-2) grid (2,2);
\draw(0,-2)--(0,2);
\draw(-2,0)--(2,0);
\foreach \x in {-2,-1.75,-1.5,...,2}{
\draw (-0.1,\x)--(0.1,\x);
\draw (\x, -0.1)-- (\x, 0.1);
}
\only<4>{\draw(2.5,2) node[DeepSkyBlue4]{$2f(x)$};}
\only<3>{
	\draw(-2,0.8) node[scale=0.8,DeepSkyBlue4
]{(-8,4)};
	\draw(-1,2.2) node[scale=0.8,DeepSkyBlue4
]{(-4,8)};
	\draw(0,-0.2) node[scale=0.8,DeepSkyBlue4
]{(0,0)};
	\draw(1,2.2) node[scale=0.8,DeepSkyBlue4
]{(4,8)};
	\draw(2,2.2) node[scale=0.8,DeepSkyBlue4
]{(8,8)};
}
\only<3-4>{
\draw[red,ultra thin](-2,0.5)--(-1,1)--(0,0)--(1,1)--(2,1);
\draw[DeepSkyBlue4,ultra thick] (-2,1)--(-1,2)--(0,0)--(1,2)--(2,2);
}
\only<5>{
\draw[red,ultra thin] (-2,1)--(-1,2)--(0,0)--(1,2)--(2,2);
\draw[DeepSkyBlue4,ultra thick] (-2,0.25)--(-1,1.25)--(0,-0.75)--(1,1.25)--(2,1.25);
\draw (2.75,1.25) node[DeepSkyBlue4]{$2f(x)-3$};
}
\end{scope}
\end{tikzpicture}
\end{center}	
\end{frame}

\begin{frame}\frametitle{Operations with functions}
\framesubtitle{Horizontal Operations}
\begin{center}
\rowcolors{1}{gray!25}{gray!25}
   \begin{tabular}{c||l} 
   \rowcolor{DeepSkyBlue4} &\\
   \rowcolor{DeepSkyBlue4}
   \textbf{Operation on }$f(x)$ & \textbf{Effect on the graph} \\
   \rowcolor{DeepSkyBlue4} & \pause \\ 
   & \\ $f(x+C)\qquad (C>0)$ & Horizontal shift (\alert{left}) by $C$ units \\
   & \pause \\ 
   \rowcolor{white} &\\ \rowcolor{white}
   $f(x-C)\qquad (C>0)$ & Horizontal shift (\alert{right}) by $C$ units \\
   \rowcolor{white} & \pause \\ 
   &\\ $f(ax)\qquad (a>1)$ & Horizontal compression by a factor of $1/a$ \\
   & \pause \\ 
   \rowcolor{white} &\\
   \rowcolor{white}
   $f(x/a)\qquad (0<a<1)$ & Horizontal stretching by a factor of $a$ \\
   \rowcolor{white} & \pause \\ 
   &\\ $f(-x)$ & Reflection with respect to the $y$--axis  \\ & \\
   \end{tabular}
   \end{center}
\end{frame}

\begin{frame}\frametitle{Operations with functions}
\framesubtitle{Horizontal Operations}
\begin{example}
Given the graph of the function $f(x)$ below, sketch the graph of the function $2-f(2x)$
\end{example}
\begin{center}
\begin{tikzpicture}
\draw[step=0.25, help lines] (-2,-2) grid (2,2);
\draw(0,-2)--(0,2);
\draw(-2,0)--(2,0);
\foreach \x in {-2,-1.75,-1.5,...,2}{
\draw (-0.1,\x)--(0.1,\x);
\draw (\x, -0.1)-- (\x, 0.1);
}
\draw[DeepSkyBlue3,ultra thick](-2,0.5)--(-1,1)--(0,0)--(1,1)--(2,1);
\draw (2.5,1) node[DeepSkyBlue3]{$f(x)$};
\only<2->{
	\draw(3,-2.5) node[DeepSkyBlue4]{$f(x) \to f(2x) \to -f(2x) \to -f(2x)+2$};}
\begin{scope}[xshift=6cm]
\draw[step=0.25, help lines] (-2,-2) grid (2,2);
\draw(0,-2)--(0,2);
\draw(-2,0)--(2,0);
\foreach \x in {-2,-1.75,-1.5,...,2}{
\draw (-0.1,\x)--(0.1,\x);
\draw (\x, -0.1)-- (\x, 0.1);
}
\only<3>{
	\draw(-1.5,0.5) node[scale=0.8,DeepSkyBlue3]{(-4,2)};
	\draw(-0.5,1.2) node[scale=0.8,DeepSkyBlue3]{(-2,4)};
	\draw(0,-0.2) node[scale=0.8,DeepSkyBlue3]{(0,0)};
	\draw(0.5,1.2) node[scale=0.8,DeepSkyBlue3]{(2,4)};
	\draw(1.25,1.2) node[scale=0.8,DeepSkyBlue3]{(4,4)};
}
\only<3-4>{
	\draw(1.5,0.6) node[DeepSkyBlue3]{$f(2x)$};
	\draw[red,ultra thin](-2,0.5)--(-1,1)--(0,0)--(1,1)--(2,1);
	\draw[DeepSkyBlue3,ultra thick] (-1,0.5)--(-0.5,1)--(0,0)--(0.5,1)--(1,1);
}
\only<5>{
	\draw[red,ultra thin] (-1,0.5)--(-0.5,1)--(0,0)--(0.5,1)--(1,1);
	\draw[DeepSkyBlue3,ultra thick] (-1,-0.5)--(-0.5,-1)--(0,0)--(0.5,-1)--(1,-1);
	\draw (1.5,-0.6) node[DeepSkyBlue3]{$-f(2x)$};
}
\only<6>{
	\draw[red,ultra thin] (-1,-0.5)--(-0.5,-1)--(0,0)--(0.5,-1)--(1,-1);
	\draw[DeepSkyBlue3,ultra thick] (-1,0)--(-0.5,-0.5)--(0,0.5)--(0.5,-0.5)--(1,-0.5);
	\draw (1.5,-1.1) node[DeepSkyBlue3]{$-f(2x)+2$};
}
\end{scope}
\end{tikzpicture}
\end{center}
\end{frame}

\begin{frame}\frametitle{Operations with functions}
\framesubtitle{Horizontal Operations}
\begin{example}
Given the graph of the function $f(x)$ below, sketch the graph of the function $f(4x-2)$
\end{example}
\begin{center}
\begin{tikzpicture}
\draw[step=0.25, help lines] (-2,-2) grid (2,2);
\draw(0,-2)--(0,2);
\draw(-2,0)--(2,0);
\foreach \x in {-2,-1.75,-1.5,...,2}{
\draw (-0.1,\x)--(0.1,\x);
\draw (\x, -0.1)-- (\x, 0.1);
}
\draw[DeepSkyBlue4,ultra thick](-2,0.5)--(-1,1)--(0,0)--(1,1)--(2,1);
\draw (2.5,1) node[DeepSkyBlue4]{$f(x)$};
\only<2->{
	\draw(3,-2.5) node[DeepSkyBlue4]{$f(x) \to f(4x) \to f(4x-2)$};}
\begin{scope}[xshift=6cm]
\draw[step=0.25, help lines] (-2,-2) grid (2,2);
\draw(0,-2)--(0,2);
\draw(-2,0)--(2,0);
\foreach \x in {-2,-1.75,-1.5,...,2}{
\draw (-0.1,\x)--(0.1,\x);
\draw (\x, -0.1)-- (\x, 0.1);
}
\only<3>{
	\draw[red,ultra thin](-2,0.5)--(-1,1)--(0,0)--(1,1)--(2,1);
	\draw[DeepSkyBlue4
,ultra thick] (-0.5,0.5)--(-0.25,1)--(0,0)--(0.25,1)--(0.5,1);
	\draw (1.5,0.6) node[DeepSkyBlue4
]{$f(4x)$};
}
\only<4>{
	\draw (2,0.6) node[DeepSkyBlue4
]{$f(4x-2)$};
	\draw[red,ultra thin] (-0.5,0.5)--(-0.25,1)--(0,0)--(0.25,1)--(0.5,1);
	\draw[DeepSkyBlue4
,ultra thick] (0,0.5)--(0.25,1)--(0.5,0)--(0.75,1)--(1,1);
}
\end{scope}
\end{tikzpicture}
\end{center}
\end{frame}

\begin{frame}\frametitle{Operations with functions}
\framesubtitle{Examples}
\begin{example}
	Write an equation for a graph obtained by \alert{vertically stretching} the graph of $y=x^3$ by a factor of 3, followed by a \alert{vertical upward shift} of 2 units.
\end{example}
\pause Solution: $y=3x^3+2$
\begin{align*}
	\text{Original function} && \text{After vertical stretch} && \text{After shift up} \\
	x^3 && \uncover<3->{3x^3} && \uncover<4->{3x^3+2}
\end{align*}
\pause\pause
\begin{example}
	What is the equation if the order of the transformations is interchanged?
\end{example}
\pause Solution: $y=3\big( x^3+2 \big)$
\begin{align*}
	\text{Original function} && \text{After shift up} && \text{After vertical stretch} \\
	x^3 && x^3+2 && 3\big( x^3+2 \big)
\end{align*}
\end{frame}

\end{document}