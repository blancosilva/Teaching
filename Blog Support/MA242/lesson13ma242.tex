\documentclass[9pt,xcolor=x11names,compress]{beamer}

%% General document %%%%%%%%%%%%%%%%%%%%%%%%%%%%%%%%%%
\usepackage{graphicx}
\usepackage{tikz}
\usetikzlibrary{decorations.fractals,lindenmayersystems}
%%%%%%%%%%%%%%%%%%%%%%%%%%%%%%%%%%%%%%%%%%%%%%%%%%%%%%


%% Beamer Layout %%%%%%%%%%%%%%%%%%%%%%%%%%%%%%%%%%
\useoutertheme[subsection=false,shadow]{miniframes}
\useinnertheme{default}
\usefonttheme{serif}
\usepackage{palatino}

\setbeamerfont{title like}{shape=\scshape}
\setbeamerfont{frametitle}{shape=\scshape}

\setbeamercolor*{lower separation line head}{bg=DeepSkyBlue4} 
\setbeamercolor*{normal text}{fg=black,bg=white} 
\setbeamercolor*{alerted text}{fg=DeepSkyBlue4} 
\setbeamercolor*{example text}{fg=black} 
\setbeamercolor*{structure}{fg=black} 

\setbeamercolor*{palette tertiary}{fg=black,bg=black!10} 
\setbeamercolor*{palette quaternary}{fg=black,bg=black!10} 

\setbeamertemplate{blocks}[rounded][shadow=true]
\setbeamercolor{block title}{bg=DeepSkyBlue4}
\setbeamercolor{block title example}{bg=DeepSkyBlue4}
\setbeamercolor{block body}{bg=black!15!white}
\setbeamercolor{block body example}{bg=black!15!white}

\setbeamertemplate{navigation symbols}{}
%%%%%%%%%%%%%%%%%%%%%%%%%%%%%%%%%%%%%%%%%%%%%%%%%%

\author[Francisco Blanco-Silva]{Francisco Blanco-Silva}
\institute[USC]{University of South Carolina}
\date{
\pgfdeclarelindenmayersystem{Funny curve}{
\rule{X -> X+YF+}
  \rule{Y -> -FX-Y}}
	\begin{tikzpicture} 
    \draw [DeepSkyBlue4]
    [l-system={Funny curve, axiom=X, order=11, step=2.5pt, angle=90}]
    lindenmayer system; 
	\end{tikzpicture}
}
\title{Lesson 13: The General Second-Order Linear Equations with Constant Coefficients: Variation of Parameters}

\begin{document}

\frame{\titlepage}

\section{Warm up}
\subsection{Integration by Parts}
\begin{frame}\frametitle{Warm-up}
\framesubtitle{Remembering Tricks using Integrations by Parts}
   \begin{block}
    	{Compute the following integral:}
    	\begin{equation*}
    		\int e^x\sin x\, dx
    	\end{equation*}
    \end{block} 
    \pause We start with integration by parts:
    \begin{align*}
    	u&=e^x & dv&=\sin x \\
    	du&=e^x &v&= -\cos x
    \end{align*}
    It is then
    \begin{equation*}
    	\int e^x\sin x\, dx = -e^x\cos x + \int e^x\cos x\, dx
    \end{equation*}
    \pause This new integral, can be also done by parts:
    \begin{align*}
    	u&=e^x &dv&=\cos x
    \end{align*}
    This gives us
    \begin{align*}
    	\alert{\int e^x\sin x} &= -e^x\cos x + \Big( e^x\sin x - \int e^x\sin x\, dx\Big) \\
    	&= -e^x\cos x + e^x\sin x - \alert{\int e^x\sin x\, dx}
    \end{align*}
\end{frame}

\begin{frame}\frametitle{Warm-up}
\framesubtitle{Remembering Tricks using Integrations by Parts}
   \begin{block}
    	{Compute the following integral:}
    	\begin{equation*}
    		\int e^x\sin x\, dx
    	\end{equation*}
    \end{block} 
    This means, that it must be
    \begin{equation*}
    2 \int e^x\sin x\, dx = -e^x\cos x + e^x\sin x,
    \end{equation*}
    or simplifying,
    \begin{equation*}
    \int e^x\sin x\, dx = \tfrac{1}{2} e^x \big( \sin x - \cos x \big)
    \end{equation*}

    \vspace{2.5cm}
\end{frame}

\subsection{Cramer's Rule}
\begin{frame}\frametitle{Warm-up}
\framesubtitle{Cramer's Rule}
Given the system of $2$ equations with $2$ variables
\begin{equation*}
	\begin{cases}
		a_{1,1} x + a_{2,1} y =b_1 \\
		a_{1,2} x + a_{2,2} y =b_2
	\end{cases}
\end{equation*}
We can code it in matrix form:
\begin{equation*}
	\underbrace{\begin{bmatrix}
	a_{1,1} & a_{2,1} \\
	a_{1,2} & a_{2,2}
	\end{bmatrix}}_{A} \cdot
	\begin{bmatrix}
		x\\ y 
	\end{bmatrix} = 
	\begin{bmatrix}
		b_1 \\ b_2 
	\end{bmatrix}
\end{equation*}
If the determinant of $A$ is not zero, then we may write the solutions to this system using Cramer's rule as
\begin{align*}
	x &= \frac{1}{\det A} \begin{vmatrix}
		b_1 & a_{2,1} \\ b_2 & a_{2,2}
	\end{vmatrix} &
	y &= \frac{1}{\det A} \begin{vmatrix}
		a_{1,1} & b_1 \\ a_{1,2} & b_2
	\end{vmatrix}
\end{align*}
\end{frame}

\begin{frame}\frametitle{Warm-up}
\framesubtitle{Cramer's Rule: Example}
\begin{block}
	{Use Cramer's rule to solve the following system}
	\begin{equation*}
		\begin{cases}
			4x - 3y = 7 \\
			3x + 5y = -2
		\end{cases}
	\end{equation*}
\end{block}
\pause We code it in matrix form:
\begin{equation*}
	\underbrace{\begin{bmatrix}
		4 & -3 \\ 3 & 5
	\end{bmatrix}}_{A}
	\begin{bmatrix}
		x \\ y
	\end{bmatrix} = 
	\begin{bmatrix}
		7 \\ -2
	\end{bmatrix}
\end{equation*}
\pause Note that the determinant of $A$ is non-zero:
\begin{equation*}
	\det A = \begin{vmatrix}
		4 & -3 \\ 3 & 5
	\end{vmatrix} = 4\cdot 5- (-3)\cdot 3 = 29
\end{equation*}
\pause The solution of the system is then
\begin{align*}
	x&=\frac{1}{29} \begin{vmatrix}
		\alert{7} & -3 \\ \alert{-2} & 5
	\end{vmatrix} = \frac{7\cdot 5 - (-2)(-3) }{29}=\frac{29}{29}=1 \\
	\uncover<5->{y&= \frac{1}{29} \begin{vmatrix}
		4 & \alert{7} \\ 3 & \alert{-2}
	\end{vmatrix} = \frac{4\cdot (-2) - 7\cdot 3}{29} = -\frac{29}{29}=-1}
\end{align*}
\end{frame}


\section{What do we know?}
\begin{frame}\frametitle{What do we know?}
\begin{columns}[T]
\begin{column}{0.45\linewidth}
\begin{itemize}
\item The concepts of \alert{differential equation} and \alert{initial value problem}
\item The concept of \alert{order} of a differential equation.
\item The concepts of \alert{general solution}, \alert{particular solution} and \alert{singular solution}.
\item \alert{Slope fields}
\item Approximations to solutions via \alert{Euler's Method} and \alert{Improved Euler's Method}
\end{itemize} 
\end{column}
\begin{column}{0.55\linewidth}
\begin{itemize}
\item First-Order Differential Equations
\begin{itemize}
\item Separable equations 
\item Homogeneous First-Order Equations 
\item Linear First-Order Equations 
\item Bernoulli Equations 
\item General Substitution Methods
\item Exact Equations 
\end{itemize}
\item Second-Order Differential Equations
\begin{itemize}
	\item Reducible Equations
	\item Linear Equations (Intro)
	\item Homogeneous with Constant Coefficients
\end{itemize}
\end{itemize}
\end{column}
\end{columns}
\end{frame}

\section[Variation of Parameters]{Second-Order Linear Equations with Constant Coefficients: Variation of Parameters}
\subsection{The Formula}

\begin{frame}\frametitle{The Method of Variation of Parameters}
\framesubtitle{The Formulas}
   Consider now the \alert{non-homogeneous} linear second-order differential equation with constant coefficients:
   \begin{equation*}
   	ay''+by'+cy=f(x)
   \end{equation*}
\pause The solution of this equation comes in the form
\begin{equation*}
   	y=A(x)y_1(x)+B(x)y_2(x),
   \end{equation*}   
   where $y_1$ and $y_2$ are the solutions of the homogeneous equation $ay''+by'+cy=0$ that we found in the previous lecture, and the functions $A(x), B(x)$ are computed as follows:
   \begin{align*}
   	A(x) &= -\frac{1}{a} \int \frac{y_2(x) f(x)}{W(y_1,y_2)}\, dx 
   	&B(x) &= \frac{1}{a} \int \frac{y_1(x) f(x)}{W(y_1,y_2)}\, dx.
   \end{align*}
   Note that there will be a different constant after each integration.
\end{frame}

\begin{frame}\frametitle{The Method of Variation of Parameters}
\framesubtitle{The Formulas}
This is then the algorithm to solve non-homogeneous second-order linear equations with constant coefficients:
\begin{equation*}
	ay''+by'+cy=f(x)
\end{equation*}
\pause
\begin{description}[<+->]
	\item[\textbf{Step \#1:}] Find two solutions $y_1(x)$, $y_2(x)$ of the homogeneous equation $ay''+by'+cy=0$, using the technique from last lecture.
	\item[\textbf{Step \#2:}] Compute their Wronkskian $W(y_1,y_2)$ (we know it is never zero!).
	\item[\textbf{Step \#3:}] Compute the integrals 
   \begin{align*}
   	A(x) &= -\frac{1}{a} \int \frac{y_2(x) f(x)}{W(y_1,y_2)}\, dx 
   	&B(x) &= \frac{1}{a} \int \frac{y_1(x) f(x)}{W(y_1,y_2)}\, dx.
   \end{align*}
	\item[\textbf{Step \#4:}] The solution is then
	\begin{equation*}
		y=A(x)y_1(x)+B(x)y_2(x)	
	\end{equation*}
\end{description}
\end{frame}

\subsection{Examples}

\begin{frame}\frametitle{The Method of Variation of Parameters}
\framesubtitle{Examples}
\begin{block}
	{Solve the differential equation}
	\begin{equation*}
		y''+3y'+2y=4e^x
	\end{equation*}
\end{block}
\pause
\begin{itemize}[<+->]
	\item Solve the homogeneous equation $y''+3y'+2y=0$:
	\begin{equation*}
		r^2+3r+2=0, \qquad r=\frac{-3\pm \sqrt{9-4\cdot 2}}{2}=\frac{-3\pm 1}{2}=\{-2,-1\}
	\end{equation*}
	We have $y_1(x)=e^{-x}$, $y_2(x)=e^{-2x}$.
	\item Compute the Wronskian:
	\begin{equation*}
		W(y_1,y_2)= \begin{vmatrix}
			e^{-x} & e^{-2x} \\ -e^{-x} & -2e^{-2x}
		\end{vmatrix} = -2e^{-x}e^{-2x} + e^{-x}e^{-2x} = -e^{-3x}
	\end{equation*}
\end{itemize}
\end{frame}

\begin{frame}\frametitle{The Method of Variation of Parameters}
\framesubtitle{Examples}
\begin{block}
	{Solve the differential equation}
	\begin{equation*}
		y''+3y'+2y=4e^x
	\end{equation*}
\end{block}
\begin{itemize}
	\item Compute the \emph{parameter} functions $A$ and $B$:
	\begin{align*}
		A(x)&= -\int \frac{y_2(x) f(x)}{W(y_1,y_2)}\, dx = -\int \frac{e^{-2x}\, 4e^{x}}{-e^{-3x}}\, dx \uncover<2->{= \int 4e^{2x}\, dx = 2e^{2x}+\alert{A}} \\
		\uncover<3->{B(x) &= \int \frac{y_1(x) f(x)}{W(y_1,y_2)}\, dx = \int \frac{e^{-x}\, 4e^{x}}{-e^{-3x}}\, dx} \uncover<4->{= \int -4e^{3x}\, dx = -\tfrac{4}{3}e^{3x}+\alert{B}}
	\end{align*}
	\item<5-> The solution is then
	\begin{align*}
		y&=\big( 2e^{2x}+A\big)e^{-x} + \big(-\tfrac{4}{3}e^{3x}+B\big)e^{-2x}\\
		\uncover<6->{&=2e^x-\tfrac{4}{3}e^x+Ae^{-x}+Be^{-2x} \\}
		\uncover<7->{&= \alert{A\underbrace{e^{-x}}_{y_1}+B\underbrace{e^{-2x}}_{y_2}+\tfrac{2}{3}e^x}}
	\end{align*}
	\end{itemize}
	\end{frame}


\begin{frame}\frametitle{The Method of Variation of Parameters}
\framesubtitle{Examples}
\begin{block}
	{Solve the differential equation}
	\begin{equation*}
		y''-y'-6y=2\sin 3x
	\end{equation*}
\end{block}
\pause \begin{itemize}[<+->]
\item Solve the homogeneous equation:
\begin{equation*}
	r^2-r-6=0,\qquad r=\frac{1\pm \sqrt{1-4\cdot (-6)}}{2}=\frac{1\pm 5}{2}=\{3,-2\}
\end{equation*}
We have $y_1(x)=e^{3x}$, $y_2(x)=e^{-2x}$.
\item  Compute the Wronskian:
\begin{equation*}
	W(y_1,y_2) = \begin{vmatrix}
		e^{3x} & e^{-2x} \\ 3e^{3x} & -2e^{-2x}
	\end{vmatrix} = -2e^{3x}e^{-2x}-3e^{3x}e^{-2x}=-5e^x
\end{equation*}
\end{itemize}
\end{frame}

\begin{frame}\frametitle{The Method of Variation of Parameters}
\framesubtitle{Examples}
\begin{block}
	{Solve the differential equation}
	\begin{equation*}
		y''-y'-6y=2\sin 3x
	\end{equation*}
\end{block}
\begin{itemize}
\item Compute the \emph{parameter} functions $A$ and $B$:
\begin{align*}
	A(x) &= -\int \frac{y_2(x) f(x)}{W(y_1,y_2)}\, dx = \int \frac{e^{-2x} 2\sin 3x}{5e^{x}}\, dx  \\
	\uncover<2->{&= \int \tfrac{2}{5} e^{-3x}\sin 3x\, dx= -\tfrac{1}{15}e^{-3x} \big( \sin 3x + \cos 3x \big)+\alert{A} \\}
	\uncover<3->{B(x) &= \int \frac{y_1(x) f(x)}{W(y_1,y_2)}\, dx = \int \frac{e^{3x}\, 2\sin 3x}{-5e^x}\, dx \\}
	\uncover<4->{&= \int -\tfrac{2}{5} e^{2x}\sin 3x\, dx = \tfrac{2}{65}e^{2x}\big( 3\cos 3x -2 \sin 3x \big) + \alert{B}}
\end{align*}
\item<5-> The solution is then
\begin{align*}
	y&=\Big( -\tfrac{1}{15}e^{-3x} \big( \sin 3x + \cos 3x \big) + A \Big) e^{3x} - \Big( \tfrac{2}{65} e^{2x} \big( 3\cos 3x - 2\sin 3x\big) + B \Big) e^{-2x} \\
	&= Ae^{3x} + Be^{-2x} - \frac{5}{39}\sin 3x + \frac{1}{39}\cos 3x
\end{align*}
\end{itemize}
\end{frame}

\subsection{Proof}

\begin{frame}\frametitle{The Method of Variation of Parameters}
\framesubtitle{Proof}
Let's see why this method works: We are looking for a function of the form $y=A(x)y_1(x) + B(x)y_2(x)$ that solves the differential equation $ay''+by'+cy=f(x)$, where $y_1$ and $y_2$ solve the homogeneous equation:
\begin{align}\label{eq:yh}
	ay''_1(x)+by'_1(x)+cy_1(x)&=0  &
	ay''_2(x)+by'_2(x)+cy_2(x)&=0 
\end{align}
\pause Let us compute the first derivative of $y$:
\begin{align*}
	y' &= A'(x)y_1(x) + A(x)y'_1(x) + B'(x)y_2(x)+ B(x)y'_2(x) \\
	\uncover<3->{&= \big( A(x) y'_1(x) + B(x) y'_2(x) \big) + \alert{\big( A'(x)y_1(x)+B'(x) y_2(x)\big)}}
\end{align*}
\pause\pause We would like to have as simple as expression as possible, so we are going to (artificially) impose that the second term is zero:
\begin{equation}\label{eq:cond1}
	A'(x)y_1+B'(x)y_2=0
\end{equation}
\pause In this case, the first derivative of $y$ reads:
\begin{equation}\label{eq:cond2}
	y' = A(x)y'_1(x)+B(x)y'_2(x)
\end{equation}
\end{frame}

\begin{frame}\frametitle{The Method of Variation of Parameters}
\framesubtitle{Proof}
The second derivative of $y$ is then
\begin{align*}
	y'' &= A'(x)y'_1(x)+A(x)y''_1(x)+B'(x)y'_2(x)+B(x)y''_2(x) \\
	\uncover<2->{&= \big(A(x)\alert{y''_1(x)}+B(x)\alert{y''_2(x)}\big) + \big( A'(x)y'_1(x)+B'(x)y'_2(x) \big)}
\end{align*}
\pause\pause By \eqref{eq:yh}, it must be
\begin{align*}
	y''_1&=-\frac{b}{a}y'_1-\frac{c}{a}y_1 &
	y''_2&=-\frac{b}{a}y'_2-\frac{c}{a}y_2
\end{align*}
And so we may re-write the second derivative of $y$ as follows:
\begin{align*}
	y''&= A(x)\Big(-\tfrac{b}{a}y'_1(x) -\tfrac{c}{a}y_1(x) \Big) + B(x) \Big( -\tfrac{b}{a}y'_2(x) -\tfrac{c}{a} y_2(x)\Big) \\
	&\quad + \big( A'(x)y'_1(x) + B'(x)y'_2(x) \big) \\
	\uncover<4->{&= -\frac{b}{a} \underbrace{\big( A(x)y'_1(x) + B(x)y'_2(x) \big)}_{y' \text{ by }\eqref{eq:cond2}} 
	-\frac{c}{a} \underbrace{\big( A(x)y_1(x) + B(x)y_2(x) \big)}_{y \text{ by definition}}  \\
	&\quad +\big( A'(x)y'_1(x) + B'(x)y'_2(x) \big) \\}
	\uncover<5->{&= \big( A'(x)y'_1(x) + B'(x)y'_2(x) \big) -\frac{b}{a}y' - \frac{c}{a}y}
\end{align*}
\end{frame}

\begin{frame}\frametitle{The Method of Variation of Parameters}
\framesubtitle{Proof}
Let us re-write this last equation:
\begin{gather*}
	y'' = -\frac{b}{a}y'-\frac{c}{a}y + \big( A'(x)y'_1(x)+B'(x)y'_2(x) \big) \\
	\uncover<2->{\underbrace{ay''+by'+cy}_{f(x)} = a\big( A'(x)y'_1(x) + B'(x)y'_2(x) \big)}
\end{gather*}
\pause\pause This gives us one new condition on $A$ and $B$:
\begin{equation}\label{eq:cond3}
	A'(x)y'_1(x) + B'(x)y'_2(x) = \frac{1}{a}f(x)
\end{equation}
\pause Note how now we can gather conditions \eqref{eq:cond1} and \eqref{eq:cond3} to form a system of two equations with two unknowns: $A'$ and $B'$.
\begin{equation*}
	\begin{cases}
		A'(x)y_1(x) + B'(x)y_2 = 0\\
		A'(x)y'_1(x) + B'(x)y'_2(x) = \tfrac{1}{a}f(x)
	\end{cases}
\end{equation*}
\pause In matrix form, this is:
\begin{equation*}
\begin{bmatrix}
y_1 & y_2 \\ y'_1 & y'_2
\end{bmatrix} \cdot
\begin{bmatrix}
A'(x) \\ B'(x)
\end{bmatrix} =
\begin{bmatrix}
0 \\ \tfrac{1}{a}f(x)
\end{bmatrix}
\end{equation*}
\end{frame}

\begin{frame}\frametitle{The Method of Variation of Parameters}
\framesubtitle{Proof}
The solution, using Cramer's rule---and noticing that the determinant of the square matrix is precisely $W(y_1,y_2)$---yields
\begin{align*}
	A'(x) &= \frac{1}{W(y_1,y_2)} \begin{vmatrix}
		0 & y_2(x) \\ \tfrac{1}{a}f(x) & y'_2(x)
	\end{vmatrix} \uncover<2->{= -\frac{y_2(x) f(x)}{a W(y_1,y_2)}} \\ \\
	\uncover<3->{B'(x) &= \frac{1}{W(y_1,y_2)} \begin{vmatrix}
		y_1(x) & 0 \\ y'_1(x) & \tfrac{1}{a}f(x)
	\end{vmatrix}} \uncover<4->{= \frac{y_1(x) f(x)}{a W(y_1,y_2)}}
\end{align*}
\pause\pause\pause\pause Taking integrals, we obtain the desired formulas:
\begin{align*}
	\alert{A(x)} &\alert{= -\frac{1}{a} \int \frac{y_2(x) f(x)}{W(y_1,y_2)}\, dx} \\
	\alert{B(x)} &\alert{= \frac{1}{a} \int \frac{y_1(x) f(x)}{W(y_1,y_2)}\, dx}
\end{align*}
\end{frame}
\end{document}