\documentclass[9pt,xcolor=x11names,compress]{beamer}

%% General document %%%%%%%%%%%%%%%%%%%%%%%%%%%%%%%%%%
\usepackage{graphicx}
\usepackage{tikz}
\usetikzlibrary{decorations.fractals,lindenmayersystems}
%%%%%%%%%%%%%%%%%%%%%%%%%%%%%%%%%%%%%%%%%%%%%%%%%%%%%%


%% Beamer Layout %%%%%%%%%%%%%%%%%%%%%%%%%%%%%%%%%%
\useoutertheme[subsection=false,shadow]{miniframes}
\useinnertheme{default}
\usefonttheme{serif}
\usepackage{palatino}

\setbeamerfont{title like}{shape=\scshape}
\setbeamerfont{frametitle}{shape=\scshape}

\setbeamercolor*{lower separation line head}{bg=DeepSkyBlue4} 
\setbeamercolor*{normal text}{fg=black,bg=white} 
\setbeamercolor*{alerted text}{fg=DeepSkyBlue4} 
\setbeamercolor*{example text}{fg=black} 
\setbeamercolor*{structure}{fg=black} 

\setbeamercolor*{palette tertiary}{fg=black,bg=black!10} 
\setbeamercolor*{palette quaternary}{fg=black,bg=black!10} 

\setbeamertemplate{blocks}[rounded][shadow=true]
\setbeamercolor{block title}{bg=DeepSkyBlue4}
\setbeamercolor{block title example}{bg=DeepSkyBlue4}
\setbeamercolor{block body}{bg=black!15!white}
\setbeamercolor{block body example}{bg=black!15!white}

\setbeamertemplate{navigation symbols}{}
%%%%%%%%%%%%%%%%%%%%%%%%%%%%%%%%%%%%%%%%%%%%%%%%%%

\author[Francisco Blanco-Silva]{Francisco Blanco-Silva}
\institute[USC]{University of South Carolina}
\date{
\pgfdeclarelindenmayersystem{Funny curve}{
  \rule{F -> FF+F+F+F+FF}}
	\begin{tikzpicture} 
    \draw [DeepSkyBlue4]
    [l-system={Funny curve, axiom=F+F+F+F, order=4, step=1.5pt, angle=90}]
    lindenmayer system -- cycle; 
	\end{tikzpicture}
}
\title{Lesson 12: Homogeneous Second-Order Linear Equations with Constant Coefficients}

\begin{document}

\frame{\titlepage}

\section{What do we know?}
\begin{frame}\frametitle{What do we know?}
\begin{columns}[T]
\begin{column}{0.45\linewidth}
\begin{itemize}
\item The concepts of \alert{differential equation} and \alert{initial value problem}
\item The concept of \alert{order} of a differential equation.
\item The concepts of \alert{general solution}, \alert{particular solution} and \alert{singular solution}.
\item \alert{Slope fields}
\item Approximations to solutions via \alert{Euler's Method} and \alert{Improved Euler's Method}
\end{itemize} 
\end{column}
\begin{column}{0.55\linewidth}
\begin{itemize}
\item First-Order Differential Equations
\begin{itemize}
\item Separable equations 
\item Homogeneous First-Order Equations 
\item Linear First-Order Equations 
\item Bernoulli Equations 
\item General Substitution Methods
\item Exact Equations 
\end{itemize}
\item Second-Order Differential Equations
\begin{itemize}
	\item Reducible Equations
	\item Linear Equations (Intro)
\end{itemize}
\end{itemize}
\end{column}
\end{columns}
\end{frame}

\section{Homogeneous Second-Order Linear Equations with Constant Coefficients}
\subsection{Definitions and Basic Results}

\begin{frame}\frametitle{Homogeneous Second-Order Linear Equations with Constant Coefficients}
\framesubtitle{Motivation}
A \alert{homogeneous second-order linear equation with constant coefficients} has the form:
\begin{equation*}
	ay''+by'+cy=0 \qquad(a\neq 0)
\end{equation*}
\pause \alert{Can you guess (intuitively) what a solution of this equation looks like?}

\pause It looks like something of the form $y_1=e^{r x}$ could do the trick.  If that is the case, we may find the value of $r$ by substitution:
\begin{align*}
y_1&=e^{rx} &y'_1&=	re^{rx} &y''_1&=r^2 e^{rx}
\end{align*}
\pause Let's try:
\begin{gather*}
	\alert<6>{ay''_1 + by'_1 + cy_1 =0} \\
	ar^2 e^{rx} + br e^{rx} + ce^{rx} =0 \\
	\uncover<5->{e^{rx} \big( ar^2+br+c \big) =0 \\}
	\uncover<6->{\alert{ar^2+br+c=0}}
\end{gather*}
\end{frame}

\begin{frame}\frametitle{Homogeneous Second-Order Linear Equations with Constant Coefficients}
 	\framesubtitle{Definition} 
\begin{definition}
Given a homogeneous second-order linear differential equation with constant coefficients $ay''+by'+cy=0$, we say that the quadratic equation $ar^2+br+c=0$ is the \alert{characteristic equation}.  
\end{definition}
\pause The roots of the characteristic equation give away the solutions of the differential equation.  We have three cases:
\begin{itemize}[<+->]
	\item Both roots are real and different: $r_1 \neq r_2$
	\begin{equation*}
		\alert{r^2-5r+6=0\qquad \{ r=2, r=3 \}}
	\end{equation*}
	\item Both roots are real and equal: $r=r_1=r_2$.
	\begin{equation*}
	\alert{r^2-2r+1=0\qquad \{r=1\} }
	\end{equation*}
	\item Both roots are complex: $r=\alpha \pm i\beta$ with $\beta>0$.
	\begin{equation*}
		\alert{r^2+1=0\qquad \{r=\pm i\}}
	\end{equation*}
	\end{itemize}

\end{frame}

\subsection{Cases}

\begin{frame}\frametitle{Homogeneous Second-Order Linear Equations with Constant Coefficients}
\framesubtitle{First Case: $r_1\neq r_2$ real roots}
	In this case, the two possible solutions are 
	\begin{align*}
	\alert{y_1}&\alert{=e^{r_1x}} &\alert{y_2}&\alert{=e^{r_2x}} \\
	\intertext{\uncover<2->{Note that the Wronskian is never zero:}}
	\uncover<2->{y'_1&=r_1 e^{r_1x} &y'_2&=r_2 e^{r_2x}.}
	\end{align*}
	\pause
	\begin{equation*}
		W\big(e^{r_1x},e^{r_2x}\big) = \left| \begin{matrix} e^{r_1x} & e^{r_2x} \\ r_1e^{r_1x} & r_2 e^{r_2x} \end{matrix} \right| \uncover<3->{= r_2e^{r_1x}e^{r_2 x}-r_1e^{r_1x}e^{r_2 x}} \uncover<4->{= \underbrace{(r_2-r_1)}_{\neq 0} e^{(r_1+r_2)x}}
	\end{equation*}
	\pause\pause\pause This means that a general solution to the differential equation has the form 
	\begin{equation*}
	y=Ae^{r_1x}+Be^{r_2x}
	\end{equation*}
\end{frame}

\begin{frame}\frametitle{Homogeneous Second-Order Linear Equations with Constant Coefficients}
\framesubtitle{Second Case: $r=r_1=r_2$ real roots}
In this case, the two possible solutions are 
\begin{align*}
\alert{y_1}&\alert{=e^{rx}} &\alert{y_2}&\alert{=xe^{rx}} \\
\intertext{\uncover<2->{The Wronskian is also non-zero:}}
\uncover<2->{y'_1&=re^{rx} &y'_2&=e^{rx}+rxe^{rx} = (1+rx)e^{rx}}
\end{align*}
\pause 
\begin{equation*}
W\big( e^{rx}, xe^{rx} \big) = \left| \begin{matrix}
e^{rx} & xe^{rx} \\ re^{rx} & (1+rx)e^{rx} \end{matrix} \right| \uncover<3->{= (1+rx)e^{2rx} - rxe^{2rx}} \uncover<4->{= e^{2rx}}
\end{equation*}
\pause\pause\pause This means that a general solution to the differential equation has the form 
\begin{equation*}
y=Ae^{rx}+Bxe^{rx} = (A+Bx)e^{rx}
\end{equation*}
\end{frame}

\begin{frame}\frametitle{Homogeneous Second-Order Linear Equations with Constant Coefficients}
\framesubtitle{Third Case: $r=\alpha \pm i\beta$ complex roots ($\beta > 0$)}
In this case, the two possible solutions are 
\begin{align*}
\alert{y_1}&\alert{=e^{\alpha x}\cos (\beta x)} &\alert{y_2}&\alert{=e^{\alpha x} \sin (\beta x)} \\
\intertext{\uncover<2->{The Wronskian is not zero:}}
\uncover<2->{y'_1&=\alpha e^{\alpha x}\cos(\beta x)-\beta e^{\alpha x}\sin (\beta x) &y'_2&=\alpha e^{\alpha x}\sin (\beta x)+\beta e^{\alpha x}\cos (\beta x)}
\end{align*}
\pause 
\begin{align*}
	W &(y_1, y_2 ) = \left| \begin{matrix}
		e^{\alpha x} \cos(\beta x) & e^{\alpha x} \sin(\beta x) \\
		e^{\alpha x} \big(\alpha \cos (\beta x) - \beta \sin (\beta x) \big) & e^{\alpha x} \big(\alpha \sin (\beta x)+ \beta \cos (\beta x) \big)
	\end{matrix} \right| \\
	\uncover<3->{&=e^{2\alpha x} \cos(\beta x) \big( \alpha \sin (\beta x) + \beta \cos(\beta x)\big) - e^{2\alpha x} \sin (\beta x) \big( \alpha \cos(\beta x) - \beta \sin (\beta x) \big) \\}
	\uncover<4->{&=\textcolor{red}{\alpha e^{2\alpha x} \sin (\beta x)\cos(\beta x)} + \beta e^{2\alpha x} \cos^2(\beta x) \textcolor{red}{- \alpha e^{2\alpha x}\sin(\beta x) \cos (\beta x)} + \beta e^{2\alpha x} \sin^2(\beta x)} \\
	\uncover<5->{&=\beta e^{2\alpha x} \big( \cos^2(\beta x) + \sin^2 (\beta x) \big) = \beta e^{2\alpha x}}
\end{align*}
\pause\pause\pause\pause This means that a general solution to the differential equation has the form
\begin{equation*}
	y=Ae^{\alpha x}\sin(\beta x)+Be^{\alpha x} \cos(\beta x)
\end{equation*}
\end{frame}

\subsection{Examples}
\begin{frame}\frametitle{Homogeneous Second-Order Linear Equations with Constant Coefficients}
\framesubtitle{Examples}
\begin{block}
	{Find a particular solution to the initial value problem}
	\begin{align*}
		y''-5y'+6y&=0 &y(0)&=1 &y'(0)&=2
	\end{align*}
\end{block}
\pause First, we seek a general solution of the differential equation.  We form the characteristic equation, and solve it to find the roots:
\begin{equation*}
	\alert{r^2-5r+6=0}\qquad 
	\uncover<3->{r=\frac{5\pm \sqrt{25-4\cdot 6}}{2}=\frac{5\pm 1}{2}=\{ 2,3 \}}
\end{equation*}
\pause The general solution (and its derivative) is then 
\begin{align*}
	y&=Ae^{2x}+Be^{3x} & y'&=2Ae^{2x}+3Be^{3x} 
\end{align*}
\pause We need to find the value of the constants $A,B$ that solve the IVP:
\begin{equation*}
\begin{cases}
	1=y(0)=A+B \\ 2=y'(0)=2A+3B
\end{cases}\qquad
\uncover<5->{\begin{cases}
	A=1 \\B=0
\end{cases}\qquad}
\uncover<6->{\alert{y=e^{2x}}}
\end{equation*}
\end{frame}

\begin{frame}\frametitle{Homogeneous Second-Order Linear Equations with Constant Coefficients}
\framesubtitle{Examples}
\begin{block}
	{Find a particular solution to the initial value problem}
	\begin{align*}
		y''-2y'+y&=0 &y(0)&=1 &y'(0)&=2
	\end{align*}
\end{block}
\pause First, we seek a general solution of the differential equation.  We form the characteristic equation, and solve it to find the roots:
\begin{equation*}
	\alert{r^2-2r+1=0}\qquad 
	\uncover<3->{r=\frac{2\pm \sqrt{4-4}}{2}=1}
\end{equation*}
\pause The general solution (and its derivative) is then 
\begin{align*}
	y&=(A+Bx)e^x & y'&=Be^x+(A+Bx)e^x=(A+B+Bx)e^x
\end{align*}
\pause We need to find the value of the constants $A,B$ that solve the IVP:
\begin{equation*}
\begin{cases}
	1=y(0)=A \\ 2=y'(0)=A+B
\end{cases}\qquad
\uncover<5->{\begin{cases}
	A=1 \\B=1
\end{cases}\qquad}
\uncover<6->{\alert{y=(1+x)e^x}}
\end{equation*}
\end{frame}

\begin{frame}\frametitle{Homogeneous Second-Order Linear Equations with Constant Coefficients}
\framesubtitle{Examples}
\begin{block}
	{Find a particular solution to the initial value problem}
	\begin{align*}
		y''&=-y &y(0)&=1 &y'(0)&=2
	\end{align*}
\end{block}
\pause First, we seek a general solution of the differential equation.  We form the characteristic equation, and solve it to find the roots:
\begin{equation*}
	\alert{r^2+1=0}\qquad 
	\uncover<3->{r=\pm i\qquad (\alpha=0, \beta=1)}
\end{equation*}
\pause The general solution (and its derivative) is then 
\begin{align*}
	y&=Ae^{0\cdot x}\cos(1\cdot x)+ Be^{0\cdot x}\sin(1\cdot x)=A\cos x + B\sin x, & y'&=-A\sin x + B\cos x
\end{align*}
\pause We need to find the value of the constants $A,B$ that solve the IVP:
\begin{equation*}
\begin{cases}
	1=y(0)=A \\ 2=y'(0)=B
\end{cases}\qquad
\uncover<5->{\alert{y=\cos x + 2\sin x}}
\end{equation*}
\end{frame}

\begin{frame}\frametitle{Homogeneous Second-Order Linear Equations with Constant Coefficients}
\framesubtitle{Examples}
\begin{block}
	{Find a particular solution to the initial value problem}
	\begin{align*}
		y''+2y'+5y&=0 &y(0)&=1 &y'(0)&=2
	\end{align*}
\end{block}
\pause First, we seek a general solution of the differential equation.  We form the characteristic equation, and solve it to find the roots:
\begin{equation*}
	\alert{r^2+2r+5=0}\qquad 
	\uncover<3->{r=\frac{-2\pm \sqrt{4-4\cdot 5}}{2}=\frac{-2\pm \sqrt{-16}}{2}=\frac{-2\pm 4i}{2} = \underbrace{-1\pm 2i}_{\alert{\alpha+\beta i}}}
\end{equation*}
\pause The general solution (and its derivative) is then 
\begin{align*}
	y&=Ae^{-x}\cos(2x)+Be^{-x}\sin(2x)=e^{-x}\big(A\cos(2x)+B\sin(2x)\big) \\
	y'&=-Ae^{-x}\cos(2x)-2Ae^{-x}\sin(2x)-Be^{-x}\sin(2x)+2Be^{-x}\cos(2x)
\end{align*}
\pause We need to find the value of the constants $A,B$ that solve the IVP:
\begin{equation*}
\begin{cases}
	1=y(0)=A \\ 2=y'(0)=-A+2B
\end{cases}\qquad
\uncover<5->{\begin{cases}
	A=1 \\B=\tfrac{3}{2}
\end{cases}\qquad}
\uncover<6->{\alert{y=e^{-x}\big(\cos(2x)+\tfrac{3}{2}\sin(2x)\big)}}
\end{equation*}
\end{frame}

\end{document}