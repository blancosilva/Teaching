\PassOptionsToPackage{table}{xcolor}
\documentclass[9pt,xcolor=x11names,compress]{beamer}

%% General document %%%%%%%%%%%%%%%%%%%%%%%%%%%%%%%%%%
\usepackage{graphicx}
\usepackage{tikz}
\usepackage{array}
\usetikzlibrary{decorations.fractals,lindenmayersystems}
%%%%%%%%%%%%%%%%%%%%%%%%%%%%%%%%%%%%%%%%%%%%%%%%%%%%%%


%% Beamer Layout %%%%%%%%%%%%%%%%%%%%%%%%%%%%%%%%%%
\useoutertheme[subsection=false,shadow]{miniframes}
\useinnertheme{default}
\usefonttheme{serif}
\usepackage{palatino}

\setbeamerfont{title like}{shape=\scshape}
\setbeamerfont{frametitle}{shape=\scshape}

\setbeamercolor*{lower separation line head}{bg=DeepSkyBlue4} 
\setbeamercolor*{normal text}{fg=black,bg=white} 
\setbeamercolor*{alerted text}{fg=DeepSkyBlue4} 
\setbeamercolor*{example text}{fg=black} 
\setbeamercolor*{structure}{fg=black} 

\setbeamercolor*{palette tertiary}{fg=black,bg=black!10} 
\setbeamercolor*{palette quaternary}{fg=black,bg=black!10} 

\setbeamertemplate{blocks}[rounded][shadow=true]
\setbeamercolor{block title}{bg=DeepSkyBlue4}
\setbeamercolor{block title example}{bg=DeepSkyBlue4}
\setbeamercolor{block body}{bg=black!15!white}
\setbeamercolor{block body example}{bg=black!15!white}

\setbeamertemplate{navigation symbols}{}
%%%%%%%%%%%%%%%%%%%%%%%%%%%%%%%%%%%%%%%%%%%%%%%%%%
% Some useful definitions
\newcommand*\Laplace[1]{\mathcal{L}\{{#1}\}}
\newcommand*\BLaplace[1]{\mathcal{L}\Big\{{#1}\Big\}}
\newcommand*\iLaplace[1]{\mathcal{L}^{-1}\{{#1}\}}
\newcommand*\ibLaplace[1]{\mathcal{L}^{-1}\big\{{#1}\big\}}
\newcommand*\iBLaplace[1]{\mathcal{L}^{-1}\Big\{{#1}\Big\}}
%%%%%%%%%%%%%%%%%%%%%%%%%%%%%%%%%%%%%%%%%%%%%%%%%%

\author[Francisco Blanco-Silva]{Francisco Blanco-Silva}
\institute[USC]{University of South Carolina}
\date{
\pgfdeclarelindenmayersystem{leaf}{
	\rule{F -> F--F+F-}
}
\begin{tikzpicture}[color=DeepSkyBlue4]
    \draw [l-system={leaf, axiom=FFF, order=7, step=0.75pt, angle=60 }]
    lindenmayer system; 
	\end{tikzpicture}
}

\title{Lesson 20: Laplace Transform of Derivatives.  Transformation of Initial Value Problems} 

\begin{document}

\frame{\titlepage}

\section{What do we know?}
\subsection{General Program}

\begin{frame}\frametitle{What do we know?}
\begin{columns}[T]
\begin{column}{0.45\linewidth}
\begin{itemize}
\item The concepts of \alert{differential equation} and \alert{initial value problem}
\item The concept of \alert{order} of a differential equation.
\item The concepts of \alert{general solution}, \alert{particular solution} and \alert{singular solution}.
\item \alert{Slope fields}
\item Approximations to solutions via \alert{Euler's Method} and \alert{Improved Euler's Method}
\end{itemize} 
\end{column}
\begin{column}{0.55\linewidth}
\begin{itemize}
\item First-Order Differential Equations
\begin{itemize}
\item Separable equations 
\item Homogeneous First-Order Equations 
\item Linear First-Order Equations 
\item Bernoulli Equations 
\item General Substitution Methods
\item Exact Equations 
\end{itemize}
\item Second-Order Differential Equations
\begin{itemize}
	\item Reducible Equations
	\item General Linear Equations (Intro)
	\item Linear Equations with Constant Coefficients
	\begin{itemize}
		\item Characteristic Equation
		\item Variation of Parameters
		\item Undetermined Coefficients
	\end{itemize}
\end{itemize}
\end{itemize}
\end{column}
\end{columns}
\end{frame}

\begin{frame}\frametitle{What do we know?}
\framesubtitle{Laplace Transforms}

\noindent\begin{tikzpicture}
\node[scale=0.7]{
\rowcolors{2}{white}{gray!25!white}
	\begin{tabular}{|m{2cm}|m{3.2cm}l||m{2cm}|m{3.2cm}l|}
	\rowcolor{DeepSkyBlue4}
		$f(x)$\raisebox{0.5cm} & $\mathcal{L}\{f\}=\int_0^\infty e^{-sx}f(x)\, dx$\raisebox{0.5cm} & &
		$f(x)$\raisebox{0.5cm} & $\mathcal{L}\{f\}=\int_0^\infty e^{-sx}f(x)\, dx$\raisebox{0.5cm} & \\[0.4cm]
		$1$ & $\dfrac{1}{s}$\raisebox{0.6cm} & $s>0$ &
		$cf(x)\pm g(x)$ & $cF(s) \pm G(s)$\raisebox{0.4cm} & $s>max(a,b)$ \\[0.4cm]
		$x^p$ & $\dfrac{\Gamma(p+1)}{s^{p+1}}$\raisebox{0.6cm} & $s>0$ &
		$x^n f(x)$ & $(-1)^n F^{(n)}$\raisebox{0.4cm} & $s>a$ \\[0.4cm]
		$e^{\alpha x}$ & $\dfrac{1}{s-\alpha}$\raisebox{0.6cm} & $s>\alpha$ & $e^{\alpha x}f(x)$ & $F(s-\alpha)$\raisebox{0.4cm} & $s>a+\alpha$ \\[0.4cm]
		$\sin \beta x$ & $\dfrac{\beta}{s^2+\beta^2}$\raisebox{0.6cm} & $s>0$ & $\dfrac{f(x)}{x}$ & $\displaystyle{\int_s^\infty F(\sigma)\, d\sigma}$\raisebox{0.4cm} & $s>a$ \\[0.4cm]
		$\cos \beta x$ & $\dfrac{s}{s^2+\beta^2}$\raisebox{0.6cm} & $s>0$ &
		$f\star g$ & $F(s) G(s)$\raisebox{0.4cm} & $s>\max(a,b)$ \\[0.4cm]
		\hline
		\end{tabular}};
	\end{tikzpicture}
\end{frame}

\section{Warm-up}
\subsection{examples}

\begin{frame}\frametitle{Warm-up}
\framesubtitle{Examples}
\begin{block}
{Compute the inverse Laplace transform}
\begin{equation*}
	F(s)=\frac{1}{(s+1)^2}	
\end{equation*}
\end{block}
\pause The easiest way to go about this one is to interpret the function as a $-1$--shift of the function $s^{-2}$ (which is the Laplace transform of $x$).  It is then
\begin{equation*}
	\iBLaplace{\frac{1}{(s+1)^2}} = xe^{-x}
\end{equation*}
\pause Another possibility is, of course, via convolution:
\begin{equation*}
	\iBLaplace{\frac{1}{(s+1)^2}} = \iBLaplace{\frac{1}{s+1}\cdot \frac{1}{s+1}} =  \int_0^x e^{-x+t}e^{-t}\, dt = e^{-x}\int_0^x dt = xe^{-x}
\end{equation*}
\end{frame}

\begin{frame}\frametitle{Warm-up}
\framesubtitle{Examples}
\begin{block}
{Compute the inverse Laplace transform}
\begin{equation*}
G(s) = \frac{2-s}{(s^2+4)^2}
\end{equation*}
\end{block}
\pause There is no shift in this expression, but the square in the denominator suggests a derivative of the Laplace transform of a sine.  \pause Note that, if $f(x)=\sin 2x$, then
\begin{align*}
	F(s) &= \frac{2}{s^2+4}, &F'(s)&= -\frac{4s}{(s^2+4)^2}
\end{align*}
\pause In our case, we have
\begin{align*}
	G(s)&=\frac{2-s}{(s^2+4)^2} = \frac{2}{(s^2+4)^2} - \frac{s}{(s^2+4)^2} \\
	\uncover<5->{&= \alert{\frac{1}{2}}\cdot \underbrace{\frac{2}{s^2+4} \cdot \frac{2}{s^2+4}}_{\sin 2x \star \sin 2x} + \alert{\frac{1}{4}} \cdot \underbrace{\frac{-4s}{(s^2+4)^2}}_{-x\sin 2x}}
\end{align*}
\end{frame}

\begin{frame}\frametitle{Warm-up}
\framesubtitle{Examples}
\begin{block}
{Compute the inverse Laplace transform}
\begin{equation*}
G(s) = \frac{2-s}{(s^2+4)^2}
\end{equation*}
\end{block}
It only remains to compute the convolution of $\sin 2x$ with itself:
\begin{align*}
	\int_0^x \sin 2(x-t)&\sin 2t\, dt 
	\uncover<2->{= \int_0^x \big(\sin 2x\cos 2t - \cos 2x\sin2t\big) \sin 2t\, dt \\}
	\uncover<3->{&= \sin 2x \int_0^x \cos 2t \sin 2t\, dt - \cos 2x \int_0^x \sin^2 2t\, dt \\}
	\uncover<4->{&= \frac{1}{2}\sin 2x \int_0^x \sin 4t\, dt - \frac{1}{8}\cos 2x \big(4x-\sin 4x\big) \\}
	\uncover<5->{&= \frac{1}{8}\sin 2x \big( 1-\cos(4x) \big) - \frac{1}{8}\cos 2x \big(4x-\sin 4x\big)}
\end{align*}
\uncover<6->{The solution of this problem is then}
\begin{equation*}
	\uncover<6->{\alert{g(x)=\frac{1}{16}\Big( \sin 2x \big( 1-\cos 4x\big) - \cos 2x \big( 4x-\sin 4x\big) \Big) - \frac{1}{4}x \sin 2x}}
\end{equation*}
\end{frame}

\section{Laplace Transform of Derivatives}
\subsection{Definition}

\begin{frame}\frametitle{Laplace Transform of Derivatives}
\begin{theorem}
	Suppose that both $f$ and $f'$ are both contiuous functions in $(0,\infty)$.  \alert<6->{Suppose further that there exists constants $K,a,M$ such that $\lvert f(x) \rvert \leq Ke^{at}$ for $x>M$.}  Then $\Laplace{f'}=s\Laplace{f}-f(0)$ for $s>a$.
\end{theorem}
\pause We prove it in the usual way:
\begin{align*}
	\uncover<2->{\Laplace{f'} &= \int_0^\infty e^{-sx} f'(x)\, dx} \uncover<3->{= \lim_{A\to \infty} \int_0^A \underbrace{e^{-sx}}_{u} \underbrace{f'(x)\, dx}_{dv} \\}
	\uncover<4->{&=\lim_{A\to \infty} \left[ \alert{f(x)e^{-sx}\Big\rvert_0^A} +s \int_0^A f(x)e^{-sx}\, dx \right]}
\end{align*}
\uncover<5->{Note that}
\begin{align*}
	\uncover<5->{\lim_{A\to\infty} \left( f(x)e^{-sx} \Big\rvert_0^A \right) &= \lim_{A\to\infty} \Big( f(A)e^{-sA} - f(0) \Big) \\}
	\uncover<6->{&\leq \lim_{A\to\infty} \alert{Ke^{aA}e^{-sA}} - f(0)} \uncover<7->{= \lim_{A\to\infty} K e^{(a-s)A} -f(0)}
\end{align*}
\end{frame}

\begin{frame}\frametitle{Laplace Transform of Derivatives}
\begin{theorem}
	Suppose that both $f$ and $f'$ are both contiuous functions in $(0,\infty)$.  Suppose further that there exists constants $K,a,M$ such that $\lvert f(x) \rvert \leq Ke^{at}$ for $x>M$.  Then $\Laplace{f'}=s\Laplace{f}-f(0)$ for $s>a$.
\end{theorem}
It must then be
\begin{equation*}
	\lim_{A\to\infty} \left( f(x)e^{-sx} \Big\rvert_0^A \right) = -f(0)
\end{equation*}
only if $a-s<0$.  The limit diverges otherwise.  

\pause In summary: If $s>a$,
\begin{equation*}
	\Laplace{f'} = -f(0) + s \underbrace{\int_0^\infty e^{-sx}f(x)\, dx}_{\Laplace{f}} = s\Laplace{f}-f(0)
\end{equation*}
\end{frame}

\begin{frame}\frametitle{Laplace Transform of Derivatives}
We may generalize this result: 
\begin{theorem}
	If $f, f', f'', \dotsc, f^{(n-1)}$ are all continuous functions in $(0,\infty)$, and they are \emph{good enough}, then 
	\begin{align*}
		\Laplace{f''}&=s^2\Laplace{f}-sf(0)-f'(0) \\
		\Laplace{f'''}&=s^3\Laplace{f}-s^2f(0)-sf'(0)-f''(0) \\
		&\vdots	 \\
		\Laplace{f^{(n)}}&=s^n\Laplace{f}-s^{n-1}f(0)-s^{n-2}f'(0)-\dotsb - f^{(n-1)}(0)
	\end{align*}
\end{theorem}
\end{frame}

\subsection{Examples}
\begin{frame}\frametitle{Laplace Transform of Derivatives}
\framesubtitle{Examples}
\begin{block}
	{Use the transform of $\sin \beta x$ and the transform of its derivative to deduct the formula}
	\begin{equation*}
	\Laplace{\cos \beta x}=\frac{s}{s^2+\beta^2}
	\end{equation*}
\end{block}
\pause The trick here is to realize that $\beta \cos \beta x$ is the derivative of $f(x)=\sin \beta x$.  In that case, 
\begin{align*}
	\Laplace{\cos \beta x} &= \frac{1}{\beta} \Laplace{\beta \cos \beta x} = \frac{1}{\beta} \Laplace{f'(x)} \\
	\uncover<3->{&=\frac{1}{\beta} \Big( s\Laplace{f(x)}-f(0) \Big) \\}
	\uncover<4->{&= \frac{1}{\beta} \Big( s\frac{\beta}{s^2+\beta^2} - \sin 0 \Big) \\}
	\uncover<5->{&= \frac{s}{s^2+\beta^2}}
\end{align*}
\end{frame}

\section[Transformation of IVPs]{Transformation of Initial Value Problems}
\subsection{Motivation}

\begin{frame}\frametitle{Laplace Transform of Derivatives}
\framesubtitle{Transformation of Initial Value Problems}
Note how we may use the Laplace Transform to compute particular solutions of an IVP:
\begin{example}
	\begin{equation*}
		y''+2y'+y=e^{3x},\quad \alert<5>{y(0)=y'(0)=0}
	\end{equation*}
\end{example}
\pause The Laplace transform of the left hand-side must be equal to the Laplace transform of the right hand-side.  Out target in this step is to compute the Laplace transform of a general solution $y$, $\Laplace{y}$:
\begin{gather*}
	\uncover<3->{\Laplace{y''+2y'+y} = \Laplace{e^{3x}} \\}
	\uncover<4->{\Laplace{y''} + 2\Laplace{y'} + \Laplace{y} = \frac{1}{s-3}\quad (s>3)\\}
	\uncover<5->{\Big( s^2 \Laplace{y} -s\alert{y(0)} -\alert{y'(0)} \Big) + 2\Big( s\Laplace{y} - \alert{y(0)} \Big) + \Laplace{y} = \frac{1}{s-3} \\}
	\uncover<6->{(s^2+2s+1)\Laplace{y} = \frac{1}{s-3} \\}
	\uncover<7->{\Laplace{y} = \frac{1}{(s-3)(s^2+2s+1)}\quad (s>3)}
\end{gather*}
\end{frame}

\begin{frame}\frametitle{Laplace Transform of Derivatives}
\framesubtitle{Transformation of Initial Value problems}
\begin{example}
	\begin{equation*}
		y''+2y'+y=e^{3x},\quad y(0)=y'(0)=0
	\end{equation*}
\end{example}
In the second stage, we compute the inverse Laplace transform of this function:
\begin{align*}
	\alert<5>{y} &= \iBLaplace{\frac{1}{(s-3)(s^2+2s+1)}} = \iBLaplace{\frac{1}{(s-3)(s+1)^2}} \\
	\uncover<2->{&= \iBLaplace{\frac{A}{s-3} + \frac{B}{s+1} + \frac{C}{(s+1)^2}} \\}
	\uncover<3->{&= \iBLaplace{\frac{1}{16}\cdot \frac{1}{s-3} - \frac{1}{16}\cdot \frac{1}{s+1} - \frac{1}{4} \cdot \frac{1}{(s+1)^2}} \\} 
	\uncover<4->{&=\frac{1}{16}\iBLaplace{\frac{1}{s-3}} - \frac{1}{16} \iBLaplace{\frac{1}{s+1}} - \frac{1}{4} \iBLaplace{\frac{1}{(s+1)^2}}} \\
	\uncover<5->{&= \alert{\frac{1}{16}e^{3x} - \frac{1}{16}e^{-x} - \frac{1}{4} xe^{-x}}}
\end{align*}
\end{frame}

\begin{frame}\frametitle{Laplace Transform of Derivatives}
\framesubtitle{Transformation of Initial Value problems}
\begin{example}
	\begin{equation*}
		y'''-2y''=x\sin 2x,\quad y(0)=0,y'(0)=1,y''(0)=0
	\end{equation*}
\end{example}
\pause We proceed as before, finding first $\Laplace{y}$ 
\begin{gather*}
	\Laplace{y'''-2y''} = \Laplace{x\sin 2x} \\
	\uncover<3->{\Laplace{y'''}-2\Laplace{y''} = \frac{4s}{(s^2+4)^2}\quad (s>0)\\}
	\uncover<4->{\Big( s^3\Laplace{y} -s^2\alert{y(0)} -s\alert{y'(0)} -\alert{y''(0)} \Big) - 2\Big( s^2\Laplace{y} -s\alert{y(0)} - \alert{y'(0)} \Big) = \frac{4s}{(s^2+4)^2} \\}
	\uncover<5->{(s^3-2s^2)\Laplace{y} -s+2 = \frac{4s}{(s^2+4)^2} \\}
	\uncover<6->{\Laplace{y} = \frac{4s}{(s^2+4)^2(s^3-2s^2)} + \frac{2-s}{s^3-2s^2} \\}
	\uncover<7->{\Laplace{y} =\frac{2-s}{s^2(s-2)} + \frac{4}{s(s-2)(s^2+4)^2}=-\frac{1}{s^2}+ \frac{4}{s(s-2)(s^2+4)^2}}
\end{gather*}
\end{frame}

\begin{frame}\frametitle{Laplace Transform of Derivatives}
\framesubtitle{Transformation of Initial Value problems}
\begin{example}
	\begin{equation*}
		y'''-2y''=x\sin 2x,\quad y(0)=0,y'(0)=1,y''(0)=0
	\end{equation*}
\end{example}
We now compute the inverse Laplace transform:
\begin{align*}
	\alert<3>{y} &= -\iBLaplace{\frac{1}{s^2}} + \iBLaplace{\frac{A}{s} + \frac{B}{s-2} + \frac{Cs+D}{s^2+4} + \frac{Es+F}{(s^2+4)^2}}\\
	\uncover<2->{&= -\iBLaplace{\frac{1}{s^2}} - \frac{1}{8} \iBLaplace{\frac{1}{s}} - \frac{1}{32} \iBLaplace{\frac{1}{s-2}} \\}
	\uncover<2->{&\quad + \frac{3}{32} \iBLaplace{\frac{s}{s^2+4}} - \frac{1}{32} \iBLaplace{\frac{2}{s^2+4}}+ \frac{1}{4} \iBLaplace{\frac{s-2}{(s^2+4)^2}} \\}
	\uncover<3->{&= \alert{-x - \frac{1}{8} - \frac{1}{32}e^{2x} + \frac{3}{32}\cos 2x - \frac{1}{32} \sin 2x}\\
	\uncover<3->{&\quad \alert{- \frac{1}{64}\Big( \sin 2x \big( 1-\cos 4x\big) - \cos 2x \big( 4x-\sin 4x\big) \Big) + \frac{1}{16}x \sin 2x}}}
\end{align*}
\end{frame}
\end{document}