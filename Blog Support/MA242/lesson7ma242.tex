\documentclass[smaller,xcolor=x11names,compress]{beamer}

%% General document %%%%%%%%%%%%%%%%%%%%%%%%%%%%%%%%%%
\usepackage{graphicx}
\usepackage{tikz}
\usetikzlibrary{decorations.fractals,lindenmayersystems}
%%%%%%%%%%%%%%%%%%%%%%%%%%%%%%%%%%%%%%%%%%%%%%%%%%%%%%


%% Beamer Layout %%%%%%%%%%%%%%%%%%%%%%%%%%%%%%%%%%
\useoutertheme[subsection=false,shadow]{miniframes}
\useinnertheme{default}
\usefonttheme{serif}
\usepackage{palatino}

\setbeamerfont{title like}{shape=\scshape}
\setbeamerfont{frametitle}{shape=\scshape}

\setbeamercolor*{lower separation line head}{bg=DeepSkyBlue4} 
\setbeamercolor*{normal text}{fg=black,bg=white} 
\setbeamercolor*{alerted text}{fg=red} 
\setbeamercolor*{example text}{fg=black} 
\setbeamercolor*{structure}{fg=black} 

\setbeamercolor*{palette tertiary}{fg=black,bg=black!10} 
\setbeamercolor*{palette quaternary}{fg=black,bg=black!10} 

\setbeamertemplate{blocks}[rounded][shadow=true]
\setbeamercolor{block title}{bg=black!50!white}
\setbeamercolor{block title example}{bg=black!50!white}
\setbeamercolor{block body}{bg=black!15!white}
\setbeamercolor{block body example}{bg=black!15!white}

\setbeamertemplate{navigation symbols}{}
%%%%%%%%%%%%%%%%%%%%%%%%%%%%%%%%%%%%%%%%%%%%%%%%%%

\author[Francisco Blanco-Silva]{Francisco Blanco-Silva}
\institute[USC]{University of South Carolina}
\date{
	\begin{tikzpicture}[decoration=Koch curve type 2] 
		\draw[DeepSkyBlue4] decorate{ decorate{ decorate{ (0,0) -- (3,0) }}}; 
	\end{tikzpicture}  
	\\
	\today
}
\title{Lesson 7: Linear First-Order and Bernoulli Equations}

\begin{document}

\frame{\titlepage}

\section{What do we know?}
\begin{frame}\frametitle{What do we know?}
\begin{columns}[T]
\begin{column}{0.6\linewidth}
\begin{itemize}
\item The concepts of \alert{differential equation} and \alert{initial value problem}
\begin{equation*}
F\big(x,y,y',\dotsc,y^{(n)}\big)=0
\end{equation*}
\item The concept of \alert{order} of a differential equation.
\item The concepts of \alert{general solution}, \alert{particular solution} and \alert{singular solution}.
\item \alert{Slope fields}
\item Approximations to solutions via \alert{Euler's Method} and \alert{Improved Euler's Method}
\end{itemize} 
\end{column}
\begin{column}{0.4\linewidth}
\begin{itemize}
\item Separable equations $y'=H_1(x) H_2(y)$
\item Homogeneous First-Order Equations $y'=H(y/x)$
\end{itemize}
\end{column}
\end{columns}
\end{frame}

\section{Linear First-Order Equations}
\subsection{Definition}
\begin{frame}\frametitle{Linear First-Order Equations}
\framesubtitle{Definition}
The linear first-order equation has the form
\begin{equation*}
	\frac{dy}{dx}+P(x)y=Q(x)	
\end{equation*}
\pause To solve it, we use an integrating factor (we disregard constant of integration now)
\begin{equation*}
\rho(x) = e^{\int P(x)\, dx} 
\end{equation*}
\pause The solution of the linear first-order equation can then be written in implicit form as
\begin{equation*}
\rho(x) y = C+ \int \rho(x) Q(x)\, dx
\end{equation*}
\pause \alert{Why?}
\end{frame}

\subsection{Proof}
\begin{frame}\frametitle{Linear First-Order Equations}
\framesubtitle{Proof}
\alert{Why does this formula work?}  We start with the differential equation
\begin{equation*}
	\frac{dy}{dx} + P(x)y = Q(x)
\end{equation*}
\pause Let us examine first the integration factor $\rho(x) = e^{\int P(x)\, dx}$. What is its derivative? \pause
\begin{equation*}
\frac{d}{dx} \big( \rho(x) ) = P(x) e^{\int P(x)\, dx} = P(x) \rho(x)
\end{equation*}
\pause The key of the formula for the solution of these equations is the product \alert{$\rho(x) y$}.
\pause If we take derivatives in this product, we get
\begin{align*}
\frac{d}{dx} \big( \rho(x) y \big) &= \frac{d}{dx} \rho(x) y + \rho(x) \frac{dy}{dx} \\
\uncover<6->{&= P(x) \alert{\rho(x)} y + \alert{\rho(x)} \frac{dy}{dx} \\}
\uncover<7->{&= \rho(x) \alert{\underbrace{\Big( P(x) y + \frac{dy}{dx} \Big)}_{\text{What is this?}}}} \uncover<8->{ = \rho(x) Q(x)}
\end{align*}
\end{frame}

\begin{frame}\frametitle{Linear First-Order Equations}
\framesubtitle{Proof}
We have just discovered that 
\begin{equation*}
	\frac{d}{dx} \big( \rho(x) y \big) = \rho(x) Q(x).
\end{equation*}
\pause Integrate both sides.
\begin{equation*}
	\rho(x) y = \int \rho(x) Q(x)\, dx \uncover<3->{+C}
\end{equation*}
\end{frame}

\subsection{Examples}
\begin{frame}\frametitle{Linear First-Order Equations}
\framesubtitle{Examples}
\begin{block}{Find a general solution}
\begin{equation*}
	\frac{dy}{dx}-y=e^{-x}	
\end{equation*}
\end{block}
\pause Let us identify $P$, $Q$, and compute $\rho$:
\begin{align*}
	\uncover<3->{\alert{P(x)}&\alert{=-1} &\alert{Q(x)}&\alert{=e^{-x}}} \uncover<4->{&\int P(x)\,dx &=-x} \uncover<5->{&\alert{\rho(x)} &\alert{=e^{-x}}}
\end{align*}
\pause\pause\pause\pause We need only one more ingredient: \pause
\begin{equation*}
	\alert{\int \rho(x)Q(x)\, dx} = \int e^{-x}e^{-x}\, dx = \int e^{-2x}\, dx = \alert{-\tfrac{1}{2}e^{-2x}}
\end{equation*}
\pause The solution is then
\begin{align*}
	\uncover<8->{\rho(x)\, y &= C + \int \rho(x)Q(x)\, dx} \\
	\uncover<9->{e^{-x} y &= C -\tfrac{1}{2}e^{-2x}}  \\
	\uncover<10->{y &= Ce^x-\tfrac{1}{2}e^{-x}}
\end{align*}
\end{frame}

\begin{frame}\frametitle{Linear First-Order Equations}
\framesubtitle{Examples}
\begin{block}{Find a general solution}
\begin{equation*}
	(x^2+1)\frac{dy}{dx}=6x-3xy
\end{equation*}
\end{block}
\pause Let us identify $P$, $Q$.  But first, we need to re-write the equation:
\begin{gather*}
	\uncover<3->{\frac{dy}{dx}=\frac{6x-3xy}{x^2+1} = \frac{6x}{x^2+1} - \frac{3x}{x^2+1}y \\}
	\uncover<4->{\frac{dy}{dx} + \alert{\underbrace{\frac{3x}{x^2+1}}_{P(x)}}\, y = \alert{\underbrace{\frac{6x}{x^2+1}}_{Q(x)}}}
\end{gather*}
\pause\pause\pause Let us compute the integrating factor:
\begin{align*}
\uncover<5->{\int P(x)\, dx &= \int \frac{3x}{x^2+1}\, dx = \tfrac{3}{2}\ln \lvert x^2+1 \rvert = \tfrac{3}{2} \ln (x^2+1) \\}
\uncover<6->{\alert{\rho(x)} &\alert{= e^{\tfrac{3}{2}\ln(x^2+1)} = (x^2+1)^{3/2}}}
\end{align*}
\end{frame}

\begin{frame}\frametitle{Linear First-Order Equations}
\framesubtitle{Examples}
\begin{block}{Find a general solution}
\begin{equation*}
	(x^2+1)\frac{dy}{dx}=6x-3xy
\end{equation*}
\end{block}
The final ingredient:
\begin{align*}
	\alert{\int \rho(x)Q(x)\, dx} \uncover<2->{&= \int (x^2+1)^{3/2} \frac{6x}{x^2+1}\, dx = \int 6x (x^2+1)^{1/2}\, dx \\}
	\uncover<3->{&= \underbrace{\int 3 u^{1/2}\, du}_{u=x^2+1} = 2 u^{3/2} = \alert{(x^2+1)^{3/2}}}
\end{align*}
\pause\pause\pause The solution is then
\begin{align*}
\rho(x)\,y &= C+\int \rho(x)Q(x)\, dx \\
\uncover<5->{(x^2+1)^{3/2} y &= (x^2+1)^{3/2} + C}  \\
\uncover<6->{y &= 1+C(x^2+1)^{-3/2}}
\end{align*}
\end{frame}

\begin{frame}\frametitle{Linear First-Order Equations}
\framesubtitle{Examples}
\begin{block}{Find a general solution}
\begin{equation*}
	y'=1+x+y+xy
\end{equation*}
\end{block}
We have seen this equation before, in the setting of \emph{separable equations}.  But we may re-write it also as a linear first-order too!
\begin{align*}
\frac{dy}{dx} &= (1+x)+y(1+x)  \uncover<2->{&\frac{dy}{dx} &\alert{\underbrace{- (1+x)}_{P(x)}}\, y = \alert{\underbrace{1+x}_{Q(x)}}}\\ 
\uncover<3->{\int P(x)\, dx &= -\int (1+x)\, dx = -x-\tfrac{1}{2}x^2} \uncover<4->{&\alert{\rho(x)} &\alert{= e^{-x-x^2/2}} }\\
\end{align*}
\begin{equation*}
\uncover<5->{\alert{\int \rho(x)Q(x)\, dx} = \int (1+x)e^{-x-x^2/2}\, dx} \uncover<6->{ =\underbrace{\int e^{-u}\, du}_{u=x+x^2/2} = -e^{-u} = \alert{-e^{-x-x^2/2}}}
\end{equation*}
\end{frame}

\begin{frame}\frametitle{Linear First-Order Equations}
\framesubtitle{Examples}
\begin{block}{Find a general solution}
\begin{equation*}
	y'=1+x+y+xy
\end{equation*}
\end{block}
The solution is then
\begin{align*}
\rho(x)\, y &= C + \int \rho(x) Q(x)\, dx \\
\uncover<2->{e^{-x-x^2/2} y & = C - e^{-x-x^2/2}}  \\
\uncover<3->{\alert{y} &\alert{= Ce^{x+x^2/2}-1}}
\end{align*}
\end{frame}

\section{Bernoulli Equation}
\subsection{Definition}
\begin{frame}\frametitle{Bernoulli Equation}
\framesubtitle{Definition}
The Bernoulli equation is very similar to the linear equation:
\begin{equation*}
	\frac{dy}{dx}+\overline{P}(x)y=\overline{Q}(x)\alert{y^n}
\end{equation*}
\pause We can turn it into one of those, by performing a substitution: $v=y^{1-n}$.
\begin{align*}
	\uncover<3->{y&=v^{1/(1-n)}} \uncover<4->{&\frac{dy}{dx} &= \frac{1}{1-n} v^{n/(1-n)} \frac{dv}{dx}}
\end{align*}
\pause\pause\pause We use this substitution in the original equation, to get rid of the $y$'s:
\begin{align*}
	\frac{dy}{dx}+\overline{P}(x)y &= \overline{Q}(x)y^n \\
	\uncover<6->{\frac{1}{1-n} v^{n/(1-n)} \frac{dv}{dx} +\overline{P}(x)v^{1/(1-n)} &= \overline{Q}(x) v^{n/(1-n)} \\}
	\uncover<7->{\frac{dv}{dx} + \alert{\underbrace{(1-n)\overline{P}(x)}_{P(x)}} v &= \alert{\underbrace{(1-n) \overline{Q}(x)}_{Q(x)}}}
\end{align*}
\end{frame}

\subsection{Examples}
\begin{frame}\frametitle{Bernoulli Equation}
\framesubtitle{Examples}
\begin{block}{Find a general solution}
\begin{equation*}
	x\frac{dy}{dx} +6y = 3xy^{4/3}
\end{equation*}
\end{block}
\pause Rewrite the equation to realize it is a Bernoulli, and find $\overline{P}$, $\overline{Q}$, $n$:
\begin{equation*}
	\uncover<3->{\frac{dy}{dx} + \frac{6}{x} y = 3y^{4/3}}
\end{equation*}
\pause It is $\overline{P}(x)=6/x$, $\overline{Q}(x)=3$, $n=4/3$.

\pause We need to apply the substitution $v=y^{1-4/3}=y^{-1/3}$.  \pause Rather than using the formula, it pays off to perform all the computations from scratch.
\begin{align*}
	\uncover<5->{y&=v^{-3}} \uncover<6->{&\frac{dy}{dx} &=-3v^{-4}\frac{dv}{dx}}
\end{align*}
\pause\pause\pause We get then
\begin{align*}
	\uncover<8->{\frac{dy}{dx} + \frac{6}{x} y &= 3 y^{4/3}}
	\uncover<9->{&-3v^{-4}\frac{dv}{dx} + \frac{6}{x} v^{-3} &= 3 v^{-4}}
	\uncover<10->{&\frac{dv}{dx} -\frac{2}{x} v &= -1}
\end{align*}
\end{frame}

\begin{frame}\frametitle{Bernoulli Equation}
\framesubtitle{Examples}
\begin{block}{Find a general solution}
\begin{equation*}
	x\frac{dy}{dx} +6y = 3xy^{4/3}
\end{equation*}
\end{block}
We need to solve now the linear first-order equation
\begin{equation*}
	\frac{dv}{dx} - \frac{2}{x}\, v = -1
\end{equation*}
Let us compute all the ingredients of the formula:
\begin{align*}
	\uncover<2->{P(x)&=-\frac{2}{x}}
	\uncover<3->{&Q(x)&=-1}
	\uncover<4->{&\int P(x)\, dx&=-2\ln \lvert x \rvert}
	\uncover<5->{&\rho(x)&=x^{-2}}
\end{align*}
\begin{equation*}
	\uncover<6->{\int \rho(x)Q(x)\, dx = \int -x^{-2} \, dx =  x^{-1}}
\end{equation*}
\pause\pause \pause\pause \pause\pause Therefore, the solution of this equation is
\begin{align*}
	\uncover<7->{x^{-2} \alert{v} &= C + x^{-1}} 
	\uncover<8->{& x^{-2} y^{-3} &= C + x^{-1}}
\end{align*}
\end{frame}

\begin{frame}\frametitle{Bernoulli Equation}
\framesubtitle{Examples}    
\begin{block}{Find a general solution}
\begin{equation*}
	(1+x^2) y'+xy=x^2y^2
\end{equation*}
\end{block}
\pause \begin{equation*}
	\frac{dy}{dx} + \alert{\underbrace{\frac{x}{1+x^2}}_{\overline{P}(x)}} y = \alert{\underbrace{\frac{x^2}{1+x^2}}_{\overline{Q}(x)}} y^{\alert{2}}
\end{equation*}
$n=2$.  We use the substitution $v=y^{1-2}=y^{-1}$.
\begin{align*}
	\uncover<3->{y&=v^{-1} &\frac{dy}{dx}&=-v^{-2}\frac{dv}{dx}}
\end{align*}
\pause\pause We get the linear equation
\begin{align*}
	\uncover<4->{\frac{dy}{dx}+\frac{x}{1+x^2}y &= \frac{x^2}{1+x^2} y^2}
	\uncover<6->{& \alert{\frac{dv}{dx} - \frac{x}{1+x^2} v}&\alert{= -\frac{x^2}{1+x^2}}} \\
	\uncover<5->{-v^{-2}\frac{dv}{dx} + \frac{x}{1+x^2}v^{-1} &= \frac{x^2}{1+x^2} v^{-2}} 
\end{align*}
\end{frame}

\begin{frame}\frametitle{Bernoulli Equation}
\framesubtitle{Examples}    
\begin{block}{Find a general solution}
\begin{equation*}
	(1+x^2) y'+xy=x^2y^2
\end{equation*}
\end{block}
Let us solve the linear equation:
\begin{equation*}
	\frac{dv}{dx}-\frac{x}{1+x^2}v = -\frac{x^2}{1+x^2}	
\end{equation*}
In this equation, $P(x)=-x/(1+x^2)$, $Q(x)=-x^2/(1+x^2)$.  We solve it in the usual way:
\begin{align*}
	\int P(x)\, dx &= -\int \frac{x}{1+x^2}\, dx = -\tfrac{1}{2} \ln \lvert 1+x^2 \rvert = \ln (1+x^2)^{-1/2} \\
	\rho(x) &= (1+x^2)^{-1/2}  \\
	\int \rho(x) Q(x)\, dx &= -\int \frac{x^2}{(1+x^2)^{3/2}}\, dx \alert{\longleftarrow \text{can you find this integral?}}
\end{align*}
\end{frame}
\end{document}