\PassOptionsToPackage{table}{xcolor}
\documentclass[9pt,xcolor=x11names,compress]{beamer}

%% General document %%%%%%%%%%%%%%%%%%%%%%%%%%%%%%%%%%
\usepackage{graphicx}
\usepackage{tikz}
\usepackage{array}
\usetikzlibrary{decorations.fractals,lindenmayersystems}
%%%%%%%%%%%%%%%%%%%%%%%%%%%%%%%%%%%%%%%%%%%%%%%%%%%%%%


%% Beamer Layout %%%%%%%%%%%%%%%%%%%%%%%%%%%%%%%%%%
\useoutertheme[subsection=false,shadow]{miniframes}
\useinnertheme{default}
\usefonttheme{serif}
\usepackage{palatino}

\setbeamerfont{title like}{shape=\scshape}
\setbeamerfont{frametitle}{shape=\scshape}

\setbeamercolor*{lower separation line head}{bg=DeepSkyBlue4} 
\setbeamercolor*{normal text}{fg=black,bg=white} 
\setbeamercolor*{alerted text}{fg=DeepSkyBlue4} 
\setbeamercolor*{example text}{fg=black} 
\setbeamercolor*{structure}{fg=black} 

\setbeamercolor*{palette tertiary}{fg=black,bg=black!10} 
\setbeamercolor*{palette quaternary}{fg=black,bg=black!10} 

\setbeamertemplate{blocks}[rounded][shadow=true]
\setbeamercolor{block title}{bg=DeepSkyBlue4}
\setbeamercolor{block title example}{bg=DeepSkyBlue4}
\setbeamercolor{block body}{bg=black!15!white}
\setbeamercolor{block body example}{bg=black!15!white}

\setbeamertemplate{navigation symbols}{}
%%%%%%%%%%%%%%%%%%%%%%%%%%%%%%%%%%%%%%%%%%%%%%%%%%
% Some useful definitions
\newcommand*\Laplace[1]{\mathcal{L}\{{#1}\}}
\newcommand*\BLaplace[1]{\mathcal{L}\Big\{{#1}\Big\}}
\newcommand*\iLaplace[1]{\mathcal{L}^{-1}\{{#1}\}}
\newcommand*\ibLaplace[1]{\mathcal{L}^{-1}\big\{{#1}\big\}}
\newcommand*\iBLaplace[1]{\mathcal{L}^{-1}\Big\{{#1}\Big\}}
%%%%%%%%%%%%%%%%%%%%%%%%%%%%%%%%%%%%%%%%%%%%%%%%%%

\author[Francisco Blanco-Silva]{Francisco Blanco-Silva}
\institute[USC]{University of South Carolina}
\date{
\pgfdeclarelindenmayersystem{levyCurve}{
	\rule{F -> F+F+}
}
\begin{tikzpicture}[color=DeepSkyBlue4,rotate=90]
    \draw [l-system={levyCurve, axiom=F, order=12, step=1pt, angle=90 }]
    lindenmayer system; 
	\end{tikzpicture}		
}

\title{Lesson 19: Integration of Transforms. Convolution} 

\begin{document}

\frame{\titlepage}

\section{What do we know?}
\subsection{General Program}

\begin{frame}\frametitle{What do we know?}
\begin{columns}[T]
\begin{column}{0.45\linewidth}
\begin{itemize}
\item The concepts of \alert{differential equation} and \alert{initial value problem}
\item The concept of \alert{order} of a differential equation.
\item The concepts of \alert{general solution}, \alert{particular solution} and \alert{singular solution}.
\item \alert{Slope fields}
\item Approximations to solutions via \alert{Euler's Method} and \alert{Improved Euler's Method}
\end{itemize} 
\end{column}
\begin{column}{0.55\linewidth}
\begin{itemize}
\item First-Order Differential Equations
\begin{itemize}
\item Separable equations 
\item Homogeneous First-Order Equations 
\item Linear First-Order Equations 
\item Bernoulli Equations 
\item General Substitution Methods
\item Exact Equations 
\end{itemize}
\item Second-Order Differential Equations
\begin{itemize}
	\item Reducible Equations
	\item General Linear Equations (Intro)
	\item Linear Equations with Constant Coefficients
	\begin{itemize}
		\item Characteristic Equation
		\item Variation of Parameters
		\item Undetermined Coefficients
	\end{itemize}
\end{itemize}
\end{itemize}
\end{column}
\end{columns}
\end{frame}

\begin{frame}\frametitle{What do we know?}
\framesubtitle{Laplace Transforms}
\begin{center}
\begin{tikzpicture}[scale=0.7]
\node[scale=0.8]{
\rowcolors{2}{gray!25!white}{white}
	\begin{tabular}{m{2cm}||m{3.5cm}l}
	\rowcolor{DeepSkyBlue4}
		$f(x)$\raisebox{0.5cm} & $\mathcal{L}\{f\}=\int_0^\infty e^{-sx}f(x)\, dx$\raisebox{0.5cm} & \\[0.4cm]
		$1$ & $\dfrac{1}{s}$\raisebox{0.6cm} & $s>0$ \\[0.4cm]
		$x^p$ & $\dfrac{\Gamma(p+1)}{s^{p+1}}$\raisebox{0.6cm} & $s>0$\\[0.4cm]
		$e^{\alpha x}$ & $\dfrac{1}{s-\alpha}$\raisebox{0.6cm} & $s>\alpha$ \\[0.4cm]
		$\sin \beta x$ & $\dfrac{\beta}{s^2+\beta^2}$\raisebox{0.6cm} & $s>0$ \\[0.4cm]
		$\cos \beta x$ & $\dfrac{s}{s^2+\beta^2}$\raisebox{0.6cm} & $s>0$\\[0.4cm]
		$cf(x)\pm g(x)$ & $cF(s) \pm G(s)$\raisebox{0.4cm} & $s>max(a,b)$ \\[0.4cm]
		$x^n f(x)$ & $(-1)^n F^{(n)}$\raisebox{0.4cm} & $s>a$ \\[0.4cm]
		$e^{\alpha x}f(x)$ & $F(s-\alpha)$\raisebox{0.4cm} & $s>a+\alpha$ \\[0.4cm]
	\end{tabular}};
	\end{tikzpicture}
\end{center}
\end{frame}

\section{Warm-up}
\subsection{Review of techniques}
\begin{frame}\frametitle{Warm-up}
    
\framesubtitle{Review of techniques}
\begin{example}
	Compute the inverse Laplace transform of the following function, using \emph{linearization} and \emph{partial fraction decomposition}
	\begin{equation*}
		F(s)=\frac{2s-3}{(s-1)(s^2+4)}\quad (s>1)
	\end{equation*}
\end{example}
\pause Note that
\begin{equation*}
	\frac{2s-3}{(s-1)(s^2+4)}=\frac{A}{s-1}+\frac{Bs+C}{s^2+4}
\end{equation*}
\pause A quick computation gives $A=-1/5$, $B=1/5$ and $C=11/5$.  It is then
\begin{align*}
	\uncover<4->{\iBLaplace{\frac{2s-3}{(s-1)(s^2+4)}}&=\iBLaplace{-\frac{1}{5}\cdot \frac{1}{s-1} + \frac{1}{5} \cdot \frac{s+11}{s^2+4}} \\}
	\uncover<5->{&=-\frac{1}{5} \iBLaplace{\frac{1}{s-1}} + \frac{1}{5} \iBLaplace{\frac{s}{s^2+4}} + \frac{11}{5} \iBLaplace{\frac{1}{s^2+4}} \\}
	\uncover<6->{&=-\frac{1}{5} \iBLaplace{\frac{1}{s-1}} + \frac{1}{5} \iBLaplace{\frac{s}{s^2+4}} + \frac{11}{\alert{10}} \iBLaplace{\frac{\alert{2}}{s^2+4}} \\}
	\uncover<7->{&=-\frac{1}{5} e^x + \frac{1}{5} \cos 2x + \frac{11}{10} \sin 2x \\}
\end{align*}
\end{frame}

\begin{frame}\frametitle{Warm-up}
    
\framesubtitle{Review of techniques}
\begin{block}
	{Compute the Laplace transform of}
	\begin{equation*}
		xe^{3x}\sin 4x
	\end{equation*}
\end{block}
We have to use two tricks here.  First, the exponential $e^{3x}$ suggests that the Laplace transform of $e^{3x}x\sin 4x$ is $F(s-3)$, where $F(s)$ is the Laplace transform of $x\sin 4x$.  

\pause But to compute the Laplace transform of $x\sin 4x$ we must use the technique of \emph{derivative of transforms}:  Set $g(x)=\sin 4x$, which gives $G(s)=\dfrac{4}{s^2+16}$ for $s>0$ its Laplace transform.  The Laplace transform of $x\sin 4x$ is then $F(s)=-G'(s)$.
\pause The derivative of $G(s)$ is
\begin{equation*}
	G'(s)=-4(s^2+16)^{-2}(2s)=\frac{-8s}{(s^2+16)^2}
\end{equation*}
We have then that the Laplace transform of $e^{3x}x\sin 4x$ is
\begin{equation*}
	\Laplace{e^{3x}x\sin 4x} = F(s-3) = -G'(s-3) = \frac{8(s-3)}{\big( (s-3)^2+16\big)^2}
\end{equation*}
\end{frame}

\section{Integration of Transforms}
\subsection{Motivation}
\begin{frame}\frametitle{Integration of Transforms}
    
At this stage, the next logical step is to be able to compute the Laplace transform of fractions, but carefully.  We cannot allow our fractions to have zeros in the denominator, in the interval $(0,\infty)$.
\begin{theorem}
	Suppose that $f(x)$ satisfies the three conditions below:
	\begin{itemize}
		\item $f(x)$ is piecewise continuous for $x\geq 0$.
		\item $\displaystyle{\lim_{x\to 0^+} \frac{f(x)}{x}}$ exists and it is finite.
		\item There exist constants $M,c$ so that $\lvert f(x) \rvert \leq Me^{cx}$ as $x\to \infty$.
	\end{itemize}
	Then for $s>c$, 
	\begin{equation*}
		\BLaplace{\frac{f(x)}{x}}=\int_s^\infty F(\sigma)\, d\sigma
	\end{equation*}
\end{theorem}
\end{frame}

\begin{frame}\frametitle{Integration of Transforms}
    
\framesubtitle{Examples}
\begin{example}
	Compute the Laplace transform of $x^{-1}\sin x$..
\end{example}
\pause Notice that we may write this function in the form $f(x)/x$, where $f(x)=\sin x$. The Laplace transform of $f(x)$ is $F(s)=(s^2+1)^{-1}$, for $s>0$.  It is then
\begin{align*}
	\alert<6>{\BLaplace{x^{-1}\sin x}}&=\int_s^\infty F(\sigma)\, d\sigma = \int_s^\infty \frac{d\sigma}{\sigma^2+1} \\
	\uncover<3->{&= \lim_{A\to\infty} \int_s^A \frac{d\sigma}{\sigma^2+1} \\}
	\uncover<4->{&= \lim_{A\to\infty} \tan^{-1}(\sigma)\Big\rvert_s^A \\}
	\uncover<5->{&= \lim_{A\to\infty} \big(\tan^{-1}(A) - \tan^{-1}(s) \big) \\}
	\uncover<6->{&\alert{= \frac{\pi}{2}-\tan^{-1}(s)}}
\end{align*}
\end{frame}

\section{Convolutions}
\subsection{Motivation}

\begin{frame}\frametitle{Convolutions}
 
We are ready to take now inverse Laplace transforms of products.  The rule is simple:
\begin{theorem}
	Given $f(x)$ and $g(x)$ \emph{good enough} functions with Laplace transforms $F(s)$ and $G(s)$ respectively, both for $s>c\geq 0$.  Then 
	\begin{equation*}
		\ibLaplace{F(s)G(s)} = \int_0^x f(x-t)g(t)\, dt = \int_0^x f(t)g(x-t)\, dt
	\end{equation*}
\end{theorem}
\pause These two integrals offer exactly the same result (one is usually easier to compute than the other).  We refer to this integral operation as the \alert{convolution} of $f$ with $g$, and denote it $(f\star g)(x)$.
\end{frame}

\subsection{Examples}

\begin{frame}\frametitle{Convolutions}
    
\framesubtitle{Examples}
\begin{block}
	{Find the inverse Laplace transform of}
	\begin{equation*}
		H(s)=\frac{5}{s^2(s^2+25)}\quad (s>0)
	\end{equation*}
\end{block}
\pause We may regard $H(s)$ as the product of $1/s^2$ and $5/(s^2+25)$ which, according to the table, are the transforms of $f(x)=x$ and $g(x)=\sin 5x$ respectively.

\pause It is then
\begin{align*}
	\alert<8>{\iBLaplace{\frac{5}{s^2(s^2+25)}}}&=\iBLaplace{\frac{1}{s^2} \cdot \frac{5}{s^2+25}}
	\uncover<4->{=(f \star g)(x) = \int_0^x f(x-t)g(t)\,dt \\}
	\uncover<5->{&= \int_0^x (x-t) \sin (5t)\, dt}
	\uncover<6->{= x\int_0^x \sin 5t\, dt - \int_0^x \underbrace{t}_u\underbrace{\sin 5t\, dt}_{dv} \\}
	\uncover<7->{&= x \bigg(-\frac{1}{5}\cos 5t \bigg\rvert_0^x \bigg) + \frac{t}{5}\cos 5t \bigg\rvert_0^x -\frac{1}{5}\int_0^x \cos 5t\, dt \\}
	\uncover<8->{&= \alert{\frac{x}{5} - \frac{1}{25}\sin 5x}}
\end{align*}

\end{frame}

\end{document}