\documentclass[12pt]{article}

\usepackage{amsmath,amsthm,amsfonts,amssymb,amsxtra}
\usepackage{pgf,tikz}
\usetikzlibrary{arrows}
\renewcommand{\theenumi}{(\alph{enumi})} 
\renewcommand{\labelenumi}{\theenumi}

\pagestyle{empty}
\setlength{\textwidth}{7in}
\setlength{\oddsidemargin}{-0.5in}
\setlength{\topmargin}{-1.0in}
\setlength{\textheight}{9.5in}

\newtheorem{problem}{Problem}

\begin{document}

\noindent{\large\bf MATH 142}\hfill{\large\bf Final Exam.}\hfill{\large\bf
  Summer II 2012}\hfill{\large\bf Page 1/10}\hrule

\bigskip
\begin{center}
  \begin{tabular}{|ll|}
    \hline & \cr
    {\bf Name: } & \makebox[12cm]{\hrulefill}\cr & \cr
    {\bf 4-digit code:} & \makebox[12cm]{\hrulefill}\cr & \cr
    \hline
  \end{tabular}
\end{center}
\begin{itemize}
\item Write your name and the last 4 digits of your SSN in the space provided above.
\item The test has sixteen (10) pages including this one.
\item Enter your answer in the box(es) provided.
\item You must show sufficient work to justify all answers unless
  otherwise stated in the problem.  Correct answers with inconsistent
  work may not be given credit.
\item Credit for each problem is given in parentheses at the right of
  the problem number.
\item No books, notes or calculators may be used on this test.
\end{itemize}
\hrule

\begin{center}
  \begin{tabular}{|c|c|c|}
    \hline
    &&\cr
    {\large\bf Page} & {\large\bf Max} & {\large\bf Points} \cr
    &&\cr
    \hline
    &&\cr
    {\Large \mbox{}} & \Large \mbox{} & \cr
    &&\cr
    \hline
    &&\cr
    {\Large 2} & \Large 10 & \cr
    &&\cr
    \hline
    &&\cr
    {\Large 3} & \Large 10 & \cr
    &&\cr
    \hline
    &&\cr
    {\Large 4} & \Large 10 & \cr
    &&\cr
    \hline
    &&\cr
    {\Large 5} & \Large 10 & \cr
    &&\cr
    \hline\hline
    &&\cr
    {\large\bf Total} & \Large 40 & \cr
    &&\cr
    \hline
  \end{tabular}
  \begin{tabular}{|c|c|c|}
    \hline
    &&\cr
    {\large\bf Page} & {\large\bf Max} & {\large\bf Points} \cr
    &&\cr
    \hline
    &&\cr
    {\Large 6} & \Large 5 & \cr
    &&\cr
    \hline
    &&\cr
    {\Large 7} & \Large 20 & \cr
    &&\cr
    \hline
    &&\cr
    {\Large 8} & \Large 10 & \cr
    &&\cr
    \hline
    &&\cr
    {\Large 9} & \Large 15 & \cr
    &&\cr
    \hline
    &&\cr
    {\Large 10} & \Large 10 & \cr
    &&\cr
    \hline\hline
    &&\cr
    {\large\bf Total} & \Large 60 & \cr
    &&\cr
    \hline
  \end{tabular}
\end{center}
\newpage

%%%%%%%%%%%%%%%%%%%%%%%%%%%%%%%%%%%%% Page 2
\noindent{\large\bf MATH 142}\hfill{\large\bf Final Exam}\hfill{\large\bf
  Summer II 2012}\hfill{\large\bf Page 2/10}\hrule

\bigskip
{\problem[5 pts] \em  Find the area of the region that is enclosed between the
curves $x=2y^2$ and $x=4+y^2$.} 
\vspace{8.5cm}
\begin{flushright}
  \begin{tikzpicture}
    \draw (-1cm,0.5cm) node {$A = $};
    \draw (0cm,-0.2cm) rectangle (5cm,1.2cm);
  \end{tikzpicture}
\end{flushright}
\hrule
{\problem[5 pts] \em Find the volume of the solid that is obtained by rotating
the region bounded by the curves $x=2\sqrt{y}$, $x=0$ and $y=9$ about the
$y$-axis.} \vspace{8.5cm}
\begin{flushright}
  \begin{tikzpicture}
    \draw (-1cm,0.5cm) node {$V = $};
    \draw (0cm,-0.2cm) rectangle (5cm,1.2cm);
  \end{tikzpicture}
\end{flushright}
\newpage

%%%%%%%%%%%%%%%%%%%%%%%%%%%%%%%%%%%%% Page 3
\noindent{\large\bf MATH 142}\hfill{\large\bf Final Exam}\hfill{\large\bf
  Summer II 2012}\hfill{\large\bf Page 3/10}\hrule

{\problem[5 pts] \em Use the Fundamental Theorem of Calculus to find the derivative of
 \begin{equation*}
f(x) = \int_{1-3x}^1 \frac{u^3}{1+u^2}\, du.
\end{equation*}
\vspace{6.5cm}
\begin{flushright}
  \begin{tikzpicture}
    \draw (-1cm,0.5cm) node {$f'(x) = $};
    \draw (0cm,-0.2cm) rectangle (5cm,1.2cm);
  \end{tikzpicture}
\end{flushright}
\hrule
{\problem[5 pts] \em Find a positive value of $k$ such that the average value
of $f(x) = 2+6x-3x^2$ over the interval $[0,k]$ is equal to $3$.} 
\vspace{8cm}
\begin{flushright}
  \begin{tikzpicture}
    \draw (-1cm,0.5cm) node {$k = $};
    \draw (0cm,-0.2cm) rectangle (5cm,1.2cm);
  \end{tikzpicture}
\end{flushright}
\newpage

%%%%%%%%%%%%%%%%%%%%%%%%%%%%%%%%%%%%% Page 4
\noindent{\large\bf MATH 142}\hfill{\large\bf Final Exam}\hfill{\large\bf
  Summer II 2012}\hfill{\large\bf Page 4/10}\hrule

\bigskip
{\problem[5 pts] \em Use \textbf{integration by parts} to evaluate the integral $\displaystyle{\int xe^{2x}\, dx}$.}
\vspace{6cm}
\begin{flushright}
  \begin{tikzpicture}
    \draw (-1.25cm,0.5cm) node {$\displaystyle{\int xe^{2x}\, dx} = $};
    \draw (0cm,-0.2cm) rectangle (5cm,1.2cm);
  \end{tikzpicture}
\end{flushright}
\hrule
{\problem[5 pts] \em Evaluate the improper integral $\displaystyle{\int_1^\infty \frac{dx}{x^3}}$.}
\vspace{10cm}
\begin{flushright}
  \begin{tikzpicture}
    \draw (-1.25cm,0.5cm) node {$\displaystyle{\int_1^\infty \frac{dx}{x^3}} = $};
    \draw (0cm,-0.2cm) rectangle (5cm,1.2cm);
  \end{tikzpicture}
\end{flushright}
\newpage

%%%%%%%%%%%%%%%%%%%%%%%%%%%%%%%%%%%%% Page 5
\noindent{\large\bf MATH 142}\hfill{\large\bf Final Exam}\hfill{\large\bf
  Summer II 2012}\hfill{\large\bf Page 5/10}\hrule

\bigskip
{\problem[10 pts] \em Use a \textbf{trigonometric substitution} to evaluate the integral $\displaystyle{\int \frac{dx}{\sqrt{x^2-9}}}$.}
\vspace{18cm}
\begin{flushright}
  \begin{tikzpicture}
    \draw (-4.5cm,0.5cm) node {$\displaystyle{\int \frac{dx}{\sqrt{x^2-9}}} = $};
    \draw (-3cm,-0.2cm) rectangle (5cm,1.2cm);
  \end{tikzpicture}
\end{flushright}
\newpage

%%%%%%%%%%%%%%%%%%%%%%%%%%%%%%%%%%%%% Page 6
\noindent{\large\bf MATH 142}\hfill{\large\bf Final Exam}\hfill{\large\bf
  Summer II 2012}\hfill{\large\bf Page 6/10}\hrule
  
\bigskip
{\problem[5 pts] \em Use \textbf{partial fractions} to evaluate the integral $\displaystyle{\int \frac{dx}{x^2+x-2}}$.}
\vspace{18cm}
\begin{flushright}
  \begin{tikzpicture}
    \draw (-5.23cm,0.5cm) node {$\displaystyle{\int \frac{dx}{x^2+x-2}} = $};
    \draw (-3cm,-0.2cm) rectangle (5cm,1.2cm);
  \end{tikzpicture}
\end{flushright}

\newpage

%%%%%%%%%%%%%%%%%%%%%%%%%%%%%%%%%%%%% Page 7
\noindent{\large\bf MATH 142}\hfill{\large\bf Final Exam}\hfill{\large\bf
  Summer II 2012}\hfill{\large\bf Page 7/10}\hrule

\bigskip
{\problem[10 pts] \em  Find a formula for the general term of the following sequences:} 
\begin{enumerate}
\item $\displaystyle{\frac{1}{2}, \frac{3}{4}, \frac{5}{6}, \frac{7}{8}, \dotsc}$
\bigskip
\begin{flushright}
  \begin{tikzpicture}
    \draw (-0.5cm,0.5cm) node {$x_n = $};
    \draw (0cm,-0.2cm) rectangle (5cm,1.2cm);
  \end{tikzpicture}
\end{flushright}
\item $\displaystyle{1-\frac{1}{2}, \frac{1}{3} - \frac{1}{2}, \frac{1}{3} - \frac{1}{4}, \frac{1}{5}-\frac{1}{4}, \dotsc}$
\vspace{3cm}
\begin{flushright}
  \begin{tikzpicture}
    \draw (-0.5cm,0.5cm) node {$x_n = $};
    \draw (0cm,-0.2cm) rectangle (5cm,1.2cm);
  \end{tikzpicture}
\end{flushright}
\end{enumerate}
\hrule
{\problem[10pts] \em Write out the first five terms of the sequence $\left\{ \displaystyle{\frac{(-1)^{n+1}}{n^2}} \right\}_{n=1}^\infty$ \newline Determine whether the sequence converges, and if so find its limit.
\vspace{7cm}
\begin{flushright}
  \begin{tikzpicture}
    \draw (-3.75cm,1.9cm) node {First five terms:};
    \draw (-2cm, 1.3cm) rectangle (5cm, 2.7cm);
    \draw (-1.25cm,0.5cm) node {$\displaystyle{\lim_{n \to \infty} x_n} = $};
    \draw (0cm,-0.2cm) rectangle (5cm,1.2cm);
  \end{tikzpicture}
\end{flushright}
\newpage


%%%%%%%%%%%%%%%%%%%%%%%%%%%%%%%%%%%%% Page 8
\noindent{\large\bf MATH 142}\hfill{\large\bf Final Exam}\hfill{\large\bf
  Summer II 2012}\hfill{\large\bf Page 8/10}\hrule

\bigskip
{\problem[10 pts] \em Determine whether the series converge, and if so find their sum:}
\begin{enumerate}
\item $\displaystyle{\sum_{k=1}^\infty \Big( -\frac{3}{4} \Big)^{k-1}}$
\vspace{7cm}
\begin{flushright}
  \begin{tikzpicture}
    \draw (-1.75cm,0.5cm) node {$\displaystyle{\sum_{k=1}^\infty \Big( -\frac{3}{4} \Big)^{k-1} }= $};
    \draw (0cm,-0.2cm) rectangle (5cm,1.2cm);
  \end{tikzpicture}
\end{flushright}
\item $\displaystyle{\sum_{k=1}^\infty \frac{1}{(k+2)(k+3)}}$
\vspace{7cm}
\begin{flushright}
  \begin{tikzpicture}
    \draw (-2cm,0.5cm) node {$\displaystyle{\sum_{k=1}^\infty \frac{1}{(k+2)(k+3)}  }= $};
    \draw (0cm,-0.2cm) rectangle (5cm,1.2cm);
  \end{tikzpicture}
\end{flushright}
\end{enumerate}
\newpage

%%%%%%%%%%%%%%%%%%%%%%%%%%%%%%%%%%%%% Page 9
\noindent{\large\bf MATH 142}\hfill{\large\bf Final Exam}\hfill{\large\bf
  Summer II 2012}\hfill{\large\bf Page 9/10}\hrule

\bigskip
{\problem[5 pts] \em Apply the \textbf{divergence test} and state what it tells you about the series.}
\begin{equation*}
\sum_{k=1}^\infty \frac{k^2+k+3}{2k^2+1}.
\end{equation*}
\vspace{2cm}
\hrule
{\problem[5 pts] \em Use the \textbf{integral test} to determine whether the series $\displaystyle{\sum_{k=1}^\infty \frac{1}{5k+2}}$ converges.}
\vspace{6.5cm}
\hrule
{\problem[5 pts] \em Use the \textbf{ratio test} to determine whether the series $\displaystyle{\sum_{k=1}^\infty \frac{4^k}{k^2}}$ converges.  If the test is inconclusive, then say so.}
\newpage

%%%%%%%%%%%%%%%%%%%%%%%%%%%%%%%%%%%%% Page 10
\noindent{\large\bf MATH 142}\hfill{\large\bf Final Exam}\hfill{\large\bf
  Summer II 2012}\hfill{\large\bf Page 10/10}\hrule
  
\bigskip

{\problem[5 pts] \em Use the \textbf{root test} to determine whether the series $\displaystyle{\sum_{k=1}^\infty \Big( \frac{3k+2}{2k-1} \Big)^k}$ converges.  If the test is inconclusive, then say so.}
\vspace{9cm}
\hrule{\problem[5 pts] \em Classify the series $\displaystyle{\sum_{k=1}^\infty (-1)^k\, \frac{4k^2+1}{k^3+2}}$ as absolutely convergent, convergent or divergent.}

\end{document}




