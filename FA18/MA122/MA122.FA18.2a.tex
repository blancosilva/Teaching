\documentclass[12pt]{article}

\usepackage{amsmath,amsthm,amsfonts,amssymb,amsxtra}
\usepackage{pgf,tikz}
\usetikzlibrary{arrows}
\renewcommand{\theenumi}{(\alph{enumi})} 
\renewcommand{\labelenumi}{\theenumi}

\pagestyle{empty}
\setlength{\textwidth}{7in}
\setlength{\oddsidemargin}{-0.5in}
\setlength{\topmargin}{-1.0in}
\setlength{\textheight}{9.5in}

\theoremstyle{definition}
\newtheorem{problem}{Problem}

\newcommand{\sectionlinetwo}[2]{%
  \nointerlineskip \vspace{.5\baselineskip}\hspace{\fill}
  {\color{#1}
    \resizebox{0.5\linewidth}{2ex}
    {{%
        {\begin{tikzpicture}
            \node  (C) at (0,0) {};
            \node (D) at (9,0) {};
            \path (C) to [ornament=#2] (D);
          \end{tikzpicture}}}}}%
  \hspace{\fill}
  \par\nointerlineskip \vspace{.5\baselineskip}
}

\makeatletter
\newcommand*{\radiobutton}{%
  \@ifstar{\@radiobutton0}{\@radiobutton1}%
}
\newcommand*{\@radiobutton}[1]{%
  \begin{tikzpicture}
    \pgfmathsetlengthmacro\radius{height("X")/2}
    \draw[radius=\radius] circle;
    \ifcase#1 \fill[radius=.6*\radius] circle;\fi
  \end{tikzpicture}%
}
\makeatother

\begin{document}

\noindent{\large\bf MATH 122}\hfill{\large\bf Exam \#2.}\hfill{\large\bf Fall 2018}\hfill{\large\bf Page 1/4}\hrule

\bigskip
\begin{center}
  \begin{tabular}{|ll|}
    \hline & \cr
             {\bf Name: } & \makebox[12cm]{\hrulefill}\cr & \cr
                                                            {\bf VIP ID:} & \makebox[12cm]{\hrulefill}\cr & \cr
                                                                                                            \hline
  \end{tabular}
\end{center}
\begin{itemize}
\item Write your name and VIP ID in the space provided above.
\item The test has four (4) pages, including this one.
\item Enter your answer in the box(es) provided.
\item You must show sufficient work to justify all answers unless otherwise stated in the problem.  Correct answers with
  inconsistent work may not be given credit.
\item Credit for each problem is given in parentheses at the right of the problem number.
\item No books or notes may be used on this test.
\item An approved calculator may be used on this test.
\end{itemize}
\hrule
\vspace{2cm}

\begin{center}
  \begin{tabular}{|c|c|c|}
    \hline
    &&\cr
       {\large\bf Page} & {\large\bf Max.~points} & {\large\bf Your points} \cr
    &&\cr
       \hline
    &&\cr
       {\Large 2} & \Large 30 & \cr
    &&\cr
       \hline
    &&\cr
       {\Large 3} & \Large 40 & \cr
    &&\cr
       \hline
    &&\cr
       {\Large 4} & \Large 30 & \cr
    &&\cr
       \hline\hline
    &&\cr
       {\large\bf Total} & \Large 100 & \cr
    &&\cr
       \hline
  \end{tabular}
\end{center}
\newpage

%%%%%%%%%%%%%%%%%%%%%%%%%%%%%%%%%%%%% Page 2
\noindent{\large\bf MATH 122}\hfill{\large\bf Exam \#2.}\hfill{\large\bf Fall 2018}\hfill{\large\bf Page 2/4}\hrule

\bigskip

\begin{problem}(5 pts each)
  Find the derivative of the following functions:
  \begin{enumerate}
  \item $f(x) = 56$
    \begin{flushright}
      \begin{tikzpicture}
        \draw (-4cm,0.5cm) node {$f'(x)=$};
        \draw (-3cm,-0.2cm) rectangle (5cm,1.2cm);
      \end{tikzpicture}
    \end{flushright}
  \item $y = t + \sqrt{t}$
    \begin{flushright}
      \begin{tikzpicture}
        \draw (-4cm,0.5cm) node {$y'(t)=$};
        \draw (-3cm,-0.2cm) rectangle (5cm,1.2cm);
      \end{tikzpicture}
    \end{flushright}
  \item $f(x) = e^x + 2^x + 3 \cdot 3^x$
    \begin{flushright}
      \begin{tikzpicture}
        \draw (-4cm,0.5cm) node {$f'(x)=$};
        \draw (-3cm,-0.2cm) rectangle (5cm,1.2cm);
      \end{tikzpicture}
    \end{flushright}
  \item $f(x) = \ln x - \ln 5$
    \begin{flushright}
      \begin{tikzpicture}
        \draw (-4cm,0.5cm) node {$f'(x)=$};
        \draw (-3cm,-0.2cm) rectangle (5cm,1.2cm);
      \end{tikzpicture}
    \end{flushright}
  \end{enumerate}
\end{problem}
\hrule

\begin{problem}[10 pts]
  Find all points at which the tangent line to the graph of the following function is horizontal:
  \begin{equation*}
    y = f(x) = 2x^3 - 3x^2 - 36x
  \end{equation*}

  \vspace{7cm}
  \begin{flushright}
    \begin{tikzpicture}
      \draw (0,-0.2cm) rectangle (12cm,1.2cm); 
    \end{tikzpicture}
  \end{flushright}
\end{problem}
\newpage

%%%%%%%%%%%%%%%%%%%%%%%%%%%%%%%%%%%%% Page 3
\noindent{\large\bf MATH 122}\hfill{\large\bf Exam \#2.}\hfill{\large\bf Fall 2018}\hfill{\large\bf Page 3/4}\hrule

\bigskip

\begin{problem}(5 pts each)
  Find the derivative of the following functions:
  \begin{enumerate}
  \item $f(x) = \big( 3x^5 \big)^2$
    \begin{flushright}
      \begin{tikzpicture}
        \draw (-4cm,0.5cm) node {$f'(x)=$};
        \draw (-3cm,-0.2cm) rectangle (5cm,1.2cm);
      \end{tikzpicture}
    \end{flushright}
  \item $f(x) = \sqrt{\dfrac{1}{x^{39}}}$
    \begin{flushright}
      \begin{tikzpicture}
        \draw (-4cm,0.5cm) node {$f'(x)=$};
        \draw (-3cm,-0.2cm) rectangle (5cm,1.2cm);
      \end{tikzpicture}
    \end{flushright}
  \item $y = 6t^5 - 10\sqrt{t} + \frac{9}{t}$
    \begin{flushright}
      \begin{tikzpicture}
        \draw (-4cm,0.5cm) node {$y'(t)=$};
        \draw (-3cm,-0.2cm) rectangle (5cm,1.2cm);
      \end{tikzpicture}
    \end{flushright}
  \item $f(x) = (2^x + x^5)(3 - \ln x)$
    \begin{flushright}
      \begin{tikzpicture}
        \draw (-4cm,0.5cm) node {$f'(x)=$};
        \draw (-3cm,-0.2cm) rectangle (5cm,1.2cm);
      \end{tikzpicture}
    \end{flushright}
  \item $f(x) = \dfrac{x^8+2}{x}$
    \begin{flushright}
      \begin{tikzpicture}
        \draw (-4cm,0.5cm) node {$f'(x)=$};
        \draw (-3cm,-0.2cm) rectangle (5cm,1.2cm);
      \end{tikzpicture}
    \end{flushright}
  \item $f(x) = \ln \big(8 - e^{-x}\big)$
    \begin{flushright}
      \begin{tikzpicture}
        \draw (-4cm,0.5cm) node {$f'(x)=$};
        \draw (-3cm,-0.2cm) rectangle (5cm,1.2cm);
      \end{tikzpicture}
    \end{flushright}
  \item $f(x) = \big( 6 + \ln x \big)^{0.6}$
    \begin{flushright}
      \begin{tikzpicture}
        \draw (-4cm,0.5cm) node {$f'(x)=$};
        \draw (-3cm,-0.2cm) rectangle (5cm,1.2cm);
      \end{tikzpicture}
    \end{flushright}
  \item $f(x) = 2e^{7x} + e^{-x^6}$
    \begin{flushright}
      \begin{tikzpicture}
        \draw (-4cm,0.5cm) node {$f'(x)=$};
        \draw (-3cm,-0.2cm) rectangle (5cm,1.2cm);
      \end{tikzpicture}
    \end{flushright}
  \end{enumerate}
\end{problem}

\newpage

%%%%%%%%%%%%%%%%%%%%%%%%%%%%%%%%%%%%% Page 4
\noindent{\large\bf MATH 122}\hfill{\large\bf Exam \#2.}\hfill{\large\bf Fall 2018}\hfill{\large\bf Page 4/4}\hrule

\bigskip

\begin{problem}[10 pts]
  Find an equation for the tangent line to the graph of $f(x) = 3x^2-5x+6$ at $x=1$.

  \vspace{4cm} 

  \begin{flushright}
    \begin{tikzpicture}
      \draw (-3.5cm,0.5cm) node {$y=$};
      \draw (-3cm,-0.2cm) rectangle (5cm,1.2cm);
    \end{tikzpicture}
  \end{flushright}
\end{problem}
\hrule

\begin{problem}[10 pts]
  Find an equation for the tangent line to the graph of $f(x) = (2x^2-1)(3x+4)$ at $x=0$.

  \vspace{5cm} 

  \begin{flushright}
    \begin{tikzpicture}
      \draw (-3.5cm,0.5cm) node {$y=$};
      \draw (-3cm,-0.2cm) rectangle (5cm,1.2cm);
    \end{tikzpicture}
  \end{flushright}
\end{problem}
\hrule

\begin{problem}[10 pts]
  Find an equation for the tangent line to the graph of $f(x) = (2x^2-8)^5$ at $x=2$.

  \vspace{5cm} 

  \begin{flushright}
    \begin{tikzpicture}
      \draw (-3.5cm,0.5cm) node {$y=$};
      \draw (-3cm,-0.2cm) rectangle (5cm,1.2cm);
    \end{tikzpicture}
  \end{flushright}
\end{problem}

\end{document}

%%% Local Variables:
%%% mode: latex
%%% TeX-master: t
%%% End:
