\documentclass[11pt]{article}

\usepackage{amsmath,amsfonts,amsthm,amssymb,amsxtra}
\usepackage{pgf,tikz}
\usepackage{tkz-euclide}
\usetkzobj{all}
\usepackage{mathrsfs, verbatim}
\usepackage{framed}

\renewcommand{\theenumi}{(\alph{enumi})} 
\renewcommand{\labelenumi}{\theenumi}

\pagestyle{empty}
\setlength{\textwidth}{7in}
\setlength{\oddsidemargin}{-0.5in}
\setlength{\topmargin}{-1.0in}
\setlength{\textheight}{9.5in}

\newtheorem*{remark}{Remark}
\theoremstyle{definition}
\newtheorem{problem}{Problem}
\newtheorem*{problem*}{Sample Problem}
\theoremstyle{theorem}
\newtheorem{proposition}{Proposition}
\newtheorem*{proposition*}{Proposition}

%%% Solutions macro 
% To include solutions, leave the first two lines and comment out the next two.
% To not include solutions, do the reverse.
%%%%%%%%%%%%%%%%%%%%%%
% \renewcommand{\proofname}{Solution:}
\renewcommand\proof\comment
%%%%%%%%%%%%%%%%%%%%%%%


\begin{document}

\noindent{\large\bf MATH 300}\hfill{\large\bf Exploratory Session \#3.}\hfill{\large\bf  Fall 2018}\hrule

\section*{Set-up}
Throughout this session, consider the values of $m$, $d$ given by your birthday (in the form \emph{m/d/Y}).  For
instance, if you were born today, then $m=10$ and $d=8$.

Consider a triangle $\triangle ABC$ that has angles with degree values $\alpha = 4\max(m,d)-2\min(m,d)+6,$ and
$\beta, \gamma \in \mathbb{N}$.

\begin{center}
  \begin{tikzpicture}
    \coordinate[label = below left:$A$] (A) at (0,0);
    \coordinate[label = above:$B$] (B) at (45:3.5);
    \coordinate[label = below right:$C$] (C) at (4, 0);
    \draw (A) -- (B) -- (C) -- cycle;
    \tkzMarkAngle[fill= orange, size=15pt, opacity=.4](B,C,A)
    \tkzLabelAngle[pos = 0.75](B,C,A){$\gamma$}

    \tkzMarkAngle[fill= orange,size=15pt, opacity=.4](C,A,B)
    \tkzLabelAngle[pos = 0.77](C,A,B){$\alpha$}

    \tkzMarkAngle[fill= orange,size=15pt, opacity=.4](A,B,C)
    \tkzLabelAngle[pos = 0.75](A,B,C){$\beta$}
  \end{tikzpicture}
\end{center}

\subsection*{Exploration Stage}
\begin{problem}[10 pts---all or nothing] 
  Describe the set $\mathcal{B}$ that contains all the possible values of $\beta$ (in degrees) for the triangle
  $\triangle ABC$ that you have just constructed.  Use set-builder notation, rather than listing its elements.
\end{problem}

\begin{problem}[10 pts---all or nothing]
  Once an angle $\beta \in \mathcal{B}$ has been chosen, there is only one possible value for the remaining angle
  $\gamma$.  Find a formula for $\gamma$ in terms of $m, d$ and $\beta$.
\end{problem}

\subsection*{Planning and Delivery}
\begin{problem}[20 pts]
  Prove that in your triangle $\triangle ABC$ with angles (in degrees) with values $\beta, \gamma \in \mathbb{N}$, and
  $\alpha=4\max(m,d)-2\min(m,d)+6$, if $\underbrace{\beta \text{ is even}}_{P}$, then
  $\underbrace{\gamma \text{ is also even}}_{Q}$.

  Start by completing the following step-by-step table, before you put your proof into words.
  \begin{center}
    \begin{tabular}{|l|p{10cm}|}
      \hline
      \textbf{Statement}                     & \textbf{Reason (Fact)} \\ \hline
      $P = ``\beta \text{ is even.}"$         & hypothesis \\ \hline
      $\exists b \in \mathbb{N}, \beta = 2b$ & Definition of even (natural) number \\ \hline
      &  \\ 
      \hspace{1cm} $\vdots$ & \hspace{4cm} $\vdots$ \\ 
      & \\ \hline
      $\gamma=2c$ for $c=\dotsb$             & $\dotsb$ \\ \hline
      $Q = ``\gamma \text{ is even.}"$        & Definition of even number \\ \hline
    \end{tabular}
  \end{center}
\end{problem}

\begin{problem}[20 pts]
  For the same triangle, prove that if $\beta$ is odd, then $\gamma$ is also odd.
\end{problem}

\begin{problem}[40 pts]
  Write a proof of the following proposition.
  \vspace*{-0.75cm}
  \begin{quote}
    \begin{proposition*}
      In any right triangle with integer-valued angles (in degrees), the non-right angles have the same parity.
    \end{proposition*}
    \end{quote}
\end{problem}

\end{document}
