\documentclass[11pt]{article}

\usepackage{amsmath,amsthm,amsfonts,amssymb,amsxtra}
\usepackage{pgf,tikz}
\usepackage{pgfplots}
\pgfplotsset{compat=1.11}
\usepackage{mathrsfs}
\renewcommand{\theenumi}{(\alph{enumi})} 
\renewcommand{\labelenumi}{\theenumi}

\pagestyle{empty}
\setlength{\textwidth}{7in}
\setlength{\oddsidemargin}{-0.5in}
\setlength{\topmargin}{-1.0in}
\setlength{\textheight}{9.5in}

\theoremstyle{definition}
\newtheorem{problem}{Problem}
\theoremstyle{theorem}
\newtheorem*{theorem}{Theorem}
\newtheorem{proposition}{Proposition}

\begin{document}

\noindent{\large\bf MATH 300}\hfill{\large\bf Test \#3}\hfill{\large\bf Fall 2018}\hfill{\large\bf Page 1/11}\hrule

\bigskip
\begin{center}
  \begin{tabular}{|ll|}
    \hline & \\
    {\bf Name: } & \makebox[12cm]{\hrulefill}\\ & \\
    {\bf VIP ID:} & \makebox[12cm]{\hrulefill}\\ & \\
    \hline
  \end{tabular}
\end{center}
\begin{itemize}
\item Write your name and VIP ID in the space provided above.
\item Each of the propositions below is worth 20 points.  Present a proof of \textbf{exactly} five of them.  Mark
  clearly in the next page which propositions you have chosen.
\item You much choose at least one of each of the following categories:
  \begin{description}
  \item[Proof by Contrapositive:] Instead of proving $P \implies Q$, try $\lnot Q \implies \lnot P$.  
  \item[Proof by Contradiction:] To prove $P \implies Q$, start with the assumption that $P$ and $\lnot Q$ are true.
  \item[If and only if statement:] To prove $P \iff Q$, prove both $P \implies Q$ and $Q \implies P$.
  \item[Existence statement:] List, construct, or prove by contradiction.
  \item[Proof by Induction:] $\forall n\geq n_0, P(n)$.  Prove the \emph{basis step} $P(n_0),$ and use the \emph{inductive
      hypothesis} to prove the \emph{inductive step} $P(n) \implies P(n+1).$
  \end{description}
\item Make sure to \textbf{box} your proofs, to differentiate them from your exploration and planning.  I will only grade
  for boxed content on each submission.
\item Books, notes and calculators are allowed.
\end{itemize}
\hrule

\begin{center}
  \begin{tabular}{|c|c|c|}
    \hline
    && \\
    {\large\bf Proposition \#} & {\large\bf Max} & {\large\bf Points} \\
    && \\ \hline
    && \\
     & {\Large 20} & \\
    && \\ \hline
    && \\
     & {\Large 20} & \\
    && \\ \hline
    && \\
     & {\Large 20} & \\
    && \\ \hline
    && \\
     & {\Large 20} & \\
    && \\ \hline
    && \\
     & {\Large 20} & \\
    && \\ \hline \hline
    && \\
    {\large\bf Total} & {\Large 100} & \\
    && \\ \hline
  \end{tabular}
\end{center}
\newpage

%%%%%%%%%%%%%%%%%%%%%%%%%%%%%%%%%%%%%%%%%%%%%%%%%%%%%%%%%%%%%%%%%%%%% Page 2
\noindent{\large\bf MATH 300}\hfill{\large\bf Test \#3}\hfill{\large\bf Fall 2018}\hfill{\large\bf Page 2/11}\hrule

\bigskip
\begin{proposition}%%% 1
  For every $n \in \mathbb{N}$, it follows that
  \begin{equation*}
    3^1 + 3^2 + 3^3 + 3^4 + \dotsb + 3^n = \frac{3^{n+1}-3}{2}.
  \end{equation*}
\end{proposition}

\begin{proposition}%%% 2
  Concerning the Fibonacci sequence $\{ F_n \}_{n\in\mathbb{N}}$, it follows that
  \begin{equation*}
    F_2+F_4+F_6+F_8+\dotsb+F_{2n} = F_{2n+1}-1.
  \end{equation*}
\end{proposition}

\begin{proposition}%%% 3
  For every $n\in\mathbb{N}$, it follows that
  \begin{equation*}
    (1+2+3+\dotsb+n)^2 = 1^3 + 2^3 + 3^3 + \dotsb + n^3.
  \end{equation*}
\end{proposition}

\begin{proposition}%%% 4
  For every $n \in \mathbb{N},$ it follows that 
  \begin{equation*}
    \frac{1}{n} + \frac{1}{n+1} + \dotsb + \frac{1}{2n} \geq \frac{1}{2}.
  \end{equation*}
\end{proposition}

\begin{proposition}%%% 5
  For any integer $n \geq 0$, it follows that $3 \vert (5^{2n}+2).$
\end{proposition}

\begin{proposition}%%% 6
  For every $n \in \mathbb{N},$ it follows that
  \begin{equation*}
    \sum_{k=1}^n (4k-3) = n(2n-1)
  \end{equation*}
\end{proposition}

\begin{proposition}%%% 7
  Given an integer $a \in \mathbb{Z}$, then $a^3+a^2+a$ is even if and only if $a$ is even.
\end{proposition}

\begin{proposition}%%% 8
  There exist three positive real numbers $x \in \mathbb{R}$ for which $x^2 < \sqrt{x}$.
\end{proposition}

\begin{proposition}%%% 9
  There exists a real number $x \in \mathbb{R}$ such that $x^3-4x^2=7.$
\end{proposition}

\begin{proposition}%%% 10
  If $n \in \mathbb{Z}$, then $4 \vert n^2$ or $4 \vert (n^2-1).$
\end{proposition}

\begin{proposition}%%% 11
  Suppose $x,y \in \mathbb{R}.$  Then $x^3+x^2y=y^2+xy$ if and only if $y=x^2$ or $y+x=0$.
\end{proposition}

\begin{proposition}%%% 12
  There exist no integers $a,b \in \mathbb{Z}$ for which $21a+30b=1$.
\end{proposition}

\begin{proposition}%%% 13
  The number $\sqrt[3]{2}$ is irrational.
\end{proposition}

\begin{proposition}%%% 14
  The number $\log_34$ is irrational.
\end{proposition}

\begin{proposition}%%% 15
  If $a, b \in \mathbb{R}$ are positive real numbers, then $a+b \geq 2\sqrt{ab}$.
\end{proposition}

\begin{proposition}%%% 16
  For each positive real number $x \in \mathbb{R}$, if $x$ is irrational, then $\sqrt{x}$ is irrational.
\end{proposition}

\begin{proposition}
  For each integer $n \in \mathbb{Z}$, $n$ is even if and only if $4 \vert n^2$.
\end{proposition}

\begin{proposition}
  If $p, q \in \mathbb{Q}$ are rational numbers with $p < q$, then there exists another rational number $x \in \mathbb{Q}$ that
  satisfies $p < x <  q$.
\end{proposition}

\end{document}



%%% Local Variables:
%%% mode: latex
%%% TeX-master: t
%%% End:

%%% Local Variables:
%%% mode: latex
%%% TeX-master: t
%%% End:
