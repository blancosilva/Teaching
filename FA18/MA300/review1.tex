\documentclass[11pt]{article}

\usepackage{amsmath,amsthm,amsfonts,amssymb,amsxtra}
\usepackage{multicol}
\usepackage{pgf,tikz}
\usepackage{mathrsfs, verbatim}
\usetikzlibrary{arrows}
\renewcommand{\theenumi}{(\alph{enumi})} 
\renewcommand{\labelenumi}{\theenumi}

\pagestyle{empty}
\setlength{\textwidth}{7in}
\setlength{\oddsidemargin}{-0.5in}
\setlength{\topmargin}{-1.0in}
\setlength{\textheight}{9.5in}

\newtheorem*{remark}{Remark}
\theoremstyle{definition}
\newtheorem{problem}{Problem}
\newtheorem*{problem*}{Sample Problem}

%%% Solutions macro 
% To include solutions, leave the first two lines and comment out the next two.
% To not include solutions, do the reverse.
%%%%%%%%%%%%%%%%%%%%%%
% \renewcommand{\proofname}{Solution:}
\renewcommand\proof\comment
%%%%%%%%%%%%%%%%%%%%%%%


\begin{document}

\noindent{\large\bf MATH 300}\hfill{\large\bf Review for Test \#1.}\hfill{\large\bf  Fall 2018}\hrule

\begin{problem}
  State DeMorgan's Laws for set complements $(A \cup B)^\complement$ and $(A \cap B)^\complement$.
\end{problem}
\begin{proof}
  $(A \cup B)^\complement = A^\complement \cap B^\complement$,   $(A \cap B)^\complement = A^\complement \cup B^\complement$.
\end{proof}

\begin{problem}
  Let $A=\{ a, b \}$, $B = \{ b, 1, 2 \}$.  Give the elements of $(A \times B) \setminus ( A \times \{ b \})$ by listing
  them within braces.
\end{problem}
\begin{proof}
  $(A\times B) \setminus (A\times \{b \}) = \big\{ (a,1), (a,2), (b,1), (b,2) \big\}$
\end{proof}

\begin{problem}
  For each $n \in \mathbb{N}$, let $I_n$ be the closed interval $\big[ -1+\tfrac{1}{n}, 1 - \tfrac{1}{n} \big]$.
  Describe the set $\bigcup_{n \in \mathbb{N}} I_n$ in either interval or set-builder notation.
\end{problem}
\begin{proof}
  $\displaystyle{\bigcup_{n \in \mathbb{N}} I_n = (-1,1)}$.
\end{proof}

\begin{problem}
  For each $n \in \mathbb{N}$, let $J_n$ be the closed interval $\big[ 1+\tfrac{1}{n}, 2 - \tfrac{1}{n} \big]$.
  Describe the set $\bigcup_{n=2}^\infty J_n$ in either interval or set-builder notation.
\end{problem}
\begin{proof}
  $\displaystyle{\bigcup_{n=2}^\infty J_n = (1,2)}$.
\end{proof}

\begin{problem}
  Let $X=\{ a, b, c, d\}$ and $S = \big\{ Y \in \mathscr{P}(X) : b \not\in Y, \lvert Y \rvert \leq 2 \big\}$.  Give the
  elements of $S$.
\end{problem}
\begin{proof}
  $S = \big\{ \emptyset, \{ a \}, \{ c \}, \{ d \}, \{ a,c \}, \{ a, d \}, \{ c, d \} \big\}$.
\end{proof}

\begin{problem}
  Let $P$ and $Q$ be statements.  Are the following statements equivalent?  Justify your answers.
  \begin{enumerate}
  \item $P \land (Q \lor \lnot Q)$ and $(\lnot P) \implies (Q \land \lnot  Q)$.
  \item $(\lnot P) \land (P \implies Q)$ and $\lnot (Q \lor P)$.
  \end{enumerate}
\end{problem}
\begin{proof}
  \begin{enumerate}
  \item They are equivalent.
    \begin{equation*}
      \begin{array}{c|c|c|c|c|c|c}
        P & Q & Q \lor \lnot Q & \lnot P & Q \land \lnot Q & P \lor (Q \lor \lnot Q) & (\lnot P ) \implies (Q \land
                                                                                       \lnot Q) \\
        \hline
        T & T & T & F & F & T & T \\
        T & F & T & F & F & T & T \\
        F & T & T & T & F & F & F \\
        F & F & T & T & F & F & F
      \end{array}
    \end{equation*}
    \item They are not.  If $P$ is false and $Q$ is true, then $\lnot(Q \lor P)$ is false, and $(\lnot P) \land (P
      \implies Q)$ is true. \qedhere
  \end{enumerate}
\end{proof}

\begin{problem}
  Consider the following statement $S$:
  \begin{quote}
    ``All foreign cars are well made.''
  \end{quote}
  Which of the following statements (there may be more than one) correctly negate $S$?
  \begin{enumerate}
  \item\label{car:a} ``All foreign cars are badly made.''
  \item\label{car:b} ``All domestic (non-foreign) cars are well made.''
  \item\label{car:c} ``There are domestic (non-foreign) cars that are well made.''
  \item\label{car:d} ``Some foreign cars are badly made.''
  \item\label{car:e} ``If a car is not foreign, then it is not well made.''
  \end{enumerate}
\end{problem}
\begin{proof}
  The statement \ref{car:d} is the only one that correctly negates $S$.
\end{proof}

\begin{problem}
  Consider the following statement $P$:
  \begin{equation*} \forall X \subset \mathbb{N}, \exists n \in \mathbb{Z}, \lvert X \rvert = n \end{equation*}
  \begin{enumerate}
  \item Rewrite $\lnot P$ as an affirmative statement (i.e.~the symbol $\lnot$ should not appear anywhere)
  \item What is $\lnot P$ saying in plain English?  Is it true or false?
  \end{enumerate}
\end{problem}
\begin{proof}
  The negation of $P$ can be written as follows:
  \begin{equation*} \exists X \subset \mathbb{N}, \forall n \in \mathbb{Z}, \vert X \rvert \neq n \end{equation*}
  In plain English, this statement indicates that there is a subset of the natural numbers whose cardinality is not an
  integer (in other words, that there exist infinite subsets of the natural numbers).  This is clearly true.
\end{proof}

\begin{problem}
  Consider the following statement $R$:
  \begin{quote} ``An integer $n$ is divisible by 15 only if it is divisible by 5.'' \end{quote}
  \begin{enumerate}
  \item Rewrite $R$ in the form $P \implies Q$.
  \item Use the word \emph{necessary} or \emph{sufficient} as appropriate:
    \begin{quote}
      ``For an integer $n$ to be divisible by 5 it is \makebox[2cm]{\hrulefill} that $n$ be divisible by 15.''
    \end{quote}
  \item Use the word \emph{necessary} or \emph{sufficient} as appropriate:
    \begin{quote}
      ``For an integer $n$ to be divisible by 15 it is \makebox[2cm]{\hrulefill} that $n$ be divisible by 5.''
    \end{quote}
  \item State the converse of $R$.
  \item State the contrapositive of $R$.
  \end{enumerate}
\end{problem}
\begin{proof}
  \begin{enumerate}
  \item $P(n) \implies Q(n)$, where $P(n)$ and $Q(n)$ are respectively ``15 divides $n$'' and ``5 divides $n$.''
  \item ``For an integer $n$ to be divisible by 5 it is \emph{sufficient} that $n$ be divisible by 15.''
  \item ``For an integer $n$ to be divisible by 15 it is \emph{necessary} that $n$ be divisible by 5.''
  \item $Q(n) \implies P(n)$: ``An integer $n$ is divisible by 5 only if it is divisible by 15.''
  \item $\lnot Q(n) \implies \lnot P(n)$: ``If an integer is not divisible by 5, then it is not divisible by 15.'' \qedhere
  \end{enumerate}
\end{proof}

\begin{problem}
  Let $A = [-1,0) \cup (0,1]$, and consider $U=\mathbb{R}$ as the universal set. Describe the set $A^\complement$.
\end{problem}
\begin{proof}
  $A^\complement = (-\infty, -1) \cup \{ 0 \} \cup (1, \infty)$.
\end{proof}
\end{document}
