\documentclass[11pt]{article}

\usepackage{amsmath,amsthm,amsfonts,amssymb,amsxtra}
\usepackage{pgf,tikz}
\usepackage{pgfplots}
\pgfplotsset{compat=1.11}
\usepackage{mathrsfs}
\renewcommand{\theenumi}{(\alph{enumi})} 
\renewcommand{\labelenumi}{\theenumi}

\pagestyle{empty}
\setlength{\textwidth}{7in}
\setlength{\oddsidemargin}{-0.5in}
\setlength{\topmargin}{-1.0in}
\setlength{\textheight}{9.5in}

\theoremstyle{definition}
\newtheorem{problem}{Problem}
\theoremstyle{theorem}
\newtheorem*{theorem}{Theorem}
\newtheorem{proposition}{Proposition}

\begin{document}

\noindent{\large\bf MATH 300}\hfill{\large\bf Test \#3 (second chances)}\hfill{\large\bf Fall 2018}\hfill{\large\bf Page
  1/1}\hrule

\bigskip
\begin{center}
  \begin{tabular}{|ll|}
    \hline & \\
    {\bf Name: } & \makebox[12cm]{\hrulefill}\\ & \\
    {\bf VIP ID:} & \makebox[12cm]{\hrulefill}\\ & \\
    \hline
  \end{tabular}
\end{center}
\begin{itemize}
\item Write your name and VIP ID in the space provided above.
\item Each of the propositions below is worth 20 points.
\item If in your previous exam there are \textbf{\textcolor{red}{-20}} comments associated to a specific kind of proof
  technique on the front page, choose a proposition below with the same characteristics, and give a proof.  You may
  choose one proposition for each such instance.
  \begin{description}
  \item[Proof by Contrapositive:] Instead of proving $P \implies Q$, try $\lnot Q \implies \lnot P$.  
  \item[Proof by Contradiction:] To prove $P \implies Q$, start with the assumption that $P$ and $\lnot Q$ are true.
  \item[If and only if statement:] To prove $P \iff Q$, prove both $P \implies Q$ and $Q \implies P$.
  \item[Existence statement:] List, construct, or prove by contradiction.
  \item[Proof by Induction:] $\forall n\geq n_0, P(n)$.  Prove the \emph{basis step} $P(n_0),$ and use the \emph{inductive
      hypothesis} to prove the \emph{inductive step} $P(n) \implies P(n+1).$
  \end{description}
\item Make sure to \textbf{box} your proofs, to differentiate them from your exploration and planning.  I will only grade
  for boxed content on each submission.
\item Books, and calculators are allowed, but not notes.
\end{itemize}

\hrule

\begin{proposition}%%% 1
  For every $n\in\mathbb{N}$, it follows that
  \begin{equation*}
    (1+2+3+\dotsb+n)^2 = 1^3 + 2^3 + 3^3 + \dotsb + n^3.
  \end{equation*}
\end{proposition}

\begin{proposition}%%% 2
  For every $n \in \mathbb{N},$ it follows that 
  \begin{equation*}
    \frac{1}{n} + \frac{1}{n+1} + \dotsb + \frac{1}{2n} \geq \frac{1}{2}.
  \end{equation*}
\end{proposition}

\begin{proposition}%%% 3
  We say that a number $p \in \mathbb{N}$ is \textbf{perfect} if it equals the sum of its positive divisors less than
  itself.  There is a perfect number less than 10.
\end{proposition}

\begin{proposition}%%% 4
  For any $n \in \mathbb{N}$, the product of any $n$ consecutive positive integers is divisible by $n!.$
\end{proposition}

\begin{proposition}%%% 5
  Suppose $a, b \in \mathbb{Z}.$ If $ab$ is odd, then $a^2+b^2$ is even.
\end{proposition}

\begin{proposition}%%% 6
  There exists a real number $x \in \mathbb{R}$ such that $x^3-4x^2=7.$
\end{proposition}

\begin{proposition}%%% 7
  If $n \in \mathbb{Z}$, then $4 \vert n^2$ or $4 \vert (n^2-1).$
\end{proposition}

\begin{proposition}%%% 8
  Suppose $x \in \mathbb{R}.$  If $x^3-x>0$ then $x>-1.$
\end{proposition}

\begin{proposition}%%% 9
  Suppose $a, b \in \mathbb{Z}$.  Then $a \equiv b \pmod{10}$ if and only if $a \equiv b \pmod{2}$ and $a \equiv b \pmod{5}$.
\end{proposition}

\begin{proposition}%%% 10
  The number $\log_2(1) + \log_2(2) + \log_2(3)$ is irrational.
\end{proposition}

\begin{proposition}%%% 11
   If $a, b \in \mathbb{R}$ are positive real numbers, then $a+b \geq 2\sqrt{ab}$.
\end{proposition}

\begin{proposition}%%% 12
   If $p, q \in \mathbb{Q}$ are rational numbers with $p < q$, then there exists another rational number $x \in \mathbb{Q}$ that
  satisfies $p < x <  q$.
\end{proposition}

\end{document}



%%% Local Variables:
%%% mode: latex
%%% TeX-master: t
%%% End:

%%% Local Variables:
%%% mode: latex
%%% TeX-master: t
%%% End:

%%% Local Variables:
%%% mode: latex
%%% TeX-master: t
%%% End:
