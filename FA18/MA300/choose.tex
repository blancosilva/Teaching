\documentclass[11pt]{article}

\usepackage{amsmath,amsfonts,amsthm,amssymb,amsxtra}
\usepackage{pgf,tikz}
\usepackage{tkz-euclide}
\usetkzobj{all}
\usepackage{mathrsfs, verbatim}
% \usepackage{framed}
\usepackage{pdfpages}
\usepackage{pgfornament}

\renewcommand{\theenumi}{(\alph{enumi})} 
\renewcommand{\labelenumi}{\theenumi}

\pagestyle{empty}
\setlength{\textwidth}{7in}
\setlength{\oddsidemargin}{-0.5in}
\setlength{\topmargin}{-1.0in}
\setlength{\textheight}{9.5in}
 
\newtheorem*{remark}{Remark}
\theoremstyle{definition}
\newtheorem*{definition*}{Definition}
\newtheorem{problem}{Problem}
\newtheorem*{problem*}{Sample Problem}
\theoremstyle{theorem}
\newtheorem*{theorem}{Theorem} 
\newtheorem{proposition}{Proposition}
\newtheorem*{proposition*}{Proposition}
\newtheorem{fact}{Fact}

\newcommand{\separator}{%
  \nointerlineskip \vspace{.5\baselineskip}\hspace{\fill}
  {\resizebox{0.5\linewidth}{0.5ex} 
    {\pgfornament[color = black]{88}
    }}%
  \hspace{\fill}
  \par\nointerlineskip \vspace{.5\baselineskip}
}

\makeatletter
\newcommand\binomialCoefficient[2]{%
  % Store values 
  \c@pgf@counta=#1% n
  \c@pgf@countb=#2% k
  % 
  % Take advantage of symmetry if k > n - k
  \c@pgf@countc=\c@pgf@counta%
  \advance\c@pgf@countc by-\c@pgf@countb%
  \ifnum\c@pgf@countb>\c@pgf@countc%
  \c@pgf@countb=\c@pgf@countc%
  \fi%
  % 
  % Recursively compute the coefficients
  \c@pgf@countc=1% will hold the result
  \c@pgf@countd=0% counter
  \pgfmathloop% c -> c*(n-i)/(i+1) for i=0,...,k-1
  \ifnum\c@pgf@countd<\c@pgf@countb%
  \multiply\c@pgf@countc by\c@pgf@counta%
  \advance\c@pgf@counta by-1%
  \advance\c@pgf@countd by1%
  \divide\c@pgf@countc by\c@pgf@countd%
  \repeatpgfmathloop%
  \the\c@pgf@countc%
}
\makeatother

%%% Solutions macro 
% To include solutions, leave the first two lines and comment out the next two.
% To not include solutions, do the reverse.
%%%%%%%%%%%%%%%%%%%%%%
% \renewcommand{\proofname}{Solution:}
\renewcommand\proof\comment
%%%%%%%%%%%%%%%%%%%%%%%


\begin{document}

\noindent{\large\bf MATH 300}\hfill{\large\bf Exploratory Session \#4.}\hfill{\large\bf  Fall 2018}\hrule

\section*{Set-up}
Let $\mathcal{S}$ be the set that contains the letters of your complete name (including middle and last name).  For
example, mine is $\mathcal{S}=\{ F,R,A,N,C,I,S,O,J,V,E,B,L \} = \{ A,B,C,E,F,I,J,L,N,O,R,S,V \}.$

As usual, let $m$ and $d$ from your birthday.  For instance, if you were born today, $m=10$, $d=17$.

\section*{Exploratory phase}
\begin{definition*}[Binomial Number]
  In $n$ and $k$ are integers, then the \emph{binomial number} $\binom{n}{k}$ denotes the number of subsets that can be
  made by choosing $k$ elements from a set with $n$ elements. The symbol $\binom{n}{k}$ is read ``$n$ choose $k$.''
\end{definition*}
\separator

\begin{problem}[10 pts]
  In our second quiz, you had to choose 5 problems out of 10, in a way that no other student in the class would have the
  same selection as you.  The question now is, how many possible selections of 5 problems out of 10 are there?  Write
  the solution as a binomial number.
\end{problem}
\separator

\begin{fact}[Basic Formula]
  If $n,k \in \mathbb{Z}$ and $0 \leq k \leq n$, then $\binom{n}{k} = \frac{n!}{k! (n-k)!}$.  Otherwise, $\binom{n}{k}=0$.
\end{fact}
\separator

\begin{problem}[10 pts]
  Let $n=\max(m,d)$ and $k=\min(m,d)$ for your particular values of $m$ and $d$.  Compute $\binom{n}{k}$.
\end{problem} 

\begin{problem}[20 pts]
  Show that if $n,k \in \mathbb{Z}$ and $0\leq k \leq n$, then $\binom{n}{k} = \binom{n}{n-k}$.
\end{problem}

\begin{problem}[10 pts]
  For your particular values of $m$, $d$ and set $\mathcal{S}$, what is the cardinality of the following set?   
  \begin{equation*}
    \big\{ X \in \mathscr{P}(\mathcal{S}) : \lvert X \rvert \leq \min(m,d)+1 \big\}
  \end{equation*}
  Write the solution both as a binomial number, and its precise value.
\end{problem}
\separator

\begin{fact}[Recursive Formula]
  If $n,k \in \mathbb{Z}$ and $0\leq k < n$, then
  \begin{equation*}
    \binom{n}{k} = \binom{n-1}{k-1} + \binom{n-1}{k} 
  \end{equation*}
\end{fact}

\begin{definition*}[Pascal's Triangle]
  Arranging all binomial numbers in order requires a triangular pattern:
  \begin{center}
    \newdimen\R
    \R=.4cm
    \newcommand\mycolor{gray}
    \begin{tabular}{ccc}
      \begin{minipage}{0.33\linewidth}
        \begin{tikzpicture}[line width=.8pt]
          \foreach \k in {0,...,7}{ 
            \begin{scope}[shift={(-60:{sqrt(3)*\R*\k})}]
              \pgfmathtruncatemacro\ystart{7-\k}
              \foreach \n in {0,...,\ystart}{
                \pgfmathtruncatemacro\newn{\n+\k}
                \ifthenelse{\k=0}{\def\mycolor{purple}}{}
                \ifthenelse{\k=1}{\def\mycolor{yellow}}{}
                \ifthenelse{\k=2}{\def\mycolor{blue}}{}
                \ifthenelse{\k=3}{\def\mycolor{green}}{}
                \begin{scope}[shift={(-120:{sqrt(3)*\R*\n})}]
                  \draw[top color=\mycolor!20,bottom color=\mycolor!60] 
                  (30:\R) \foreach \x in {90,150,...,330} {
                    -- (\x:\R)}
                  --cycle (90:0) node {\tiny $\boldsymbol{\binom{\newn}{\k}}$};
                \end{scope}
              }
            \end{scope}
          }
        \end{tikzpicture}
      \end{minipage}& $=$ &
                            \begin{minipage}{0.33\linewidth}
                              \begin{tikzpicture}[line width=.8pt]
                                \foreach \k in {0,...,7}{ 
                                  \begin{scope}[shift={(-60:{sqrt(3)*\R*\k})}]
                                    \pgfmathtruncatemacro\ystart{7-\k}
                                    \foreach \n in {0,...,\ystart}{
                                      \pgfmathtruncatemacro\newn{\n+\k}
                                      \ifthenelse{\k=0}{\def\mycolor{purple}}{}
                                      \ifthenelse{\k=1}{\def\mycolor{yellow}}{}
                                      \ifthenelse{\k=2}{\def\mycolor{blue}}{}
                                      \ifthenelse{\k=3}{\def\mycolor{green}}{}
                                      \begin{scope}[shift={(-120:{sqrt(3)*\R*\n})}]
                                        \draw[top color=\mycolor!20,bottom color=\mycolor!60] 
                                        (30:\R) \foreach \x in {90,150,...,330} {
                                          -- (\x:\R)}
                                        --cycle (90:0) node {\tiny $\mathbf{\binomialCoefficient{\newn}{\k}}$};
                                      \end{scope}
                                    }
                                  \end{scope}
                                }
                              \end{tikzpicture}
                            \end{minipage} \\
    \end{tabular}
  \end{center} 
\end{definition*}

\newpage

\noindent{\large\bf MATH 300}\hfill{\large\bf Exploratory Session \#4.}\hfill{\large\bf  Fall 2018}\hrule

\begin{theorem}[Binomial Theorem]
  If $n$ is a non-negative integer, then
  \begin{equation*}
    (x+y)^n = \sum_{k=0}^n \tbinom{n}{k} x^{n-k}y^k = \tbinom{n}{0}x^n + \tbinom{n}{1}x^{n-1}y + \tbinom{n}{2} x^{n-2}y^2 +
    \dotsb + \tbinom{n}{n-1}x y^{n-1} + \tbinom{n}{n}y^n
  \end{equation*}
\end{theorem}
\separator

\begin{problem}[10 pts]
  Write (an expansion of) the polynomial $(1+x)^5$ using the formula from the Binomial Theorem---but do not leave your
  answer in terms of binomial numbers.  Compute these numbers before providing your answer.
\end{problem}

\begin{problem}[20 pts]
  Show that if $n,k \in \mathbb{Z}$, and $1 \leq k \leq n$, then
  \begin{equation*}
    k\binom{n}{k} = n \binom{n-1}{k-1}
  \end{equation*}
\end{problem}

\begin{problem}[10 pts]
  Use the Binomial Theorem to find the numerical value of the coefficient of $x^6 y^3$ in the polynomial $(3x-2y)^9$.
\end{problem}

\begin{problem}[10 pts]
  In the Pascal's triangle depicted above, we have highlighted four diagonal sequences:
  \begin{quote}
    \begin{description}
    \item[Purple] $\{ 1, 1, 1, \dotsc \} = \{ 1 \}_{n=0}^\infty$.
    \item[Yellow] $\{ 1, 2, 3, \dotsc \} = \{ n \}_{n=1}^\infty$.
    \item[Blue] $\{1, 3, 6, 10, 15, 21, \dotsc \} = \big\{ \tbinom{n}{2} \big\}_{n=2}^\infty$.
    \item[Green] $\{ 1, 4, 10, 20, 35, \dotsc \} = \big\{ \tbinom{n}{3} \big\}_{n=3}^\infty$.
    \end{description}
  \end{quote}
  Find simplified formulas for the terms of the blue and green sequences without using binomial numbers or factorials in
  your expressions.
  
\end{problem}
\end{document}

%%% Local Variables:
%%% mode: latex
%%% TeX-master: t
%%% End

