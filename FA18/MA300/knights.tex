\documentclass[11pt]{article}

\usepackage{amsmath,amsthm,amsfonts,amssymb,amsxtra}
\usepackage{multicol}
\usepackage{pgf,tikz}
\usetikzlibrary{arrows}
\renewcommand{\theenumi}{(\alph{enumi})} 
\renewcommand{\labelenumi}{\theenumi}

\pagestyle{empty}
\setlength{\textwidth}{7in}
\setlength{\oddsidemargin}{-0.5in}
\setlength{\topmargin}{-1.0in}
\setlength{\textheight}{9.5in}

\newtheorem*{remark}{Remark}
\theoremstyle{definition}
\newtheorem{problem}{Problem}
\newtheorem*{problem*}{Sample Problem}

\renewcommand{\proofname}{Solution:}


\begin{document}
\noindent{\large\bf MATH 300}\hfill{\large\bf Exploratory Session \#2.}\hfill{\large\bf  Fall 2018}\hrule

\section*{Rules}
There are 10 problems assigned in this session, each of them worth 20 points.
\begin{enumerate}
\item You must submit \textbf{only five of those problems.}
\item But beware: if two or more students choose exactly the same five problems, then all those $n$ students will
  receive a $20(n-1)$-point penalty.  You must therefore coordinate with every single other student in the class, to
  avoid penalties.

  For extra credit (10 points): how many possible selections are there?
\item You are also required not to submit any three consecutive problems. For instance, it is acceptable to submit
  problems 1, 2, 4, 5 and 9.  It is not acceptable to submit problems 2, 3, 4, 6 and 9 (since problems 2, 3, 4 are
  consecutive).

  For extra credit (10 points): How many possible selections are there with this extra constraint? 
\end{enumerate}
\hrule

\subsection*{The Island of Knights and Knaves\footnote{All problems in this assignment come from Robert Smullian's
  \emph{What is the Name of this Book?}, Chapter 3.}}
In the Island of Knights and Knaves, ``knights'' always tell the truth, and ``knaves'' always lie.  Every inhabitant of
the island is either a knight or a knave.  You are a visitor.

\begin{problem*}
  Two of the inhabitants, A and B, standing together in a garden.  You ask A, ``Are you a knight or a knave?''  A
  answered, but rather indistinctly and you cannot make out what he said.  You then ask B, ``What did A say?''  B
  replies ``A said that he is a knave.''
  What is B? Can you determine what A is?
\end{problem*}
\begin{proof}
A can never say ``I am a knave.'' (Why?)  This implies that B must be a knave.  We cannot determine whether A is a
knight or a knave with the given information.
\end{proof}

\subsubsection*{Three people A, B, C in a garden}

\begin{problem}
  Three people, A, B, C, each of whom is either a knight or a knave.  A and B make the following statements:
  \begin{quote}
    \begin{description}
    \item[A:] All of us are knaves.
    \item[B:] Exactly one of us is a knight.
    \end{description}
  \end{quote}
  What are A, B, C?
\end{problem}

\begin{problem}
  Suppose instead, A and B say the following:
  \begin{quote}
    \begin{description}
    \item[A:] All of us a knaves.
    \item[B:] Exactly one of us is a knave.
    \end{description}
  \end{quote}
  Can it be determined what B is?  Can it be determined what C is?
\end{problem}

\begin{problem}
  Two people are said to be of the \emph{same type} if they are both knights or both knaves.  A and B make the following
  statements:
  \begin{quote}
    \begin{description}
    \item[A:] B is a knave.
    \item[B:] A and C are of the same type.
    \end{description}
  \end{quote}
  What is C?
\end{problem}
\newpage

\noindent{\large\bf MATH 300}\hfill{\large\bf Exploratory Session \#2.}\hfill{\large\bf  Fall 2018}\hrule

\subsubsection*{Two people A,B in a garden}

\begin{problem}
  A makes the following statement: ``At least one of us is a knave.''  What are A and B?
\end{problem}

\begin{remark}
  Recall the definition of \emph{exclusive OR} (\texttt{XOR}), and the difference with \emph{OR}:
  \begin{equation*}
    \begin{array}{c|c|c}
      \boldsymbol{P} & \boldsymbol{Q} & \boldsymbol{P \lor Q} \\ \hline
      T & T & T \\
      T & F & T \\ 
      F & T & T \\ 
      F & F & F
    \end{array}
    \qquad
    \begin{array}{c|c|c}
      \boldsymbol{P} & \boldsymbol{Q} & \boldsymbol{P \oplus Q} \\ \hline
      T & T & F \\
      T & F & T \\ 
      F & T & T \\ 
      F & F & F
    \end{array}
  \end{equation*}
\end{remark}

\begin{problem}
  Suppose A says, ``Either I am a knave or B is a knight.''  What are A and B?
\end{problem}

\begin{problem}
  Suppose A says, ``Either I am a knave or else two plus two equals five.''  What would you conclude?
\end{problem}

\subsection*{Knights, Knaves and Normals}
Three types of people in a new island: knights, who always tell the truth; knaves, who always lie; and normal people,
who sometimes lie and sometimes tell the truth.  On this island, knaves are said to be of the \emph{lowest rank},
normals of \emph{middle rank}, and knights of \emph{highest rank}.

\begin{problem}
  We are given three people, A, B, C, one of whom is a knight, one a knave, and one normal (but not necessarily in that
  order).  They make the following statements:
  \begin{quote}
    \begin{description}
    \item[A:] I am normal.
    \item[B:] That is true.
    \item[C:] I am not normal.
    \end{description}
  \end{quote}
  What are A, B, and C?
\end{problem}

\begin{problem}
  Given two people A, B, each of whom is a knight, a knave or a normal, they make the following statements:
  \begin{quote}
    \begin{description}
    \item[A:] I am of lower rank than B.
    \item[B:] That's not true!
    \end{description}
  \end{quote}
  Can the ranks of either A or B be determined?  Can it be determined of these statements whether it is true or false?
\end{problem}

\begin{problem}
  Given three people A, B, C, one of whom is a knight, one a knave, and one normal.  A, B make the following statements:
  \begin{quote}
    \begin{description}
    \item[A:] B is of higher rank than C.
    \item[B:] C is of higher rank than A.
    \end{description}
  \end{quote}
  Then C is asked: ``Who has a higher rank, A or B?''  What does C answer?
\end{problem}

\subsection*{The Island of Bahava}

In the island of Bahava women are also called knights, knaves or normal.  A knight can only marry a knave, and a knave
can only marry a knight.  (Hence a normal can only marry a normal.)

\begin{problem}
  Consider a married couple, Mr.~and Mrs.~A.  They make the following statements:
  \begin{quote}
    \begin{description}
    \item[Mr.~A:] My wife is not normal.
    \item[Mrs.~A:] My husband is not normal.
    \end{description}
  \end{quote}
  What are Mr.~and Mrs.~A?
\end{problem}
\end{document}
