\documentclass[12pt]{article}

\usepackage{amsmath,amsthm,amsfonts,amssymb,amsxtra}
\usepackage{pgf,tikz}
\usetikzlibrary{arrows}
\renewcommand{\theenumi}{(\alph{enumi})} 
\renewcommand{\labelenumi}{\theenumi}

\pagestyle{empty}
\setlength{\textwidth}{7in}
\setlength{\oddsidemargin}{-0.5in}
\setlength{\topmargin}{-1.0in}
\setlength{\textheight}{9.5in}

\theoremstyle{definition}
\newtheorem{problem}{Problem}

\begin{document}

\noindent{\large\bf MATH 142}\hfill{\large\bf Exam\#4.}\hfill{\large\bf
  Spring 2015}\hfill{\large\bf Page 1/5}\hrule

\bigskip
\begin{center}
  \begin{tabular}{|ll|}
    \hline & \cr
    {\bf Name: } & \makebox[12cm]{\hrulefill}\cr & \cr
    {\bf 4-digit code:} & \makebox[12cm]{\hrulefill}\cr & \cr
    \hline
  \end{tabular}
\end{center}
\begin{itemize}
\item Write your name and the last 4 digits of your SSN in the space provided above.
\item The test has five (5) pages, including this one.
\item You must show sufficient work to justify all answers unless otherwise stated in the problem.  Correct answers with inconsistent work may not be given credit.
\item Credit for each problem is given in parentheses at the right of the problem number.
\item No books, notes or calculators may be used on this test.
\end{itemize}
\hrule

\begin{center}
  \begin{tabular}{|c|c|c|}
    \hline
    &&\cr
    {\large\bf Page} & {\large\bf Max.~points} & {\large\bf Your points} \cr
    &&\cr
    \hline
    &&\cr
    {\Large 2} & \Large 25 & \cr
    &&\cr
    \hline
    &&\cr
    {\Large 3} & \Large 25 & \cr
    &&\cr
    \hline
    &&\cr
    {\Large 4} & \Large 20 & \cr
    &&\cr
    \hline
    &&\cr
    {\Large 5} & \Large 30 & \cr
    &&\cr
    \hline\hline
    &&\cr
    {\large\bf Total} & \Large 100 & \cr
    &&\cr
    \hline
  \end{tabular}
\end{center}
\newpage

%%%%%%%%%%%%%%%%%%%%%%%%%%%%%%%%%%%%% Page 2
\noindent{\large\bf MATH 142}\hfill{\large\bf Exam\#4.}\hfill{\large\bf
  Spring 2015}\hfill{\large\bf Page 2/5}\hrule

\bigskip

\begin{problem}[10 pts]
For what values of $x$ is the following series convergent?
\begin{equation*}
\sum_{n=1}^\infty \frac{(x-\pi)^n}{n}
\end{equation*}
\vspace{7cm}
\end{problem}
\hrule

\begin{problem}[15 pts]
Find the radius of convergence and interval of convergence of the series 
\begin{equation*}
\sum_{n=1}^\infty \frac{(-\pi)^n x^n}{\sqrt{n}}
\end{equation*}
\end{problem}


\newpage


%%%%%%%%%%%%%%%%%%%%%%%%%%%%%%%%%%%%% Page 3
\noindent{\large\bf MATH 142}\hfill{\large\bf Exam\#4.}\hfill{\large\bf
  Spring 2015}\hfill{\large\bf Page 3/5}\hrule

\bigskip
\begin{problem}[10 pts]
Assume known that 
\begin{equation*}
\frac{1}{1-x} = \sum_{n=0}^\infty x^n \text{ for } \lvert x \rvert <1.
\end{equation*}
Express the function $f(x)=1/(1-x)^2$ as a power series by differentiating the previous equation.  What is the radius of convergence?
\vspace{8cm}
\end{problem}
\hrule

\begin{problem}[15 pts]
Find a power series representation for $f(x) = \ln(1-x)$ and its radius of convergence.
\end{problem}
\newpage

%%%%%%%%%%%%%%%%%%%%%%%%%%%%%%%%%%%%% Page 4
\noindent{\large\bf MATH 142}\hfill{\large\bf Exam\#4.}\hfill{\large\bf
  Spring 2015}\hfill{\large\bf Page 4/5}\hrule

\bigskip

\begin{problem}[10 pts]
Find the Taylor series for $f(x) = e^x$ at $a=\pi$.
\vspace{10cm}
\end{problem}
\hrule

\begin{problem}[10 pts]
Find the Maclaurin series for the function $f(x) = x^2 \sin x$
\end{problem}
\newpage

%%%%%%%%%%%%%%%%%%%%%%%%%%%%%%%%%%%%% Page 5
\noindent{\large\bf MATH 142}\hfill{\large\bf Exam\#4.}\hfill{\large\bf
  Spring 2015}\hfill{\large\bf Page 5/5}\hrule
  
\bigskip
\begin{problem}[15 pts]
Find the first three nonzero terms in the Maclaurin series for $f(x)=e^x\cos x$.
\vspace{10cm}
\end{problem}
\hrule

\begin{problem}[15 pts]
Evaluate $\int e^{-x^2}\, dx$ as an infinite series.
\end{problem}

\end{document}
