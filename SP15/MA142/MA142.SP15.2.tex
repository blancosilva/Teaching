\documentclass[12pt]{article}

\usepackage{amsmath,amsthm,amsfonts,amssymb,amsxtra}
\usepackage{pgf,tikz}
\usetikzlibrary{arrows}
\renewcommand{\theenumi}{(\alph{enumi})} 
\renewcommand{\labelenumi}{\theenumi}

\pagestyle{empty}
\setlength{\textwidth}{7in}
\setlength{\oddsidemargin}{-0.5in}
\setlength{\topmargin}{-1.0in}
\setlength{\textheight}{9.5in}

\theoremstyle{definition}
\newtheorem{problem}{Problem}

\begin{document}

\noindent{\large\bf MATH 142}\hfill{\large\bf Second Midterm.}\hfill{\large\bf
  Spring 2015}\hfill{\large\bf Page 1/4}\hrule

\bigskip
\begin{center}
  \begin{tabular}{|ll|}
    \hline & \cr
    {\bf Name: } & \makebox[12cm]{\hrulefill}\cr & \cr
    {\bf 4-digit code:} & \makebox[12cm]{\hrulefill}\cr & \cr
    \hline
  \end{tabular}
\end{center}
\begin{itemize}
\item Write your name and the last 4 digits of your SSN in the space provided above.
\item The test has four (4) pages, including this one.
\item Show sufficient work to justify all answers unless otherwise
  stated in the problem.  Correct answers with inconsistent work may
  not be given credit. 
\item Credit for each problem is given at the right of each problem
  number. 
\item No books, notes or calculators may be used on this test.
\end{itemize}
\hrule

\begin{center}
  \begin{tabular}{|c|c|c|}
    \hline
    &&\cr
    {\large\bf Page} & {\large\bf Max} & {\large\bf Points} \cr
    &&\cr
    \hline
    &&\cr
    {\Large 2} & \Large 30 & \cr
    &&\cr
    \hline
    &&\cr
    {\Large 3} & \Large 35 & \cr
    &&\cr
    \hline
    &&\cr
    {\Large 4} & \Large 35 & \cr
    &&\cr
    \hline\hline
    &&\cr
    {\large\bf Total} & \Large 100 & \cr
    &&\cr
    \hline
  \end{tabular}
\end{center}
\newpage

%%%%%%%%%%%%%%%%%%%%%%%%%%%%%%%%%%%%% Page 2
\noindent{\large\bf MATH 142}\hfill{\large\bf Second Midterm.}\hfill{\large\bf
  Spring 2015}\hfill{\large\bf Page 2/4}\hrule

\bigskip
\begin{problem}[30 pts]
Evaluate the integrals below

\begin{enumerate}
\item $\displaystyle{\int_0^{\pi} \cos \theta  \sin \theta\, d\theta}$
\vspace{1.5cm}
\begin{flushright}
  \begin{tikzpicture}
    \draw (0cm,-0.2cm) rectangle (5cm,1.2cm);
  \end{tikzpicture}
\end{flushright}

\item $\displaystyle{\int_0^{\infty} \frac{ \sin ( \tfrac{\pi}{2} e^{-x}) }{e^x} \, dx}$
\vspace{2.5cm}
\begin{flushright}
  \begin{tikzpicture}
    \draw (0cm,-0.2cm) rectangle (5cm,1.2cm);
  \end{tikzpicture}
\end{flushright}
\item $\displaystyle{\int_0^{\pi^2/4} \frac{\cos \sqrt{t}}{\sqrt{t}}\, dt}$
\vspace{5cm}
\begin{flushright}
  \begin{tikzpicture}
    \draw (0cm,-0.2cm) rectangle (5cm,1.2cm);
  \end{tikzpicture}
\end{flushright}
\end{enumerate}
\end{problem}
\newpage

%%%%%%%%%%%%%%%%%%%%%%%%%%%%%%%%%%%%% Page 3
\noindent{\large\bf MATH 142}\hfill{\large\bf Second Midterm.}\hfill{\large\bf
  Spring 2015}\hfill{\large\bf Page 3/4}\hrule

\bigskip
\begin{problem}[10 pts]
Compute the area of the region bounded by the graphs of $y=\cos x$ and $y=\sin x$ between $x=0$ and $x=2\pi$.
\vspace{8.5cm}
\begin{flushright}
  \begin{tikzpicture}
    \draw (0cm,-0.2cm) rectangle (5cm,1.2cm);
  \end{tikzpicture}
\end{flushright}
\end{problem}
\hrule
\begin{problem}[20 pts]
Find the area of the region bounded by the graphs of $y=\dfrac{1}{x}$, $y=\dfrac{1}{x^2}$ and $x=2$.
\vspace{8.5cm}
\begin{flushright}
  \begin{tikzpicture}
    \draw (0cm,-0.2cm) rectangle (5cm,1.2cm);
  \end{tikzpicture}
\end{flushright}
\end{problem}
\newpage

%%%%%%%%%%%%%%%%%%%%%%%%%%%%%%%%%%%%% Page 4
\noindent{\large\bf MATH 142}\hfill{\large\bf Second Midterm.}\hfill{\large\bf
  Spring 2015}\hfill{\large\bf Page 4/4}\hrule

\bigskip
\begin{problem}
Find the volume of the solid obtained by rotating the region bounded by the curve $y=\sqrt{x}$ between $x=1$ and $x=9$ about the $x$--axis.
\vspace{8cm}
\begin{flushright}
  \begin{tikzpicture}
    \draw (0cm,-0.2cm) rectangle (5cm,1.2cm);
  \end{tikzpicture}
\end{flushright}
\end{problem}
\hrule
\begin{problem}[20 pts]
Find the volume of the solid obtained by rotating the curve $x=4y^2-y^3$ about the $y$--axis.

\noindent\textbf{Hint:} You need to find first the interval of integration, by computing where the given curve intersects the $y$--axis.
\vspace{8.5cm}
\begin{flushright}
  \begin{tikzpicture}
    \draw (0cm,-0.2cm) rectangle (5cm,1.2cm);
  \end{tikzpicture}
\end{flushright}
\end{problem}
\end{document}
