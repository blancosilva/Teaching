\documentclass[12pt]{article}

\usepackage{amsmath,amsthm,amsfonts,amssymb,amsxtra}
\usepackage{pgf,tikz}
\usetikzlibrary{arrows}
\renewcommand{\theenumi}{(\alph{enumi})} 
\renewcommand{\labelenumi}{\theenumi}

\pagestyle{empty}
\setlength{\textwidth}{7in}
\setlength{\oddsidemargin}{-0.5in}
\setlength{\topmargin}{-1.0in}
\setlength{\textheight}{9.5in}

\theoremstyle{definition}
\newtheorem{problem}{Problem}

\begin{document}

\noindent{\large\bf MATH 141}\hfill{\large\bf Exam\#2.}\hfill{\large\bf Spring 2015}\hfill{\large\bf Page 1/4}\hrule

\bigskip
\begin{center}
  \begin{tabular}{|ll|}
    \hline & \cr
    {\bf Name: } & \makebox[12cm]{\hrulefill}\cr & \cr
    {\bf 4-digit code:} & \makebox[12cm]{\hrulefill}\cr & \cr
    \hline
  \end{tabular}
\end{center}
\begin{itemize}
\item Write your name and the last 4 digits of your SSN in the space provided above.
\item The test has four (4) pages, including this one.
\item Enter your answer in the box(es) provided.
\item You must show sufficient work to justify all answers unless
  otherwise stated in the problem.  Correct answers with inconsistent
  work may not be given credit.
\item Credit for each problem is given in parentheses at the right of
  the problem number.
\item No books, notes or calculators may be used on this test.
\end{itemize}
\hrule

\begin{center}
  \begin{tabular}{|c|c|c|}
    \hline
    &&\cr
    {\large\bf Page} & {\large\bf Max.~points} & {\large\bf Your points} \cr
    &&\cr
    \hline
    &&\cr
    {\Large 2} & \Large 30 & \cr
    &&\cr
    \hline
    &&\cr
    {\Large 3} & \Large 40 & \cr
    &&\cr
    \hline
    &&\cr
    {\Large 4} & \Large 30 & \cr
    &&\cr
   \hline\hline
    &&\cr
    {\large\bf Total} & \Large 100 & \cr
    &&\cr
    \hline
  \end{tabular}
\end{center}
\newpage

%%%%%%%%%%%%%%%%%%%%%%%%%%%%%%%%%%%%% Page 1
\noindent{\large\bf MATH 141}\hfill{\large\bf Exam\#2.}\hfill{\large\bf
  Spring 2015}\hfill{\large\bf Page 2/4}\hrule

\bigskip
\begin{problem}[10 pts] 
Use the limit definition of the derivative to
compute $f'(x)$ for $f(x)=2x - x^2$
\begin{equation*}
f'(x) = \lim_{h \to 0} \frac{ f(x+h)-f(x)}{h}
\end{equation*}
\vspace{2cm}
\begin{flushright}
  \begin{tikzpicture}
    \draw (-1cm,0.5cm) node {$f'(x) =$};
    \draw (0cm,0cm) rectangle (5cm,1.2cm);
  \end{tikzpicture}
\end{flushright}
\end{problem}
\hrule
\begin{problem}[10 pts]
Use implicit differentiation to compute $y'$ if $y$ is a function of $x$ that satisfies $x^3+y^2=4xy$.
\vspace{5cm}
\begin{flushright}
  \begin{tikzpicture}
    \draw (-4cm,0.5cm) node {$y'=$};
    \draw (-3.5cm,0cm) rectangle (5cm,1.2cm);
  \end{tikzpicture}
\end{flushright}
\end{problem}
\hrule
\begin{problem}[10 pts]
Use logarithmic differentiation to compute the derivative of the following function:
\begin{equation*}
y = \frac{\cos^2 x \ln x}{\sqrt{9+x^2} \tan x}
\end{equation*}
\vspace{5cm}
\begin{flushright}
  \begin{tikzpicture}
    \draw (-6.25cm,0.5cm) node {$y'=$};
    \draw (-5.5cm,0cm) rectangle (5cm,1.2cm);
  \end{tikzpicture}
\end{flushright}
\end{problem}
\newpage

%%%%%%%%%%%%%%%%%%%%%%%%%%%%%%%%%%%%% Page 2
\noindent{\large\bf MATH 141}\hfill{\large\bf Exam\#2.}\hfill{\large\bf
  Spring 2015}\hfill{\large\bf Page 3/4}\hrule

\bigskip
\noindent Find the derivative of the following functions:
{\problem[5 pts] $f(x)=7x$}
\begin{flushright}
  \begin{tikzpicture}
    \draw (-1.25cm,0.5cm) node {$ f'(x) =$};
    \draw (0cm,0cm) rectangle (5cm,1.2cm);
  \end{tikzpicture}
\end{flushright}
\hrule
{\problem[5 pts] $f(x)=7x^{21}$}
\begin{flushright}
  \begin{tikzpicture}
    \draw (-1.25cm,0.5cm) node {$ f'(x) =$};
    \draw (0cm,0cm) rectangle (5cm,1.2cm);
  \end{tikzpicture}
\end{flushright}
\hrule
{\problem[5 pts] $f(x)=7(x^2+1)^{21}$}
\begin{flushright}
  \begin{tikzpicture}
    \draw (-1.25cm,0.5cm) node {$ f'(x) =$};
    \draw (0cm,0cm) rectangle (5cm,1.2cm);
  \end{tikzpicture}
\end{flushright}
\hrule
{\problem[5 pts] $f(x)=e^x$}
\begin{flushright}
  \begin{tikzpicture}
    \draw (-1.25cm,0.5cm) node {$ f'(x) =$};
    \draw (0cm,0cm) rectangle (5cm,1.2cm);
  \end{tikzpicture}
\end{flushright}
\hrule
{\problem[5 pts] $f(x)=e^x x^{21}$}
\begin{flushright}
  \begin{tikzpicture}
    \draw (-1.25cm,0.5cm) node {$ f'(x) =$};
    \draw (0cm,0cm) rectangle (5cm,1.2cm);
  \end{tikzpicture}
\end{flushright}
\hrule
{\problem[5 pts] $f(x)=e^x (x^2+1)^{21}$}
\vspace{2cm}
\begin{flushright}
  \begin{tikzpicture}
    \draw (-1.25cm,0.5cm) node {$ f'(x) =$};
    \draw (0cm,0cm) rectangle (5cm,1.2cm);
  \end{tikzpicture}
\end{flushright}
\hrule
{\problem[10 pts] $f(x)=e^x (x^2+1)^{21} \ln x$}
\vspace{2cm}
\begin{flushright}
  \begin{tikzpicture}
    \draw (-1.25cm,0.5cm) node {$ f'(x) =$};
    \draw (0cm,0cm) rectangle (5cm,1.2cm);
  \end{tikzpicture}
\end{flushright}
\newpage

%%%%%%%%%%%%%%%%%%%%%%%%%%%%%%%%%%%%% Page 3
\noindent{\large\bf MATH 141}\hfill{\large\bf Exam\#2.}\hfill{\large\bf
  Spring 2015}\hfill{\large\bf Page 4/4}\hrule

\bigskip
{\problem[5 pts] $f(x)=\sqrt{x}$}
\begin{flushright}
  \begin{tikzpicture}
    \draw (-1.25cm,0.5cm) node {$ f'(x) =$};
    \draw (0cm,0cm) rectangle (5cm,1.2cm);
  \end{tikzpicture}
\end{flushright}
\hrule
{\problem[5 pts] $f(x)=\displaystyle{\frac{1}{\sqrt{x}}}$}
\begin{flushright}
  \begin{tikzpicture}
    \draw (-1.25cm,0.5cm) node {$ f'(x) =$};
    \draw (0cm,0cm) rectangle (5cm,1.2cm);
  \end{tikzpicture}
\end{flushright}
\hrule
{\problem[5 pts] $f(x)=\displaystyle{\frac{\pi}{\sqrt{x}}}$}
\begin{flushright}
  \begin{tikzpicture}
    \draw (-1.25cm,0.5cm) node {$ f'(x) =$};
    \draw (0cm,0cm) rectangle (5cm,1.2cm);
  \end{tikzpicture}
\end{flushright}
\hrule
{\problem[5 pts] $f(x)=\displaystyle{\frac{\pi}{\sqrt{x}}} \tan x$}
\vspace{4cm}
\begin{flushright}
  \begin{tikzpicture}
    \draw (-1.25cm,0.5cm) node {$ f'(x) =$};
    \draw (0cm,0cm) rectangle (5cm,1.2cm);
  \end{tikzpicture}
\end{flushright}
\hrule
{\problem[10 pts] $f(x)=\displaystyle{\frac{\pi}{\sqrt{x}}} \tan
\vspace{4cm}
(\pi x)$}
\begin{flushright}
  \begin{tikzpicture}
    \draw (-1.25cm,0.5cm) node {$ f'(x) =$};
    \draw (0cm,0cm) rectangle (5cm,1.2cm);
  \end{tikzpicture}
\end{flushright}
\newpage

\end{document}
