\documentclass[12pt]{article}

\usepackage{amsmath,amsthm,amsfonts,amssymb,amsxtra}
\usepackage{pgf,tikz}
\usetikzlibrary{arrows}
\renewcommand{\theenumi}{(\alph{enumi})} 
\renewcommand{\labelenumi}{\theenumi}

\pagestyle{empty}
\setlength{\textwidth}{7in}
\setlength{\oddsidemargin}{-0.5in}
\setlength{\topmargin}{-1.0in}
\setlength{\textheight}{9.5in}

\theoremstyle{definition}
\newtheorem{problem}{Problem}

\begin{document}

\noindent{\large\bf MATH 141}\hfill{\large\bf Exam\#3.}\hfill{\large\bf
  Spring 2015}\hfill{\large\bf Page 1/5}\hrule

\bigskip
\begin{center}
  \begin{tabular}{|ll|}
    \hline & \cr
    {\bf Name: } & \makebox[12cm]{\hrulefill}\cr & \cr
    {\bf 4-digit code:} & \makebox[12cm]{\hrulefill}\cr & \cr
    \hline
  \end{tabular}
\end{center}
\begin{itemize}
\item Write your name and the last 4 digits of your SSN in the space provided above.
\item The test has five (5) pages, including this one.
\item You have fifty (50) minutes to complete the exam.
\item Enter your answer in the box(es) provided.
\item You must show sufficient work to justify all answers unless
  otherwise stated in the problem.  Correct answers with inconsistent
  work may not be given credit.
\item Credit for each problem is given in parentheses at the right of
  the problem number.
\item No books, notes or calculators may be used on this test.
\end{itemize}
\hrule

\begin{center}
  \begin{tabular}{|c|c|c|}
    \hline
    &&\cr
    {\large\bf Page} & {\large\bf Max.~points} & {\large\bf Your points} \cr
    &&\cr
    \hline
    &&\cr
    {\Large 2} & \Large 30 & \cr
    &&\cr
    \hline
    &&\cr
    {\Large 4} & \Large 20 & \cr
    &&\cr
    \hline
    &&\cr
    {\Large 5} & \Large 20 & \cr
    &&\cr
    \hline
    &&\cr
    {\Large 6} & \Large 30 & \cr
    &&\cr
   \hline\hline
    &&\cr
    {\large\bf Total} & \Large 100 & \cr
    &&\cr
    \hline
  \end{tabular}
\end{center}
\newpage

%%%%%%%%%%%%%%%%%%%%%%%%%%%%%%%%%%%%% Page 1
\noindent{\large\bf MATH 141}\hfill{\large\bf Exam\#3.}\hfill{\large\bf
  Spring 2015}\hfill{\large\bf Page 2/5}\hrule

\bigskip
\begin{problem}[10 pts]
Find the equation of the tangent lines to the curve $y = (\ln x)/x$ at the points $(1,0)$ and $(e,1/e)$.

\vspace{5cm}

\begin{center}
  \begin{tikzpicture}
    \draw (8.5cm,0.5cm) node {$y = $};
    \draw (9cm,0cm) rectangle (14cm,1.2cm);
    \draw (-0.5cm,0.5cm) node {$y = $};
    \draw (0cm,0cm) rectangle (5cm,1.2cm);
  \end{tikzpicture}
\end{center}
\end{problem}

\hrule

\begin{problem}[10 pts]
Find an equation of the normal line to the curve $y=\sin x + \sin^2 x$ at the point $(0,0)$

\vspace{5cm}

\begin{flushright}
  \begin{tikzpicture}
    \draw (-0.5cm, 0.5cm) node {$y=$};
    \draw (0cm,0cm) rectangle (5cm,1.2cm);
  \end{tikzpicture}
\end{flushright}
\end{problem}

\hrule

\begin{problem}[10 pts]
The curve with equation $y^2=x^3+3x^2$ is called the \textbf{Tschirnhausen cubic}.  At what points does this curve have horizontal tangents?

\vspace{5cm}

\begin{flushright}
  \begin{tikzpicture}
    \draw (0cm,0cm) rectangle (10cm,1.2cm);
  \end{tikzpicture}
\end{flushright}
\end{problem}

\newpage

%%%%%%%%%%%%%%%%%%%%%%%%%%%%%%%%%%%%% Page 2
\noindent{\large\bf MATH 141}\hfill{\large\bf Exam\#3.}\hfill{\large\bf
  Spring 2015}\hfill{\large\bf Page 3/5}\hrule

\bigskip
\begin{problem}[10 pts]
Find the critical values of the function $h(x)=\dfrac{x-2}{x^2+1}$.  You \textbf{do not} have to indicate whether they are local maxima, local minima, or neither.

\vspace{10cm}

\begin{flushright}
  \begin{tikzpicture}
    \draw (0cm,0cm) rectangle (10cm,1.2cm);
  \end{tikzpicture}
\end{flushright}
\end{problem}
\hrule
\begin{problem}[10 pts]
Find the absolute extrema of the function $f(x)=x^3-3x+1$ on the interval $[-3,0]$.

\vspace{5cm}
\begin{center}
  \begin{tikzpicture}
    \draw (-0.5cm,0.5cm) node {$M=$};
    \draw (0cm,0cm) rectangle (5cm,1.2cm);
    \draw (8.5cm,0.5cm) node {$m=$};
    \draw (9cm,0cm) rectangle (14cm,1.2cm);
  \end{tikzpicture}
\end{center}
\end{problem}
\newpage

%%%%%%%%%%%%%%%%%%%%%%%%%%%%%%%%%%%%% Page 3
\noindent{\large\bf MATH 141}\hfill{\large\bf Exam\#3.}\hfill{\large\bf
  Spring 2015}\hfill{\large\bf Page 4/5}\hrule

\bigskip
\begin{problem}[20 pts]
Find the global extrema of the function $f(x) = x^2/(x-2)$ on the interval $(2,5]$.  If any of the extrema does not exist, indicate so.

\vspace{18cm}
\begin{center}
  \begin{tikzpicture}
    \draw (-0.5cm,0.5cm) node {$M=$};
    \draw (0cm,0cm) rectangle (5cm,1.2cm);
    \draw (8.5cm,0.5cm) node {$m=$};
    \draw (9cm,0cm) rectangle (14cm,1.2cm);
  \end{tikzpicture}
\end{center}
\end{problem}

\newpage


%%%%%%%%%%%%%%%%%%%%%%%%%%%%%%%%%%%%% Page 4
\noindent{\large\bf MATH 141}\hfill{\large\bf Exam\#3.}\hfill{\large\bf
  Spring 2015}\hfill{\large\bf Page 5/5}\hrule

\bigskip
{\problem \em Compute the following limits:}
\begin{itemize}
  \item[] [5 pts] $\displaystyle{\lim_{x\to 0^+} x \ln (x)} =$ \framebox[2cm][c]{\textcolor{white}{$\bigg ( $}}
  \vspace{3cm}
  \item[] [5 pts] $\displaystyle{\lim_{x\to\infty} \frac{1-x-x^2}{5x^2-9}} = $ \framebox[2cm][c]{\textcolor{white}{$\bigg ( $}}
  \vspace{3cm}
  \item[] [10 pts] $\displaystyle{\lim_{x\to\infty} \Big(1-\frac{3}{x} \Big)^{4x}} = $ \framebox[2cm][c]{\textcolor{white}{$\bigg ( $}}
  \vspace{6cm}
  \item[] [10 pts] $\displaystyle{\lim_{x\to\infty} x^{1/x}} = $ \framebox[2cm][c]{\textcolor{white}{$\bigg ( $}}
\end{itemize}

\end{document}
