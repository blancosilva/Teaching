\documentclass[12pt]{article}

\usepackage{amsmath,amsthm,amsfonts,amssymb,amsxtra}
\usepackage{pgf,tikz}
\usetikzlibrary{arrows}
\renewcommand{\theenumi}{(\alph{enumi})} 
\renewcommand{\labelenumi}{\theenumi}

\pagestyle{empty}
\setlength{\textwidth}{7in}
\setlength{\oddsidemargin}{-0.5in}
\setlength{\topmargin}{-1.0in}
\setlength{\textheight}{9.5in}

\theoremstyle{definition}
\newtheorem{problem}{Problem}

\begin{document}

\noindent{\large\bf MATH 141}\hfill{\large\bf Final Exam.}\hfill{\large\bf
  Spring 2015}\hfill{\large\bf Page 1/6}\hrule

\bigskip
\begin{center}
  \begin{tabular}{|ll|}
    \hline & \cr
    {\bf Name: } & \makebox[12cm]{\hrulefill}\cr & \cr
    {\bf 4-digit code:} & \makebox[12cm]{\hrulefill}\cr & \cr
    \hline
  \end{tabular}
\end{center}
\begin{itemize}
\item Write your name and the last 4 digits of your SSN in the space provided above.
\item The test has six (6) pages, including this one.
\item Enter your answers in the boxes provided.
\item You must show sufficient work to justify all answers unless
  otherwise stated in the problem.  Correct answers with inconsistent
  work may not be given credit.
\item Credit for each problem is given in parentheses at the right of
  the problem number.
\item No books, notes or calculators may be used on this test.
\end{itemize}
\hrule

\begin{center}
  \begin{tabular}{|c|c|c|}
    \hline
    &&\cr
    {\large\bf Page} & {\large\bf Max.~points} & {\large\bf Your points} \cr
    &&\cr
    \hline
    &&\cr
    {\Large 2} & \Large 20 & \cr
    &&\cr
    \hline
    &&\cr
    {\Large 3} & \Large 20 & \cr
    &&\cr
    \hline
    &&\cr
    {\Large 4} & \Large 20 & \cr
    &&\cr
    \hline
    &&\cr
    {\Large 5} & \Large 20 & \cr
    &&\cr
	\hline
    &&\cr
    {\Large 6} & \Large 20 & \cr
    &&\cr
  \hline\hline
    &&\cr
    {\large\bf Total} & \Large 100 & \cr
    &&\cr
    \hline
  \end{tabular}
\end{center}
\newpage

%%%%%%%%%%%%%%%%%%%%%%%%%%%%%%%%%%%%% Page 2
\noindent{\large\bf MATH 141}\hfill{\large\bf Final Exam.}\hfill{\large\bf
  Spring 2015}\hfill{\large\bf Page 2/6}\hrule

\bigskip
{\problem[5 pts---all or nothing] Find the domain of $f(x) = \sqrt{(1-x)(2-x)}$.}
\vspace{2cm}
\begin{flushright}
  \begin{tikzpicture}
    \draw (-1cm,0.5cm) node {domain $=$};
    \draw (0cm,0cm) rectangle (5cm,1.2cm);
  \end{tikzpicture}
\end{flushright}
\hrule

{\problem[5 pts---all or nothing] Let $f(x) = x^2+4$, $g(x) = \sqrt{x}$.  Find $g \circ f$
\vspace{2cm}
\begin{flushright}
  \begin{tikzpicture}
    \draw (-1.25cm,0.5cm) node {$ \big(g \circ f\big)(x) =$};
    \draw (0cm,0cm) rectangle (5cm,1.2cm);
  \end{tikzpicture}
\end{flushright}
\hrule

\begin{problem}[5pts]
Evaluate the following limit:
\begin{equation*}
\lim_{n \to \infty} \sum_{k=1}^n \frac{5k}{n^2}
\end{equation*}

\vspace{3cm}
\begin{flushright}
  \begin{tikzpicture}
    \draw (0cm,0cm) rectangle (5cm,1.2cm);
  \end{tikzpicture}
\end{flushright}
\end{problem}
\hrule

{\problem[5 pts] Assume $y$ is a function of $x$ given implicitly by $\sin(x+y)=x+y$.  Find $y'$.}
\vspace{4.9cm}
\begin{flushright}
  \begin{tikzpicture}
    \draw (0cm,0cm) rectangle (10cm,1.2cm);
  \end{tikzpicture}
\end{flushright}
\newpage


%%%%%%%%%%%%%%%%%%%%%%%%%%%%%%%%%%%%% Page 3
\noindent{\large\bf MATH 141}\hfill{\large\bf Final Exam.}\hfill{\large\bf
  Spring 2015}\hfill{\large\bf Page 3/6}\hrule

\bigskip


{\problem [10 pts] Use logarithmic differentiation to find the
  derivative of the function 
\begin{equation*}
y=\frac{\tan^2 x \sin^4 x}{e^{3x}(x^2+1)}
\end{equation*}
\vspace{8cm}
\begin{flushright}
  \begin{tikzpicture}
    \draw (-0.7cm,0.5cm) node {$y'=$ };
    \draw (0cm,-0.2cm) rectangle (10cm,1.2cm);
  \end{tikzpicture}
\end{flushright}
\hrule

{\problem[10 pts] Find an equation of the tangent line to the curve
$y=\ln( x e^x)$ at the point $(1,1)$.}
\vspace{8cm}
\begin{flushright}
  \begin{tikzpicture}
    \draw (0cm,-0.2cm) rectangle (5cm,1.2cm);
  \end{tikzpicture}
\end{flushright}

\newpage

%%%%%%%%%%%%%%%%%%%%%%%%%%%%%%%%%%%%% Page 4
\noindent{\large\bf MATH 141}\hfill{\large\bf Final Exam.}\hfill{\large\bf
  Spring 2015}\hfill{\large\bf Page 4/6}\hrule

\bigskip
{\problem[20 pts] Sketch the graph of the rational function $f(x)
  = \displaystyle{\frac{x^2}{1-x^2}}$.}
\begin{quotation}
{\small \noindent
Indicate clearly: Domain; $x$- and $y$-intercepts; vertical and horizontal asymptotes; intervals of increase, decrease and different concavity. Indicate also the location of relative extrema, if any.}
\end{quotation}

\newpage

%%%%%%%%%%%%%%%%%%%%%%%%%%%%%%%%%%%%% Page 5
\noindent{\large\bf MATH 141}\hfill{\large\bf Final Exam.}\hfill{\large\bf
  Spring 2015}\hfill{\large\bf Page 5/6}\hrule
  
\bigskip

{\problem[10 pts] The volume of a cube is increasing at a rate of
  $300~\text{cm}^3/\text{min}$.  How fast are the edges increasing
  when the length of an edge is $10~\text{cm}$?}
\vspace{8cm}
\begin{flushright}
  \begin{tikzpicture}
    \draw (-4cm,0.5cm) node {The edges are increasing at a speed of};
    \draw (0cm,-0.2cm) rectangle (5cm,1.2cm);
  \end{tikzpicture}
\end{flushright}
\hrule
{\problem[10 pts] A farmer wants to fence an area of 1.5 million square feet in a rectangular field and then divide it in half with a fence parallel to one of the sides of the rectangle.  How can he do this so as to minimize the cost of the fence?}
\vspace{8cm}
\begin{flushright}
  \begin{tikzpicture}
    \draw (-3.7cm,0.5cm) node {Dimensions of most economic fence: };
    \draw (0cm,-0.2cm) rectangle (5cm,1.2cm);
  \end{tikzpicture}
\end{flushright}
\newpage

%%%%%%%%%%%%%%%%%%%%%%%%%%%%%%%%%%%%% Page 6
\noindent{\large\bf MATH 141}\hfill{\large\bf Final Exam.}\hfill{\large\bf
  Spring 2015}\hfill{\large\bf Page 6/6}\hrule

\bigskip
{\problem[1/2/3/4 pts] Compute the following limits:}

\begin{enumerate}
\item $\displaystyle{\lim_{x \to -\infty} \frac{x^2-2x-8}{x^2-4}} =$
\vspace{1cm}
\item $\displaystyle{\lim_{x \to 2} \frac{x^2+2x-8}{x^2-4}} = $
\vspace{1cm}
\item $\displaystyle{\lim_{x \to -2} \frac{x^2-2x-8}{x^2-4}} =$
\vspace{2cm}
\item $\displaystyle{\lim_{x \to 0} x^{1/x}} = $
\end{enumerate}
\vspace{4cm}

\hrule

{\problem[3/3/4 pts] Evaluate each integral:} 
\begin{enumerate}
\item $\displaystyle{\int \big( \tfrac{1}{x} - 2^x \big)\, dx = }$
\vspace{1cm}
\item $\displaystyle{\int \big( 3\sin x - 2\cos x \big)\, dx = }$
\vspace{1cm}
\item $\displaystyle{\int_1^2 \big( 5x + \frac{2}{3x^5} - \sqrt{2} e^x \big)\, dx = }$

\end{enumerate}

\end{document}
