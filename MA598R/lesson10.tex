\documentclass[12pt,reqno]{amsart}
\usepackage{pgf,tikz}

\renewcommand{\theenumi}{(\roman{enumi})} 
\renewcommand{\labelenumi}{\theenumi}
\def\field#1{\mathbb{#1}}
\def\abs#1{\lvert {#1} \rvert}
\def\Lpnorm#1#2{\lVert {#1} \rVert_{#2}}
\theoremstyle{definition}
\newtheorem{problem}{Problem}
\theoremstyle{remark}
\newtheorem*{solution}{Solution}

\begin{document}
\title{Advanced problems}
\maketitle

\begin{problem}
  Suppose $f \in L_1[a,b]$ and that $\int_a^c f(t)\, dt = 0$ for all
  $c \in [a,b]$. Prove that $f(x) = 0$ a.e.
\end{problem}
\begin{problem}
  Suppose $f$ is a real-valued function defined on an interval $I$,
  and that all the Dini numbers of $f$ for $x$ in $I$ lie between $-K$
  and $K$, where $K$ is some positive constant.  Must $f$ be Lipschitz
  on $I$?  If so, what is the relation between the Lispchitz constant
  of $f$ and $K$?
\end{problem}
\begin{problem}
  Compute the Dini numbers of the Cantor-Lebesgue function at each
  point $x$ of $[0,1]$.
\end{problem}
\begin{problem}
  Does there exist a strictly increasing functions $f$ defined on an
  interval $I$ so that $f' = 0$~a.e.~on $I$?
\end{problem}
\begin{problem}
  Suppose $f$ is a real-valued function defined on $I$.  Show that if
  $f$ is not constant and $f'=0$~a.e., then $f$ cannot be Lipschitz on
  $I$.
\end{problem}
\begin{problem}
  Let $f$ be a real-valued continuous function defined on $I=[a,b]$,
  and suppose that $f$ is AC on $[a,d]$, for any $d<b$.  Show that $f$
  is AC on $I$.
\end{problem}
\begin{problem}
  Let $f$ be AC on $I$, and $f(I) \subseteq J$.  If $\phi\colon J \to
  \field{R}$ is Lipschitz, show that $\phi \circ f$ is AC on $I$.
\end{problem}
\begin{problem}
  Suppose $f$ is a non-decreasing, AC function on $I$, and $f(I)
  \subseteq J$.  Show that if $\phi$ is AC on $J$, then $\phi \circ
  f$ is AC on $I$.
\end{problem}

\begin{problem}
  Given that $f \in L_1(\field{R})$ and that $\int_{\field{R}}
  \int_{\field{R}} f(4x) f(x+y)\, dx\, dy = 1$, calculate
  $\int_{\field{R}} f(x)\, dx$.
\end{problem}
\begin{problem}
  Calculate $\displaystyle{ \int_0^\infty \int_0^{\sqrt{\pi}}
    \frac{x^3 y^3 \cos(y^2)}{(x^4 + y^4)^{3/2}}\, dy\, dx }$.
\end{problem}
\begin{problem}
  Given $f \in L_1(\field{R})$ and $h.0$, let $\phi_h(x) =
  \frac{1}{2h} \int_x^{x+h} f(t)\, dt$.  Prove that $\phi_h \in
  L_1(\field{R})$ and $\int_{\field{R}} \abs{\phi_h(x)}\, dx \leq
  \Lpnorm{f}{1}$.
\end{problem}

\begin{problem}
  Suppose $g \in L_1[0,1]$, $1 \leq p < \infty$ and that there exists
  a constant $M>0$ such that $\big\lvert \int_0^1 g(x)s(x)\, dx
  \big\rvert \leq M \Lpnorm{s}{p}$ for all simple functions $s$.
  Prove that $g \in L_q[0,1]$ and $\Lpnorm{g}{q} \leq M$, where $q$
  satisfies $1/p + 1/q = 1$.
\end{problem}
\begin{problem}
  Let $\varphi \geq 0$ with $\int_{\field{R}^n} \varphi(y)\, dy =1$.
  Denote $\varphi_\varepsilon (x) = \varepsilon^n
  \varphi(x/\varepsilon)$. Prove the following statements:
  \begin{enumerate}
  \item $\displaystyle{\int_{\field{R}^n} \varphi_\varepsilon (x)\, dx
      = \int_{\field{R}^n} \varphi(x)\, dx}$.
  \item For any $\delta >0$, $\displaystyle{\lim_{\varepsilon \to 0}
      \int_{\{ \abs{x}> \delta \}} \varphi_\varepsilon (x)\, dx = 0}$.
  \item \textcolor{blue}{If $f \in L_p(\field{R}^n)$, $1 \leq p <
      \infty$, then $\lim_{\varepsilon \to 0} \Lpnorm{f \ast
        \varphi_\varepsilon - f}{p} = 0$.}
  \item \textcolor{blue}{If $f \in L_\infty(\field{R}^n)$, then
      $\lim_{\varepsilon \to 0} f \ast \varphi_\varepsilon (x) =
      f(x)$.}
  \end{enumerate}
\end{problem}
\begin{problem}
  \textcolor{blue}{Compute the Fourier transform of $H(x) =
    (4\pi)^{-n/2} e^{-\abs{x}^2/4}$, $x \in \field{R}^n$.}
\end{problem}
\begin{tikzpicture}
  \draw(0.5\linewidth,1cm) node[shade, text justified, text width=
  0.9\linewidth, draw,rounded corners]{ \textbf{Hint:} Show that
    $\phi(\xi) = (4\pi)^{-1/2} \int_{\field{R}} \cos (2\pi x \xi)
    e^{-x^2/4}\, dx$ satisfies $\phi'(\xi) = -8\pi^2 \xi \phi(\xi)$.};
\end{tikzpicture}
\begin{problem}
  Prove that for all $f \in L_1(\field{R}^n)$, and a.e.~$x \in
  \field{R}^n$,
  \begin{equation*}
    f(x) = \int_{\field{R}^n} \widehat{f}(\xi) e^{-2\pi i x \cdot
    \xi}\, d\xi.
  \end{equation*}
\end{problem}
\begin{tikzpicture}
  \draw(0.5\linewidth,1cm) node[shade, text justified, text width=
  0.9\linewidth, draw,rounded corners]{ \textbf{Hint:} Use the
  approximation to the identity given by $\varphi = H$.};
\end{tikzpicture}
\begin{problem}
  Let $f \in L_1(\field{R}^n)$. Prove the following statements:
  \begin{enumerate}
  \item If $f$ is non-negative, then $\Lpnorm{\widehat{f}}{\infty} =
    \widehat{f}(0) = \Lpnorm{f}{1}$.
  \item If $f$ is continuous at $0$ and $\widehat{f}$ is non-negative,
    then $\Lpnorm{\widehat{f}}{1} = f(0)$.
  \end{enumerate}
\end{problem}
\begin{problem}
  Let $f \in C_c^\infty(\field{R}^n)$ be a radial function.  Prove
  that its Fourier transform is also radial.
\end{problem}
\begin{problem}
  Let $f$ be a function on the real line $\field{R}$ such that both
  $f$ and $g(x)=xf(x)$ are in $L_2(\field{R})$.  Prove that $f \in
  L_1(\field{R})$ and $\Lpnorm{f}{1}^2 \leq 8 \Lpnorm{f}{2}
  \Lpnorm{g}{2}$.
\end{problem}
\begin{problem}
  Let $F$ be a closed set in $\field{R}^n$ with $m(\field{R}^n
  \setminus F) < \infty$.  Let $\delta_F(x) = \inf \{ \abs{x - y} : y
  \in F\}$ denote the distance from the point $x$ to the set $F$.
  Prove that there exists a constant $C>0$ such that $\int_F
  \mathfrak{I}_F(x)\, dx \leq C m(\field{R}^n \setminus F)$, where
  \begin{equation*}
    \mathfrak{I}_F (x) = \int_{\field{R}^n}
    \frac{\delta_F(y)}{\abs{x-y}^{n+1}}\, dy.
  \end{equation*}
\end{problem}
\end{document}
