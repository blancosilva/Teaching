\documentclass[12pt]{amsart}
\usepackage{pgf,tikz}

\renewcommand{\theenumi}{(\roman{enumi})} 
\renewcommand{\labelenumi}{\theenumi}
\def\field#1{\mathbb{#1}}
\def\abs#1{\lvert {#1} \rvert}
\def\Lpnorm#1#2{\lVert {#1} \rVert_{#2}}
\theoremstyle{definition}
\newtheorem{problem}{Problem}
\theoremstyle{remark}
\newtheorem*{solution}{Solution}

\begin{document}
\title{Abstract and Lebesgue Measures. The Cantor Set.}
\maketitle

\begin{problem}
  \textcolor{blue}{Prove that $(\limsup_n A_n) \cap (\limsup_n B_n)
    \supseteq \limsup_n (A_n \cap B_n)$, and that $(\limsup_n A_n)
    \cup (\limsup_n B_n) = \limsup_n (A_n \cup B_n)$.  What are the
    corresponding statements for the $\liminf$?}
\end{problem}
\begin{problem}
  Find $\sigma(\{ \emptyset \})$.  If $\emptyset \subseteq A \subseteq
  X$, what is $\sigma(\{ A \})$? If $A_1, A_2$ are distinct subsets of
  $X$, show that $\sigma(\{ A_1, A_2 \})$ consists of, at most,
  sixteen sets.
\end{problem}
\begin{problem}
  \textcolor{blue}{ Let $\mathcal{O}$ be an open set of $\field{R}^n$.
    Show that there is a sequence of nonoverlapping closed
    $n$-dimensional cubes $\{ I_k \}$ such that $\mathcal{O} = \cup_k
    I_k$.}
\end{problem}
\begin{problem}
  Prove that $\mathcal{B}_n \times \mathcal{B}_m = \mathcal{B}_{n+m}$.
\end{problem}
\begin{problem}
  Let $(X, \mathcal{F}, \mu)$ be a measure space, and let $\{ E_n \}
  \subseteq \mathcal{F}$.  Show that if $\mu(\cup_n E_n) < \infty$,
  and $\mu( E_n ) \geq \eta >0$ for infinitely many $n$'s, then
  $\mu(\limsup_n E_n) > 0$.  By means of an example show that the
  condition $\mu( \cup_n E_n) < \infty$ cannot be removed.
\end{problem}
\begin{problem}
  \textcolor{blue}{ Let $(X, \mathcal{F}, \mu)$ be a measure space,
    and $\{ E_n \} \subseteq \mathcal{F}$.  Show that $\mu(\liminf_n
    E_n) \leq \liminf_n \mu(E_n)$ and, provided that $\mu(\cup_n E_n)
    < \infty$, $\limsup_n \mu(E_n) \leq \mu(\limsup_n E_n)$.}
  \textcolor{gray}{By means of examples show that we may have strict
    inequalities above.}
\end{problem}
\begin{problem}
  \textcolor{blue}{Let $\mu$ be a measure on $(\field{R},
    \mathcal{B}_1)$ with the property that $\mu(I) < \infty$ for every
    finite interval $I$.  Let $y \in \field{R}$ and put
    \begin{equation*}
      F_y(x) = \begin{cases}
        \mu\big( (y,x] \big) &\text{if } x>y, \\
        0 &\text{if } x=y, \\
        -\mu\big( (x,y] \big) &\text{if } x<y.
      \end{cases}
    \end{equation*}
    Show that $F_y$ is a nondecreasing right-continuous function.}
\end{problem}
\begin{problem}[\textcolor{red}{Fall'05}]
  Let $(X, \mathcal{F}, \mu)$ be a measure space with $\mu(X)=1$.  Fix
  $1 \leq n \leq m$, and let $E_1, \dotsc, E_m$ be measurable sets
  with the property that almost every $x \in X$ belongs to at least
  $n$ of these sets.  Prove that at least one of these sets must have
  $\mu$-measure greater than or equal to $n/m$.
\end{problem}
\begin{problem}
  Suppose $A$, $B$ are not Lebesgue measurable.  Is the same true of
  $A\cup B$?
\end{problem}
\begin{problem}
  Assume that $\abs{N} = 0$ and show that $\{ x^3 : x \in N\}$ is a
  null Lebesgue set.
\end{problem}
\begin{problem}[\textcolor{red}{Fall'89}]
  Suppose that $E$ is a Lebesgue measurable subset of $\field{R}$ such
  that $m(E)<\infty$.  Define $f(x) = m\big( (E+x) \cap E \big)$.
  Prove that $f$ is a continuous function on $\field{R}$ and that
  $\lim f(x) = 0$ as $x \to \infty$.
\end{problem}
\begin{problem}[\textcolor{red}{Fall'92}]
  Let $\{ I_n \}_{n \in \Gamma}$ be a collection of closed intervals
  in $\field{R}$.  Show that $\cup_{n \in \Gamma} I_n \setminus
  \sup_{n \in \Gamma} \operatorname{Int}{I}_n$ is countable.
\end{problem}
\begin{problem}[\textcolor{red}{Fall'92}]
  Let $A \subseteq [0,1]$ be a measurable set of positive measure.
  Show that there exist $x \neq y \in A$ such that $x-y \in
  \field{Q}$.
\end{problem}
\begin{problem}
  Let $A = \{ x\ in [0,1] : x = 0.a_1a_2\dotsc;\, a_n \neq 7, \text{
    all }n\}$.  Prove that $\abs{A} = 0$.  \textcolor{gray}{Generalize
    this result to different configurations of $a_n$'s and to dyadic,
    triadic expansions.}
\end{problem}

\begin{problem}[\textcolor{red}{Spring'03}]
  Let $A$ and $B$ (not necessarily Lebesgue measurable) subsets of
  $\field{R}$ and let $\abs{\cdot}_e$ stand for Lebesgue outer
  measure.  Prove that if $\abs{A}_e =1$ and $\abs{B}_e=1$ and $\abs{A
  \cup B}_e = 2$, then $\abs{ A \cap B }_e = 0$.
\end{problem}
\begin{problem}[\textcolor{red}{Fall'04}]
  Let $0 < \varepsilon <1$.  Construct a closed subset $S_\varepsilon
  \subset [0,1]$ which has empty interior but has Lebesgue measure
  greater than $\varepsilon$.
\end{problem}
\begin{problem}
  If $-1 \leq r \leq 1$, show there exists $x, y \in \mathfrak{C}$
  such that $y - x = r$.
\end{problem}
\begin{problem}
  Construct a Cantor-like subset of $[0,1]$ which consists entirely of
  irrational numbers.
\end{problem}
\begin{problem}
  \textcolor{blue}{Prove that there is no Lebesgue measurable subset
    $A$ of $\field{R}$ such that $a \abs{I} \leq \abs{A \cap I} \leq b
    \abs{I}$ for all bounded open intervals $I \subset \field{R}$, and
    $0 < a \leq b < 1$.}

  \textcolor{gray}{Do that by proving the following two assertions:}
  \begin{enumerate}
  \item \textcolor{gray}{If $\abs{A \cap I} \leq b\abs{I}$ for all
      open intervals $I \subset \field{R}$ and $b < 1$, then $\abs{A}
      = 0$.}
  \item \textcolor{gray}{If $a\abs{I} \leq \abs{A \cap I}$ for all
      open intervals $I \subset \field{R}$ and $a >0$, then $\abs{A} =
      \infty$.}
  \end{enumerate}
\end{problem}
\begin{problem}
  Prove that there exists a Lebesgue measurable set $E \subset
  \field{R}$ such that $0 < \abs{E \cap I} < \abs{I}$, all bounded
  intervals $I \subset \field{R}$.
\end{problem}
\begin{problem}[\textcolor{red}{Spring'04}]
  Prove that
  \begin{equation*}
    m^\star (E_1) - m^\star(E_2) \leq 2 m^\star(E_1
    \operatorname{\Delta} E_2) + 2 m^\star (E_1 \cap E_2),
  \end{equation*}
  where $m^\star$ is the Lebesgue outer measure on $\field{R}$, and
  $E_1, E_2 \subset \field{R}$.
\end{problem}
\begin{problem}
  \textcolor{blue}{ Assume $A$ is a Lebesgue measurable subset of
    $\field{R}$ of finite measure and put $\phi(x) = \abs{A \cap
      (-\infty, x]}$. Show that $\phi$ is continuous at each of $x \in
    \field{R}$.}
\end{problem}
\begin{problem}[\textcolor{red}{Spring'04}]
  Let $A \subset \field{R}$ be a Lebesgue measurable set.  Show that
  if $0 \leq b \leq m(A)$, then there is a Lebesgue measurable set $B
  \subset A$ with $m(B) = b$.
\end{problem}
\begin{problem}[\textcolor{red}{Spring'07}]
Answer the following questions:
\begin{enumerate}
\item Suppose that $f\colon [0,1] \to \field{R}$ is non-decreasing
  with $f(0)=0$ and $f(1)=1$. For $a>0$, let $A$ be the set of all $x
  \in (0,1)$ for which
  \begin{equation*}
    \limsup_{h \to 0} \frac{f(x+h) - f(x)}{h} > a.
  \end{equation*}
  Prove that $m^\star(A) < 1/a$, where $m^\star$ denotes the Lebesgue
  outer measure.
\item Prove that there is no Lebesgue measurable set $A$ in $[0,1]$
  with the property that $m(A\cap I) = m(I)/4$ for every interval $I$.
\end{enumerate}
\end{problem}
\begin{problem}[\textcolor{red}{Spring'07}]
  Let $\{ E_n \}_{n=1}^\infty$ be Lebesgue-measurable sets in $[0,1]$,
  let $E= \displaystyle{\cup_{n=1}^\infty} E_n$ and suppose there
  is an $\varepsilon > 0$ such that $\displaystyle{\sum_{n=1}^\infty}
  m(E_n) \leq m(E) + \varepsilon$.
\begin{enumerate}
\item Show that for all measurable sets $A \subset [0,1]$,
  \begin{equation*}
    \sum_{n=1}^\infty m(A \cap E_n) \leq m(A \cap E) + \varepsilon.
  \end{equation*}
\item Let $A$ be the set of all $x \in [0,1]$ which are in at least
  two of the $E_n$'s. Prove that $m(A) \leq \varepsilon$.
\end{enumerate}
\end{problem}

\end{document}
