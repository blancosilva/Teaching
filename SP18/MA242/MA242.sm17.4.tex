\documentclass[12pt]{article}

\usepackage{amsmath,amsthm,amsfonts,amssymb,amsxtra}
\usepackage{tikz,array}
\usetikzlibrary{arrows}
\renewcommand{\theenumi}{(\alph{enumi})} 
\renewcommand{\labelenumi}{\theenumi}

\pagestyle{empty}
\setlength{\textwidth}{7in}
\setlength{\oddsidemargin}{-0.5in}
\setlength{\topmargin}{-1.0in}
\setlength{\textheight}{9.5in}

\theoremstyle{definition}
\newtheorem{problem}{Problem}

\begin{document}

\noindent{\large\bf MATH 242}\hfill{\large\bf Test \#4}\hfill{\large\bf
  Summer 2017}\hfill{\large\bf Page 1/6}\hrule

\bigskip
\begin{center}
  \begin{tabular}{|ll|}
    \hline & \cr
    {\bf Name: } & \makebox[12cm]{\hrulefill}\cr & \cr
    {\bf VIP ID:} & \makebox[12cm]{\hrulefill}\cr & \cr
    \hline
  \end{tabular}
\end{center}
\begin{itemize}
\item Write your name and your VIP ID in the space provided above.
\item The test has six (6) pages, including this one and two pages of scratch paper at the end.
\item \textbf{Do not answer} any problem in the scratch paper.  All solutions must be provided on pages 2--4 where it proceeds.
\item \textbf{Do not detach} the scratch paper from the booklet.  If you use it, indicate clearly the problem number associated with the work.
\item Show sufficient work to justify all answers unless otherwise stated in the problem.  Correct answers with inconsistent work may not be given credit.
\item Credit for each problem is given at the right of each problem
  number. 
\end{itemize}
\hrule

\begin{center}
  \begin{tabular}{|c|c|c|}
    \hline
    &&\cr
    {\large\bf Page} & {\large\bf Max} & {\large\bf Points} \cr
    &&\cr
    \hline
    &&\cr
    {\Large 2} & \Large 30 & \cr
    &&\cr
    \hline
    &&\cr
    {\Large 3} & \Large 40 & \cr
    &&\cr
    \hline
    &&\cr
    {\Large 4} & \Large 30 & \cr
    &&\cr
    \hline\hline
    &&\cr
    {\large\bf Total} & \Large 100 & \cr
    &&\cr
    \hline
  \end{tabular}
\end{center}
\newpage

%%%%%%%%%%%%%%%%%%%%%%%%%%%%%%%%%%%%% Page 2
\noindent{\large\bf MATH 242}\hfill{\large\bf Test \#4}\hfill{\large\bf Summer 2017}\hfill{\large\bf Page 2/6}\hrule

\bigskip
\begin{problem}[30 pts---10 pts each part]
In 1885, the population of a certain country was 50 million and was growing at a rate of 750,000 people per year.  In 1940 its population was 100 million and was growing at the rate of 1 million per year.  Assume this population satisfies the logistic equation.
\begin{enumerate}
  \item Determine the limiting population
  \vspace{4cm}
  \begin{flushright}
  \begin{tikzpicture}
  \draw (-2cm, 0.5cm) node{Limiting population:};
  \draw (0cm,-0.2cm) rectangle (5cm,1.2cm);
  \end{tikzpicture}
  \end{flushright}
  \item What is the predicted population in 2,018?
  \vspace{4cm}
  \begin{flushright}
  \begin{tikzpicture}
  \draw (-2cm, 0.5cm) node{Population in 2,018:};
  \draw (0cm,-0.2cm) rectangle (5cm,1.2cm);
  \end{tikzpicture}
  \end{flushright}
  \item Sketch the diagram with solution curves for the corresponding equation (we referred to this as \emph{studying the model}).  Make sure to indicate what are the equilibria, and whether they are \emph{stable}, \emph{unstable} or \emph{semistable}
\end{enumerate}
\end{problem} 

\newpage

%%%%%%%%%%%%%%%%%%%%%%%%%%%%%%%%%%%%% Page 3
\noindent{\large\bf MATH 242}\hfill{\large\bf Test \#4}\hfill{\large\bf
  Summer 2017}\hfill{\large\bf Page 3/6}\hrule

\bigskip
\begin{problem}[40 pts---10 pts each]
For a logistic population $P(t)$ of fish on a lake, suppose that $k=1$ and $M=4$, measured in hundreds after $t$ years.  Suppose that 300 fish are removed annually by fishing at a constant rate throughout the year.  
\begin{enumerate}
\item Find the \emph{threshold populations} $H < M$ so that we can write this model in the form \newline $P'(t) = k(N-P)(P-H)$.  
\vspace{1.54cm}
\begin{flushright}
\begin{tikzpicture}
\draw (-0.5cm, 0.5cm) node{$H=$};
\draw (0cm,-0.2cm) rectangle (3cm,1.2cm);
\begin{scope}[xshift=5cm]
\draw (-0.5cm, 0.5cm) node{$N=$};
\draw (0cm,-0.2cm) rectangle (3cm,1.2cm);
\end{scope}
\end{tikzpicture}
\end{flushright}
\item Sketch the diagram of solution curves.  Indicate the equilibria, and classify their stabilily.
\vspace{3.75cm}
\item Assume the initial population of fish in that lake was 150.  Will the population die or thrive?  If it dies, indicate how long will this take.  If it thrives, indicate when the population will reach 90\% of the limiting population.
\vspace{3.1cm}
\begin{flushright}
\begin{tikzpicture}
\draw (-0.5cm, 0.5cm) node{$t = $};
\draw (0cm,-0.2cm) rectangle (4cm,1.2cm);
\end{tikzpicture}
\end{flushright}
\item Assume the initial population of fish in that lake was 50.  Will the population die or thrive?  If it dies, indicate how long will this take.  If it thrives, indicate when the population will reach 90\% of the limiting population.
\vspace{2.1cm}
\begin{flushright}
\begin{tikzpicture}
\draw (-0.5cm, 0.5cm) node{$t = $};
\draw (0cm,-0.2cm) rectangle (4cm,1.2cm);
\end{tikzpicture}
\end{flushright}
\end{enumerate}
\end{problem}

\newpage

%%%%%%%%%%%%%%%%%%%%%%%%%%%%%%%%%%%%% Page 4
\noindent{\large\bf MATH 242}\hfill{\large\bf Test \#4}\hfill{\large\bf
  Summer 2017}\hfill{\large\bf Page 4/6}\hrule

\bigskip
\begin{problem}[30 pts---10 pts each]
A baseball moving through the air at a moderate speed $v$ (less than 300~ft/s) encounters air resistance that is approximately proportional to $v$ (but we do not know the constant of proportion $\rho>0$)
\begin{enumerate}
\item Find an equation of $v$ that models the total acceleration of the ball.
\vspace{2cm}
\begin{flushright}
\begin{tikzpicture}
\draw (0cm,-0.2cm) rectangle (5cm,1.2cm);
\end{tikzpicture}
\end{flushright}
\item Suppose that you drop a baseball straight downward from a helicopter hovering at an altitude of 450~feet (assume $v(0)=0$).  Estimate the speed with which the ball will land in ft/s.
\vspace{10cm}
\begin{flushright}
\begin{tikzpicture}
\draw (-0.5cm, 0.5cm) node{$v = $};
\draw (0cm,-0.2cm) rectangle (4cm,1.2cm);
\draw (3.5cm,0.5cm) node{ft/s}; 
\end{tikzpicture}
\end{flushright}
\item What is that speed in mph?
\vspace{0.8cm}
\begin{flushright}
\begin{tikzpicture}
\draw (-0.5cm, 0.5cm) node{$v = $};
\draw (0cm,-0.2cm) rectangle (4cm,1.2cm);
\draw (3.5cm,0.5cm) node{mph}; 
\end{tikzpicture}
\end{flushright}
\end{enumerate} 
\end{problem}

%%%%%%%%%%%%%%%%%%%%%%%%%%%%%%%%%%%%% Page 5
\noindent{\large\bf MATH 242}\hfill{\large\bf Test \#4}\hfill{\large\bf
  Summer 2017}\hfill{\large\bf Page 5/6}\hrule

\bigskip
\Large Scratch paper

\newpage

%%%%%%%%%%%%%%%%%%%%%%%%%%%%%%%%%%%%% Page 6
\noindent{\large\bf MATH 242}\hfill{\large\bf Test \#4}\hfill{\large\bf
  Summer 2017}\hfill{\large\bf Page 6/6}\hrule

\bigskip
\Large Scratch paper
\end{document}
