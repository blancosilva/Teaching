%!TEX root = main.tex

% Continuous and Differentiable functions.  Symmetric matrices

\section*{Exercises}
\begin{problem}[Basic]\index{Function!continuous}
Consider the function
\begin{equation*}
f(x,y) = \dfrac{x+y}{2+\cos x}
\end{equation*}
At what points $(x,y) \in \field{R}^2$ is this function continuous?
\end{problem}

\begin{problem}[Intermediate]\index{Matrix!Symmetric}\index{Matrix!Symmetric!Positive Definite}\index{Matrix!Symmetric!Positive Semidefinite}\index{Matrix!Symmetric!Negative Definite}\index{Matrix!Symmetric!Negative Semidefinite}\index{Matrix!Symmetric!Indefinite}
Give an example of a $2 \times 2$ symmetric matrix of each kind below:
\begin{enumerate}
	\item positive definite, 
	\item positive semidefinite, 
	\item negative definite, 
	\item negative semidefinite,
	\item indefinite.
\end{enumerate}
\end{problem}

\begin{problem}[Basic]\cite[p.31, \#2]{peressini1988mathematics}\index{Matrix!Symmetric}\index{Matrix!Symmetric!Positive Definite}\index{Matrix!Symmetric!Positive Semidefinite}\index{Matrix!Symmetric!Negative Definite}\index{Matrix!Symmetric!Negative Semidefinite}\index{Matrix!Symmetric!Indefinite}
Classify the following matrices according to whether they are positive or negative definite or semidefinite or indefinite:
\begin{align*}
\mathrm{(a)} & \begin{bmatrix} 1 & 0 & 0 \\ 0 & 3 & 0 \\ 0 & 0 & 5 \end{bmatrix}  &
\mathrm{(b)} & \begin{bmatrix} -1 & 0 & 0 \\ 0 & -3 & 0 \\ 0 & 0 & -2 \end{bmatrix} &
\mathrm{(c)} & \begin{bmatrix} 7 & 0 & 0 \\ 0 & -8 & 0 \\ 0 & 0 & 5 \end{bmatrix}  \\
\mathrm{(d)} & \begin{bmatrix} 3 & 1 & 2 \\ 1 & 5 & 3 \\ 2 & 3 & 7 \end{bmatrix}  &
\mathrm{(e)} & \begin{bmatrix} -4 & 0 & 1 \\ 0 & -3 & 2 \\ 1 & 2 & -5 \end{bmatrix}  &
\mathrm{(f)} & \begin{bmatrix} 2 & -4 & 0 \\ -4 & 8 & 0 \\ 0 & 0 & -3 \end{bmatrix} 
\end{align*}
\end{problem}

\begin{problem}[Basic]\cite[p.31, \#3]{peressini1988mathematics}\index{Quadratic Form}
Write the quadratic form $\quadratic{A}(\x)$ associated with each of the following matrices $\boldsymbol{A}$:
\begin{align*}
\mathrm{(a)} & \begin{bmatrix} -1 & 2 \\ 2 & 3 \end{bmatrix} &
\mathrm{(b)} & \begin{bmatrix} 1 & -1 & 0 \\ -1 & -2 & 2 \\ 0 & 2 & 3 \end{bmatrix} \\
\mathrm{(c)} & \begin{bmatrix} 2 & -3 \\ -3 & 0 \end{bmatrix} &
\mathrm{(d)} & \begin{bmatrix} -3 & 1 & 2 \\ 1 & 2 & -1 \\ 2 & -1 & 4 \end{bmatrix} \\
\end{align*}
\end{problem}

\begin{problem}[Basic]\cite[p.32, \#4]{peressini1988mathematics}\index{Quadratic Form}
Write each of the quadratic forms below in the form $\x \boldsymbol{A} \transpose{\x}$ for an appropriate symmetric matrix $\boldsymbol{A}$:
\begin{enumerate}
\item $3x^2-xy+2y^2$.
\item $x^2+2y^2-3z^2+2xy-4xz+6yz$.
\item $2x^2-4z^2+xy-yz$.
\end{enumerate}
\end{problem}

% Coercive Functions

\begin{problem}[Intermediate]\index{Function!coercive}\index{Function!Rosenbrock}
Identify which of the following real-valued functions are coercive.  Explain the reason.
\begin{enumerate}
	\item $f(x,y) = \sqrt{x^2+y^2}$.
	\item $f(x,y) = x^2 + 9y^2 - 6xy$.
	\item $f(x,y) = x^4 - 3xy +y^4$.
	\item Rosenbrock functions $\mathcal{R}_{a,b}$.
\end{enumerate}
\end{problem}

\begin{problem}[Advanced]\cite[p.36, \#32]{peressini1988mathematics}
Find an example of a continuous, real-valued, non-coercive function $f\colon \field{R}^2 \to \field{R}$ that satisfies, for all $t \in \field{R}$,
\begin{equation*}
\lim_{x \to \infty} f(x, tx) = \lim_{y \to \infty} f(ty, y) = \infty.
\end{equation*}
\end{problem}

% Convex functions

\begin{problem}[Basic]\cite[p.77, \#1,2,7]{peressini1988mathematics}\index{Function!convex}\index{Convex!function}
Determine whether the given functions are convex, concave, strictly convex or strictly concave on the specified domains:
\begin{enumerate}
	\item $f(x) = \log(x)$ on $(0,\infty)$.
	\item $f(x) = e^{-x}$ on $\field{R}$.
	\item $f(x) = \abs{x}$ on $[-1,1]$.
	\item $f(x) = \abs{x^3}$ on $\field{R}$.
	\item $f(x,y) = 5x^2+2xy+y^2-x+2x+3$ on $\field{R}^2$.
	\item $f(x,y) = x^2/2+3y^2/2+\sqrt{3}xy$ on $\field{R}^2$.
	\item $f(x,y) = 4e^{3x-y}+5e^{x^2+y^2}$ on $\field{R}^2$.
	\item $f(x,y,z) = x^{1/2} + y^{1/3} + z^{1/5}$ on $C = \{ (x,y,z) : x>0, y>0, z>0 \}$.
\end{enumerate}
\end{problem}

\begin{problem}[Intermediate]\cite[p.79 \#11]{peressini1988mathematics}\index{Epigraph}
Sketch the epigraph of the following functions
\begin{enumerate}
	\item $f(x) = e^x$.
	\item $f(x,y)=x^2+y^2$.
\end{enumerate}
\end{problem}

% Existence

\begin{problem}[Advanced]\index{Theorem!Bounded Value}\index{Theorem!Extreme Value}
Prove the Bounded Value and Extreme Value Theorems (Theorems \ref{theorem:BVT} and \ref{theorem:EVT}).
\end{problem}

\begin{problem}[Intermediate]
For the following optimization problems, state whether existence of a solution is guaranteed:
\begin{enumerate}
	\item $f(x) = (1+x)/x$ over $[1,\infty)$
	\item $f(x) = 1/x$ over $[1,2)$
	\item The following piecewise function over $[1,2]$
	\begin{equation*}
	f(x) = \begin{cases}
	1/x, &1\leq x<2 \\
	1,   &x=2
	\end{cases}
	\end{equation*}
\end{enumerate}
\end{problem}

% Characterization

\begin{problem}[Advanced]
State and prove equivalent results to Theorems \ref{theorem:criticalGivesMinima}, \ref{theorem:necessaryMinima} and \ref{theorem:sufficientMinima} to describe necessary and sufficient conditions for the characterization of \emph{maxima}.
\end{problem}

\begin{problem}[Basic]\cite[p.32, \#7]{peressini1988mathematics}\index{Theorem!Principal Minor Criteria}
Use the \emph{Principal Minor Criteria} (Theorem \ref{theorem:PrincipalMinors}) to determine---if possible---the nature of the critical points of the following functions:
\begin{enumerate}
	\item $f(x,y) = x^3+y^3-3x-12y+20$.
	\item $f(x,y,z) = 3x^2+2y^2+2z^2+2xy+2yz+2xz$.
	\item $f(x,y,z) = x^2+y^2+z^2-4xy$.
	\item $f(x,y) = x^4+y^4-x^2-y^2+1$.
	\item $f(x,y) = 12x^3+36xy-2y^3+9y^2-72x+60y+5$.
\end{enumerate}
\end{problem}

\begin{problem}[Intermediate]\cite[p.35 \#26]{peressini1988mathematics}\label{problem:tricky}
Show that the function 
\begin{equation*}
f(x,y,z) = e^{x^2+y^2+z^2}-x^4-y^6-z^6
\end{equation*} 
has a global minimum on $\field{R}^3$.
\end{problem}

\begin{problem}[Intermediate]\cite[p.36 \#33]{peressini1988mathematics}
Consider the function
\begin{equation*}
f(x,y) = x^3 + e^{3y} -3xe^y.
\end{equation*}
Show that $f$ has exactly one critical point, and that this point is a local minimum but not a global minimum.
\end{problem}

\begin{problem}[Basic]
Let $f(x,y) = -\log(1-x-y)-\log x -\log y$.
\begin{enumerate}
	\item Find the domain $D$ of $f$.
	\item Prove that $D$ is a convex set.
	\item Prove that $f$ is strictly convex on $D$.
	\item Find the strict global minimum.
\end{enumerate}
\end{problem}

\begin{problem}[Basic]\cite[p.81 \#27]{peressini1988mathematics}
Find local and global minima in $\field{R}^3$ (if they exist) for the function 
\begin{equation*}
f(x,y) = e^{x+z-y}+e^{y-x-z}
\end{equation*}
\end{problem}

