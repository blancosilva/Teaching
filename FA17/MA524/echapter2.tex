%!TEX root = notes.tex

\section*{Exercises}
\begin{problem}[Basic]
Consider the function
\begin{equation*}
f(x,y) = \dfrac{x+y}{2+\cos x}
\end{equation*}
At what points $(x,y) \in \field{R}^2$ is this function continuous?
\end{problem}

\begin{problem}[Intermediate]
Give an examples of $2 \times 2$ symmetric matrices of each kind below:
\begin{enumerate}
	\item positive definite, 
	\item positive semidefinite, 
	\item negative definite, 
	\item negative semidefinite,
	\item indefinite.
\end{enumerate}
\end{problem}

\begin{problem}[Basic]\cite[p.31, \#2]{peressini1988mathematics}
Classify the following matrices according to whether they are positive or negative definite or semidefinite or indefinite:
\begin{align*}
\mathrm{(a)} & \begin{bmatrix} 1 & 0 & 0 \\ 0 & 3 & 0 \\ 0 & 0 & 5 \end{bmatrix}  &
\mathrm{(b)} & \begin{bmatrix} -1 & 0 & 0 \\ 0 & -3 & 0 \\ 0 & 0 & -2 \end{bmatrix} &
\mathrm{(c)} & \begin{bmatrix} 7 & 0 & 0 \\ 0 & -8 & 0 \\ 0 & 0 & 5 \end{bmatrix}  \\
\mathrm{(d)} & \begin{bmatrix} 3 & 1 & 2 \\ 1 & 5 & 3 \\ 2 & 3 & 7 \end{bmatrix}  &
\mathrm{(e)} & \begin{bmatrix} -4 & 0 & 1 \\ 0 & -3 & 2 \\ 1 & 2 & -5 \end{bmatrix}  &
\mathrm{(f)} & \begin{bmatrix} 2 & -4 & 0 \\ -4 & 8 & 0 \\ 0 & 0 & -3 \end{bmatrix} 
\end{align*}
\end{problem}

\begin{problem}[Basic]\cite[p.31, \#3]{peressini1988mathematics}
Write the quadratic form $\quadratic{A}(\x)$ associated with each of the following matrices $\boldsymbol{A}$:
\begin{align*}
\mathrm{(a)} & \begin{bmatrix} -1 & 2 \\ 2 & 3 \end{bmatrix} &
\mathrm{(b)} & \begin{bmatrix} 1 & -1 & 0 \\ -1 & -2 & 2 \\ 0 & 2 & 3 \end{bmatrix} \\
\mathrm{(c)} & \begin{bmatrix} 2 & -3 \\ -3 & 0 \end{bmatrix} &
\mathrm{(d)} & \begin{bmatrix} -3 & 1 & 2 \\ 1 & 2 & -1 \\ 2 & -1 & 4 \end{bmatrix} \\
\end{align*}
\end{problem}

\begin{problem}[Basic]\cite[p.32, \#4]{peressini1988mathematics}
Write each of the quadratic forms below in the form $\x \boldsymbol{A} \transpose{\x}$ for an appropriate symmetric matrix $\boldsymbol{A}$:
\begin{enumerate}
\item $3x^2-xy+2y^2$.
\item $x^2+2y^2-3z^2+2xy-4xz+6yz$.
\item $2x^2-4z^2+xy-yz$.
\end{enumerate}
\end{problem}

\begin{problem}[Intermediate]
Identify which of the following real-valued functions are coercive.  Explain the reason.
\begin{enumerate}
	\item $f(x,y) = \sqrt{x^2+y^2}$.
	\item $f(x,y) = x^2 + 9y^2 - 6xy$.
	\item $f(x,y) = x^4 - 3xy +y^4$.
	\item Rosenbrock functions $\mathcal{R}_{a,b}$.
\end{enumerate}
\end{problem}

\begin{problem}[Advanced]\cite[p.36, \#32]{peressini1988mathematics}
Find an example of a continuous, real-valued, non-coercive function $f\colon \field{R}^2 \to \field{R}$ that satisfies, for all $t \in \field{R}$,
\begin{equation*}
\lim_{x \to \infty} f(x, tx) = \lim_{y \to \infty} f(ty, y) = \infty
\end{equation*}
\end{problem}

\begin{problem}[Intermediate]
For the following optimization problems, state whether existence of a solution is guaranteed:
\begin{enumerate}
	\item $f(x) = \frac{1+x}{2x}$ over $[1,\infty)$
	\item $f(x) = 1/x$ over $[1,2)$
	\item The piecewise function $f(x)$ below over $[1,2]$
	\begin{equation*}
	f(x) = \begin{cases}
	1/x, &x<2 \\
	1,   &x=2
	\end{cases}
	\end{equation*}
\end{enumerate}
\end{problem}

\begin{problem}[Advanced]
State and prove equivalent results to Theorems \ref{theorem:criticalGivesMinima}, \ref{theorem:necessaryMinima} and \ref{theorem:sufficientMinima} to describe necessary and sufficient conditions for the characterization of \emph{maxima}.
\end{problem}

\begin{problem}[Intermediate]\cite[p.32, \#7]{peressini1988mathematics}
Use the \emph{Principal Minor Criteria} (Theorem \ref{theorem:PrincipalMinors}) to determine---if possible---the nature of the critical points of the following functions:
\begin{enumerate}
	\item $f(x,y) = x^3+y^3-3x-12y+20$.
	\item $f(x,y,z) = 3x^2+2y^2+2z^2+2xy+2yz+2xz$.
	\item $f(x,y,z) = x^2+y^2+z^2-4xy$.
	\item $f(x,y) = x^4+y^4-x^2-y^2+1$.
	\item $f(x,y) = 12x^3+36xy-2y^3+9y^2-72x+60y+5$.
\end{enumerate}
\end{problem}


