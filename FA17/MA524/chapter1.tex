%!TEX root = notes.tex


\chapter{Background}

Our starting point is, for any positive integer $d \in \field{N}$, the Cartesian products:
\begin{equation*}
\field{R}^d = \field{R} \times \overset{(d)}{\dotsb} \times \field{R} =\{ \x = (x_1, \dotsc, x_d) : x_k \in \field{R} \text{ for } 1\leq k \leq d\}.
\end{equation*}
We consider in this set the following two (closed) operations:
\begin{enumerate}
	\item Addition: For $\x, \y \in \field{R}^d, \x + \y = (x_1+y_1, \dotsc, x_d+y_d) \in \field{R}^d$.
	\item Scalar multiplication: For $\x \in \field{R}^d$ and $\lambda \in \field{R}$, $\lambda \cdot \x = \lambda \x = (\lambda x_1, \dotsc, \lambda x_d) \in \field{R}^d$.
\end{enumerate}
With these two operations, $\big( \field{R}^d, +, \cdot\big)$ turns into a vector space: Given $\x, \y, \z \in \field{R}^d$, $\lambda, \mu \in \field{R}$,
\begin{enumerate}
	\item The addition is commutative: $\x + \y = \y + \x$.
	\item Existence of identity elements for addition: Let $\boldsymbol{0} = (0, \dotsc, 0)$. $\x + \boldsymbol{0} = \x$. 
	\item The addition is associative: $\x + (\y + \z) = (\x + \y) + \z$.
	\item Existence of inverse elements for addition: If $\x = (x_1, \dotsc, x_d)$, the element $-\x = (-x_1, \dotsc, -x_d)$ satisfies $\x + (-\x) = \boldsymbol{0}$.  We write $\x - \y$ instead of $\x + (-\y)$.
	\item Scalar multiplication is compatible with field multiplication: $\lambda (\mu \x) = (\lambda \mu) \x$.
	\item Existence of identity for scalar multiplication: $1 \cdot \x = \x$.
	\item Scalar multiplication is distributive with respect to addition: $\lambda (\x + \y) = \lambda \x + \lambda \y$.
	\item Scalar multiplication is distributive with respect to field addition: $(\lambda + \mu)\x = \lambda \x + \mu\x$.
\end{enumerate}

A \emph{basis} of $\field{R}^d$ is a finite set $\mathcal{B}=\{ \boldsymbol{b}_k : 1\leq k \leq d \}$ satisfying two properties:
\begin{enumerate}
\item Spanning property: For all $\x \in \field{R}^d$ there exist $d$ scalars $\{ \lambda_1, \dotsc, \lambda_d \}$ so that $\x = \sum_{k=1}^d \lambda_k \boldsymbol{b}_k$.
\item Linear independence: If $\{ \lambda_1, \dotsc, \lambda_d\}$ satisfy $\sum_{k=1}^d \lambda_k \boldsymbol{b}_k = \boldsymbol{0}$, then it must be $\lambda_k=0$ for all $1\leq k \leq d$.
\end{enumerate}

\begin{problem}\label{problem:basisRd}
Define in $\field{R}^d$, for each $1\leq k \leq d$ the element $\e_k$ to be the ordered $d$-tuple with $k$-th entry equal to one, and zeros on all other entries.
\begin{enumerate}
	\item Prove that $\{ \e_k : 1\leq k \leq d\}$ is a basis for $\field{R}^d$.
	\item Set $\boldsymbol{b}_k = \e_k - \e_{k+1}$ for $1\leq k < d$, $\boldsymbol{b}_d = \e_d$.  Is $\{ \boldsymbol{b}_k : 1\leq k \leq d\}$ a basis for $\field{R}^d$?
\end{enumerate}
\end{problem}

\section{Functions}

Given sets $X, Y$, we define a \emph{function} $f\colon X \to Y$ to be a subset of $X \times Y$ subject to the following condition: for every $\x \in X$ there is exactly one element $\y \in Y$ such that the ordered pair $(\x, \y)$ is contained in the subset defining $f$. The sets $X$ and $Y$ are called respectively the \emph{domain} and \emph{codomain} of $f$.  

If $A$ is any subset of the domain $X$, then $f(A)$ is the subset of the codomain $Y$ consisting of all images of elements of $A$. We say that $f(A)$ is the \emph{image} of $A$ under $f$. The image of $f$ is given by $f(X)$.  The \emph{inverse image} of a subset $B$ of the codomain $Y$ under a function $f$ is the subset of the domain $X$ defined by $f^{-1}(B) = \{ \x \in X : f(x) \in B\}$.

For sets $X, Y, Z$, the \emph{function composition} of $f\colon X \to Y$ with $g\colon Y \to Z$ is the function $g\circ f\colon X \to Z$ defined by $\big( g \circ f \big)(\x) = g \big( f(\x) \big)$.

If $Y \subset \field{R}$, we say that the function $f$ is real-valued.  For a real-valued function $f\colon \field{R}^d \to \field{R}$, we may regard the corresponding ordered pairs $(\x, y) \in \field{R}^d\times\field{R}$ as points in a $(d+1)$--dimensional space.  We call this set the \emph{graph} of $f$.

Unless specifically stated, all functions in these notes are real-valued functions $f \colon \field{R}^d \to \field{R}$.

\begin{example}[Linear Functions]\label{example:linearFunction}
We say that a real-valued function is \emph{linear} if it preserves the operations in $\field{R}^d$: 
\begin{equation*}
f(\x + \lambda \y) = f(\x) + \lambda f(\y) \text{ for }\x, \y \in \field{R}^d, \lambda \in \field{R}.
\end{equation*}

With this definition, the function $f(x) = 3x$ is indeed a linear function, but $g(x)=3x+5$ is not!  
\end{example}

\begin{example}[Inner products]\label{example:innerprod}
We say that a function $\langle \cdot, \cdot \rangle \colon \field{R}^d \to \field{R}$ is an \emph{inner product} if it satisfies the following five properties for all $\x, \y, \z \in \field{R}^d$ and $\lambda \in \field{R}$.
\begin{enumerate}
\item $\langle \x+\y, \z \rangle = \langle \x, \z \rangle + \langle \y, \z \rangle$.
\item $\langle \lambda \x, \y \rangle = \lambda \langle \x, \y \rangle$.
\item $\langle \x, \y \rangle = \langle \y, \x \rangle$.
\item $\langle \x, \x \rangle \geq 0$.  
\item $\langle \x, \x \rangle = 0$ if and only if $\x = 0$.
\end{enumerate}
\end{example}

\begin{problem}\label{problem:innerprodRd}
Consider the real-valued function $\langle \cdot, \cdot \rangle \colon \field{R}^d \times \field{R}^d \to \field{R}$ as follows: Given $\x = (x_1, \dotsc, x_d), \y = (y_1, \dotsc, y_d) \in \field{R}^d$, 
\begin{equation*}
\langle x, y \rangle = \sum_{k=1}^d x_k y_k.
\end{equation*}
Prove this is a well-defined function:
\begin{enumerate}
	\item The domain of $\langle \cdot, \cdot \rangle$ is $\field{R}^d \times \field{R}^d$.
	\item For all $\x, \y \in \field{R}^d$, $\langle \x, \y \rangle \in \field{R}$.
	\item  If $\langle \x, \y \rangle = \lambda_1$ and $\langle \x, \y \rangle = \lambda_2$, then $\lambda_1=\lambda_2$.
\end{enumerate}
Prove that this function is an inner product.
\end{problem}

\begin{problem}\label{problem:linearFunction}
Prove that, if $f$ is a linear function in the sense of Example \ref{example:linearFunction}, then there exist a unique $\boldsymbol{a}_0 \in \field{R}^d$ so that $f(\x)=\langle \boldsymbol{a}_0, \x \rangle$ for all $\x \in \field{R}^d$.
\end{problem}

\begin{problem}\label{problem:affineFunction}
We say that $\tau\colon \field{R}^d \to \field{R}^d$ is a translation if there exist a fixed $\x_0 \in \field{R}^d$ so that $\tau(\x) = \x + \x_0$ for all $\x \in \field{R}^d$.

An \emph{affine function} $h\colon \field{R}^d \to \field{R}$ is a composition of a linear function $f\colon \field{R}^d \to \field{R}$ with a translation $\tau\colon \field{R} \to \field{R}$.

Prove that for each affine function $h$ there exist a unique $\boldsymbol{a}_0 \in \field{R}^d$ and a unique $\lambda_0 \in \field{R}$ so that $h(\x) = \lambda_0 + \langle \boldsymbol{a}_0, \x \rangle$ for all $\x \in \field{R}^d$.  Use this result to prove that the graph of an affine function is a hyperplane in $\field{R}^{d+1}$.
\end{problem}

\begin{example}[Norms]\label{example:norm}
A \emph{norm} in $\field{R}^d$ is a function $\norm{\cdot} \colon \field{R}^d \to \field{R}$ that satisfies the following properties:
For all $\x, \y \in \field{R}^d$, and for all $\lambda \in \field{R}$,
\begin{enumerate}
	\item $\norm{\x} \geq 0$.
	\item $\norm{\x} = 0$ if and only if $\x = \boldsymbol{0}$
	\item $\norm{ \lambda \x} = \abs{\lambda} \norm{\x}$.
	\item Triangle inequality: $\norm{\x + \y} \leq \norm{\x} + \norm{\y}$.
\end{enumerate}
\end{example}

\begin{problem}\label{problem:norm}
Consider the function $\norm{\cdot} \colon \field{R}^d \to \field{R}$ defined by 
\begin{equation*}
\norm{\x} = \langle \x, \x \rangle^{1/2}.
\end{equation*}
\begin{enumerate}
	\item Prove that $\norm{\cdot}$ is a norm
	\item Prove the \emph{Cauchy-Schwartz inequality}: For all $\x, \y \in \field{R}^d$, 
	\begin{equation*}
	\big\lvert \langle \x, \y \rangle \big\rvert \leq \norm{\x} \norm{\y}.
	\end{equation*}
\end{enumerate}
\end{problem}


\section{Topology}
The norm introduced in Example \ref{example:norm} induces a \emph{metric} $\dist \colon \field{R}^d \times \field{R}^d \to \field{R}^d$ on the space $\field{R}^d$: 
\begin{equation*}
\dist(\x,\y) = \norm{\x - \y} \text{ for any }\x, \y \in \field{R}^d.
\end{equation*}
Given $\x, \y, \z \in \field{R}^d$,
\begin{enumerate}
	\item Separation property: $\dist(\x,\y) \geq 0$.
	\item Identity of indiscernibles: $\dist(\x, \y) = 0$ if and only if $\x = \y$.
	\item Symmetry: $\dist(\x, \y) = \dist(\y, \x)$.
	\item Triangle inequality: $\dist(\x,\z) \leq \dist(\x,\y) + \dist(\y,\z)$. 
\end{enumerate}
We say then that $\big( \field{R}^d, \dist(\cdot,\cdot) \big)$ is a \emph{metric space}.  Metric spaces inherit a \emph{topology} in a natural manner, as explained below.

We define the \emph{open ball} of radius $r>0$ about $\x$ as the set $B_d(\x,r) = \big\{ \y \in \field{R}^d : \norm{\x- \y} < r \big\}$.  We say $\x$ is an interior point of $D \subset \field{R}^d$ if $\x \in D$ and there exists $r>0$ so that $B_d(\x, r) \subset D$.  A subset $G \subset \field{R}^d$ is said to be open if all its points are interior.

A \emph{neighborhood} of the point $\x$ is any subset of $\field{R}^d$ that contains an open ball about $\x$ as subset. 

A \emph{sequence} $\sequence{\x}{n}$ in $\field{R}^d$ is an enumerated collection of elements of $\field{R}^d$ in which repetitions are allowed.  A sequence is said to \emph{converge} to the limit $\x \in \field{R}^d$ if and only if for every $\varepsilon>0$ there exists $N=N(\varepsilon) \in \field{N}$ so that $\norm{\x_n - \x} < \varepsilon$ for all $n \geq N$.  We write then 
\begin{equation*}
\x = \lim_n \x_n, \text{ or } \lim_n \norm{\x_n - \x} = 0.
\end{equation*}

We say that a sequence $\sequence{\x}{n}$ is a \emph{Cauchy sequence} if for every $\varepsilon>0$ there exists $N = N(\varepsilon) \in \field{N}$ so that for any $m,n \geq N$, $\norm{\x_n - \x_m} < \varepsilon$.  In $\field{R}^d$, all Cauchy sequences converge (this is direct consequence of the completeness of $\field{R}$).

The complement of an open set is called \emph{closed}. In $\field{R}^d$, all subsets $F$ are closed if and only if they are \emph{sequentially closed}: If $\x_n\in F$ for all $n \in \field{N}$ and $\lim_n \norm{\x_n - \x} = 0$, then $\x \in F$.

We say $D$ is \emph{bounded} if there exists $M>0$ so that $D \subset B_d(\boldsymbol{0}, M)$.  A bounded and closed subset of $\field{R}^d$ is called \emph{compact}.

\begin{theorem}[Bolzano-Weierstrass]
Every sequence in a compact subset $K \subset \field{R}^d$ contains a convergent subsequence.
\end{theorem}

\section{Analysis}

A real-valued function $f\colon \field{R}^d \to \field{R}$ is continuous at $\x_0$ if for any $\varepsilon>0$ there exists $\delta = \delta(\varepsilon)>0$ so that $\abs{f(\x) - f(\x_0)} < \varepsilon$ for all $x \in B_d(\x_0, \delta)$.

Equivalently, $f\colon \field{R}^d \to \field{R}$ is continuous at $\x_0$ if $\lim_n f(\x_n) = f(\x_0)$ for any sequence $\sequence{\x}{n}$ satisfying $\lim_n \x_n = \x_0$.  

We say that $f$ is continuous in $D \subset \field{R}^d$ if $f$ is continuous at all points $\x \in D$.

A real-valued function $f\colon \field{R}^d \to \field{R}$



Given a set $D \subset \mathbb{R}^d$, and a real-valued function $f\colon D \to \field{R}$, we say that a point $\xstar \in D$ is:
\begin{enumerate}
	\item A \emph{global minimum} for $f$ on $D$ if $f(\xstar) \leq f(\x)$ for all $\x \in D$.
	\item A \emph{strict global minimum} for $f$ on $D$ if $f(\xstar) < f(\x)$ for all $\x \in D \setminus \{ \xstar \}$.
	\item A \emph{local minimum} for $f$ on $D$ if there exists $\delta>0$ so that  $f(\xstar) \leq f(\x)$ for all $\x \in B_\delta(\xstar)\cap D$.
	\item A \emph{local minimum} for $f$ on $D$ if there exists $\delta>0$ so that  $f(\xstar) < f(\x)$ for all $\x \in B_\delta(\xstar)\cap D$, $\x \neq \xstar$.
\end{enumerate}

A real-valued function $f\colon \field{R}^d \to \field{R}$ is \emph{continuous} at a point $\x$ 



