%!TEX root = main.tex

\section{Sufficient Conditions}
To explore sufficient conditions, we find useful to study generalizations of convex functions: \emph{quasi-convexity} and \emph{pseudo-convexity}.

\begin{definition}[Quasi-convex functions]\index{Function!quasi-convex}\index{Function!quasi-concave}
Given a convex set $C \subseteq \field{R}^d$, we say that a real-valued function $f\colon C \to \field{R}$ is \emph{quasi-convex} if for all $\x,\y \in C$ and all $0 \leq \lambda \leq 1$,
\begin{equation*}
f\big( \lambda \y + (1-\lambda)\x \big) \leq \max\{ f(\x), f(\y) \}.
\end{equation*}
We say that $f$ is \emph{quasi-concave} if 
\begin{equation*}
f\big( \lambda \y + (1-\lambda)\x \big) \geq \min\{ f(\x), f(\y) \}.
\end{equation*}
\end{definition}

\begin{definition}[Pseudo-convex functions]\index{Function!pseudo-convex}
Given a convex set $C \subseteq \field{R}^d$, we say that a real-valued differentiable function $f \colon C \to \field{R}$ is \emph{pseudo-convex} if for all $\x, \y \in C$ satisfying $\langle \gradient{f}(\x), \x-\y \rangle \geq 0$, it must be $f(\y) \geq f(\x)$
\end{definition}

\begin{remark}
The three functions are related as follows:
\begin{itemize}
	\item Differentiable convex functions are pseudo-convex.
	\item Convex functions are quasi-convex.
	\item Pseudo-convex functions are quasi-convex.
\end{itemize}
\end{remark}

\separator 

Quasi-convex functions have a very nice characterization by means of their \emph{level sets}\index{Level set}:

\begin{theorem}
Given a convex set $C \subseteq \field{R}^d$, a real-valued function $f \colon C \to \field{R}$ is quasi-convex if and only if its level sets 
$\Lambda_t(f) = \{ \x \in C : f(x) \leq t \}$ are convex for all $t \in \field{R}$.
\end{theorem}

We are now ready to state the main result on this section:

\begin{theorem}[KKT Sufficient Conditions]\label{theorem:KKTsufficient}\index{Theorem!KKT sufficient conditions}\index{Theorem!Karush-Kuhn-Tucker}
Let $\x \in S$ be a feasible point of the consistent program $(P)$ for which there are multipliers $u_k \geq 0$ ($1\leq k \leq m)$ and $v_k \in \field{R}$ ($1\leq k \leq \ell$) satisfying the conditions \ref{item:KKTnecessary1}, \ref{item:KKTnecessary2} and \ref{item:KKTnecessary3} of Theorem \ref{theorem:KKTnecessary}. If $f$ is pseudo-convex, $g_k$ is quasi-convex for all $1\leq k \leq m$, and $h_k$ is linear for all $1\leq k \leq \ell$, then $\x$ is a global optimal solution of $(P)$.
\end{theorem}

\begin{remark}\index{KKT conditions}
The conditions \ref{item:KKTnecessary1}, \ref{item:KKTnecessary2} and \ref{item:KKTnecessary3} of Theorem \ref{theorem:KKTnecessary} are called \emph{the KKT conditions} of the program $(P)$ in the literature.
\end{remark}

\begin{example}
We saw that the point $(0,0)$ satisfies the KKT conditions for the super-consistent convex program $(P)$ in Example \ref{example:feasibleP1}.  As a consequence of Theorems \ref{theorem:KKTnecessary} and \ref{theorem:KKTsufficient}, this point must be the optimal global minimum of $(P)$.
\end{example}