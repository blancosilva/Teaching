%!TEX root = main.tex

\section{Linear Programing: The simplex method}\index{Simplex method}\index{Linear programming}

A general linear program $(LP)$ is usually expressed in the following standard form:
\begin{equation*}
(LP): \begin{cases}
\displaystyle{\max_{\x \in \field{R}^d} \langle \boldsymbol{c}, \x \rangle} \\
\langle \boldsymbol{a}_j, \x \rangle = b_j &(1 \leq k \leq \ell) \\
x_k \geq 0 &(1 \leq k \leq d) 
\end{cases}
\end{equation*}
for a given $\boldsymbol{c}=[c_1, \dotsc, c_d], \boldsymbol{a}_k = [a_{k1}, \dots, a_{kd} ] \in \field{R}^d$, $b_k \in \field{R}$ for $1 \leq k \leq \ell$.  We accomplish this by performing the following operations, where necessary:
\begin{description}
	\item[Objective function] If the original program requests $\min f(\x)$, convert it to $\max \big( -f(\x) \big)$.
	\item [Slack variables] If the original program contains an inequality constraint of the form $\langle \boldsymbol{a}, \x \rangle \leq b$ with $\boldsymbol{a} = [a_1, \dotsc, a_d]$, convert it to an equality constraint by adding a non-negative \emph{slack variable}\index{Simplex method!slack variable} $s$. The resulting constraint is 
	\begin{equation*}
	a_1 x_1 + \dotsb + a_d x_d + s = b, \quad s \geq 0.
	\end{equation*}
	\item [Surplus variables] If the original program contains an inequality constraint of the form $\langle \boldsymbol{a}, \x \rangle \geq b$, convert it to an equality constraint by subtracting a non-negative \emph{surplus variable}\index{Simplex method!surplus variable} $s$.  The resulting constraint is
	\begin{equation*}
	a_1 x_1 + \dotsb + a_d x_d - s = b, \quad s \geq 0.
	\end{equation*}
	\item[Unrestricted variables in sign]\index{Simplex method!Unrestricted variables in sign} If some variable $x_k$ is unrestricted in sign, replace it everywhere in the formulation by $x_k^+ - x_k^-$, where $x_k^+ \geq 0$ and $x_k^- \geq 0$.
\end{description}

\separator 

Once in standard form, we usually represent a linear program by its corresponding \emph{tableau}:\index{Simplex method!tableau}\index{tableau}
\begin{equation*}
\begin{bmatrix} 1 & -\boldsymbol{c} & 0 \\ \transpose{\boldsymbol{0}} & \boldsymbol{A} & \transpose{\boldsymbol{b}} \end{bmatrix},
\end{equation*}
where
\begin{equation*}
\boldsymbol{A} = \begin{bmatrix} a_{11} & \dotsb & a_{1d} \\ \vdots & \ddots & \vdots \\ a_{\ell 1} & \dotsb & a_{\ell d} \end{bmatrix}
\end{equation*}
and $\boldsymbol{b} = [b_1, \dotsc, b_\ell]$.

\begin{example}\label{example:FirstLinearProgram}
Consider the linear program $(LP)$ given by the no-standard formulation
\begin{equation*}
(LP): \begin{cases}
\displaystyle{\min_{(x,y,z) \in \field{R}^3} \big( 3y-2x \big)} \\
x - 3y + 2z \leq 3, \\
2y-x \geq 2, \\
y \geq 0, z \geq 0 \\
\end{cases}
\end{equation*}
We can easily convert the objective function to a maximum, and the first two inequality constraints into equality constraints by introducing a slack variable $s_1 \geq 0$ and a surplus variable $s_2 \geq 0$.  Notice that the variable $x$ is unrestricted in sign.  We convert it to two non-negative variables $x^+ - x^-$:
\begin{equation*}
\left\{
\begin{matrix}
\displaystyle{\max_{(x,y,z) \in \field{R}^3}} & \big( 2x^+ & -2x^- &-3y \big) \\
& x^+ & -x_- & -3y & +2z & +s_1 & & = 3, \\
& -x^+ & +x^- & +2y & & &-s_2 &= 2, \\
&x^+ \geq 0, &x^- \geq 0, &y \geq 0, &z \geq 0, &s_1 \geq 0, &s_2 \geq 0 
\end{matrix}
\right.
\end{equation*}
The corresponding tableau is as follows:
\begin{equation*}
\begin{tabular}{|r|rrrrrr|r|}
\hline 
1 & -2 & 2  & 3  & 0 & 0 & 0  & 0 \\ \hline
0 & 1  & -1 & -3 & 2 & 1 & 0  & 3 \\
0 & -1 &  1 & 2  & 0 & 0 & -1 & 2 \\ \hline \hline
$z$ & $x^+$ & $x^-$ & $y$ & $z$ & $s_1$ & $s_2$ & \\ \hline
\end{tabular}
\end{equation*}
\end{example}

\separator

The \emph{Simplex method} to find the optimal solution of a linear program in standard form is based on the following two rules:
\begin{description}
	\item[Rule 1] If all variables have a nonnegative coefficient on the first row, the current basic solution (the last column) is optimal.  Otherwise, pick a variable $x_k$ with a negative coefficient $(-c_k)$ in the first row---the \emph{entering variable}\index{Simplex method!entering variable}---and \emph{pivot} it with another row.

	\item[Rule 2] The selection of row to perform the pivot for the entering variable $x_k$ is performed by choosing among the rows $j$ for which $a_{jk} > 0$, the one with the minimum ratio $b_j/a_{jk}$.
\end{description}

\begin{example}
Let's illustrate this process with the following linear program
\begin{equation*}
(LP): \begin{cases}
\displaystyle{\max_{(x,y) \in \field{R}^2} \big( x+y \big)} \\
2x+y \leq 4 \\
x+2y \leq 3 \\
x \geq 0, y \geq 0
\end{cases}
\end{equation*}
We start by converting to standard.  For ease of computations below, we first rename $x=x_1$ and $y=x_2$.  We introduce the slack variables $x_3$ and $x_4$ as they are needed.  We finish the preparation step by finding the tableau of this program.
\begin{equation*}
(LP): \left\{
\begin{matrix}
\displaystyle{\max_{(x,y) \in \field{R}^2}} &\big( x_1   & +x_2 \big) \\
                                            &2x_1        & +x_2        & +x_3        &          &= 4 \\
                                            &x_1         & +2x_2       &             & +x_4     &= 3 \\
                                            &x_1 \geq 0, & x_2 \geq 0, & x_3 \geq 0, &x_4 \geq 0
\end{matrix}
\right.
\end{equation*}
\begin{equation*}
\begin{bmatrix}
1 & \boldsymbol{-1}  & -1  & 0   & 0   & 0\\
0 & 2                & 1   & 1   & 0   & 4 \\
0 & 1                & 2   & 0   & 1   & 3 \\ \hline
z & x_1              & x_2 & x_3 & x_4 
\end{bmatrix}
\end{equation*}
At this stage, we have the following initial situation:
\begin{equation*}
z=0 \qquad \underbrace{x_3=4, \quad x_4=3}_{\text{basic solutions}} \qquad \underbrace{x_1=x_2=0}_{\text{non-basic solutions}}
\end{equation*}
The first entering variable is $x_1$.  We have two choices to pivot.  Rule 2 indicates that we must use the second row, since for this row, we have the ratio $4/2 = 2$, while the third row offers a bigger ratio: $3/1=3$.
\begin{equation*}
\begin{bmatrix}
1 & -1  & -1  & 0   & 0   & 0\\
0 & 1   & 1/2 & 1/2 & 0   & 2 \\
0 & 1   & 2   & 0   & 1   & 3 \\ \hline
z & x_1 & x_2 & x_3 & x_4 
\end{bmatrix} \to 
\begin{bmatrix}
1 & 0   & \boldsymbol{-1/2} & 1/2  & 0   & 2\\
0 & 1   & 1/2               & 1/2  & 0   & 2 \\
0 & 0   & 3/2               & -1/2 & 1   & 1 \\ \hline
z & x_1 & x_2               & x_3 & x_4 
\end{bmatrix}
\end{equation*}
At this stage, we have the following situation:
\begin{equation*}
z = 2 \qquad \underbrace{x_1 = 2, \quad x_4 =1}_{\text{basic solution}} \qquad \underbrace{x_2=x_3=0}_{\text{non-basic solution}}
\end{equation*}
The second entering variable is $x_2$.  We again have two choices to pivot.  Rule 2 indicates that we must use the third row, since for this row the ratio is $1/(3/2)=2/3$.  For the second row, the ratio is larger: $2/(1/2)=4$.
\begin{equation*}
\begin{bmatrix}
1 & 0   & -1/2 & 1/2  & 0   & 2\\
0 & 1   & 1/2  & 1/2  & 0   & 2 \\
0 & 0   & 1    & -1/3 & 2/3 & 2/3 \\ \hline
z & x_1 & x_2  & x_3  & x_4 
\end{bmatrix} \to 
\begin{bmatrix}
1 & 0   & 0   & 1/3  & 1/3  & 7/3 \\
0 & 1   & 0   & 2/3  & -1/3 & 5/3 \\
0 & 0   & 1   & -1/3 & 2/3  & 2/3 \\ \hline
z & x_1 & x_2 & x_3  & x_4 
\end{bmatrix}
\end{equation*}
There are no more negative coefficients on the first row.  This leads to an solution of the tableau given by $z=7/3$, $x_1=5/3$. $x_2=2/3$, $x_3=x_4=0$.  The global maximum of the function $f(x,y)=x+y$ on the set $S = \{ (x,y) \in \field{R}^2: x\geq 0, y \geq 0, 2x+y \leq 4, x+2y \leq 3 \}$ is attained at the point $(5/3, 2/3)$.  The corresponding maximum value is thus $7/3$.  An illustration of the three steps carried in these computations can be observed in Figure \ref{figure:simplexmethod}.
\begin{figure}[ht!]
\includegraphics[width=0.75\linewidth]{images/simplex.png}
\caption{Illustration of the simplex method for Example \ref{example:FirstLinearProgram}}
\label{figure:simplexmethod}
\end{figure}
\end{example}

\begin{example}
What happens if the program does not have a unique solution?  Can the simplex method offer this information? Consider for $f(x,y)=x+\tfrac{1}{2}y$ the program
\begin{equation*}
(LP): \begin{cases} 
\displaystyle{\max_{(x,y)\in\field{R}^s} f(x,y)} \\
2x+y \leq 4 \\
x+2y \leq 3 \\
x \geq 0, y \geq 0
\end{cases}
\end{equation*}
Once in standard form, this program has the tableau
\begin{equation*}
\begin{bmatrix} 
1 & -1  & -1/2 & 0   & 0   & 0 \\
0 & 2   & 1    & 1   & 0   & 4 \\
0 & 1   & 2    & 0   & 1   & 3 \\ \hline
z & x_1 & x_2  & x_3 & x_4
\end{bmatrix}
\end{equation*}
The initial solution gives 
\begin{equation*}
z=0 \qquad \underbrace{x_3=4, \quad x_4=3}_{\text{basic solutions}} \qquad \underbrace{x_1=x_2=0}_{\text{non-basic solutions}}
\end{equation*}
There is an entering variable at $x_1$, that has to be pivoted with the second row:
\begin{equation*}
\begin{bmatrix} 
1 & -1  & -1/2 & 0   & 0   & 0 \\
0 & 1   & 1/2  & 1/2 & 0   & 2 \\
0 & 1   & 2    & 0   & 1   & 3 \\ \hline
z & x_1 & x_2  & x_3 & x_4
\end{bmatrix} \to 
\begin{bmatrix} 
1 & 0   & 0    & 1/2  & 0   & 2 \\
0 & 1   & 1/2  & 1/2  & 0   & 2 \\
0 & 0   & 3/2  & -1/2 & 1   & 1 \\ \hline
z & x_1 & x_2  & x_3  & x_4
\end{bmatrix}
\end{equation*}
The solution at this stage---which already offers an optimal solution of the program $(LP)$---gives
\begin{equation*}
z=2 \qquad \underbrace{x_1=2, \quad x_4=1}_{\text{basic solutions}} \qquad \underbrace{x_2=x_3=0}_{\text{non-basic solutions}}
\end{equation*}
Notice now that at this point we could increase the value of the coefficient of the variable $x_2$ without changing the value of $z$.
\begin{equation*}
\begin{bmatrix} 
1 & 0   & 0    & 1/2  & 0   & 2 \\
0 & 1   & 1/2  & 1/2  & 0   & 2 \\
0 & 0   & 1    & -1/3 & 2/3 & 2/3 \\ \hline
z & x_1 & x_2  & x_3  & x_4
\end{bmatrix} \to 
\begin{bmatrix} 
1 & 0   & 0    & 1/2  & 0    & 2 \\
0 & 1   & 0    & 2/3  & -1/3 & 5/3 \\
0 & 0   & 1    & -1/3 & 2/3  & 2/3 \\ \hline
z & x_1 & x_2  & x_3  & x_4
\end{bmatrix}
\end{equation*}
The solution at this stage gives
\begin{equation*}
z=2 \qquad \underbrace{x_1=5/3, \quad x_2=2/3}_{\text{basic solutions}} \qquad \underbrace{x_3=x_4=0}_{\text{non-basic solutions}}
\end{equation*}
We have found two different optimal solutions of the program $(LP)$ using the simplex method: $(2,0)$ and $(5/3, 2/3)$.  Notice that in this case, any other point in the segment joining those two points, must also be a solution.  Namely: for any $t \in [0,1]$, the point $\big(2-\tfrac{1}{3}t, \tfrac{2}{3}t \big)$ satisfies
\begin{align*}
&f\big(2-\tfrac{1}{3}t, \tfrac{2}{3}t \big) = 2 - \tfrac{1}{3}t + 2\tfrac{1}{3}t = 2. &&\text{(The value is always 4)} \\
&2\big( 2-\tfrac{1}{3}t \big) + \tfrac{2}{3}t = 4 &&(\text{The first constraint is satisfied}) \\
&2-\tfrac{1}{3}t + 2 \tfrac{2}{3}t = 2 + t \leq 3 &&(\text{The second constraint is satisfied}) \\
&2 - \tfrac{1}{3}t \geq \tfrac{5}{3} > 0 &&(\text{The third constraint is satisfied}) \\
&\tfrac{2}{3}t \geq 0 &&  (\text{The fourth constraint is satisfied})
\end{align*}
\end{example}


\begin{example}
What happens if we are unable to employ Rule 1 from the simplex method?  This situation arises on unbounded programs.  Consider the one below:
\begin{equation*}
(LP): \begin{cases}
\displaystyle{\max_{(x,y) \in \field{R}^2} \big( 2x+y \big)} \\
-x+y \leq 1 \\
x - 2y \leq 2 \\
x \geq 0, y \geq 0
\end{cases}
\end{equation*}
Once in standard form, its tableau is as follows:
\begin{equation*}
\begin{bmatrix}
1 & -2  & -1  & 0   & 0   & 0 \\
0 & -1  &  1  & 1   & 0   & 1 \\
0 &  1  & -2  & 0   & 1   & 2 \\ \hline
z & x_1 & x_2 & x_3 & x_4
\end{bmatrix}
\end{equation*}
At this stage, an initial solution is given by
\begin{equation*}
z=0 \qquad \underbrace{x_3=1, \quad x_4=2}_{\text{basic solutions}} \qquad \underbrace{x_1=x_2=0}_{\text{non-basic solutions}}
\end{equation*}
The entering variable $x_1$ must be pivoted with the third row.
\begin{equation*}
\begin{bmatrix}
1 &  0  & -5  & 0   & 2   & 4 \\
0 &  0  & -1  & 1   & 1   & 3 \\
0 &  1  & -2  & 0   & 1   & 2 \\ \hline
z & x_1 & x_2 & x_3 & x_4
\end{bmatrix}
\end{equation*}
At this stage, we have
\begin{equation*}
z=4 \qquad \underbrace{x_1=2, \quad x_3=3}_{\text{basic solution}} \qquad \underbrace{x_2=x_4=0}_{\text{non-basic solutions}}
\end{equation*}
But notice that at this new stage we are unable to apply rule 1, since there are no positive coefficients for $x_2$. Any pivot operation that we apply using the second row will change the values of $z$; in particular, we should be able to perform enough changes to make $z$ as large as we desire.  For instance: what row operations would you perform to get $z=19$, with the feasible point $(8,3)$?
\end{example}

\begin{example}\index{scipy}
In Python there is an implementation of the simplex algorithm in the libraries \texttt{sympy.optimize}: the routine \texttt{linprog} with the option \texttt{method='simplex'}.  It is smart enough to indicate, among other things,
\begin{itemize}
	\item If the program terminates successfully, is infeasible or unbounded.
	\item The value of the slack or surplus variables, when these are used.
	\item In case of failure, the coordinates of the last point obtained by the algorithm.
	\item The different tableau computed on each stage.
	\item The index of the tableau selected as pivot on each stage.
\end{itemize} 
The following session illustrates how to use it to solve some of the examples we have shown in this section.

\begin{minted}[frame=single, fontsize=\footnotesize, linenos, mathescape]{python}
import numpy as np, matplotlib.pyplot as plt 
from scipy.optimize import linprog, lingrog_verbose_callback

# First program: $\max(x+y)$ with $2x+y \leq 4, x+2y \leq 3, x \geq 0, y \geq 0$
# We obtained the value $7/3$ at the point $(5/3,2/3)$
c1 = [-1, -1]
A1 = [[2,1], [1,2]]
b1 = [4,3]
x0_bnds = (0, None) # This means literally $0 \leq x_0 < \infty$
x1_bnds = (0, None) # and this, $0 \leq x_1 < \infty$

# Second program: $\max(2x+y)$ with $-x+y \leq 1, x-2y\leq 2, x\geq 0, y\geq 0$
# The program is unbounded
c2 = [-2, -1]
A2 = [[-1, 1], [1, -2]]
b2 = [1, 2]
\end{minted}

We call the optimization \texttt{linprog} with the values of $\boldsymbol{c}$ first (as a list), then the values of the matrix $\boldsymbol{A}$ (as a list of lists), and the values of $\boldsymbol{b}$ (as a list).  We use the option \texttt{bounds=} to provide a collection of bounds \texttt{(min\_value, max\_value)} for each of the relevant variables.  For instance if we request the first variable, $x_0$ to satisfy $a \leq x_0 \leq b$, we input \texttt{(a,b)}.  If any of the bounds is infinite, we signal it with \texttt{None}.  A simple output looks like these.

\begin{minted}[frame=single,fontsize=\footnotesize, mathescape]{python}
>>> linprog(c1, A1, b1, bounds=(x0_bnds, x1_bnds), method='simplex')
     fun: -2.3333333333333335
 message: 'Optimization terminated successfully.'
     nit: 2
   slack: array([ 0.,  0.])
  status: 0
 success: True
       x: array([ 1.66666667,  0.66666667])

 >>> linprog(c2, A2, b2, bounds=(x0_bnds, x1_bnds), method='simplex')
      fun: -4.0
 message: 'Optimization failed. The problem appears to be unbounded.'
     nit: 1
   slack: array([ 3.,  0.])
  status: 3
 success: False
       x: array([ 2.,  0.])

\end{minted}
\end{example}

For extra information, we issue the option \texttt{callback=} with a user-defined callback function, or the default callback \texttt{linprog\_verbose\_callback} already defined in \texttt{scipy.optimize}
\begin{minted}[frame=single,fontsize=\footnotesize, mathescape]{python}
>>> linprog(c1, A1, b1, bounds=(x0_bounds, x1_bounds), \
...         method='simplex', callback=linprog_verbose_callback)
--------- Initial Tableau - Phase 1 ----------

[[      2.0000       1.0000       1.0000       0.0000       4.0000]
 [      1.0000       2.0000       0.0000       1.0000       3.0000]
 [     -1.0000      -1.0000       0.0000       0.0000       0.0000]
 [      0.0000       0.0000       0.0000       0.0000       0.0000]]

Pivot Element: T[nan, nan]

Basic Variables: [2 3]

Current Solution:
x =  [      0.0000       0.0000]

Current Objective Value:
f =  -0.0

--------- Initial Tableau - Phase 2 ----------

[[      2.0000       1.0000       1.0000       0.0000       4.0000]
 [      1.0000       2.0000       0.0000       1.0000       3.0000]
 [     -1.0000      -1.0000       0.0000       0.0000       0.0000]]

Pivot Element: T[0, 0]

Basic Variables: [2 3]

Current Solution:
x =  [      0.0000       0.0000]

Current Objective Value:
f =  -0.0

--------- Iteration 1  - Phase 2 --------

Tableau:
[[      1.0000       0.5000       0.5000       0.0000       2.0000]
 [      0.0000       1.5000      -0.5000       1.0000       1.0000]
 [      0.0000      -0.5000       0.5000       0.0000       2.0000]]

Pivot Element: T[1, 1]

Basic Variables: [0 3]

Current Solution:
x =  [      2.0000       0.0000]

Current Objective Value:
f =  -2.0

--------- Iteration Complete - Phase 2 -------

Tableau:
[[      1.0000       0.0000       0.6667      -0.3333       1.6667]
 [      0.0000       1.0000      -0.3333       0.6667       0.6667]
 [      0.0000       0.0000       0.3333       0.3333       2.3333]]

Basic Variables: [0 1]

Current Solution:
x =  [      1.6667       0.6667]

Current Objective Value:
f =  -2.33333333333

\end{minted}
