%!TEX root = main.tex

\section*{Exercises}

\begin{problem}[Basic]
Use a projection method to find the global minimum of the function $f(x,y,z)=x^2+2y^2+3z^2$ over the plane $x+y+z=1$.
\end{problem}

\begin{problem}[Basic]
Use a projection method to find the global minimum of the function $f(x,y,z,t)=x^2+2y^2+2z^2+t^2$ over the plane 
\begin{equation*}
S = \{ (x,y,z,t) \in \field{R}^4 : x+y+z=t, x+y=4 \}
\end{equation*}
\end{problem}

\begin{problem}[Basic]
Solve the following linear program by the simplex method:
\begin{equation*}
(LP): \begin{cases}
\displaystyle{\max_{(x,y,z) \in \field{R}^3} \big( 4x+y-z \big)} \\
x+3z \leq 6 \\
3x+y+3z \leq 9 \\
x \geq 0, y\geq 0, z\geq 0
\end{cases}
\end{equation*}
\end{problem}

\begin{problem}[Basic]
The following tableaux were obtained in the course of solving linear programs with two non-negative variables $x_1$ and $x_2$, two inequality constraints for which slack or surplus variables $x_3$ and $x_4$ were needed.  In each case, indicate whether the corresponding linear program has a unique optimum solution, has several optimum solutions (and in that case find them all), it is unbounded, or degenerate.
\begin{enumerate}
\item $\displaystyle{%
\begin{bmatrix}
1 &   0 &   3 &   2 &   0 & 20 \\
0 &   1 &  -2 &  -1 &   0 &  4 \\
0 &   0 &  -1 &   0 &   1 &  2 \\ \hline
z & x_1 & x_2 & x_3 & x_4
\end{bmatrix}
}$\smallskip
\item $\displaystyle{%
\begin{bmatrix}
1 &   0 &  -1 &   0 &   2 & 20 \\
0 &   0 &   0 &   1 &  -2 &  5 \\
0 &   1 &  -2 &   0 &   3 &  6 \\ \hline
z & x_1 & x_2 & x_3 & x_4
\end{bmatrix}
}$\smallskip
\item $\displaystyle{%
\begin{bmatrix}
1 &   2 &   0 &   0 &   1 &  8 \\
0 &   3 &   1 &   0 &  -2 &  4 \\
0 &  -2 &   0 &   1 &   1 &  0 \\ \hline
z & x_1 & x_2 & x_3 & x_4
\end{bmatrix}
}$\smallskip
\item $\displaystyle{%
\begin{bmatrix}
1 &   0 &   0 &   2 &   0 &  5 \\
0 &   0 &  -1 &   1 &   1 &  4 \\
0 &   1 &   1 &  -1 &   0 &  4 \\ \hline
z & x_1 & x_2 & x_3 & x_4
\end{bmatrix}
}$
\end{enumerate}
\end{problem}

\begin{problem}[Intermediate]
Consider the following linear program
\begin{equation*}
(LP): \begin{cases}
\displaystyle{\max_{(x,y,z) \in \field{R}^3} \big( 5x+3y+z \big)} \\
x+y+z \leq 6 \\
5x+3y+6z \leq 15 \\
x \geq 0, y \geq 0, z \geq 0
\end{cases}
\end{equation*}
Assume the following is an associated tableau:
\begin{equation*}
\begin{bmatrix}
1 &   0 &   0 &    5 &   0 &   1  & 15 \\
0 &   0 & 0.4 & -0.2 &   1 & -0.2 &  3 \\
0 &   1 & 0.6 &  1.2 &   0 &  0.2 &  3 \\ \hline
z & x_1 & x_2 &  x_3 & x_4 &  x_5  
\end{bmatrix}
\end{equation*}
\begin{enumerate}
\item What basic solution does this tableau represent? Is this solution optimal? Explain why or why not.
\item Does this tableau represent a unique optimal solution? If not, find at least three alternative optimal solutions.
\end{enumerate}
\end{problem}

\begin{problem}[CAS]
In a computer language or CAS of your choice, design a routine that gathers the following as input:
\begin{itemize}
	\item A $d$-dimensional vector $\boldsymbol{c}=[c_1, \dotsc, c_d]$ representing a linear function $f(\x) = \langle \boldsymbol{c}, \x \rangle$.
	\item A matrix $\boldsymbol{A}$ with dimension $\ell \times d$, and an $\ell$-dimensional vector $\boldsymbol{b} = [b_1, \dotsc, b_\ell]$.  These two objects represent $\ell$ linear inequality constraints of the form $\boldsymbol{A}\transpose{\x} \leq \transpose{\boldsymbol{b}}$. 
\end{itemize}
The routine attempts to find a solution to the following linear program:
\begin{equation*}  
(LP): \begin{cases} \displaystyle{\min_{\x \in S} \langle \boldsymbol{c}, \x \rangle} \\ S = \{ \x \in \field{R}^d : \boldsymbol{A}\transpose{\x} \leq \transpose{\boldsymbol{b}}, x_k \geq 0, 1 \leq k \leq d \} \end{cases}
\end{equation*}
If a solution is found, the code outputs it.  If the program is unbounded, the code indicates so.  
\end{problem}

\begin{problem}[Basic]
Use the Frank-Wolfe method to find the minimum value of the function $f(x,y)=(x-1)^2+(y+5)^2$ over the square with vertices at $(2,2)$, $(3,3)$, $(4,2)$ and $(3,1)$. Use the center of the square as initial guess.  Illustrate graphically each step of the method.
\end{problem}
