%!TEX root = main.tex

\section*{Exercises}

\begin{problem}[Basic]
Consider the following problem: Find the global minimum of the function $f(x,y) = 6(x-10)^2+4(y-12.5)^2$ on the set $S = \{ (x,y) \in \field{R}^2 : x^2+(y-5)^2 \leq 50, x^2+3y^2\leq 200, (x-6)^2+y^2 \leq 37 \}$.
\begin{enumerate}
	\item Write the statement of this problem as a program with the notation from equation \ref{equation:programP}.  Label the objective function, as well as the inequality constraints accordingly.
	\item Is the objective function $f$ pseudo-convex? Why or why not?
	\item Are the inequality constraints quasi-convex?  Why or why not?
	\item Sketch the feasibility region.  Label all relevant objects involved.
	\item Is the point $(7,6)$ feasible?  Why or why not?
	\item Employ Theorem \ref{theorem:KKTnecessary} to write a necessary condition for optimality and verify that is satisfied by the point $(7,6)$.
	\item Employ Theorem \ref{theorem:KKTsufficient} to decide whether this point is an optimal solution of $(P)$.
\end{enumerate}
\end{problem}
% The point $(7,6)$ is feasible.  To see this, set $g_1(x,y)=x^2+(y-5)^2-50$, $g_2(x,y)=x^2+3y^2-200$, $g_3(x,y)=(x-6)^2+y^2-37$, and notice that
% \begin{equation*}
% g_1(7,6) = 0, g_2(7,6) = -43, g_3(7,6) = 0.
% \end{equation*}
% We need to check if the gradients line up nicely for this point:
% \begin{align*}
% \gradient{f}(x,y)   &= [ 12(x-10), 8(y-12.5) ] &\gradient{f}(7,6)   &= [-36, -52], \\
% \gradient{g_1}(x,y) &= [ 2x, 2(y-5) ]          &\gradient{g_1}(7,6) &= [ 14,   2], \\
% \gradient{g_2}(x,y) &= [ 2x, 6y ]              &\gradient{g_2}(7,6) &= [ 14,  36], \\
% \gradient{g_3}(x,y) &= [ 2(x-6), 2y ]          &\gradient{g_3}(7,6) &= [  2,  12],
% \end{align*}
% Looking for $u_k \geq 0$ (not all of them equal to zero) so that the following linear combination is equal to $[0,0]$.
% \begin{align*}
% \gradient{f}(0,0) + u_1 \gradient{g_1}(0,0) + u_2 \gradient{g_2}(0,0) + u_3 \gradient{g_3}(0,0) &= [0,0], \\
% [-36,-52] + u_1[14,2] + u_2[14,36] +u_3[2,12] &= [0,0],
% \end{align*}
% or equivalently
% \begin{equation*}
% \begin{cases}
% 7 u_1 +  7 u_2 = -18 - u_3 \\
%   u_1 + 18 u_2 = -26 - 6 u_3
% \end{cases}
% \end{equation*}
% Among all possible solutions, pick for instance $u_1=2$, $u_2=0$ and $u_3=4$ to prove that the point $(7,6)$ is indeed a good candidate for the optimal solution of $(P)$.

\begin{problem}[Basic]\cite[lec6\_constr\_opt, 10]{Freund2004nonlinear}
Let $f(x,y)=(x-4)^2+(y-6)^2$.  Consider the program $(P)$ to find the global minimum of $f$ on the set $S = \{ (x,y) \in \field{R}^2 : y-x^2\geq 0, y\leq 4 \}$.
\begin{enumerate}
	\item Write the statement of this problem as a program with the notation from equation \ref{equation:programP}.  Label the objective function, as well as the inequality constraints accordingly.
	\item Is the objective function $f$ pseudo-convex? Why or why not?
	\item Are the inequality constraints quasi-convex?  Why or why not?
	\item Sketch the feasibility region.  Label all relevant objects involved.
	\item Is the point $(7,6)$ feasible?  Why or why not?
	\item Employ Theorem \ref{theorem:KKTnecessary} to write a necessary condition for optimality and verify that is satisfied by the point $(7,6)$.
	\item Employ Theorem \ref{theorem:KKTsufficient} to decide whether this point is an optimal solution of $(P)$.
\end{enumerate}
\end{problem}


