%!TEX root = main.tex

\chapter{Rates of Convergence}\label{appendix:convergence}

\begin{definition}
Consider a convergent sequence $\{ \x_n \}_{n\in\field{N}} \subset \field{R}^d$ with $\xstar=\lim_n \x_n$.  We say that this sequence exhibits
\begin{description}
	\item [Linear Convergence] If there exists $0<\delta<1$ so that 
	\begin{equation*}
	\lim_n \frac{\norm{\x_{n+1} - \xstar}}{\norm{\x_n - \xstar}}=\delta.
	\end{equation*}
	We refer to $\delta$ as the \emph{rate of convergence}.
	\item [Superlinear Convergence] If 
	\begin{equation*}
	\lim_n \frac{\norm{\x_{n+1} - \xstar}}{\norm{\x_n - \xstar}}=0.
	\end{equation*}
	\item [Sublinear Convergence] If
	\begin{equation*}
	\lim_n \frac{\norm{\x_{n+1} - \xstar}}{\norm{\x_n - \xstar}}=1.
	\end{equation*}
	If, additionally, 
	\begin{equation*}
	\lim_n \frac{\norm{\x_{n+2} - \x_{n+1}}}{\norm{\x_{n+1} - \x_n}}=1,
	\end{equation*}
	we say the the sequence exhibits \emph{logarithmic convergence} to $\xstar$.
	\item [Convergence of order $q>1$] If $\x_n$ is exhibits superlinear convergence, and there exists $q>1$, $0<\delta<1$ so that 
	\begin{equation*}
	\lim_n \frac{\norm{\x_{n+1} - \xstar}}{\norm{\x_n - \xstar}^q}=\delta.
	\end{equation*}
	In particular, 
	\begin{itemize}
		\item Convergence with $q=2$ is said to be \emph{quadratic}.
		\item Convergence with $q=2$ is said to be \emph{cubic}.
		\item etc.
	\end{itemize}
\end{description}
\end{definition}

A practical method to calculate the rate of convergence of a sequence is to calculate the following sequence, which converges to $q$:
\begin{equation}\label{equation:estimateQ}
q \approx \frac{\log\big\lvert \frac{x_{n+1}-x_n}{x_n-x_{n-1}} \big\rvert}{\log\big\lvert \frac{x_n-x_{n-1}}{x_{n-1}-x_{n-2}} \big\rvert}
\end{equation}

\begin{example}
The sequence $x_n = 1/n!$ exhibits superlinear convergence, since $\lim_n \frac{1}{n!}=0$ and 
\begin{equation*}
\lim_n \frac{x_{n+1}}{x_n} = \lim_n \frac{1}{n+1} = 0.
\end{equation*}
\end{example}

\begin{example}
Given $a \in \field{R}$, $0<r<1$, the geometric sequence $\x_n = ar^n$ exhibits linear convergence, since $\lim_n ar^n = 0$ and
\begin{equation*}
\lim_n \frac{x_{n+1}}{x_n} = r < 1.
\end{equation*}
The rate of convergence is precisely $r$.
\end{example}

\begin{example}
The sequence $x_n = 2^{-2^n}$ converges to zero and is superlinear:
\begin{equation*}
\lim_n \frac{x_{n+1}}{x_n} = \lim_n 2^{-2^n} = 0
\end{equation*}
Using the estimation for $q$ given by the formula in \eqref{equation:estimateQ}, we obtain that this sequence exhibits quadratic convergence.
\end{example}

\begin{example}
The sequence $x_n = 1/n$ converges to zero and is sublinear, since
\begin{equation*}
\lim_n \frac{x_{n+1}}{x_n} = \lim_n \frac{n}{n+1} = 1.
\end{equation*}
Notice 
\begin{equation*}
\lim_n \frac{\abs{\x_{n+2} - \x_{n+1}}}{\abs{\x_{n+1} - \x_n}} = \lim_n \frac{n}{n+2} = 1;
\end{equation*}
therefore, this sequence exhibits logarithmic convergence.
\end{example}
