%!TEX root = main.tex

\section{Broyden's Secant Method}
Both Newton-Raphson and Steepest descent are sound methods, with their pros and their cons.  Steepest descent always converges to a local minimum, yet slowly.  Newton has a faster convergence, but we cannot always guarantee convergence.  Another drawback of both methods is the fact that we do need expressions for both the function itself and its derivative.  The \emph{Broydent Secant method} offers an improvement to some of these issues.

\subsection{A secant method to search for roots of univariate functions}
To explain how it works, let's once again try to find an accurate value of $\sqrt{2}$ as the root of the polynomial $p(x) = x^2-2$.
\begin{enumerate}
	\item Consider two initial guesses $x_0=3$, $x_1=2.8$.  Notice $f(3)= 7 \neq 5.84 = f(2.8)$.
	\item The line that joins the points $(3, 7)$ and $( 2,8, 5.84)$ has equation
	\begin{align*}
	y - 7 &= \frac{5.84-7}{2.8-3}(x-3), \\ 
	y &= 5.8x-10.4
	\end{align*}
	and intersects the $x$--axis at
	\begin{equation*}
	x_2 = \frac{10.4}{5.8} \approx 1.7931034483
	\end{equation*}
	\item Repeat this process to get a sequence $x_n$.
\end{enumerate}
\begin{example}
Observe the result of applying this recursive process, and compare with the similar experiment we conducted using the Newton-Raphson method in page \pageref{table:Newton-Raphson}.
\begin{center}
\begin{tabular}{|r|r|r|} \hline
$n$ & $x_n$ & $f(x_n)$ \\ \hline \hline
$0$ & $3.000000000000000$ & $7.0000E+00$ \\ \hline
$1$ & $2.800000000000000$ & $5.8400E+00$ \\ \hline
$2$ & $1.793103448275862$ & $1.2152E+00$ \\ \hline
$3$ & $1.528528528528528$ & $3.3640E-01$ \\ \hline
$4$ & $1.427253172054743$ & $3.7052E-02$ \\ \hline
$5$ & $1.414717869757887$ & $1.4267E-03$ \\ \hline
$6$ & $1.414215876250105$ & $6.5446E-06$ \\ \hline
$7$ & $1.414213562785585$ & $1.1667E-09$ \\ \hline
$8$ & $1.414213562373095$ & $8.8818E-16$ \\ \hline
\end{tabular}
\end{center}
\begin{figure}[ht!]
\includegraphics[width=0.6\linewidth]{images/secant.png}
\caption{Secant iterative method}\label{figure:SecantMethod}
\end{figure}
\end{example}

\begin{definition}\label{def:SecantMethod}\index{Secant method}\index{Secant method!iteration}\index{Secant method!recursive formula}
Given a function $f \colon \field{R} \to \field{R}$ and two initial guesses $x_0 \neq x_1$ satisfying $f(x_0) \neq f(x_1)$, we define the \emph{Secant method iteration} to be the sequence given by the following recursive formula
\begin{equation*}\label{equation:SecantMethod}\index{Secant method!iteration}
	x_{n+1} = x_{n-1} - \frac{x_n - x_{n-1}}{f(x_n) - f(x_{n-1})}f(x_{n-1})
\end{equation*}
The \emph{Secant method} refers to employing this sequence to search and approximate roots of the equation $f(x)=0$.
\end{definition}