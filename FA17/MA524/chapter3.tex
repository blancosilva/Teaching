%!TEX root = main.tex

Although technically any characterization result finds the exact value of the extrema of a function, computationally this is hardly feasible (specially for functions of very high dimension).  See the following session based on problem \ref{problem:tricky} for an example, where we try to find the critical points of the function $f(x,y,z)=e^{x^2+y^2+z^2}-x^4-y^y-z^6$ symbolically in Python with the \texttt{sympy} libraries:

% framesep=2mm,
% baselinestretch=1.2,
\begin{minted}[frame=single, fontsize=\footnotesize, linenos ]{python}
# Importing necessary symbols/libraries/functions
from sympy.abc import x,y,z
from sympy import Matrix, solve, exp
from sympy.tensor.array import derive_by_array

# Description of f, computation of its gradient and Hessian
f = exp(x**2 + y**2 + z**2) - x**4 -y**6 - z**6
gradient = derive_by_array(f, [x,y,z])
hessian  = Matrix([derive_by_array(gradient, a) for a in [x,y,z]])
\end{minted}
While the correct expressions for $\gradient{f}$ and $\Hess{f}$ are quickly computed, trying to find critical points results in an error:
\begin{minted}[frame=single,fontsize=\footnotesize,mathescape]{python}
>>> solve(gradient) # Search of critical points by solving $\gradient{f}=0$
NotImplementedError: could not solve 
4*x**2*sqrt(-log(exp(x**2)/(2*x**2))) - 6*(-log(exp(x**2)/(2*x**2)))**(5/2)
\end{minted}

Too complex a task to be performed symbolically, although the obvious answer is $(0,0,0)$.  A better way to approach this is by trying to approximate this minimum using the structure of the graph of $f$.  In these notes we are going to explore several strategies to accomplish this task, based on the concept of \emph{iterative methods for finding zeros of real-valued functions}.