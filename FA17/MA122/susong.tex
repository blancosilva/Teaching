\documentclass{amsart}

\theoremstyle{definition}
\newtheorem{problem}{Problem}


\begin{document}
\begin{problem}
Write the function $P=11 e^{0.38t}$ in the form $P=P_0 a^t$.  Is this exponential growth or exponential decay?
\end{problem}
\hrule

\begin{problem}
Put the function $P=16(15)^t$ in the form $P=P_0 e^{kt}$. Do not use your calculator---provide an exact answer.
\end{problem}
\hrule

\begin{problem}
Put the function $P=12(1.8)^t$ in the form $P=P_0 e^{kt}$. Do not use your calculator---provide an exact answer.
\end{problem}
\hrule

\begin{problem}
Put the function $P=160(0.44)^t$ in the form $P=P_0 e^{kt}$. Do not use your calculator---provide an exact answer.
\end{problem}
\hrule

\begin{problem}
A quantity $P$ is an exponential function of time $t$, such that $P=60$ when $t=4$ and $P=90$ when $t=3$.  Use the given information about the function $P=P_0 e^{kt}$ to:
\begin{itemize}
\item Find values for the parameters $k$ and $P_0$ (round your answers to three decimal places).
\item State the initial quantity and the continuous percent rate of growth or decay.  Round your answer for the initial quantity to three decimal places.  Is the quantity growing or decaying?
\end{itemize}
\end{problem}

\end{document}