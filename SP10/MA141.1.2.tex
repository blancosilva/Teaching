\documentclass[12pt]{article}

\usepackage{amsmath,amsthm,amsfonts,amssymb,amsxtra}
\usepackage{pgf,tikz}
\usetikzlibrary{arrows}
\renewcommand{\theenumi}{(\alph{enumi})} 
\renewcommand{\labelenumi}{\theenumi}

\pagestyle{empty}
\setlength{\textwidth}{7in}
\setlength{\oddsidemargin}{-0.5in}
\setlength{\topmargin}{-1.0in}
\setlength{\textheight}{9.5in}

\newtheorem{problem}{Problem}

\begin{document}

\noindent{\large\bf MATH 141}\hfill{\large\bf Exam\#1B.}\hfill{\large\bf
  Fall 2009}\hfill{\large\bf Page 1/6}\hrule

\bigskip
\begin{center}
  \begin{tabular}{|ll|}
    \hline & \cr
    {\bf Name: } & \makebox[12cm]{\hrulefill}\cr & \cr
    {\bf 4-digit code:} & \makebox[12cm]{\hrulefill}\cr & \cr
    \hline
  \end{tabular}
\end{center}
\begin{itemize}
\item Write your name and the last 4 digits of your SSN in the space provided above.
\item The test has six (6) pages, including this one.
\item Enter your answer in the box(es) provided.
\item You must show sufficient work to justify all answers unless
  otherwise stated in the problem.  Correct answers with inconsistent
  work may not be given credit.
\item Credit for each problem is given in parentheses at the right of
  the problem number.
\item No books, notes or calculators may be used on this test.
\end{itemize}
\hrule

\begin{center}
  \begin{tabular}{|c|c|c|}
    \hline
    &&\cr
    {\large\bf Page} & {\large\bf Max.~points} & {\large\bf Your points} \cr
    &&\cr
    \hline
    &&\cr
    {\Large 2} & \Large 20 & \cr
    &&\cr
    \hline
    &&\cr
    {\Large 3} & \Large 20 & \cr
    &&\cr
    \hline
    &&\cr
    {\Large 4} & \Large 15 & \cr
    &&\cr
    \hline
    &&\cr
    {\Large 5} & \Large 25 & \cr
    &&\cr
    \hline
    &&\cr
    {\Large 6} & \Large 20 & \cr
    &&\cr
   \hline\hline
    &&\cr
    {\large\bf Total} & \Large 100 & \cr
    &&\cr
    \hline
  \end{tabular}
\end{center}
\newpage

%%%%%%%%%%%%%%%%%%%%%%%%%%%%%%%%%%%%% Page 1
\noindent{\large\bf MATH 141}\hfill{\large\bf Exam\#1B.}\hfill{\large\bf
  Fall 2009}\hfill{\large\bf Page 2/6}\hrule

\bigskip
{\problem[5 pts] \em Find $f(0)$ and $f(\pi/2)$ for}
$f(x) = 
\begin{cases}
 \sqrt{x+1} & \text{if } x\geq 1,\\
3 & \text{if } x<1.
\end{cases}$
\vspace{1cm}
\begin{flushright}
  \begin{tikzpicture}
    \draw (-1cm,0.5cm) node {$f(\pi/2) =$};
    \draw (0cm,0cm) rectangle (5cm,1.2cm);
    \draw (-1cm,2cm) node {$f(0) =$};
    \draw (0cm,1.4cm) rectangle (5cm,2.6cm);
  \end{tikzpicture}
\end{flushright}
\hrule
{\problem[10 pts] \em Find the domain of $f(x) = \sqrt{(x-1)(x-2)}$.}
\vspace{6cm}
\begin{flushright}
  \begin{tikzpicture}
    \draw (-1cm,0.5cm) node {domain $=$};
    \draw (0cm,0cm) rectangle (5cm,1.2cm);
  \end{tikzpicture}
\end{flushright}
\hrule
{\problem[5 pts] \em Express the function $f(x) = \lvert x-1 \rvert$
  in piecewise form without using absolute values.}     
\vspace{4cm}
\begin{flushright}
  \begin{tikzpicture}
    \draw (-1cm,0.5cm) node {$f(x) = {\Biggr\{}$};
    \draw (0cm,-0.6cm) rectangle (5cm,1.8cm);
  \end{tikzpicture}
\end{flushright}
\newpage

%%%%%%%%%%%%%%%%%%%%%%%%%%%%%%%%%%%%% Page 2
\noindent{\large\bf MATH 141}\hfill{\large\bf Exam\#1B.}\hfill{\large\bf
  Fall 2009}\hfill{\large\bf Page 3/6}\hrule

\bigskip
{\problem[10 pts] \em Let $f(x) = x^2 + 4$ and $g(x) = \sqrt{x}$. Find
  $\big( g \circ f\big)(x)$.}
\vspace{6cm}
\begin{flushright}
  \begin{tikzpicture}
    \draw (-1.25cm,0.5cm) node {$ \big(g \circ f\big)(x) =$};
    \draw (0cm,0cm) rectangle (5cm,1.2cm);
  \end{tikzpicture}
\end{flushright}

\hrule
{\problem[10 pts] \em How many tangent lines to the curve $y=x/(x+1)$
  pass through the point $(0,0)$.}      

\noindent{\textbf{HINT:} \em You do not have to compute the equations of
  the lines.}

\newpage

%%%%%%%%%%%%%%%%%%%%%%%%%%%%%%%%%%%%% Page 3
\noindent{\large\bf MATH 141}\hfill{\large\bf Exam\#1B.}\hfill{\large\bf
  Fall 2009}\hfill{\large\bf Page 4/6}\hrule

\bigskip
{\problem[5 pts] \em Solve for $x$:}
\begin{equation*}
  \ln x + \ln (x-1) = 1
\end{equation*}
\vspace{3cm}
\begin{flushright}
  \begin{tikzpicture}
    \draw (-0.75cm,0.5cm) node {$x =$};
    \draw (0cm,0cm) rectangle (5cm,1.2cm);
  \end{tikzpicture}
\end{flushright}
\hrule
 
{\problem[10 pts] \em Compute the derivatives of the following functions.}
\begin{enumerate}
\item $f(x) = \pi \sqrt{x} (x^4 - 4 x^3 + 6 x^2 -4 x^1 + 1 - x^{-1})$
\vspace{4cm}
\begin{flushright}
  \begin{tikzpicture}
    \draw (-0.75cm,0.5cm) node {$f'(x) =$};
    \draw (0cm,0cm) rectangle (5cm,1.2cm);
  \end{tikzpicture}
\end{flushright}
\item $\displaystyle{g(t) = \frac{t^2-5}{t^{-1}}}$
\vspace{4cm}
\begin{flushright}
  \begin{tikzpicture}
    \draw (-0.75cm,0.5cm) node {$g'(t) =$};
    \draw (0cm,0cm) rectangle (5cm,1.2cm);
  \end{tikzpicture}
\end{flushright}
\end{enumerate}

\newpage

%%%%%%%%%%%%%%%%%%%%%%%%%%%%%%%%%%%%% Page 4
\noindent{\large\bf MATH 141}\hfill{\large\bf Exam\#1B.}\hfill{\large\bf
  Fall 2009}\hfill{\large\bf Page 5/6}\hrule

\bigskip
{\problem[15 pts] \em Compute the following limits:}
\vspace{1cm}

\noindent
\begin{tikzpicture}
\draw (4cm,14cm) node{
(a) $\displaystyle{\lim_{x \to 2} \frac{x^2-2x-8}{x^2-4}} = \mbox{}$ };
\draw (6cm,13.4cm) rectangle (11cm,14.6cm);
\draw (4cm, 10cm) node{
(b) $\displaystyle{\lim_{x \to -\infty} \frac{x^2-2x-8}{x^2-4}} =
\mbox{}$}; 
\draw (6.2cm, 9.4cm) rectangle (11.1cm, 10.6cm); 
\draw (4cm, 6cm) node{
(b) $\displaystyle{\lim_{x \to -2} \frac{x^2-2x-8}{x^2-4}} =
\mbox{}$}; 
\draw (6.1cm, 5.4cm) rectangle (11.1cm, 6.6cm); 
\end{tikzpicture}

\vspace{3cm}
\hrule

{\problem[10 pts] \em Find the value of the constant $k$ for which the
  following function is continuous everywhere:} 
\begin{equation*}
f(x) = \begin{cases}
2k^2x^3 &\text{if }x<2, \\
x+32k-18 &\text{if }x \geq 2.
\end{cases}
\end{equation*}
\vspace{5cm}
\begin{flushright}
  \begin{tikzpicture}
    \draw (-1cm,0.5cm) node {$k =$};
    \draw (0cm,0cm) rectangle (5cm,1.2cm);
  \end{tikzpicture}
\end{flushright}

\newpage


%%%%%%%%%%%%%%%%%%%%%%%%%%%%%%%%%%%%% Page 5
\noindent{\large\bf MATH 141}\hfill{\large\bf Exam\#1B.}\hfill{\large\bf
  Fall 2009}\hfill{\large\bf Page 6/6}\hrule

\bigskip
{\problem[20 pts] \em Find equations of the tangent lines to the curve}
\begin{equation*}
  y = \frac{x-1}{x+1}
\end{equation*}
that are parallel to the line $x-\tfrac{9}{2} y=3$.
\end{document}
