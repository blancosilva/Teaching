\documentclass[12pt]{article}

\usepackage{amsmath,amsthm,amsfonts,amssymb,amsxtra}
\usepackage{pgf,tikz}
\usetikzlibrary{arrows}
\renewcommand{\theenumi}{(\alph{enumi})} 
\renewcommand{\labelenumi}{\theenumi}

\pagestyle{empty}
\setlength{\textwidth}{7in}
\setlength{\oddsidemargin}{-0.5in}
\setlength{\topmargin}{-1.0in}
\setlength{\textheight}{9.5in}

\newtheorem{problem}{Problem}

\begin{document}

\noindent{\large\bf MATH 122}\hfill{\large\bf First Midterm.}\hfill{\large\bf
  Spring 2012}\hfill{\large\bf Page 1/5}\hrule

\bigskip
\begin{center}
  \begin{tabular}{|ll|}
    \hline & \cr
    {\bf Name: } & \makebox[12cm]{\hrulefill}\cr & \cr
    {\bf 4-digit code:} & \makebox[12cm]{\hrulefill}\cr & \cr
    \hline
  \end{tabular}
\end{center}
\begin{itemize}
\item Write your name and the last 4 digits of your SSN in the space provided above.
\item The test has five (5) pages, including this one.
\item For multiple-choice questions, circle the answer you select.  On
  the other problems, you should enter your answer in the box(es)
  provided. 
\item Show sufficient work to justify all answers unless otherwise
  stated in the problem.  Correct answers with inconsistent work may
  not be given credit. 
\item Credit for each problem is given at the right of each problem
  number. 
\item No books or notes may be used on this test.  Calculators are
  allowed, provided they don't have a computer algebra system.
\end{itemize}
\hrule

\begin{center}
  \begin{tabular}{|c|c|c|}
    \hline
    &&\cr
    {\large\bf Page} & {\large\bf Max} & {\large\bf Points} \cr
    &&\cr
    \hline
    &&\cr
    {\Large 2} & \Large 20 & \cr
    &&\cr
    \hline
    &&\cr
    {\Large 3} & \Large 30 & \cr
    &&\cr
    \hline
    &&\cr
    {\Large 4} & \Large 25 & \cr
    &&\cr
    \hline
    &&\cr
    {\Large 5} & \Large 25 & \cr
    &&\cr
    \hline\hline
    &&\cr
    {\large\bf Total} & \Large 100 & \cr
    &&\cr
    \hline
  \end{tabular}
\end{center}
\newpage

%%%%%%%%%%%%%%%%%%%%%%%%%%%%%%%%%%%%% Page 2
\noindent{\large\bf MATH 122}\hfill{\large\bf First Midterm.}\hfill{\large\bf
  Spring 2012}\hfill{\large\bf Page 2/5}\hrule

\bigskip
{\problem[10 pts] \em The solid waste generated each year in the
  cities of the US is increasing.  The solid waste generated (in
  millions of tons) was 238.3 in 2000 and 251.3 in 2006.}
\begin{itemize}
\item Assume that it increases {\bf linearly}. Find a formula for
  this function.
\vspace{3.5cm}
\begin{flushright}
  \begin{tikzpicture}
    \draw (0cm,-0.2cm) rectangle (5cm,1.2cm);
  \end{tikzpicture}
\end{flushright}
\item Use this formula to predict the amount of solid waste generated
  in the year 2020.
\vspace{3.5cm}
\begin{flushright}
  \begin{tikzpicture}
    \draw (0cm,-0.2cm) rectangle (5cm,1.2cm);
  \end{tikzpicture}
\end{flushright}
\end{itemize}
\hrule
{\problem[10 pts] \em Winning height in men's Olympic pole vault}
\begin{center}
\begin{tabular}{l|r|r|r|c|r|r|r|}
\hline
year & 1960 & 1964 & 1968 & $\cdots$ & 1992 & 1996 & 2000 \\
\hline
height (in) & 185 & 201 & 213 & $\cdots$ & 228 & 233 & 232 \\
\hline
\end{tabular}
\end{center}
What is the average rate of change in height from 1992 to the year 2000?
\begin{enumerate}
\item $\dfrac{8}{213-185}$
\item $\dfrac{213-185}{8}$
\item $\dfrac{8}{232-228}$
\item $\dfrac{232-228}{8}$
\end{enumerate}
\newpage

%%%%%%%%%%%%%%%%%%%%%%%%%%%%%%%%%%%%% Page 3
\noindent{\large\bf MATH 122}\hfill{\large\bf First Midterm.}\hfill{\large\bf
  Spring 2012}\hfill{\large\bf Page 3/5}\hrule

\bigskip
{\problem[10 pts] \em Find the average rate of change of $y=f(x) =
  \sqrt{x}$ between $x=4$ and $x=9$.  Show work to get credit.}
\begin{enumerate}
\item $3$
\item $5$
\item $\tfrac{1}{3}$
\item $\tfrac{1}{5}$
\end{enumerate}

\vspace{2cm}
\hrule
{\problem[10 pts] \em Determine the slope and $y$--intercept of the
  line with equation $3x=6y+12.$}
\begin{tabular}{rl}
Slope $m$: \\
(a) & $m=\tfrac{1}{4}$ \\ \\
(b) & $m=-\tfrac{1}{4}$ \\ \\
(c) &$m=\tfrac{1}{2}$ \\ \\
(d) &$m=-\tfrac{1}{2}$
\end{tabular} \hspace{3cm}
\begin{tabular}{rl}
$y$--intercept $b$: \\
(a) & $b=-2$ \\ \\
(b) &$b=\tfrac{1}{3}$ \\ \\
(c) &$b=2$ \\ \\
(d) &$b=-\tfrac{1}{3}$
\end{tabular}
\vspace{3.5cm}
\hrule
{\problem[10 pts] \em For certain antibiotic, 40\% of the drug is eliminated
every hour (not continuosly).  Find a function that expresses the quantity of
antibiotic in the blood after $t$ hours, assuming that the initial dose is
$200$ mg.}
\vspace{3.5cm}
\begin{flushright}
  \begin{tikzpicture}
    \draw (0cm,-0.2cm) rectangle (5cm,1.2cm);
  \end{tikzpicture}
\end{flushright}
\newpage

%%%%%%%%%%%%%%%%%%%%%%%%%%%%%%%%%%%%% Page 4
\noindent{\large\bf MATH 122}\hfill{\large\bf First Midterm.}\hfill{\large\bf
  Spring 2012}\hfill{\large\bf Page 4/5}\hrule

\bigskip
{\problem[15 pts] \em The fixed cost for factory and machinery needed
  to begin production in a wallet company is \$24,000.  The cost of
  labor and raw material amounts to \$7 per manufactured wallet.
  What is the total cost function?}
\vspace{1cm}

\begin{flushright}
  \begin{tikzpicture}
    \draw (0cm,-0.2cm) rectangle (5cm,1.2cm);
  \end{tikzpicture}
\end{flushright}
\noindent
If wallets sell for \$15 each, what is the revenue function?
\vspace{1cm}

\begin{flushright}
  \begin{tikzpicture}
    \draw (0cm,-0.2cm) rectangle (5cm,1.2cm);
  \end{tikzpicture}
\end{flushright}
\noindent
What is the break even point?
\vspace{3cm}

\begin{flushright}
  \begin{tikzpicture}
    \draw (0cm,-0.2cm) rectangle (5cm,1.2cm);
  \end{tikzpicture}
\end{flushright}
\hrule
{\problem[10 pts] \em The gross domestic product of Switzerland, $G$,
  was 310 billion dollars in 2007.  Give a formula for $G$ (in
  billions of dollars) $t$ years after 2007 if $G$ increases by 40\% a
  year (not continuously).}
\vspace{4cm}
\begin{flushright}
  \begin{tikzpicture}
    \draw (0cm,-0.2cm) rectangle (5cm,1.2cm);
  \end{tikzpicture}
\end{flushright}
\newpage

%%%%%%%%%%%%%%%%%%%%%%%%%%%%%%%%%%%%% Page 5
\noindent{\large\bf MATH 122}\hfill{\large\bf First Midterm.}\hfill{\large\bf
  Spring 2012}\hfill{\large\bf Page 5/5}\hrule

\bigskip
{\problem[15 pts] \em One of the tables below represents a supply
  curve; the other represents a demand curve.}
\begin{center}
\begin{tabular}{l|r|r|r|r|r|r|r|r|}
\hline
$p$ & 182 & 167 & 153 & 143 & 133 & 125 & 118 \\
\hline
$q$ & 5 & 10 & 15 & 20 & 25 & 30 & 35 \\
\hline \\



\hline
$p$ & 6 & 35 & 66 & 110 & 166 & 235 & 316 \\
\hline
$q$ & 5 & 10 & 15 & 20 & 25 & 30 & 35 \\
\hline
\end{tabular}
\end{center}
\noindent
Which  one is the demand curve?
\begin{enumerate}
\item The first table.
\item The second table.
\end{enumerate}
\noindent
At a price of \$155, approximately how many items would consumers
purchase?
\begin{enumerate}
\item between 10 and 15.
\item between 20 and 25.
\item The question cannot be answered with the offered data.
\end{enumerate}
\noindent
What would the price have to be if you wanted consumers to buy at
least 20 items?
\begin{enumerate}
\item 143
\item 10
\item between 5 and 10
\end{enumerate}
\hrule
{\problem[10 pts] \em Find a formula for the population of Nevada as a
  function of time, assuming that the population increases
  exponentially, and the following data:  In 2000, the population of
  Nevada was 2.02 million.  In 2006, it was 2.498 million.}
\vspace{2.5cm}
\begin{flushright}
  \begin{tikzpicture}
    \draw (0cm,-0.2cm) rectangle (5cm,1.2cm);
  \end{tikzpicture}
\end{flushright}
\noindent
When will the population reach 3 million?
\vspace{2.5cm}
\begin{flushright}
  \begin{tikzpicture}
    \draw (0cm,-0.2cm) rectangle (5cm,1.2cm);
  \end{tikzpicture}
\end{flushright}
\end{document}
