\documentclass[12pt]{article}

\usepackage{amsmath,amsthm,amsfonts,amssymb,amsxtra}
\usepackage{pgf,tikz}
\usetikzlibrary{arrows}
\renewcommand{\theenumi}{(\alph{enumi})} 
\renewcommand{\labelenumi}{\theenumi}

\pagestyle{empty}
\setlength{\textwidth}{7in}
\setlength{\oddsidemargin}{-0.5in}
\setlength{\topmargin}{-1.0in}
\setlength{\textheight}{9.5in}

\newtheorem{problem}{Problem}

\begin{document}

\noindent{\large\bf MATH 141}\hfill{\large\bf Exam\#3.}\hfill{\large\bf
  Fall 2008}\hfill{\large\bf Page 1/5}\hrule

\bigskip
\begin{center}
  \begin{tabular}{|ll|}
    \hline & \cr
    {\bf Name: } & \makebox[12cm]{\hrulefill}\cr & \cr
    {\bf 4-digit code:} & \makebox[12cm]{\hrulefill}\cr & \cr
    \hline
  \end{tabular}
\end{center}
\begin{itemize}
\item Write your name and the last 4 digits of your SSN in the space provided above.
\item The test has five (5) pages, including this one.
\item Enter your answer in the box(es) provided.
\item You must show sufficient work to justify all answers unless
  otherwise stated in the problem.  Correct answers with inconsistent
  work may not be given credit.
\item Credit for each problem is given in parentheses at the right of
  the problem number.
\item No books, notes or calculators may be used on this test.
\end{itemize}
\hrule

\begin{center}
  \begin{tabular}{|c|c|c|}
    \hline
    &&\cr
    {\large\bf Page} & {\large\bf Max.~points} & {\large\bf Your points} \cr
    &&\cr
    \hline
    &&\cr
    {\Large 2} & \Large 30 & \cr
    &&\cr
    \hline
    &&\cr
    {\Large 3} & \Large 30 & \cr
    &&\cr
    \hline
    &&\cr
    {\Large 4} & \Large 20 & \cr
    &&\cr
    \hline
    &&\cr
    {\Large 5} & \Large 20 & \cr
    &&\cr
    \hline\hline
    &&\cr
    {\large\bf Total} & \Large 100 & \cr
    &&\cr
    \hline
  \end{tabular}
\end{center}
\newpage

%%%%%%%%%%%%%%%%%%%%%%%%%%%%%%%%%%%%% Page 2
\noindent{\large\bf MATH 141}\hfill{\large\bf Exam\#3.}\hfill{\large\bf
  Fall 2008}\hfill{\large\bf Page 2/5}\hrule

\bigskip
{\problem[20 pts] \em  Evaluate each integral:} 
\begin{enumerate}
\item $\displaystyle{\int \big( 5x + \frac{2}{3x^5} - \sqrt{2} e^x \big)\, dx}$
\vspace{2cm}
\item $\displaystyle{\int \big( 3\sin x - 2\sec^2 x \big)\, dx}$
\vspace{2cm}
\item $\displaystyle{\int ( 1 + \sin t)^{90} \cos t\, dt}$
\vspace{2cm}
\item $\displaystyle{\int \frac{5x^4}{(x^5+1)^2}\, dx}$
\vspace{3cm}
\end{enumerate}
\hrule
{\problem[10 pts] \em Suppose that a point moves along a curve $y=f(x)$ in the $xy$--plane in such a way that at each point $(x,y)$ on the curve the tangent line has slope $\sin x$.  Find an equation for the curve, given that it passes through the point $(0,-2)$.}
\newpage

%%%%%%%%%%%%%%%%%%%%%%%%%%%%%%%%%%%%% Page 3
\noindent{\large\bf MATH 141}\hfill{\large\bf Exam\#3.}\hfill{\large\bf
  Fall 2008}\hfill{\large\bf Page 3/5}\hrule

\bigskip
{\problem[30 pts] \em Sketch the graph of the rational function $f(x) = \displaystyle{\frac{2x^2-8}{x^2-16}}$.}
\begin{quotation}
Indicate clearly:
\begin{itemize}
\item $x$- and $y$-intercepts.
\item Vertical and horizontal asymptotes.
\item Intervals of increase, decrease and different concavity.
\item Location of relative extrema and inflection points. 
\end{itemize}
\end{quotation}
\newpage

%%%%%%%%%%%%%%%%%%%%%%%%%%%%%%%%%%%%% Page 4
\noindent{\large\bf MATH 141}\hfill{\large\bf Exam\#3.}\hfill{\large\bf
  Fall 2008}\hfill{\large\bf Page 4/5}\hrule

\bigskip
{\problem[10 pts] \em Find the absolute extrema of $f(x) = 6x^{4/3} - 3x^{1/3}$ on the interval $[-1,1]$.}
\vspace{9cm}
\begin{flushright}
  \begin{tikzpicture}
    \draw (-2.25cm,2.5cm) node {Absolute maximum at };
    \draw (0cm,1.8cm) rectangle (5cm,3.2cm);
    \draw (-2.25cm,0.5cm) node {Absolute minimum at };
    \draw (0cm,-0.2cm) rectangle (5cm,1.2cm);
  \end{tikzpicture}
\end{flushright}
\hrule
{\problem[10 pts] \em Let $f(x) = \cos x$. Verify that the hypotheses of Rolle's Theorem are satisfied on the interval $\big[\tfrac{\pi}{2}, 3\tfrac{\pi}{2}\big]$.}
\newpage

%%%%%%%%%%%%%%%%%%%%%%%%%%%%%%%%%%%%% Page 5
\noindent{\large\bf MATH 141}\hfill{\large\bf Exam\#3.}\hfill{\large\bf
  Fall 2008}\hfill{\large\bf Page 5/5}\hrule

\bigskip
\noindent{\Large \textbf{Choose one of the following two story-problems:}}
{\problem[20 pts] \em The function $s(t) = t^4 - 4t^2 + 4$ describes the position of a particle moving along a coordinate line, where $s$ is in feet and $t \geq 0$ is in seconds.}
\begin{enumerate}
\item Find the velocity and acceleration functions.
\item When is the particle speeding up? Slowing down? (Justify your answers with sign charts).
\end{enumerate}
\vspace{9cm}
\hrule
{\problem[20 pts] \em A container with square base, vertical sides, and open top is to be made from 1000 ft$^2$ of material.  Find the dimensions of the container with greatest volume.}
\vspace{7cm}
\begin{flushright}
  \begin{tikzpicture}
    \draw (-2.7cm,0.5cm) node {Dimensions of container: };
    \draw (0cm,-0.2cm) rectangle (5cm,1.2cm);
  \end{tikzpicture}
\end{flushright}
\end{document}
