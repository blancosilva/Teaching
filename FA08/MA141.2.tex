\documentclass[12pt]{article}

\usepackage{amsmath,amsthm,amsfonts,amssymb,amsxtra}
\usepackage{pgf,tikz}
\usetikzlibrary{arrows}
\renewcommand{\theenumi}{(\alph{enumi})} 
\renewcommand{\labelenumi}{\theenumi}

\pagestyle{empty}
\setlength{\textwidth}{7in}
\setlength{\oddsidemargin}{-0.5in}
\setlength{\topmargin}{-1.0in}
\setlength{\textheight}{9.5in}

\newtheorem{problem}{Problem}

\begin{document}

\noindent{\large\bf MATH 141}\hfill{\large\bf Exam\#2.}\hfill{\large\bf
  Fall 2008}\hfill{\large\bf Page 1/6}\hrule

\bigskip
\begin{center}
  \begin{tabular}{|ll|}
    \hline & \cr
    {\bf Name: } & \makebox[12cm]{\hrulefill}\cr & \cr
    {\bf 4-digit code:} & \makebox[12cm]{\hrulefill}\cr & \cr
    \hline
  \end{tabular}
\end{center}
\begin{itemize}
\item Write your name and the last 4 digits of your SSN in the space provided above.
\item The test has six (6) pages, including this one.
\item Enter your answer in the box(es) provided.
\item You must show sufficient work to justify all answers unless
  otherwise stated in the problem.  Correct answers with inconsistent
  work may not be given credit.
\item Credit for each problem is given in parentheses at the right of
  the problem number.
\item No books, notes or calculators may be used on this test.
\end{itemize}
\hrule

\begin{center}
  \begin{tabular}{|c|c|c|}
    \hline
    &&\cr
    {\large\bf Page} & {\large\bf Max.~points} & {\large\bf Your points} \cr
    &&\cr
    \hline
    &&\cr
    {\Large 2} & \Large 20 & \cr
    &&\cr
    \hline
    &&\cr
    {\Large 3} & \Large 10 & \cr
    &&\cr
    \hline
    &&\cr
    {\Large 4} & \Large 30 & \cr
    &&\cr
    \hline
    &&\cr
    {\Large 5} & \Large 20 & \cr
    &&\cr
    \hline
    &&\cr
    {\Large 6} & \Large 20 & \cr
    &&\cr
    \hline\hline
    &&\cr
    {\large\bf Total} & \Large 100 & \cr
    &&\cr
    \hline
  \end{tabular}
\end{center}
\newpage

%%%%%%%%%%%%%%%%%%%%%%%%%%%%%%%%%%%%% Page 1
\noindent{\large\bf MATH 141}\hfill{\large\bf Exam\#2.}\hfill{\large\bf
  Fall 2008}\hfill{\large\bf Page 2/6}\hrule

\bigskip
{\problem[15 pts] \em  Compute the following limit without using L'H\^{o}pital's rule:} 

\bigskip
\begin{tikzpicture}
\draw (4cm,14cm) node{
$\displaystyle{ \lim_{x\to 0} \frac{\tan(5x^2) + \sin^2(2x)}{x^2} = \mbox{}}$ };
\draw (6.5cm,13.4cm) rectangle (11.5cm,14.6cm);
\end{tikzpicture}

\noindent{\bf Hint:} $\displaystyle{\lim_{x\to 0} \frac{\sin x}{x}=0}.$
\vspace{14cm}
\hrule
{\problem[5 pts] \em Compute the following limit:}

\bigskip
\begin{tikzpicture}
\draw (4cm,14cm) node{
$\displaystyle{ \lim_{x\to 0} \frac{e^x-1}{\tan x} = \mbox{}}$ };
\draw (5.5cm,13.4cm) rectangle (10.5cm,14.6cm);
\end{tikzpicture}
\newpage

%%%%%%%%%%%%%%%%%%%%%%%%%%%%%%%%%%%%% Page 2
\noindent{\large\bf MATH 141}\hfill{\large\bf Exam\#2.}\hfill{\large\bf
  Fall 2008}\hfill{\large\bf Page 3/6}\hrule

\bigskip
{\problem[10 pts] \em Answer the following questions:}
\begin{enumerate}
\item Use the definition of derivative to find $f'(x)$ for $f(x) = 2x^3+2$.
\vspace{10cm}
\begin{flushright}
  \begin{tikzpicture}
    \draw (-1cm,0.5cm) node {$f'(x) = $};
    \draw (0cm,-0.2cm) rectangle (5cm,1.2cm);
  \end{tikzpicture}
\end{flushright}
\item Find the point-slope equation of the tangent line to the graph of $f(x)$ at $x=1$.
\end{enumerate}
\newpage

%%%%%%%%%%%%%%%%%%%%%%%%%%%%%%%%%%%%% Page 3
\noindent{\large\bf MATH 141}\hfill{\large\bf Exam\#2.}\hfill{\large\bf
  Fall 2008}\hfill{\large\bf Page 4/6}\hrule

\bigskip
{\problem[30 pts] \em Find the derivative of the following functions:}
\begin{enumerate}
\item $\displaystyle{y = \frac{x^3 + x^2 + x - 1}{x^{3/2}}}$.
\vspace{4cm}
\begin{flushright}
  \begin{tikzpicture}
    \draw (-1cm,0.5cm) node {$\displaystyle{\frac{dy}{dx}} = $};
    \draw (0cm,-0.2cm) rectangle (5cm,1.2cm);
  \end{tikzpicture}
\end{flushright}
\item $f(x) = \sec^2 (x^2) - \tan^2 (x^2)$.
\vspace{4cm}
\begin{flushright}
  \begin{tikzpicture}
    \draw (-1cm,0.5cm) node {$f'(x) = $};
    \draw (0cm,-0.2cm) rectangle (5cm,1.2cm);
  \end{tikzpicture}
\end{flushright}
\item $g(t) = 3\cos^{-1}(t^5)$.

\noindent
{\bf Hint}: The derivative of $y=\cos^{-1}t$ is $\displaystyle{\frac{dy}{dt}} = \frac{-1}{\sqrt{1-t^2}}$.
\vspace{4cm}
\begin{flushright}
  \begin{tikzpicture}
    \draw (-1cm,0.5cm) node {$g'(t) = $};
    \draw (0cm,-0.2cm) rectangle (5cm,1.2cm);
  \end{tikzpicture}
\end{flushright}
\end{enumerate}
\newpage

%%%%%%%%%%%%%%%%%%%%%%%%%%%%%%%%%%%%% Page 4
\noindent{\large\bf MATH 141}\hfill{\large\bf Exam\#2.}\hfill{\large\bf
  Fall 2008}\hfill{\large\bf Page 5/6}\hrule

\bigskip
{\problem[10 pts] \em An aircraft is climbing at $30^o$ angle to the horizontal.  How fast is the aircraft gaining altitude if its speed is 500 mi/h?}
\vspace{10cm}
\begin{flushright}
  \begin{tikzpicture}
    \draw (-4.25cm,0.5cm) node {The aircraft is gaining altitude at a speed of};
    \draw (0cm,-0.2cm) rectangle (5cm,1.2cm);
  \end{tikzpicture}
\end{flushright}
\hrule
{\problem[10 pts] \em Show that $y=e^{2x} - e^{-4x}$ satisfies the equation}
\begin{equation*}
y'' + 2y' -8y = 0.
\end{equation*}
\newpage

%%%%%%%%%%%%%%%%%%%%%%%%%%%%%%%%%%%%% Page 5
\noindent{\large\bf MATH 141}\hfill{\large\bf Exam\#2.}\hfill{\large\bf
  Fall 2008}\hfill{\large\bf Page 6/6}\hrule

\bigskip
{\problem[10 pts] \em Find $\displaystyle{\frac{dy}{dy}}$ by implicit differentiation.}
\begin{equation*}
\sin \big( x^3 y^3 \big) = y.
\end{equation*}
\vspace{6cm}
\begin{flushright}
  \begin{tikzpicture}
    \draw (-0.7cm,0.5cm) node {$\displaystyle{\frac{dy}{dx}} = $};
    \draw (0cm,-0.2cm) rectangle (5cm,1.2cm);
  \end{tikzpicture}
\end{flushright}
\hrule
{\problem[10 pts] \em Use logarithmic differentiation to find $\displaystyle{\frac{dy}{dx}}$.}
\begin{equation*}
y = \frac{\sin x \cos x \tan^3 x}{\sqrt{x}}.
\end{equation*}
\vspace{4cm}
\begin{flushright}
  \begin{tikzpicture}
    \draw (-0.7cm,0.5cm) node {$\displaystyle{\frac{dy}{dx}} = $};
    \draw (0cm,-0.2cm) rectangle (5cm,1.2cm);
  \end{tikzpicture}
\end{flushright}
\end{document}
