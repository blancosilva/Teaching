\documentclass[12pt]{article}

\usepackage{amsmath,amsthm,amsfonts,amssymb,amsxtra}
\usepackage{pgf,tikz}
\usetikzlibrary{arrows}
\renewcommand{\theenumi}{(\alph{enumi})} 
\renewcommand{\labelenumi}{\theenumi}

\pagestyle{empty}
\setlength{\textwidth}{7in}
\setlength{\oddsidemargin}{-0.5in}
\setlength{\topmargin}{-1.0in}
\setlength{\textheight}{9.5in}

\newtheorem{problem}{Problem}

\begin{document}

\noindent{\large\bf MATH 242}\hfill{\large\bf Second Midterm.}\hfill{\large\bf
  Fall 2012}\hfill{\large\bf Page 1/4}\hrule

\bigskip
\begin{center}
  \begin{tabular}{|ll|}
    \hline & \cr
    {\bf Name: } & \makebox[12cm]{\hrulefill}\cr & \cr
    {\bf 4-digit code:} & \makebox[12cm]{\hrulefill}\cr & \cr
    \hline
  \end{tabular}
\end{center}
\begin{itemize}
\item Write your name and the last 4 digits of your SSN in the space provided above.
\item The test has four (4) pages, including this one.
\item Write the answers in the boxes provided, where applicable.
\item Show sufficient work to justify all answers unless otherwise
stated in the problem.  Correct answers with inconsistent work may not
be given credit. 
\item Credit for each problem is given at the right of each problem
  number. 
\item No books, notes or calculators may be used on this test.
\end{itemize}
\hrule

\begin{center}
  \begin{tabular}{|c|c|c|}
    \hline
    &&\cr
    {\large\bf Page} & {\large\bf Max} & {\large\bf Points} \cr
    &&\cr
    \hline
    &&\cr
    {\Large 2} & \Large 35 & \cr
    &&\cr
    \hline
    &&\cr
    {\Large 3} & \Large 30 & \cr
    &&\cr
    \hline
    &&\cr
    {\Large 4} & \Large 35 & \cr
    &&\cr
    \hline\hline
    &&\cr
    {\large\bf Total} & \Large 100 & \cr
    &&\cr
    \hline
  \end{tabular}
\end{center}
\newpage

%%%%%%%%%%%%%%%%%%%%%%%%%%%%%%%%%%%%% Page 2
\noindent{\large\bf MATH 242}\hfill{\large\bf Second Midterm.}\hfill{\large\bf
  Fall 2012}\hfill{\large\bf Page 2/4}\hrule

\bigskip
{\problem[30 pts] \em Use the method of \textbf{variation of parameters}
  to solve the initial value problem $y''+3y'+2y=x$ that satisfies
  $y(0)=0, y'(0)=\tfrac{1}{2}$.} 
\vspace{14cm}
\begin{flushright}
  \begin{tikzpicture}
    \draw (-0.8cm,0.6cm) node{$y(x)=$};
    \draw (0cm,-0.2cm) rectangle (5cm,1.2cm);
  \end{tikzpicture}
\end{flushright}
\hrule
{\problem[5pts] \em Decide whether the functions $y_1(x)=1+\lvert x^3
\rvert$
and $y_2(x)=1+x^3$ are linearly dependent or independent in the interval
$(-3,3)$.}
\newpage
%%%%%%%%%%%%%%%%%%%%%%%%%%%%%%%%%%%%% Page 3
\noindent{\large\bf MATH 242}\hfill{\large\bf Second Midterm.}\hfill{\large\bf
  Fall 2012}\hfill{\large\bf Page 3/4}\hrule

\bigskip
{\problem[20 pts] \em Use exclusively the technique of
\textbf{undetermined coefficients} to find a general solution of the
differential equation $y''+3y'+2y=x$.}
\vspace{11cm}
\begin{flushright}
  \begin{tikzpicture}
    \draw (-0.8cm,0.6cm) node{$y(x)=$};
    \draw (0cm,-0.2cm) rectangle (5cm,1.2cm);
  \end{tikzpicture}
\end{flushright}
\hrule
{\problem[10pts] \em Use the Wronskian of the functions $y_1(x)=e^{-3x}$,
$y_2(x)=\cos 2x$, and $y_3(x)=\sin 2x$, to prove that they are linearly
independent.}
\vspace{6cm}
\begin{flushright}
  \begin{tikzpicture}
    \draw (-1.4cm,0.6cm) node{$W(y_1,y_2,y_3)=$};
    \draw (0cm,-0.2cm) rectangle (5cm,1.2cm);
  \end{tikzpicture}
\end{flushright}
\newpage

%%%%%%%%%%%%%%%%%%%%%%%%%%%%%%%%%%%%% Page 4
\noindent{\large\bf MATH 242}\hfill{\large\bf Second Midterm.}\hfill{\large\bf
  Fall 2012}\hfill{\large\bf Page 4/4}\hrule

\bigskip
{\problem[15 pts] \em Find the solution to the differential equation
$y''-4y'+5y=0$ that satisfies $y(0)=1$ and $y'(0)=5$.}
\vspace{6cm}
\begin{flushright}
  \begin{tikzpicture}
    \draw (-0.8cm,0.6cm) node{$y(x)=$};
    \draw (0cm,-0.2cm) rectangle (5cm,1.2cm);
  \end{tikzpicture}
\end{flushright}
\hrule
{\problem[20] \em Find a particular solution $y_p$ to the differential
equation $y''+y=\sin x$.}
\vspace{11.5cm}
\begin{flushright}
  \begin{tikzpicture}
    \draw (-0.8cm,0.6cm) node{$y_p(x)=$};
    \draw (0cm,-0.2cm) rectangle (5cm,1.2cm);
  \end{tikzpicture}
\end{flushright}

\end{document}
