\documentclass[12pt]{article}

\usepackage{amsmath,amsthm,amsfonts,amssymb,amsxtra}
\usepackage{pgf,tikz}
\usetikzlibrary{arrows}
\renewcommand{\theenumi}{(\alph{enumi})} 
\renewcommand{\labelenumi}{\theenumi}

\pagestyle{empty}
\setlength{\textwidth}{7in}
\setlength{\oddsidemargin}{-0.5in}
\setlength{\topmargin}{-1.0in}
\setlength{\textheight}{9.5in}

\newtheorem{problem}{Problem}

\begin{document}

\noindent{\large\bf MATH 241}\hfill{\large\bf Exam\#3.}\hfill{\large\bf
  Fall 2012}\hfill{\large\bf Page 1/4}\hrule

\bigskip
\begin{center}
  \begin{tabular}{|ll|}
    \hline & \cr
    {\bf Name: } & \makebox[12cm]{\hrulefill}\cr & \cr
    {\bf 4-digit code:} & \makebox[12cm]{\hrulefill}\cr & \cr
    \hline
  \end{tabular}
\end{center}
\begin{itemize}
\item Write your name and the last 4 digits of your SSN in the space provided above.
\item The test has four (4) pages, including this one.
\item The test is fifty (50) minutes long.
\item Enter your answer in the box(es) provided.
\item You must show sufficient work to justify all answers unless
  otherwise stated in the problem.  Correct answers with inconsistent
  work may not be given credit.
\item Credit for each problem is given in parentheses at the right of
  the problem number.
\item No books, notes or calculators may be used on this test.
\end{itemize}
\hrule

\begin{center}
  \begin{tabular}{|c|c|c|}
    \hline
    &&\cr
    {\large\bf Page} & {\large\bf Max.~points} & {\large\bf Your points} \cr
    &&\cr
    \hline
    &&\cr
    {\Large 2} & \Large 30 & \cr
    &&\cr
    \hline
    &&\cr
    {\Large 3} & \Large 30 & \cr
    &&\cr
    \hline
    &&\cr
    {\Large 4} & \Large 40 & \cr
    &&\cr
    \hline\hline
    &&\cr
    {\large\bf Total} & \Large 100 & \cr
    &&\cr
    \hline
  \end{tabular}
\end{center}
\newpage

%%%%%%%%%%%%%%%%%%%%%%%%%%%%%%%%%%%%% Page 2
\noindent{\large\bf MATH 241}\hfill{\large\bf Exam\#3.}\hfill{\large\bf
  Fall 2012}\hfill{\large\bf Page 2/4}\hrule

\bigskip
{\problem[15 pts] \em Calculate the double integral
$\displaystyle{\iint_R \frac{1+x^2}{1+y^2}}\, dA$, for the rectangle
$R=[0,1]\times[0,1]$.}
\vspace{8cm}
\begin{flushright}
  \begin{tikzpicture}
    \draw (0cm,-0.2cm) rectangle (5cm,1.2cm);
  \end{tikzpicture}
\end{flushright}
\hrule
{\problem[15 pts] \em Use a double integral to compute the volume under
the surface $z=xy$ and above the region bounded by $x=y^2$ and $x=y^3$.}
\vspace{8cm}
\begin{flushright}
  \begin{tikzpicture}
    \draw (0cm,-0.2cm) rectangle (5cm,1.2cm);
  \end{tikzpicture}
\end{flushright}
\newpage

%%%%%%%%%%%%%%%%%%%%%%%%%%%%%%%%%%%%% Page 3
\noindent{\large\bf MATH 241}\hfill{\large\bf Exam\#3.}\hfill{\large\bf
  Fall 2012}\hfill{\large\bf Page 3/4}\hrule

\bigskip
{\problem[15 pts] \em Evaluate $\displaystyle{\iint_R (3x+4y^2)\, dA}$,
where $R$ is the region in the upper half-plane bounded by the circles
$x^2+y^2=1$ and $x^2+y^2=4$.}
\vspace{8cm}
\begin{flushright}
  \begin{tikzpicture}
    \draw (0cm,-0.2cm) rectangle (5cm,1.2cm);
  \end{tikzpicture}
\end{flushright}
\hrule
{\problem[15 pts] \em Electric charge is distributed over the square
$\{(x,y) : 0\leq x \leq 1, 1\leq y \leq 2\}$ so that the charge density
at ($(x,y)$ is $\sigma(x,y)=e^{x+e^x}$ (measured in Coulombs per square
meter).  Find the total charge of the disk.
\vspace{7.5cm}
\begin{flushright}
  \begin{tikzpicture}
    \draw (0cm,-0.2cm) rectangle (5cm,1.2cm);
  \end{tikzpicture}
\end{flushright}
\newpage

%%%%%%%%%%%%%%%%%%%%%%%%%%%%%%%%%%%%% Page 4
\noindent{\large\bf MATH 241}\hfill{\large\bf Exam\#3.}\hfill{\large\bf
  Fall 2012}\hfill{\large\bf Page 4/4}\hrule

\bigskip
{\problem[5 pts] \em Evaluate $\displaystyle{\int_{-2}^2
\int_{-\sqrt{4-x^2}}^{\sqrt{4-x^2}} \int_{\sqrt{x^2+y^2}}^{2}
(x^2+y^2)\, dz\, dy\, dx}$.}
\vspace{6cm}
\begin{flushright}
  \begin{tikzpicture}
    \draw (0cm,-0.2cm) rectangle (5cm,1.2cm);
  \end{tikzpicture}
\end{flushright}
\hrule
{\problem[25 pts] \em Evaluate $\displaystyle{\iiint_E xyz\, dV}$, where
$E$ lies between the spheres $\rho=2$, $\rho=4$ and above the cone
$\phi=\pi/3$.}
\vspace{10cm}
\begin{flushright}
  \begin{tikzpicture}
    \draw (0cm,-0.2cm) rectangle (5cm,1.2cm);
  \end{tikzpicture}
\end{flushright}
\end{document}
