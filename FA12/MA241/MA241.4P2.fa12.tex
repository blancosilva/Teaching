\documentclass[12pt]{amsart}

\usepackage{amsmath,amsthm,amsfonts,amssymb,amsxtra}
\usepackage{pgf,tikz}
\usetikzlibrary{arrows}
\renewcommand{\theenumi}{(\alph{enumi})} 
\renewcommand{\labelenumi}{\theenumi}

\pagestyle{empty}
\setlength{\textwidth}{7in}
\setlength{\oddsidemargin}{-0.5in}
\setlength{\topmargin}{-1.0in}
\setlength{\textheight}{9.5in}

\newtheorem{problem}{Problem}
\theoremstyle{definition}
\newtheorem*{defn}{Definition}

\begin{document}
\thispagestyle{empty}
\noindent{\large\bf MATH 241}\hfill{\large\bf Exam\#4. Part II}\hfill{\large\bf
  Fall 2012}\hrule

\bigskip
\noindent We are going to define two operations that can be performed on vector
fields.  Each operation resembles differentiation, but one produces a
vector field, whereas the other produces a scalar field.

\begin{defn}
If $\boldsymbol{F}=\langle P,Q,R \rangle$ is a vector field on
$\mathbb{R}^3$ and the partial derivatives of $P$, $Q$ and $R$ all
exists, then the {\em curl} of $\boldsymbol{F}$ is the vector field on
$\mathbb{R}^3$ defined by
\begin{equation*}
\operatorname{curl} \boldsymbol{F} = \bigg( \frac{\partial R}{\partial y} -
\frac{\partial Q}{\partial z} \bigg) \boldsymbol{i} + \bigg(
\frac{\partial P}{\partial z} - \frac{\partial R}{\partial x} \bigg)
\boldsymbol{j} + \bigg( \frac{\partial Q}{\partial x} - \frac{\partial
P}{\partial y} \bigg) \boldsymbol{k}.
\end{equation*}
The {\em divergence} of $\boldsymbol{F}$ is the function of three
variables defined by
\begin{equation*}
\operatorname{div} \boldsymbol{F} = \frac{\partial P}{\partial x} +
\frac{\partial Q}{\partial y} + \frac{\partial R}{\partial z}.
\end{equation*}
\end{defn}
\hrule

\problem[5 pts]{\em Assume that $f$ is a function of three variables
that satisfies
\begin{equation*}
\frac{\partial^2 f}{\partial x \partial y} = \frac{\partial^2
f}{\partial y \partial x}, \quad
\frac{\partial^2 f}{\partial x \partial z} = \frac{\partial^2
f}{\partial z \partial x}, \quad
\frac{\partial^2 f}{\partial z \partial y} = \frac{\partial^2
f}{\partial y \partial z}.
\end{equation*}
Prove that in this case, $\operatorname{curl}(\nabla f) = \boldsymbol{0}$.}

\vspace{4cm}
\hrule
\problem[5 pts]{\em Note that the previous problem implies that
conservative vector fields will have zero curl.  Determine, using this
trick, whether $\boldsymbol{F}=\langle y^2z^3, 2xyz^3, 3xy^2z^2 \rangle$
is a conservative vector field or not.} 

\vspace{7cm}
\hrule
\problem[15 pts]{\em For the vector field $\boldsymbol{F} = \cos(xz)
\boldsymbol{j} - \sin(xy) \boldsymbol{k}$, compute:}
\begin{enumerate}
\item $\operatorname{curl} \boldsymbol{F}$
\item $\operatorname{div} \boldsymbol{F}$
\item $\operatorname{div} \operatorname{curl} \boldsymbol{F}$
\end{enumerate}
Use the back of this page for this problem
\end{document}
