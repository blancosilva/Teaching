\documentclass[12pt]{article}

\usepackage{amsmath,amsthm,amsfonts,amssymb,amsxtra}
\usepackage{pgf,tikz}
\usetikzlibrary{arrows}
\renewcommand{\theenumi}{(\alph{enumi})} 
\renewcommand{\labelenumi}{\theenumi}

\pagestyle{empty}
\setlength{\textwidth}{7in}
\setlength{\oddsidemargin}{-0.5in}
\setlength{\topmargin}{-1.0in}
\setlength{\textheight}{9.5in}

\theoremstyle{definition}
\newtheorem{problem}{Problem}

\begin{document}

\noindent{\large\bf MATH 142}\hfill{\large\bf Exam\#4.}\hfill{\large\bf
  Fall 2015}\hfill{\large\bf Page 1/5}\hrule

\bigskip
\begin{center}
  \begin{tabular}{|ll|}
    \hline & \cr
    {\bf Name: } & \makebox[12cm]{\hrulefill}\cr & \cr
    {\bf 4-digit code:} & \makebox[12cm]{\hrulefill}\cr & \cr
    \hline
  \end{tabular}
\end{center}
\begin{itemize}
\item Write your name and the last 4 digits of your SSN in the space provided above.
\item The test has five (5) pages, including this one.
\item You must show sufficient work to justify all answers unless otherwise stated in the problem.  Correct answers with inconsistent work may not be given credit.
\item Credit for each problem is given in parentheses at the right of the problem number.
\item No books or notes may be used on this test.  The use of graphing calculators is acceptable.
\end{itemize}
\hrule

\begin{center}
  \begin{tabular}{|c|c|c|}
    \hline
    &&\cr
    {\large\bf Page} & {\large\bf Max.~points} & {\large\bf Your points} \cr
    &&\cr
    \hline
    &&\cr
    {\Large 2} & \Large 25 & \cr
    &&\cr
    \hline
    &&\cr
    {\Large 3} & \Large 25 & \cr
    &&\cr
    \hline
    &&\cr
    {\Large 4} & \Large 25 & \cr
    &&\cr
    \hline
    &&\cr
    {\Large 5} & \Large 25 & \cr
    &&\cr
    \hline\hline
    &&\cr
    {\large\bf Total} & \Large 100 & \cr
    &&\cr
    \hline
  \end{tabular}
\end{center}
\newpage

%%%%%%%%%%%%%%%%%%%%%%%%%%%%%%%%%%%%% Page 2
\noindent{\large\bf MATH 142}\hfill{\large\bf Exam\#4.}\hfill{\large\bf
  Fall 2015}\hfill{\large\bf Page 2/5}\hrule

\bigskip

\begin{problem}[25 pts]
Sketch and identify the curve defined by the parametric equations 
\begin{equation*}
\begin{cases}
x=t^2-2t, \\ 
y=t+1, 
\end{cases} \text{ for } -2 \leq t \leq 2.
\end{equation*}
\end{problem}

\newpage

%%%%%%%%%%%%%%%%%%%%%%%%%%%%%%%%%%%%% Page 3
\noindent{\large\bf MATH 142}\hfill{\large\bf Exam\#4.}\hfill{\large\bf
  Fall 2015}\hfill{\large\bf Page 3/5}\hrule

\bigskip

\begin{problem}[25 pts]
Consider the curve with polar equation $r=2\cos \theta - 2 \sin \theta$.
\begin{enumerate}
\item Find a cartesian equation for this curve.
\item Sketch its graph.  Make sure that the sketch carries all relevant information.
\end{enumerate}
\end{problem}
\newpage

%%%%%%%%%%%%%%%%%%%%%%%%%%%%%%%%%%%%% Page 4
\noindent{\large\bf MATH 142}\hfill{\large\bf Exam\#4.}\hfill{\large\bf
  Fall 2015}\hfill{\large\bf Page 4/5}\hrule

\bigskip
\begin{problem}[25 pts]
Compute the length of the cardioid $r=1+\cos \theta$.
\end{problem}
\newpage

%%%%%%%%%%%%%%%%%%%%%%%%%%%%%%%%%%%%% Page 5
\noindent{\large\bf MATH 142}\hfill{\large\bf Exam\#4.}\hfill{\large\bf
  Fall 2015}\hfill{\large\bf Page 5/5}\hrule
\bigskip

\begin{problem}[25 pts]
Find the area enclosed by one loop of the four-leaved rose $r = \cos 2\theta$.
\end{problem}

\end{document}
