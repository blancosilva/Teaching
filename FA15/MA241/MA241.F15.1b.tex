\documentclass[12pt]{article}

\usepackage{amsmath,amsthm,amsfonts,amssymb,amsxtra}
\usepackage{pgf,tikz}
\usetikzlibrary{arrows}
\renewcommand{\theenumi}{(\alph{enumi})} 
\renewcommand{\labelenumi}{\theenumi}

\pagestyle{empty}
\setlength{\textwidth}{7in}
\setlength{\oddsidemargin}{-0.5in}
\setlength{\topmargin}{-1.0in}
\setlength{\textheight}{9.5in}

\newtheorem{problem}{Problem}

\begin{document}

\noindent{\large\bf MATH 241}\hfill{\large\bf Exam\#1.}\hfill{\large\bf
  Fall 2015}\hfill{\large\bf Page 1/4}\hrule

\bigskip
\begin{center}
  \begin{tabular}{|ll|}
    \hline & \cr
    {\bf Name: } & \makebox[12cm]{\hrulefill}\cr & \cr
    {\bf 4-digit code:} & \makebox[12cm]{\hrulefill}\cr & \cr
    \hline
  \end{tabular}
\end{center}
\begin{itemize}
\item Write your name and the last 4 digits of your SSN in the space provided above.
\item The test has four (4) pages, including this one.
\item You have fifty (50) minutes to complete the test.
\item Enter your answer in the box(es) provided.
\item You must show sufficient work to justify all answers unless
  otherwise stated in the problem.  Correct answers with inconsistent
  work may not be given credit.
\item Credit for each problem is given in parentheses at the right of
  the problem number.
\item No books, notes or calculators may be used on this test.
\end{itemize}
\hrule

\begin{center}
  \begin{tabular}{|c|c|c|}
    \hline
    &&\cr
    {\large\bf Page} & {\large\bf Max.~points} & {\large\bf Your points} \cr
    &&\cr
    \hline
    &&\cr
    {\Large 2} & \Large 55 & \cr
    &&\cr
    \hline
    &&\cr
    {\Large 3} & \Large 25 & \cr
    &&\cr
    \hline
    &&\cr
    {\Large 4} & \Large 20 & \cr
    &&\cr
    \hline\hline
    &&\cr
    {\large\bf Total} & \Large 100 & \cr
    &&\cr
    \hline
  \end{tabular}
\end{center}
\newpage

%%%%%%%%%%%%%%%%%%%%%%%%%%%%%%%%%%%%% Page 2
\noindent{\large\bf MATH 241}\hfill{\large\bf Exam\#1.}\hfill{\large\bf
  Fall 2015}\hfill{\large\bf Page 2/4}\hrule

\bigskip
{\problem[15 pts] \em  Find the distances from the point $(3,7,-5)$ to the three coordinate axes.} 
\vspace{4cm}
\begin{center}
  \begin{tikzpicture}
    \draw (0cm,-0.2cm) rectangle (5cm,1.2cm);
    \draw (5.5cm,-0.2cm) rectangle (10.5cm,1.2cm);
    \draw (11cm,-0.2cm) rectangle (16cm,1.2cm);
  \end{tikzpicture}
\end{center}
\hrule

{\problem[10 pts] \em Find an exact expression for the angle $\theta$ between
the vectors $\boldsymbol{v}=\langle 3, -1, 5\rangle$ and
$\boldsymbol{w}=\langle -6, 2, -15\rangle$.} 
\vspace{2cm}
\begin{flushright}
  \begin{tikzpicture}
    \draw (-0.5cm,0.5cm) node {$\theta = $};
    \draw (0cm,-0.2cm) rectangle (5cm,1.2cm);
  \end{tikzpicture}
\end{flushright}
\hrule

{\problem[15 pts] \em Find a unit vector $\boldsymbol{v}$ that is orthogonal
to both $\boldsymbol{i} + \boldsymbol{j}$ and $\boldsymbol{i} - \boldsymbol{j} +
\boldsymbol{k}$.} 
\vspace{4cm}
\begin{flushright}
  \begin{tikzpicture}
    \draw (-0.5cm,0.5cm) node {$\boldsymbol{v} = $};
    \draw (0cm,-0.2cm) rectangle (5cm,1.2cm);
  \end{tikzpicture}
\end{flushright}
\hrule

{\problem[15 pts] \em Determine whether the points $A=(0,-5,5)$, $ B=
(1,-2,4)$, $C=(0,0,0)$ and $D=(3,4,2)$ are coplanar.}
\newpage

%%%%%%%%%%%%%%%%%%%%%%%%%%%%%%%%%%%%% Page 3
\noindent{\large\bf MATH 241}\hfill{\large\bf Exam\#1.}\hfill{\large\bf
  Fall 2015}\hfill{\large\bf Page 3/4}\hrule

\bigskip
{\problem[15 pts] \em Consider the sphere that goes through the origin, and whose center is the point $P=(1,3,2)$.  Find the equation of the circle of intersection of this sphere with the $xy$--plane.} 
\vspace{5cm}
\begin{flushright}
  \begin{tikzpicture}
    \draw (-1cm,0.5cm) node {sphere: };
    \draw (0cm,-0.2cm) rectangle (5cm,1.2cm);
  \end{tikzpicture}
\end{flushright}
\hrule
{\problem[10 pts] \em Consider the point $P=(0,1,1)$ and the line $\ell$ with parametric equations}
\begin{equation*}
\begin{cases}
x = 3+t \\ y = 2t \\ z = 1-t
\end{cases}
\end{equation*}
\begin{enumerate}
\item Find the equation of a plane that goes through $P$ and is perpendicular to $\ell$.
\vspace{2cm}
\begin{flushright}
  \begin{tikzpicture}
    \draw (-1cm,0.5cm) node {plane: };
    \draw (0cm,-0.2cm) rectangle (5cm,1.2cm);
  \end{tikzpicture}
\end{flushright}
\item  Compute the intersection of the line $\ell$ with that plane.
\vspace{3cm}
\begin{flushright}
  \begin{tikzpicture}
    \draw (-1cm,0.5cm) node {point: };
    \draw (0cm,-0.2cm) rectangle (5cm,1.2cm);
  \end{tikzpicture}
\end{flushright}
\end{enumerate}
\newpage

%%%%%%%%%%%%%%%%%%%%%%%%%%%%%%%%%%%%% Page 4
\noindent{\large\bf MATH 241}\hfill{\large\bf Exam\#1.}\hfill{\large\bf
  Fall 2015}\hfill{\large\bf Page 4/4}\hrule

\bigskip
{\problem[20 pts] \em Find parametric equations for the line of intersection
of the planes $x+y+z=1$ and $x+2y+2z=1$.  Find the angle $\theta$ between the
two planes.}
\vspace{18.5cm}
\begin{flushright}
  \begin{tikzpicture}
    \draw (-2.5cm,0.5cm) node {line: };
    \draw (-2cm,-0.2cm) rectangle (5cm,1.2cm);
    \draw (-0.5cm,2.5cm) node {$\theta = $};
    \draw (0cm,1.7cm) rectangle (5cm,3.2cm);
  \end{tikzpicture}
\end{flushright}
\end{document}
