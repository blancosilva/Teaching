\documentclass[12pt]{article}

\usepackage{amsmath,amsthm,amsfonts,amssymb,amsxtra}
\usepackage{pgf,tikz}
\usetikzlibrary{arrows}
\renewcommand{\theenumi}{(\alph{enumi})} 
\renewcommand{\labelenumi}{\theenumi}

\pagestyle{empty}
\setlength{\textwidth}{7in}
\setlength{\oddsidemargin}{-0.5in}
\setlength{\topmargin}{-1.0in}
\setlength{\textheight}{9.5in}

\theoremstyle{definition}
\newtheorem{problem}{Problem}

\begin{document}

\noindent{\large\bf MATH 241}\hfill{\large\bf Exam \#2 (take home)}\hfill{\large\bf
  Fall 2015}\hfill{\large\bf Page 1/5}\hrule

\bigskip
\begin{center}
  \begin{tabular}{|ll|}
    \hline & \cr
    {\bf Name: } & \makebox[12cm]{\hrulefill}\cr & \cr
    {\bf 4-digit code:} & \makebox[12cm]{\hrulefill}\cr & \cr
    \hline
  \end{tabular}
\end{center}
\begin{itemize}
\item Write your name and your VIP ID in the space provided above.
\item The test has five (5) pages, including this one.
\item Enter your answer in the box(es) provided.
\item You must show sufficient work to justify all answers unless
  otherwise stated in the problem.  Correct answers with inconsistent
  work may not be given credit.
\item Credit for each problem is given in parentheses at the right of
  the problem number.
\end{itemize}
\hrule

\begin{center}
  \begin{tabular}{|c|c|c|}
    \hline
    &&\cr
    {\large\bf Page} & {\large\bf Max.~points} & {\large\bf Your points} \cr
    &&\cr
    \hline
    &&\cr
    {\Large 2} & \Large 30 & \cr
    &&\cr
    \hline
    &&\cr
    {\Large 3} & \Large 25 & \cr
    &&\cr
    \hline
    &&\cr
    {\Large 4} & \Large 25 & \cr
    &&\cr
    \hline
    &&\cr
    {\Large 5} & \Large 20 & \cr
    &&\cr
    \hline\hline
    &&\cr
    {\large\bf Total} & \Large 100 & \cr
    &&\cr
    \hline
  \end{tabular}
\end{center}
\newpage

%%%%%%%%%%%%%%%%%%%%%%%%%%%%%%%%%%%%% Page 2
\noindent{\large\bf MATH 241}\hfill{\large\bf Exam \#2 (take home)}\hfill{\large\bf
  Fall 2015}\hfill{\large\bf Page 2/5}\hrule

\bigskip
{\problem[15 pts] Find the domain and range of
$g(x)=\sqrt{9-x^2-y^2}$}
\vspace{2.5cm}
\begin{flushright}
  \begin{tikzpicture}
    \draw (-1.0cm,2.1cm) node {domain:};
    \draw (0cm,1.4cm) rectangle (5cm,2.8cm);
    \draw (-1.0cm,0.5cm) node {range:};
    \draw (0cm,-0.2cm) rectangle (5cm,1.2cm);
  \end{tikzpicture}
\end{flushright}
Sketch the level curves for $k=0,1,2,3$:
\vspace{8cm}
\hrule
{\problem[15 pts] If $f(x,y)=\displaystyle{\frac{xy^2}{x^2+y^4}}$, does
$\displaystyle{\lim_{(x,y)\to(0,0)} f(x,y)}$ exist?  Why?
\newpage

%%%%%%%%%%%%%%%%%%%%%%%%%%%%%%%%%%%%% Page 3
\noindent{\large\bf MATH 241}\hfill{\large\bf Exam \#2 (take home)}\hfill{\large\bf
  Fall 2015}\hfill{\large\bf Page 3/5}\hrule

\bigskip
{\problem[15 pts] Find the tangent plane to the elliptic paraboloid
$z=2x^2+y^2$ at the point $(1,1,3)$.}
\vspace{9.5cm}
\begin{flushright}
  \begin{tikzpicture}
    \draw (0cm,-0.2cm) rectangle (5cm,1.2cm);
  \end{tikzpicture}
\end{flushright}
\hrule
{\problem[10 pts] The dimensions of a rectangular box are measured
to be 75 cm, 60 cm, and 40 cm, and each measurement is correct within
0.2 cm.  Use differentials to estimate the largest possible error when
the volume of the box is calculated from these measurements.}
\vspace{7.5cm}
\begin{flushright}
  \begin{tikzpicture}
    \draw (0cm,-0.2cm) rectangle (5cm,1.2cm);
  \end{tikzpicture}
\end{flushright}
\newpage

%%%%%%%%%%%%%%%%%%%%%%%%%%%%%%%%%%%%% Page 4
\noindent{\large\bf MATH 241}\hfill{\large\bf Exam \#2 (take home)}\hfill{\large\bf
  Fall 2015}\hfill{\large\bf Page 4/5}\hrule

\bigskip
{\problem[5 pts] If $z=x^2y+3xy^4$, where $x=\sin 2t$ and $y=\cos
t$, find $dz/dt$ when $t=0$ \textbf{without calculating explicitly the
derivative from the composition}.  Use the chain rule in the form that
was explained in class.
\vspace{3cm}
\begin{flushright}
  \begin{tikzpicture}
    \draw (-0.7cm,0.5cm) node {$\displaystyle{\frac{dz}{dt}}= $};
    \draw (0cm,-0.2cm) rectangle (10cm,1.2cm);
  \end{tikzpicture}
\end{flushright}
\hrule
{\problem[10 pts] If $f(x,y,z)=x\sin yz$, find the directional
derivative of $f$ at $(1,3,0)$ in the direction of $\boldsymbol{v} =
\boldsymbol{i}+2\boldsymbol{j}-\boldsymbol{k}$.}
\vspace{3cm}
\begin{flushright}
  \begin{tikzpicture}
    \draw (-1.5cm,0.5cm) node {$D_{\boldsymbol{v}}f(1,3,0) = $};
    \draw (0cm,-0.2cm) rectangle (5cm,1.2cm);
  \end{tikzpicture}
\end{flushright}
\hrule
{\problem[10 pts] Find the local maxima, minima and saddle points
of $f(x)=x^4+y^4-4xy+1.$}
\vspace{6cm}
\begin{center}
  \begin{tabular}{ccc}
  \begin{tikzpicture}
    \draw (0cm,-0.2cm) rectangle (5cm,1.2cm);
    \draw (0.5cm,1cm) node[scale=0.8]{max};
  \end{tikzpicture} &
  \begin{tikzpicture}
    \draw (0cm,-0.2cm) rectangle (5cm,1.2cm);
    \draw (0.5cm,1cm) node[scale=0.8]{min};
  \end{tikzpicture} &
  \begin{tikzpicture}
    \draw (0cm,-0.2cm) rectangle (5cm,1.2cm);
    \draw (1cm,1cm) node[scale=0.8]{saddle pts.};
  \end{tikzpicture}
  \end{tabular}
\end{center}
\newpage

%%%%%%%%%%%%%%%%%%%%%%%%%%%%%%%%%%%%% Page 5
\noindent{\large\bf MATH 241}\hfill{\large\bf Exam \#2 (take home)}\hfill{\large\bf
  Fall 2015}\hfill{\large\bf Page 5/5}\hrule

\bigskip
\begin{problem}[10 pts]
Find the absolute maximum and minimum values of the function $f(x,y) = 4x+6y-x^2-y^2+7$ on the set $D = \big\{ (x,y) : 0 \leq x \leq 4, 0 \leq y \leq 5 \big\}$.  Make sure to sketch the set $D$ and indicate the different borders.
\end{problem}

\vspace{13cm}
\hrule
\begin{problem}[10 pts]
Use Lagrange multipliers to find the maximum and minimum values of the function $f(x,y)=x^2y$ subject to the constraint curve $x^2+2y^2=6$.
\end{problem}
\end{document}
