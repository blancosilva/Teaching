\documentclass[12pt]{article}

\usepackage{amsmath,amsthm,amsfonts,amssymb,amsxtra}
\usepackage{pgf,tikz}
\usetikzlibrary{arrows}
\renewcommand{\theenumi}{(\alph{enumi})} 
\renewcommand{\labelenumi}{\theenumi}

\pagestyle{empty}
\setlength{\textwidth}{7in}
\setlength{\oddsidemargin}{-0.5in}
\setlength{\topmargin}{-1.0in}
\setlength{\textheight}{9.5in}

\newtheorem{problem}{Problem}

\begin{document}

\noindent{\large\bf MATH 141}\hfill{\large\bf Exam\#5.}\hfill{\large\bf
  Fall 2013}\hfill{\large\bf Page 1/2}\hrule

\bigskip
\begin{center}
  \begin{tabular}{|ll|}
    \hline & \cr
    {\bf Name: } & \makebox[12cm]{\hrulefill}\cr & \cr
    {\bf 4-digit code:} & \makebox[12cm]{\hrulefill}\cr & \cr
    \hline
  \end{tabular}
\end{center}
\begin{itemize}
\item Write your name and the last 4 digits of your SSN in the space provided above.
\item The test has four (4) pages, including this one.
\item You have fifty (50) minutes to complete the exam.
\item Enter your answer in the box(es) provided.
\item You must show sufficient work to justify all answers unless
  otherwise stated in the problem.  Correct answers with inconsistent
  work may not be given credit.
\item Credit for each problem is given in parentheses at the right of
  the problem number.
\item No books, notes or calculators may be used on this test.
\end{itemize}
\hrule

\begin{center}
  \begin{tabular}{|c|c|c|}
    \hline
    &&\cr
    {\large\bf Page} & {\large\bf Max.~points} & {\large\bf Your points} \cr
    &&\cr
    \hline
    &&\cr
    {\Large 2} & \Large 40 & \cr
    &&\cr
    \hline
    &&\cr
    {\Large 3} & \Large 30 & \cr
    &&\cr
    \hline
    &&\cr
    {\Large 4} & \Large 30 & \cr
    &&\cr
    \hline
    \hline
    &&\cr
    {\Large -} & \Large 100 & \cr
    &&\cr
    \hline
  \end{tabular}
\end{center}
\newpage

%%%%%%%%%%%%%%%%%%%%%%%%%%%%%%%%%%%%% Page 1
\noindent{\large\bf MATH 141}\hfill{\large\bf Exam\#5.}\hfill{\large\bf
  Fall 2013}\hfill{\large\bf Page 2/4}\hrule

\bigskip

{\problem[10 pts] \em Find an antiderivative $F(x)$ of the function $f(x)=x^2-x+1-\dfrac{1}{x}+\dfrac{1}{\sqrt{x}}$ that satisfies $F(1)=5$.}

\vspace{3cm}
\begin{flushright}
  \begin{tikzpicture}
    \draw (0cm,0cm) rectangle (5cm,1.2cm);
  \end{tikzpicture}
\end{flushright}
\hrule 

{\problem[10 pts each] \em Compute the following integrals:}
\begin{enumerate}
\item $\displaystyle{\int \frac{dx}{\sqrt{6x}}}$
\vspace{1cm}
\begin{flushright}
  \begin{tikzpicture}
    \draw (0cm,0cm) rectangle (5cm,1.2cm);
  \end{tikzpicture}
\end{flushright}
\item $\displaystyle{\int_0^1 2^x-5e^x\, dx}$
\vspace{2.5cm}
\begin{flushright}
  \begin{tikzpicture}
    \draw (0cm,0cm) rectangle (5cm,1.2cm);
  \end{tikzpicture}
\end{flushright}
\item $\displaystyle{\int_0^1 (x^2+3)^3\, dx}$
\vspace{3cm}
\begin{flushright}
  \begin{tikzpicture}
    \draw (0cm,0cm) rectangle (5cm,1.2cm);
  \end{tikzpicture}
\end{flushright}
\end{enumerate}
\newpage

%%%%%%%%%%%%%%%%%%%%%%%%%%%%%%%%%%%%% Page 2
\noindent{\large\bf MATH 141}\hfill{\large\bf Exam\#5.}\hfill{\large\bf
  Fall 2013}\hfill{\large\bf Page 3/4}\hrule

\bigskip

{\problem[5 pts each] \em Given the following sequences, find the next two elements, and the general term:}
\begin{enumerate}
\item $-\frac{1}{2},\frac{1}{3},-\frac{1}{4},\frac{1}{5},-\frac{1}{6},\dotsc$

\vspace{0.5cm}
\begin{flushright}
  \begin{tikzpicture}
    \draw (0cm,0cm) rectangle (5cm,1.2cm);
    \draw (5.5cm,0cm) rectangle (10.5cm,1.2cm);
  \end{tikzpicture}
\end{flushright}
\item $\frac{1}{2}, \frac{3}{4}, \frac{6}{8}, \frac{9}{16}, \frac{12}{32}, \dotsc$

\vspace{0.5cm}
\begin{flushright}
  \begin{tikzpicture}
    \draw (0cm,0cm) rectangle (5cm,1.2cm);
    \draw (5.5cm,0cm) rectangle (10.5cm,1.2cm);
  \end{tikzpicture}
\end{flushright}
\end{enumerate}
\hrule

{\problem[10 pts each] \em Compute the following:}
\begin{enumerate}
\item $\displaystyle{\sum_{k=1}^{50} (3k^2-7k+1)}$
\vspace{2cm}
\begin{flushright}
  \begin{tikzpicture}
    \draw (0cm,0cm) rectangle (5cm,1.2cm);
  \end{tikzpicture}
\end{flushright}
\item $\displaystyle{\lim_{n\to \infty} \sum_{k=1}^n \frac{2}{n} \bigg( \frac{2k}{n}\bigg)^3}$
\vspace{3.5cm}
\begin{flushright}
  \begin{tikzpicture}
    \draw (0cm,0cm) rectangle (5cm,1.2cm);
  \end{tikzpicture}
\end{flushright}
\end{enumerate}
\newpage

%%%%%%%%%%%%%%%%%%%%%%%%%%%%%%%%%%%%% Page 2
\noindent{\large\bf MATH 141}\hfill{\large\bf Exam\#5.}\hfill{\large\bf
  Fall 2013}\hfill{\large\bf Page 4/4}\hrule
  
\bigskip

{\problem[10 pts each] \em Estimate the area under the graph of $f(x)=8+2x^2$ from $x=-1$ to $x=2$ using three rectangles:}
\begin{enumerate}
\item Using right endpoints.

\vspace{4cm}
\begin{flushright}
  \begin{tikzpicture}
    \draw (0cm,0cm) rectangle (5cm,1.2cm);
  \end{tikzpicture}
\end{flushright}
\item Using left endpoints.
\vspace{4cm}
\begin{flushright}
  \begin{tikzpicture}
    \draw (0cm,0cm) rectangle (5cm,1.2cm);
  \end{tikzpicture}
\end{flushright}
\item Using midpoints.
\vspace{4cm}
\begin{flushright}
  \begin{tikzpicture}
    \draw (0cm,0cm) rectangle (5cm,1.2cm);
  \end{tikzpicture}
\end{flushright}
\end{enumerate}
\end{document}
