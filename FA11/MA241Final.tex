\documentclass[12pt]{article}

\usepackage{amsmath,amsthm,amsfonts,amssymb,amsxtra}
\usepackage{pgf,tikz}
\usetikzlibrary{arrows}
\renewcommand{\theenumi}{(\alph{enumi})} 
\renewcommand{\labelenumi}{\theenumi}

\pagestyle{empty}
\setlength{\textwidth}{7in}
\setlength{\oddsidemargin}{-0.5in}
\setlength{\topmargin}{-1.0in}
\setlength{\textheight}{9.5in}

\newtheorem{problem}{Problem}

\begin{document}

\noindent{\large\bf MATH 241}\hfill{\large\bf Final Exam.}\hfill{\large\bf
  Fall 2011}\hfill{\large\bf Page 1/14}\hrule

\bigskip
\begin{center}
  \begin{tabular}{|ll|}
    \hline & \cr
    {\bf Name: } & \makebox[12cm]{\hrulefill}\cr & \cr
    {\bf 4-digit code:} & \makebox[12cm]{\hrulefill}\cr & \cr
    \hline
  \end{tabular}
\end{center}
\begin{itemize}
\item Write your name and the last 4 digits of your SSN in the space provided above.
\item The test has fourteen (14) pages, including this one, and your
  help sheet. 
\item For multi-choice questions, you should circle the answer you
  select.  On the other problems, you should enter your answer in the
  box(es) provided.
\item You must show sufficient work to justify all answers unless
  otherwise stated in the problem.  Correct answers with inconsistent
  work may not be given credit.
\item Credit for each problem is given in parentheses at the right of
  the problem number.
\item No books, notes or calculators may be used on this test.
\item \textbf{A:} 243--270 pts. \textbf{B+:} 230--242 pts. \textbf{B:} 216--229 pts. \textbf{C+:} 203--215 pts. \textbf{C:} 189--202 pts.\\ \textbf{D+:} 175--188 pts. \textbf{D:} 160--174 pts. \textbf{F}: less than 160 pts.
\end{itemize}
\hrule

\begin{center}
  \begin{tabular}{|c|c|c|}
    \hline
    &&\cr
    {\large\bf Page} & {\large\bf Max} & {\large\bf Points} \cr
    &&\cr
    \hline
    &&\cr
    {\Large 2} & \Large 30 & \cr
    &&\cr
    \hline
    &&\cr
    {\Large 3} & \Large 25 & \cr
    &&\cr
    \hline
    &&\cr
    {\Large 4} & \Large 25 & \cr
    &&\cr
    \hline
    &&\cr
    {\Large 5} & \Large 20 & \cr
    &&\cr
    \hline\hline
    &&\cr
    {\large\bf Total} & \Large 100 & \cr
    &&\cr
    \hline
  \end{tabular}
  \begin{tabular}{|c|c|c|}
    \hline
    &&\cr
    {\large\bf Page} & {\large\bf Max} & {\large\bf Points} \cr
    &&\cr
    \hline
    &&\cr
    {\Large 6} & \Large 30 & \cr
    &&\cr
    \hline
    &&\cr
    {\Large 7} & \Large 25 & \cr
    &&\cr
    \hline
    &&\cr
    {\Large 8} & \Large 25 & \cr
    &&\cr
    \hline
    &&\cr
    {\Large 9} & \Large 20 & \cr
    &&\cr
    \hline\hline
    &&\cr
    {\large\bf Total} & \Large 100 & \cr
    &&\cr
    \hline
  \end{tabular}
  \begin{tabular}{|c|c|c|}
    \hline
    &&\cr
    {\large\bf Page} & {\large\bf Max} & {\large\bf Points} \cr
    &&\cr
    \hline
    &&\cr
    {\Large 10} & \Large 30 & \cr
    &&\cr
    \hline
    &&\cr
    {\Large 11} & \Large 25 & \cr
    &&\cr
    \hline
    &&\cr
    {\Large 12} & \Large 25 & \cr
    &&\cr
    \hline
    &&\cr
    {\Large 13} & \Large 20 & \cr
    &&\cr
    \hline\hline
    &&\cr
    {\large\bf Total} & \Large 100 & \cr
    &&\cr
    \hline
  \end{tabular}
\end{center}
\newpage

%%%%%%%%%%%%%%%%%%%%%%%%%%%%%%%%%%%%% Page 2
\noindent{\large\bf MATH 241}\hfill{\large\bf Final Exam.}\hfill{\large\bf
  Fall 2011}\hfill{\large\bf Page 2/14}\hrule

\bigskip
{\problem[15 pts] \em  Find the distance $d$ from the point $(3,7,-5)$ to the $z$--axis.} 
\vspace{8.5cm}
\begin{flushright}
  \begin{tikzpicture}
    \draw (-0.5cm,0.5cm) node {$d = $};
    \draw (0cm,-0.2cm) rectangle (5cm,1.2cm);
  \end{tikzpicture}
\end{flushright}
\hrule
{\problem[15 pts] \em Find an exact expression for the angle $\theta$ between
the vectors $\boldsymbol{v}=\langle 3, -1, 5\rangle$ and
$\boldsymbol{w}=\langle -2, 4, 3\rangle$.} 
\vspace{8.5cm}
\begin{flushright}
  \begin{tikzpicture}
    \draw (-0.5cm,0.5cm) node {$\theta = $};
    \draw (0cm,-0.2cm) rectangle (5cm,1.2cm);
  \end{tikzpicture}
\end{flushright}
\newpage

%%%%%%%%%%%%%%%%%%%%%%%%%%%%%%%%%%%%% Page 3
\noindent{\large\bf MATH 241}\hfill{\large\bf Final Exam.}\hfill{\large\bf
  Fall 2011}\hfill{\large\bf Page 3/14}\hrule

\bigskip
{\problem[15 pts] \em Find the length $\ell$ of the curve $\boldsymbol{r}(t) =
\boldsymbol{i} + t^2 \boldsymbol{j} + t^3 \boldsymbol{k}$ for $0 \leq t \leq
1$.} 
\vspace{9.5cm}
\begin{flushright}
  \begin{tikzpicture}
    \draw (-0.5cm,0.5cm) node {$\ell = $};
    \draw (0cm,-0.2cm) rectangle (5cm,1.2cm);
  \end{tikzpicture}
\end{flushright}
\hrule
{\problem[10 pts] \em At what points does the helix $\boldsymbol{r}(t) = \langle \sin t, \cos t, t \rangle$ intersect the sphere $x^2+y^2+z^2=5$?}
\vspace{7.5cm}
\begin{flushright}
  \begin{tikzpicture}
    \draw (-1cm,0.5cm) node {points: };
    \draw (0cm,-0.2cm) rectangle (5cm,1.2cm);
  \end{tikzpicture}
\end{flushright}
\newpage

%%%%%%%%%%%%%%%%%%%%%%%%%%%%%%%%%%%%% Page 4
\noindent{\large\bf MATH 241}\hfill{\large\bf Final Exam.}\hfill{\large\bf
  Fall 2011}\hfill{\large\bf Page 4/14}\hrule

\bigskip
{\problem[15 pts] \em Find a unit vector $\boldsymbol{v}$ that is orthogonal
to both $\boldsymbol{i} + \boldsymbol{j}$ and $\boldsymbol{i} +
\boldsymbol{k}$.} 
\vspace{11cm}
\begin{flushright}
  \begin{tikzpicture}
    \draw (-0.5cm,0.5cm) node {$\boldsymbol{v} = $};
    \draw (0cm,-0.2cm) rectangle (5cm,1.2cm);
  \end{tikzpicture}
\end{flushright}
\hrule
{\problem[10 pts] \em Determine whether the points $A=(0,-5,5)$, $ B=
(1,-2,4)$ and $C=(3,4,2)$ lie on a straight line.}
\newpage

%%%%%%%%%%%%%%%%%%%%%%%%%%%%%%%%%%%%% Page 5
\noindent{\large\bf MATH 241}\hfill{\large\bf Final Exam.}\hfill{\large\bf
  Fall 2011}\hfill{\large\bf Page 5/14}\hrule

\bigskip
{\problem[20 pts] \em Find parametric equations for the line of intersections
of the planes $x+y+z=1$ and $x+2y+2z=1$.  Find the angle $\theta$ between the
two planes.}
\vspace{18.5cm}
\begin{flushright}
  \begin{tikzpicture}
    \draw (-2cm,-0.2cm) rectangle (5cm,1.2cm);
    \draw (-0.5cm,2.5cm) node {$\theta = $};
    \draw (0cm,1.7cm) rectangle (5cm,3.2cm);
  \end{tikzpicture}
\end{flushright}

\newpage

%%%%%%%%%%%%%%%%%%%%%%%%%%%%%%%%%%%%% Page 6
\noindent{\large\bf MATH 241}\hfill{\large\bf Final Exam.}\hfill{\large\bf
  Fall 2011}\hfill{\large\bf Page 6/14}\hrule

\bigskip
{\problem[15 pts] \em  Sketch the domain of $f(x,y)=\displaystyle{\frac{\sqrt{4-x^2}}{y^2+3}}$.}
\vspace{11.5cm}
%\begin{flushright}
%  \begin{tikzpicture}
%    \draw (-0.5cm,0.5cm) node {$D = $};
%    \draw (0cm,-0.2cm) rectangle (5cm,1.2cm);
%  \end{tikzpicture}
%\end{flushright}
\hrule
{\problem[15 pts] \em Evaluate the limit, if it exists}
\begin{equation*}
\lim_{(x,y) \to (0,0)} (x^2+y^2) \ln \big( x^2 + y^2 \big)
\end{equation*}
\vspace{6.5cm}
\begin{flushright}
  \begin{tikzpicture}
    %\draw (-0.5cm,0.5cm) node {$\theta = $};
    \draw (0cm,-0.2cm) rectangle (5cm,1.2cm);
  \end{tikzpicture}
\end{flushright}
\newpage

%%%%%%%%%%%%%%%%%%%%%%%%%%%%%%%%%%%%% Page 7
\noindent{\large\bf MATH 241}\hfill{\large\bf Final Exam.}\hfill{\large\bf
  Fall 2011}\hfill{\large\bf Page 7/14}\hrule

\bigskip
{\problem[15 pts] \em The volume of a right circular cone of radius $r$ and height $h$ is $V = \tfrac{1}{3}\pi r^2h$.  Show that if the height remains constant while the radius changes, then the volume satisfies}
\begin{equation*}
\frac{\partial V}{\partial r} = \frac{2V}{r}.
\end{equation*}
\vspace{10.5cm}
%\begin{flushright}
%  \begin{tikzpicture}
%    \draw (-0.5cm,0.5cm) node {$\ell = $};
%    \draw (0cm,-0.2cm) rectangle (5cm,1.2cm);
%  \end{tikzpicture}
%\end{flushright}
\hrule
{\problem[10 pts] \em Use the method of Lagrange multipliers to find the
dimensions of a rectangle with perimeter $p$ and maximum area.}
\vspace{4.5cm}
\begin{flushright}
  \begin{tikzpicture}
    \draw (-1cm,2.0cm) node {width: };
    \draw (0cm, 1.3cm) rectangle (5cm,2.7cm);
    \draw (-1cm,0.5cm) node {height: };
    \draw (0cm,-0.2cm) rectangle (5cm,1.2cm);
  \end{tikzpicture}
\end{flushright}
\newpage

%%%%%%%%%%%%%%%%%%%%%%%%%%%%%%%%%%%%% Page 8
\noindent{\large\bf MATH 241}\hfill{\large\bf Final Exam.}\hfill{\large\bf
  Fall 2011}\hfill{\large\bf Page 8/14}\hrule

\bigskip
{\problem[15 pts] \em Recall the formula for the volume of a right circular cone of radius $r$ and height $h$.  Suppose that the height decreases from 20 to 19.95 inches, and the radius increases from 4 to 4.05 inches.  Compare the change in volume of the cone with an approximation of this change using a total differential.}
\vspace{9cm}
\begin{flushright}
  \begin{tikzpicture}
    \draw (-0.6cm,0.5cm) node {$\Delta V = $};
    \draw (0cm,-0.2cm) rectangle (5cm,1.2cm);
    \draw (-0.6cm,2.5cm) node {$dV = $};
    \draw (0cm,1.8cm) rectangle (5cm,3.2cm);
  \end{tikzpicture}
\end{flushright}
\hrule
{\problem[10 pts] \em Find an equation for the tangent plane to the surface $z=xe^{-y}$ at the point $P=(1,0,1)$.}
\vspace{4cm}
\begin{flushright}
  \begin{tikzpicture}
    \draw (-1.4cm,0.5cm) node {tangent plane: };
    \draw (0cm,-0.2cm) rectangle (5cm,1.2cm);
  \end{tikzpicture}
\end{flushright}
\newpage

%%%%%%%%%%%%%%%%%%%%%%%%%%%%%%%%%%%%% Page 9
\noindent{\large\bf MATH 241}\hfill{\large\bf Final Exam.}\hfill{\large\bf
  Fall 2011}\hfill{\large\bf Page 9/14}\hrule

\bigskip
{\problem[20 pts] \em Find the absolute extrema of the function $f(x,y) = xy-x-3y$ on the triangular region $R$ with vertices $(0,0)$, $(0,4)$ and $(5,0)$.}
\vspace{18.5cm}
\begin{flushright}
  \begin{tikzpicture}
    \draw (-1.5cm,0.5cm) node {absolute min:};
    \draw (0cm,-0.2cm) rectangle (5cm,1.2cm);
    \draw (-1.5cm,2.5cm) node {absolute max:};
    \draw (0cm,1.7cm) rectangle (5cm,3.2cm);
  \end{tikzpicture}
\end{flushright}
\newpage


%%%%%%%%%%%%%%%%%%%%%%%%%%%%%%%%%%%%% Page 10
\noindent{\large\bf MATH 241}\hfill{\large\bf Final Exam.}\hfill{\large\bf
  Fall 2011}\hfill{\large\bf Page 10/14}\hrule

\bigskip
{\problem[15 pts] \em  Evaluate $\iint_R y \sin(xy)\, dA$, where
  $R=[1,2] \times [0,\pi]$.}
\vspace{8cm}
\begin{flushright}
 \begin{tikzpicture}
   \draw (0cm,-0.2cm) rectangle (5cm,1.2cm);
 \end{tikzpicture}
\end{flushright}
\hrule
{\problem[15 pts] \em Find the volume of the solid $S$ that is bounded
  by the elliptic paraboloid $x^2+2y^2+z=16$, the planes $x=2$ and
  $y=2$, and the three coordinate planes.}
\vspace{9cm}
\begin{flushright}
  \begin{tikzpicture}
    \draw (-0.5cm,0.5cm) node {$V = $};
    \draw (0cm,-0.2cm) rectangle (5cm,1.2cm);
  \end{tikzpicture}
\end{flushright}
\newpage

%%%%%%%%%%%%%%%%%%%%%%%%%%%%%%%%%%%%% Page 11
\noindent{\large\bf MATH 241}\hfill{\large\bf Final Exam.}\hfill{\large\bf
  Fall 2011}\hfill{\large\bf Page 11/14}\hrule

\bigskip
{\problem[15 pts] \em Evaluate the integral $\iint_D \sin(y^2)\, dA$
  where $D$ is the triangle with vertices $(0,0)$, $(1,1)$ and
  $(0,1)$.}
\vspace{8cm}
\begin{flushright}
 \begin{tikzpicture}
   \draw (0cm,-0.2cm) rectangle (5cm,1.2cm);
 \end{tikzpicture}
\end{flushright}
\hrule
{\problem[10 pts] \em Find the volume of the solid that lies under the
  paraboloid $z=x^2+y^2$, above the $xy$--plane, and inside the
  cylinder $x^2+y^2=2x$.}
\vspace{8cm}
\begin{flushright}
  \begin{tikzpicture}
    \draw (-1cm,0.5cm) node {$V =$};
    \draw (0cm,-0.2cm) rectangle (5cm,1.2cm);
  \end{tikzpicture}
\end{flushright}
\newpage

%%%%%%%%%%%%%%%%%%%%%%%%%%%%%%%%%%%%% Page 12
\noindent{\large\bf MATH 241}\hfill{\large\bf Final Exam.}\hfill{\large\bf
  Fall 2011}\hfill{\large\bf Page 12/14}\hrule

\bigskip
{\problem[10 pts] \em Evaluate $\iiint_E \sqrt{x^2+z^2}\, dV$, where
  $E$ is the region bounded by the paraboloid $y=x^2+z^2$ and the
  plane $y=4$.}
\vspace{8cm}
\begin{flushright}
  \begin{tikzpicture}
    \draw (0cm,1.8cm) rectangle (5cm,3.2cm);
  \end{tikzpicture}
\end{flushright}
\hrule
{\problem[15 pts] \em Evaluate $\int_{-2}^2
  \int_{-\sqrt{4-x^2}}^{\sqrt{4-x^2}} \int_{\sqrt{x^2+y^2}}^2
  (x^2+y^2)\, dz\, dy\, dx$ by passing the description of the region
  $E$ in terms of cylindrical coordinates (Trust me, it is
  \textbf{way} easier than evaluating the integral above directly)}
\vspace{8cm}
\begin{flushright}
  \begin{tikzpicture}
    \draw (0cm,-0.2cm) rectangle (5cm,1.2cm);
  \end{tikzpicture}
\end{flushright}
\newpage

%%%%%%%%%%%%%%%%%%%%%%%%%%%%%%%%%%%%% Page 13
\noindent{\large\bf MATH 241}\hfill{\large\bf Final Exam.}\hfill{\large\bf
  Fall 2011}\hfill{\large\bf Page 13/14}\hrule

\bigskip
{\problem[20 pts] \em A transformation is defined by the equations
  $x=u^2-v^2, y=2uv$.}
\begin{enumerate}
\item Find the image of the square $S=\big\{ (u,v) : 0 \leq u \leq 1,
  0 \leq v \leq 1 \big\}$.
\item Use the same change of variables to evaluate the integral
  $\iint_R y\, dA$, where $R$ is the region bounded by the $x$--axis
  and the parabolas $y^2=4-4x$ and $y^2=4+4x$, $y\geq 0$.
\end{enumerate}
\vspace{14.5cm}
\begin{flushright}
  \begin{tikzpicture}
    \draw (-1.5cm,0.5cm) node {$\iint_R y\, dA =$};
    \draw (0cm,-0.2cm) rectangle (5cm,1.2cm);
    \draw (-1.5cm,2.5cm) node {Image of $S$:};
    \draw (0cm,1.7cm) rectangle (5cm,3.2cm);
  \end{tikzpicture}
\end{flushright}
\end{document}
