\documentclass[12pt]{article}

\usepackage{amsmath,amsthm,amsfonts,amssymb,amsxtra}
\usepackage{pgf,tikz}
\usetikzlibrary{arrows}
\renewcommand{\theenumi}{(\alph{enumi})} 
\renewcommand{\labelenumi}{\theenumi}

\pagestyle{empty}
\setlength{\textwidth}{7in}
\setlength{\oddsidemargin}{-0.5in}
\setlength{\topmargin}{-1.0in}
\setlength{\textheight}{9.5in}

\newtheorem{problem}{Problem}

\begin{document}

\noindent{\large\bf MATH 241}\hfill{\large\bf Exam\#2.}\hfill{\large\bf
  Fall 2011}\hfill{\large\bf Page 1/5}\hrule

\bigskip
\begin{center}
  \begin{tabular}{|ll|}
    \hline & \cr
    {\bf Name: } & \makebox[12cm]{\hrulefill}\cr & \cr
    {\bf 4-digit code:} & \makebox[12cm]{\hrulefill}\cr & \cr
    \hline
  \end{tabular}
\end{center}
\begin{itemize}
\item Write your name and the last 4 digits of your SSN in the space provided above.
\item The test has five (5) pages, including this one.
\item Enter your answer in the box(es) provided.
\item You must show sufficient work to justify all answers unless
  otherwise stated in the problem.  Correct answers with inconsistent
  work may not be given credit.
\item Credit for each problem is given in parentheses at the right of
  the problem number.
\item No books, notes or calculators may be used on this test.
\end{itemize}
\hrule

\begin{center}
  \begin{tabular}{|c|c|c|}
    \hline
    &&\cr
    {\large\bf Page} & {\large\bf Max.~points} & {\large\bf Your points} \cr
    &&\cr
    \hline
    &&\cr
    {\Large 2} & \Large 30 & \cr
    &&\cr
    \hline
    &&\cr
    {\Large 3} & \Large 25 & \cr
    &&\cr
    \hline
    &&\cr
    {\Large 4} & \Large 25 & \cr
    &&\cr
    \hline
    &&\cr
    {\Large 5} & \Large 20 & \cr
    &&\cr
    \hline\hline
    &&\cr
    {\large\bf Total} & \Large 100 & \cr
    &&\cr
    \hline
  \end{tabular}
\end{center}
\newpage

%%%%%%%%%%%%%%%%%%%%%%%%%%%%%%%%%%%%% Page 2
\noindent{\large\bf MATH 241}\hfill{\large\bf Exam\#2.}\hfill{\large\bf
  Fall 2011}\hfill{\large\bf Page 2/5}\hrule

\bigskip
{\problem[15 pts] \em  Sketch the domain of $f(x,y)=\displaystyle{\frac{\sqrt{4-x^2}}{y^2+3}}$.}
\vspace{11.5cm}
%\begin{flushright}
%  \begin{tikzpicture}
%    \draw (-0.5cm,0.5cm) node {$D = $};
%    \draw (0cm,-0.2cm) rectangle (5cm,1.2cm);
%  \end{tikzpicture}
%\end{flushright}
\hrule
{\problem[15 pts] \em Evaluate the limit, if it exists}
\begin{equation*}
\lim_{(x,y) \to (0,0)} (x^2+y^2) \ln \big( x^2 + y^2 \big)
\end{equation*}
\vspace{6.5cm}
\begin{flushright}
  \begin{tikzpicture}
    %\draw (-0.5cm,0.5cm) node {$\theta = $};
    \draw (0cm,-0.2cm) rectangle (5cm,1.2cm);
  \end{tikzpicture}
\end{flushright}
\newpage

%%%%%%%%%%%%%%%%%%%%%%%%%%%%%%%%%%%%% Page 3
\noindent{\large\bf MATH 241}\hfill{\large\bf Exam\#2.}\hfill{\large\bf
  Fall 2011}\hfill{\large\bf Page 3/5}\hrule

\bigskip
{\problem[15 pts] \em The volume of a right circular cone of radius $r$ and height $h$ is $V = \tfrac{1}{3}\pi r^2h$.  Show that if the height remains constant while the radius changes, then the volume satisfies}
\begin{equation*}
\frac{\partial V}{\partial r} = \frac{2V}{r}.
\end{equation*}
\vspace{10.5cm}
%\begin{flushright}
%  \begin{tikzpicture}
%    \draw (-0.5cm,0.5cm) node {$\ell = $};
%    \draw (0cm,-0.2cm) rectangle (5cm,1.2cm);
%  \end{tikzpicture}
%\end{flushright}
\hrule
{\problem[10 pts] \em Use the method of Lagrange multipliers to find the
dimensions of a rectangle with perimeter $p$ and maximum area.}
\vspace{4.5cm}
\begin{flushright}
  \begin{tikzpicture}
    \draw (-1cm,2.0cm) node {width: };
    \draw (0cm, 1.3cm) rectangle (5cm,2.7cm);
    \draw (-1cm,0.5cm) node {height: };
    \draw (0cm,-0.2cm) rectangle (5cm,1.2cm);
  \end{tikzpicture}
\end{flushright}
\newpage

%%%%%%%%%%%%%%%%%%%%%%%%%%%%%%%%%%%%% Page 4
\noindent{\large\bf MATH 241}\hfill{\large\bf Exam\#2.}\hfill{\large\bf
  Fall 2011}\hfill{\large\bf Page 4/5}\hrule

\bigskip
{\problem[15 pts] \em Recall the formula for the volume of a right circular cone of radius $r$ and height $h$.  Suppose that the height decreases from 20 to 19.95 inches, and the radius increases from 4 to 4.05 inches.  Compare the change in volume of the cone with an approximation of this change using a total differential.}
\vspace{9cm}
\begin{flushright}
  \begin{tikzpicture}
    \draw (-0.6cm,0.5cm) node {$\Delta V = $};
    \draw (0cm,-0.2cm) rectangle (5cm,1.2cm);
    \draw (-0.6cm,2.5cm) node {$dV = $};
    \draw (0cm,1.8cm) rectangle (5cm,3.2cm);
  \end{tikzpicture}
\end{flushright}
\hrule
{\problem[10 pts] \em Find an equation for the tangent plane to the surface $z=xe^{-y}$ at the point $P=(1,0,1)$.}
\vspace{4cm}
\begin{flushright}
  \begin{tikzpicture}
    \draw (-1.4cm,0.5cm) node {tangent plane: };
    \draw (0cm,-0.2cm) rectangle (5cm,1.2cm);
  \end{tikzpicture}
\end{flushright}
\newpage

%%%%%%%%%%%%%%%%%%%%%%%%%%%%%%%%%%%%% Page 5
\noindent{\large\bf MATH 241}\hfill{\large\bf Exam\#2.}\hfill{\large\bf
  Fall 2011}\hfill{\large\bf Page 5/5}\hrule

\bigskip
{\problem[20 pts] \em Find the absolute extrema of the function $f(x,y) = xy-x-3y$ on the triangular region $R$ with vertices $(0,0)$, $(0,4)$ and $(5,0)$.}
\vspace{18.5cm}
\begin{flushright}
  \begin{tikzpicture}
    \draw (-1.5cm,0.5cm) node {absolute min:};
    \draw (0cm,-0.2cm) rectangle (5cm,1.2cm);
    \draw (-1.5cm,2.5cm) node {absolute max:};
    \draw (0cm,1.7cm) rectangle (5cm,3.2cm);
  \end{tikzpicture}
\end{flushright}
\end{document}
