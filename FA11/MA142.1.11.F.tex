\documentclass[12pt]{article}

\usepackage{amsmath,amsthm,amsfonts,amssymb,amsxtra}
\usepackage{pgf,tikz}
\usetikzlibrary{arrows}
\renewcommand{\theenumi}{(\alph{enumi})} 
\renewcommand{\labelenumi}{\theenumi}

\pagestyle{empty}
\setlength{\textwidth}{7in}
\setlength{\oddsidemargin}{-0.5in}
\setlength{\topmargin}{-1.0in}
\setlength{\textheight}{9.5in}

\newtheorem{problem}{Problem}

\begin{document}

\noindent{\large\bf MATH 142}\hfill{\large\bf Exam\#1.}\hfill{\large\bf
  Fall 2011}\hfill{\large\bf Page 1/5}\hrule

\bigskip
\begin{center}
  \begin{tabular}{|ll|}
    \hline & \cr
    {\bf Name: } & \makebox[12cm]{\hrulefill}\cr & \cr
    {\bf 4-digit code:} & \makebox[12cm]{\hrulefill}\cr & \cr
    \hline
  \end{tabular}
\end{center}
\begin{itemize}
\item Write your name and the last 4 digits of your SSN in the space provided above.
\item The test has five (5) pages, including this one.
\item Enter your answer in the box(es) provided.
\item You must show sufficient work to justify all answers unless
  otherwise stated in the problem.  Correct answers with inconsistent
  work may not be given credit.
\item Credit for each problem is given in parentheses at the right of
  the problem number.
\item No books, notes or calculators may be used on this test.
\end{itemize}
\hrule

\begin{center}
  \begin{tabular}{|c|c|c|}
    \hline
    &&\cr
    {\large\bf Page} & {\large\bf Max.~points} & {\large\bf Your points} \cr
    &&\cr
    \hline
    &&\cr
    {\Large 2} & \Large 25 & \cr
    &&\cr
    \hline
    &&\cr
    {\Large 3} & \Large 25 & \cr
    &&\cr
    \hline
    &&\cr
    {\Large 4} & \Large 25 & \cr
    &&\cr
    \hline
    &&\cr
    {\Large 5} & \Large 25 & \cr
    &&\cr
    \hline\hline
    &&\cr
    {\large\bf Total} & \Large 100 & \cr
    &&\cr
    \hline
  \end{tabular}
\end{center}
\newpage

%%%%%%%%%%%%%%%%%%%%%%%%%%%%%%%%%%%%% Page 2
\noindent{\large\bf MATH 142}\hfill{\large\bf Exam\#1.}\hfill{\large\bf
  Fall 2011}\hfill{\large\bf Page 2/5}\hrule

\bigskip
{\problem[10 pts] \em  Find the area of the region that is enclosed between the
curves $x=2y^2$ and $x=4+y^2$.} 
\vspace{8.5cm}
\begin{flushright}
  \begin{tikzpicture}
    \draw (-1cm,0.5cm) node {$A = $};
    \draw (0cm,-0.2cm) rectangle (5cm,1.2cm);
  \end{tikzpicture}
\end{flushright}
\hrule
{\problem[15 pts] \em Find the volume of the solid that is obtained by rotating
the region bounded by the curves $x=2\sqrt{y}$, $x=0$ and $y=9$ about the
$y$-axis.} \vspace{8.5cm}
\begin{flushright}
  \begin{tikzpicture}
    \draw (-1cm,0.5cm) node {$V = $};
    \draw (0cm,-0.2cm) rectangle (5cm,1.2cm);
  \end{tikzpicture}
\end{flushright}
\newpage

%%%%%%%%%%%%%%%%%%%%%%%%%%%%%%%%%%%%% Page 3
\noindent{\large\bf MATH 142}\hfill{\large\bf Exam\#1.}\hfill{\large\bf
  Fall 2011}\hfill{\large\bf Page 3/5}\hrule

\bigskip
{\problem[15 pts] \em Find the volume of the solid generated when the region
enclosed by the curves $x=1+y^2$, $y=1$, $y=2$ and $x=0$ is revolved about the
$x$-axis.}
\vspace{9.5cm}
\begin{flushright}
  \begin{tikzpicture}
    \draw (-1cm,0.5cm) node {$V = $};
    \draw (0cm,-0.2cm) rectangle (5cm,1.2cm);
  \end{tikzpicture}
\end{flushright}
\hrule
{\problem[10 pts] \em Use the Fundamental Theorem of Calculus to find the derivative of
 \begin{equation*}
f(x) = \int_{1-3x}^1 \frac{u^3}{1+u^2}\, du.
\end{equation*}
\vspace{6.5cm}
\begin{flushright}
  \begin{tikzpicture}
    \draw (-1cm,0.5cm) node {$f'(x) = $};
    \draw (0cm,-0.2cm) rectangle (5cm,1.2cm);
  \end{tikzpicture}
\end{flushright}
\newpage

%%%%%%%%%%%%%%%%%%%%%%%%%%%%%%%%%%%%% Page 4
\noindent{\large\bf MATH 142}\hfill{\large\bf Exam\#1.}\hfill{\large\bf
  Fall 2011}\hfill{\large\bf Page 4/5}\hrule

\bigskip
{\problem[15 pts] \em Find a positive value of $k$ such that the average value
of $f(x) = 2+6x-3x^2$ over the interval $[0,k]$ is equal to $3$.} 
\vspace{8cm}
\begin{flushright}
  \begin{tikzpicture}
    \draw (-1cm,0.5cm) node {$k = $};
    \draw (0cm,-0.2cm) rectangle (5cm,1.2cm);
  \end{tikzpicture}
\end{flushright}
\hrule
{\problem[10 pts] \em Evaluate the integral $\displaystyle{\int \sec^3x \tan
x\, dx}$.} \newpage

%%%%%%%%%%%%%%%%%%%%%%%%%%%%%%%%%%%%% Page 5
\noindent{\large\bf MATH 142}\hfill{\large\bf Exam\#1.}\hfill{\large\bf
  Fall 2011}\hfill{\large\bf Page 5/5}\hrule

\bigskip
{\problem[15 pts] \em Evaluate the integral $\displaystyle{\int_0^{1/\sqrt{3}} \frac{x^2-1}{x^4-1}\, dx}$.}
\vspace{9.5cm}
\begin{flushright}
  \begin{tikzpicture}
    \draw (-1cm,0.5cm) node {$A = $};
    \draw (0cm,-0.2cm) rectangle (5cm,1.2cm);
  \end{tikzpicture}
\end{flushright}
\hrule
{\problem[10 pts] \em Find the average value of the function $f(x) = 1/x$ over the interval $[1,e]$.}
\vspace{7.5cm}
\begin{flushright}
  \begin{tikzpicture}
    \draw (-1cm,0.5cm) node {$f_{ave} = $};
    \draw (0cm,-0.2cm) rectangle (5cm,1.2cm);
  \end{tikzpicture}
\end{flushright}

\end{document}
