\documentclass[12pt]{article}

\usepackage{amsmath,amsthm,amsfonts,amssymb,amsxtra}
\usepackage{pgf,tikz}
\usetikzlibrary{arrows}
\renewcommand{\theenumi}{(\alph{enumi})} 
\renewcommand{\labelenumi}{\theenumi}

\pagestyle{empty}
\setlength{\textwidth}{7in}
\setlength{\oddsidemargin}{-0.5in}
\setlength{\topmargin}{-1.0in}
\setlength{\textheight}{9.5in}

\newtheorem{problem}{Problem}

\begin{document}

\noindent{\large\bf MATH 141}\hfill{\large\bf Final Exam.}\hfill{\large\bf
  Fall 2009}\hfill{\large\bf Page 1/14}\hrule

\bigskip
\begin{center}
  \begin{tabular}{|ll|}
    \hline & \cr
    {\bf Name: } & \makebox[12cm]{\hrulefill}\cr & \cr
    {\bf 4-digit code:} & \makebox[12cm]{\hrulefill}\cr & \cr
    \hline
  \end{tabular}
\end{center}
\begin{itemize}
\item Write your name and the last 4 digits of your SSN in the space provided above.
\item The test has fourteen (14) pages, including this one.
\item You must show sufficient work to justify all answers unless
  otherwise stated in the problem.  Correct answers with inconsistent
  work may not be given credit. Credit for each problem is given in
  parentheses at the right of the problem number.
\item No books, notes or calculators may be used on this test.
\item \textbf{A:} 243--270 pts. \textbf{B+:} 230--242 pts. \textbf{B:} 216--229 pts. \textbf{C+:} 203--215 pts. \textbf{C:} 189--202 pts.\\ \textbf{D+:} 175--188 pts. \textbf{D:} 160--174 pts. \textbf{F}: less than 160 pts.
\end{itemize}
\hrule

\begin{center}
  \begin{tabular}{|c|c|c|}
    \hline
    &&\cr
    {\large\bf Page} & {\large\bf Max.~points} & {\large\bf Your points} \cr
    &&\cr
    \hline
    &&\cr
    {\Large 1} & \Large --- & \cr
    &&\cr
    \hline
    &&\cr
    {\Large 2} & \Large 20 & \cr
    &&\cr
    \hline
    &&\cr
    {\Large 3} & \Large 15 & \cr
    &&\cr
    \hline
    &&\cr
    {\Large 4} & \Large 15 & \cr
    &&\cr
    \hline
    &&\cr
    {\Large 5} & \Large 25 & \cr
    &&\cr
    \hline
    &&\cr
    {\Large 6} & \Large 25 & \cr
    &&\cr
    \hline
    &&\cr
    {\Large 7} & \Large 20 & \cr
    &&\cr
    \hline\hline
    &&\cr
    {\large\bf Total} & \Large 120 & \cr
    &&\cr
    \hline
  \end{tabular}
  \begin{tabular}{|c|c|c|}
    \hline
    &&\cr
    {\large\bf Page} & {\large\bf Max.~points} & {\large\bf Your points} \cr
    &&\cr
    \hline
    &&\cr
    {\Large 8} & \Large 30 & \cr
    &&\cr
    \hline
    &&\cr
    {\Large 9} & \Large 20 & \cr
    &&\cr
    \hline
    &&\cr
    {\Large 10} & \Large 30 & \cr
    &&\cr
    \hline
    &&\cr
    {\Large 11} & \Large 25 & \cr
    &&\cr
    \hline
    &&\cr
    {\Large 12} & \Large 30 & \cr
    &&\cr
    \hline
    &&\cr
    {\Large 13} & \Large 20 & \cr
    &&\cr
    \hline
    &&\cr
    {\Large 14} & \Large 25 & \cr
    &&\cr
    \hline\hline
    &&\cr
    {\large\bf Total} & \Large 180 & \cr
    &&\cr
    \hline
  \end{tabular}
\end{center}
\newpage

%%%%%%%%%%%%%%%%%%%%%%%%%%%%%%%%%%%%% First Midterm
%%%%%%%%%%%%%%%%%%%%%%%%%%%%%%%%%%%%% Page 2
\noindent{\large\bf MATH 141}\hfill{\large\bf Final Exam.}\hfill{\large\bf
  Fall 2009}\hfill{\large\bf Page 2/14}\hrule

\bigskip
{\problem[5 pts] \em Find $f(0)$ and $f(\pi/2)$ for}
$f(x) = 
\begin{cases}
 \sqrt{x+1} & \text{if } x\geq 1,\\
3 & \text{if } x<1.
\end{cases}$
\vspace{1cm}
\begin{flushright}
  \begin{tikzpicture}
    \draw (-1cm,0.5cm) node {$f(\pi/2) =$};
    \draw (0cm,0cm) rectangle (5cm,1.2cm);
    \draw (-1cm,2cm) node {$f(0) =$};
    \draw (0cm,1.4cm) rectangle (5cm,2.6cm);
  \end{tikzpicture}
\end{flushright}
\hrule
{\problem[10 pts] \em Find the domain of $f(x) = \sqrt{(x-1)(x-2)}$.}
\vspace{6cm}
\begin{flushright}
  \begin{tikzpicture}
    \draw (-1cm,0.5cm) node {domain $=$};
    \draw (0cm,0cm) rectangle (5cm,1.2cm);
  \end{tikzpicture}
\end{flushright}
\hrule
{\problem[5 pts] \em Express the function $f(x) = \lvert x-1 \rvert$
  in piecewise form without using absolute values.}     
\vspace{4cm}
\begin{flushright}
  \begin{tikzpicture}
    \draw (-1cm,0.5cm) node {$f(x) = {\Biggr\{}$};
    \draw (0cm,-0.6cm) rectangle (5cm,1.8cm);
  \end{tikzpicture}
\end{flushright}
\newpage

%%%%%%%%%%%%%%%%%%%%%%%%%%%%%%%%%%%%% Page 3
\noindent{\large\bf MATH 141}\hfill{\large\bf Final Exam.}\hfill{\large\bf
  Fall 2009}\hfill{\large\bf Page 3/14}\hrule

\bigskip
{\problem[5 pts] \em Let $f(x) = x^2 + 4$ and $g(x) = \sqrt{x}$. Find
  $\big( g \circ f\big)(x)$.}
\vspace{2cm}
\begin{flushright}
  \begin{tikzpicture}
    \draw (-1.25cm,0.5cm) node {$ \big(g \circ f\big)(x) =$};
    \draw (0cm,0cm) rectangle (5cm,1.2cm);
  \end{tikzpicture}
\end{flushright}

\hrule
{\problem[10 pts] \em Recall the ``$\varepsilon$--$\delta$'' definition of limit:
\begin{center}
\begin{tikzpicture}
\draw (0.5\linewidth, 0cm) node[text justified, text width=0.5\linewidth, draw, rounded corners] {
We write $\displaystyle{\lim_{x\to a}} f(x) = L$ if for all $\varepsilon
> 0$ there exists $\delta >0$ such that $\lvert x - a \rvert < \delta$
implies $\lvert f(x) - L \rvert < \varepsilon$.}; 
\end{tikzpicture}
\end{center}
Use this definition to prove that $\displaystyle{\lim_{x \to 2} (-x-2) =
  -4}$.}
\newpage

%%%%%%%%%%%%%%%%%%%%%%%%%%%%%%%%%%%%% Page 4
\noindent{\large\bf MATH 141}\hfill{\large\bf Final Exam.}\hfill{\large\bf
  Fall 2009}\hfill{\large\bf Page 4/14}\hrule

\bigskip
{\problem[5 pts] \em Solve for $x$:}
\begin{equation*}
  \ln x + \ln (x-1) = 1
\end{equation*}
\vspace{3cm}
\begin{flushright}
  \begin{tikzpicture}
    \draw (-0.75cm,0.5cm) node {$x =$};
    \draw (0cm,0cm) rectangle (5cm,1.2cm);
  \end{tikzpicture}
\end{flushright}
\hrule
 
{\problem[10 pts] \em Compute the derivatives of the following functions.}
\begin{enumerate}
\item $f(x) = \pi \sqrt{x} (x^4 - 4 x^3 + 6 x^2 -4 x^1 + 1 - x^{-1})$
\vspace{4cm}
\begin{flushright}
  \begin{tikzpicture}
    \draw (-0.75cm,0.5cm) node {$f'(x) =$};
    \draw (0cm,0cm) rectangle (5cm,1.2cm);
  \end{tikzpicture}
\end{flushright}
\item $\displaystyle{g(t) = \frac{t^2-5}{t^{-1}}}$
\vspace{4cm}
\begin{flushright}
  \begin{tikzpicture}
    \draw (-0.75cm,0.5cm) node {$g'(t) =$};
    \draw (0cm,0cm) rectangle (5cm,1.2cm);
  \end{tikzpicture}
\end{flushright}
\end{enumerate}

\newpage

%%%%%%%%%%%%%%%%%%%%%%%%%%%%%%%%%%%%% Page 5
\noindent{\large\bf MATH 141}\hfill{\large\bf Final Exam.}\hfill{\large\bf
  Fall 2009}\hfill{\large\bf Page 5/14}\hrule

\bigskip
{\problem[15 pts] \em Compute the following limits:}
\vspace{1cm}

\noindent
\begin{tikzpicture}
\draw (4cm,14cm) node{
(a) $\displaystyle{\lim_{x \to 2} \frac{x^2-2x-8}{x^2-4}} = \mbox{}$ };
\draw (6cm,13.4cm) rectangle (11cm,14.6cm);
\draw (4cm, 10cm) node{
(b) $\displaystyle{\lim_{x \to -\infty} \frac{x^2-2x-8}{x^2-4}} =
\mbox{}$}; 
\draw (6.2cm, 9.4cm) rectangle (11.1cm, 10.6cm); 
\draw (4cm, 6cm) node{
(b) $\displaystyle{\lim_{x \to -2} \frac{x^2-2x-8}{x^2-4}} =
\mbox{}$}; 
\draw (6.1cm, 5.4cm) rectangle (11.1cm, 6.6cm); 
\end{tikzpicture}

\vspace{3cm}
\hrule

{\problem[10 pts] \em Find the value of the constant $k$ for which the
  following function is continuous everywhere:} 
\begin{equation*}
f(x) = \begin{cases}
2k^2x^3 &\text{if }x<2, \\
x+32k-18 &\text{if }x \geq 2.
\end{cases}
\end{equation*}
\vspace{5cm}
\begin{flushright}
  \begin{tikzpicture}
    \draw (-1cm,0.5cm) node {$k =$};
    \draw (0cm,0cm) rectangle (5cm,1.2cm);
  \end{tikzpicture}
\end{flushright}

\newpage


%%%%%%%%%%%%%%%%%%%%%%%%%%%%%%%%%%%%% Page 6
\noindent{\large\bf MATH 141}\hfill{\large\bf Final Exam.}\hfill{\large\bf
  Fall 2009}\hfill{\large\bf Page 6/14}\hrule

\bigskip
{\problem[15 pts] \em Find equations of the tangent lines to the curve}
\begin{equation*}
  y = \frac{x-1}{x+1}
\end{equation*}
that are parallel to the line $x-\tfrac{9}{2} y=3$.
\vspace{10cm}
\hrule
{\problem[10 pts] \em How many tangent lines to the curve $y=x/(x+1)$
  pass through the point $(0,0)$.}      

\noindent{\textbf{HINT:} \em You do not have to compute the equations of
  the lines.}
\newpage
%%%%%%%%%%%%%%%%%%%%%%%%%%%%%%%%%%%%% Second Midterm
%%%%%%%%%%%%%%%%%%%%%%%%%%%%%%%%%%%%% Page 7
\noindent{\large\bf MATH 141}\hfill{\large\bf Final Exam.}\hfill{\large\bf
  Fall 2009}\hfill{\large\bf Page 7/14}\hrule

\bigskip
{\problem[10 pts] \em  Evaluate each limit:} 

\bigskip
\begin{tikzpicture}
\draw (4cm,14cm) node{
$\displaystyle{ \lim_{x\to 0} \frac{e^x-1}{\tan x} = \mbox{}}$ };
\draw (5.5cm,13.4cm) rectangle (8.5cm,14.6cm);
\end{tikzpicture}

\vspace{4cm}
\begin{tikzpicture}
\draw (4cm,14cm) node{
$\displaystyle{ \lim_{x\to 0} \frac{\tan(5x^2)}{x^2} = \mbox{}}$ };
\draw (5.5cm,13.4cm) rectangle (8.5cm,14.6cm);
\end{tikzpicture}
\vspace{5cm}

\hrule
{\problem[10 pts] \em Find an equation of the normal line to the
  curve $y=\ln(x e^{x^2})$ at the point $(1,1)$.} 
\newpage

%%%%%%%%%%%%%%%%%%%%%%%%%%%%%%%%%%%%% Page 8
\noindent{\large\bf MATH 141}\hfill{\large\bf Final Exam.}\hfill{\large\bf
  Fall 2009}\hfill{\large\bf Page 8/14}\hrule

\bigskip
{\problem[30 pts] \em Sketch the graph of the rational function $f(x)
  = \displaystyle{\frac{2x^2-8}{x^2-16}}$.}
\begin{quotation}
Indicate clearly:
\begin{itemize}
\item Domain
\item $x$- and $y$-intercepts.
\item Vertical and horizontal asymptotes (any holes?).
\item Intervals of increase, decrease and different concavity.
\item Location of relative extrema and inflection points. 
\end{itemize}
\end{quotation}
\newpage

%%%%%%%%%%%%%%%%%%%%%%%%%%%%%%%%%%%%% Page 9
\noindent{\large\bf MATH 141}\hfill{\large\bf Final Exam.}\hfill{\large\bf
  Fall 2009}\hfill{\large\bf Page 9/14}\hrule

\bigskip
{\problem[10 pts] \em Find the absolute extrema of $f(x) = 6x^{4/3} - 3x^{1/3}$ on the interval $[-1,1]$.}
\vspace{6cm}
\begin{flushright}
  \begin{tikzpicture}
    \draw (-2.25cm,2.5cm) node {Absolute maxima at };
    \draw (0cm,1.8cm) rectangle (5cm,3.2cm);
    \draw (-2.25cm,0.5cm) node {Absolute minima at };
    \draw (0cm,-0.2cm) rectangle (5cm,1.2cm);
  \end{tikzpicture}
\end{flushright}
\hrule
{\problem[10 pts] \em Use logarithmic differentiation to find the
  derivative of the function 
\begin{equation*}
y=\frac{\sin^2 x \tan^4 x}{(x^2+1)^2}
\end{equation*}
\vspace{7cm}
\begin{flushright}
  \begin{tikzpicture}
    \draw (-0.7cm,0.5cm) node {$\displaystyle{\frac{dy}{dx}}=$ };
    \draw (0cm,-0.2cm) rectangle (5cm,1.2cm);
  \end{tikzpicture}
\end{flushright}
\newpage

%%%%%%%%%%%%%%%%%%%%%%%%%%%%%%%%%%%%% Page 10
\noindent{\large\bf MATH 141}\hfill{\large\bf Final Exam.}\hfill{\large\bf
  Fall 2009}\hfill{\large\bf Page 10/14}\hrule

\bigskip
{\problem[10 pts] \em An aircraft is climbing at $30^o$ angle to the horizontal.  How fast is the aircraft gaining altitude if its speed is 500 mi/h?}
\vspace{6cm}
\begin{flushright}
  \begin{tikzpicture}
    \draw (-4.25cm,0.5cm) node {The aircraft is gaining altitude at a speed of};
    \draw (0cm,-0.2cm) rectangle (5cm,1.2cm);
  \end{tikzpicture}
\end{flushright}
\hrule
{\problem[20 pts] \em A container with square base, vertical sides, and open top is to be made from 300 ft$^2$ of material.  Find the dimensions of the container with greatest volume.}
\vspace{11cm}
\begin{flushright}
  \begin{tikzpicture}
    \draw (-2.7cm,0.5cm) node {Dimensions of container: };
    \draw (0cm,-0.2cm) rectangle (5cm,1.2cm);
  \end{tikzpicture}
\end{flushright}
\newpage

%3M%%%%%%%%%%%%%%%%%%%%%%%%%%%%%%%%%%%% Third Midterm
%%%%%%%%%%%%%%%%%%%%%%%%%%%%%%%%%%%%% Page 11
\noindent{\large\bf MATH 141}\hfill{\large\bf Final Exam.}\hfill{\large\bf
  Fall 2009}\hfill{\large\bf Page 11/14}\hrule

\bigskip
{\problem[25 pts] \em  Evaluate each integral:} 
\begin{enumerate}
\item $\displaystyle{\int_0^2 \big( 5x + \frac{2}{3x^5} - \sqrt{2} e^x \big)\, dx}$
\vspace{2.5cm}
\item $\displaystyle{\int \big( 3\sin x - 2\sec^2 x \big)\, dx}$
\vspace{2.5cm}
\item $\displaystyle{\int ( 1 + \sin t)^{90} \cos t\, dt}$
\vspace{2.5cm}
\item $\displaystyle{\int_0^1 \frac{5x^4}{(x^5+1)^2}\, dx}$
\vspace{2.5cm}
\item $\displaystyle{\int \frac{3x-2}{(x-1)(x+1)^2}\, dx}$
\end{enumerate}
\newpage

%%%%%%%%%%%%%%%%%%%%%%%%%%%%%%%%%%%%% Page 12
\noindent{\large\bf MATH 141}\hfill{\large\bf Final Exam.}\hfill{\large\bf
  Fall 2009}\hfill{\large\bf Page 12/14}\hrule

\bigskip
{\problem[20 pts] \em Express the following functions of $n$ in closed
  form and then find the limit.}
\begin{enumerate}
\item  $\displaystyle{\lim_{n \to \infty} \frac{1^2+2^2+3^2+ \dotsb +
      n^2}{n^3}}$
\vspace{5cm}
\item  $\displaystyle{\lim_{n \to \infty} \sum_{k=1}^n \frac{5k}{n^2}}$
\vspace{5cm}
\end{enumerate}
{\problem[10 pts] \em Use the definition of \textbf{definite integral} to
  express $\int_{-\pi/2}^{\pi/2} (1+ \cos x)\, dx$ as a limit.}
\newpage

%%%%%%%%%%%%%%%%%%%%%%%%%%%%%%%%%%%%% Page 13
\noindent{\large\bf MATH 141}\hfill{\large\bf Final Exam.}\hfill{\large\bf
  Fall 2009}\hfill{\large\bf Page 13/14}\hrule

\bigskip
{\problem[10 pts] \em Use the Fundamental Theorem of Calculus to find
  the derivative of the following functions.}
\begin{enumerate}
\item $\displaystyle{g(x) = \int_1^x \frac{1}{t^3+1}\, dt}$
\vspace{2cm}
\begin{flushright}
  \begin{tikzpicture}
    \draw (-1cm,2.5cm) node {$g'(x) =$};
    \draw (0cm,1.8cm) rectangle (5cm,3.2cm);
 \end{tikzpicture}
\end{flushright}
\item $\displaystyle{g(y) = \int_x^\pi \sqrt{1+ \sec t}\, dt}$
\vspace{2cm}
\begin{flushright}
  \begin{tikzpicture}
    \draw (-1cm,2.5cm) node {$g'(y) =$};
    \draw (0cm,1.8cm) rectangle (5cm,3.2cm);
 \end{tikzpicture}
\end{flushright}
\end{enumerate}
\hrule
{\problem[10 pts] \em Find the antiderivative $F$ of $f(x) =
  4-3(1+x^2)^{-1}$ that satisfies $F(1) = 0$.} 
\vspace{5cm}
\begin{flushright}
  \begin{tikzpicture}
    \draw (-1cm,2.5cm) node {$F(x) =$};
    \draw (0cm,1.8cm) rectangle (5cm,3.2cm);
 \end{tikzpicture}
\end{flushright}
\newpage

%%%%%%%%%%%%%%%%%%%%%%%%%%%%%%%%%%%%% Page 14
\noindent{\large\bf MATH 141}\hfill{\large\bf Final Exam.}\hfill{\large\bf
  Fall 2009}\hfill{\large\bf Page 14/14}\hrule

\bigskip
{\problem[25 pts] \em Sketch the region enclosed by the curves
  $y=x^2$, $y=4x-x^2$, and find the corresponding area.}
\vspace{20cm}
\begin{flushright}
  \begin{tikzpicture}
    \draw (-1cm,0.5cm) node {Area: };
    \draw (0cm,-0.2cm) rectangle (5cm,1.2cm);
  \end{tikzpicture}
\end{flushright}
\end{document}
