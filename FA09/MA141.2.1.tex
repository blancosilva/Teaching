\documentclass[12pt]{article}

\usepackage{amsmath,amsthm,amsfonts,amssymb,amsxtra}
\usepackage{pgf,tikz}
\usetikzlibrary{arrows}
\renewcommand{\theenumi}{(\alph{enumi})} 
\renewcommand{\labelenumi}{\theenumi}

\pagestyle{empty}
\setlength{\textwidth}{7in}
\setlength{\oddsidemargin}{-0.5in}
\setlength{\topmargin}{-1.0in}
\setlength{\textheight}{9.5in}

\newtheorem{problem}{Problem}

\begin{document}

\noindent{\large\bf MATH 141}\hfill{\large\bf Exam\#2(A).}\hfill{\large\bf
  Fall 2009}\hfill{\large\bf Page 1/5}\hrule

\bigskip
\begin{center}
  \begin{tabular}{|ll|}
    \hline & \cr
    {\bf Name: } & \makebox[12cm]{\hrulefill}\cr & \cr
    {\bf 4-digit code:} & \makebox[12cm]{\hrulefill}\cr & \cr
    \hline
  \end{tabular}
\end{center}
\begin{itemize}
\item Write your name and the last 4 digits of your SSN in the space provided above.
\item The test has five (5) pages, including this one.
\item Enter your answer in the box(es) provided.
\item You must show sufficient work to justify all answers unless
  otherwise stated in the problem.  Correct answers with inconsistent
  work may not be given credit.
\item Credit for each problem is given in parentheses at the right of
  the problem number.
\item No books, notes or calculators may be used on this test.
\end{itemize}
\hrule

\begin{center}
  \begin{tabular}{|c|c|c|}
    \hline
    &&\cr
    {\large\bf Page} & {\large\bf Max.~points} & {\large\bf Your points} \cr
    &&\cr
    \hline
    &&\cr
    {\Large 2} & \Large 20 & \cr
    &&\cr
    \hline
    &&\cr
    {\Large 3} & \Large 30 & \cr
    &&\cr
    \hline
    &&\cr
    {\Large 4} & \Large 20 & \cr
    &&\cr
    \hline
    &&\cr
    {\Large 5} & \Large 30 & \cr
    &&\cr
    \hline\hline
    &&\cr
    {\large\bf Total} & \Large 100 & \cr
    &&\cr
    \hline
  \end{tabular}
\end{center}
\newpage

%%%%%%%%%%%%%%%%%%%%%%%%%%%%%%%%%%%%% Page 2
\noindent{\large\bf MATH 141}\hfill{\large\bf Exam\#2(A).}\hfill{\large\bf
  Fall 2009}\hfill{\large\bf Page 2/5}\hrule

\bigskip
{\problem[10 pts] \em  Evaluate each limit:} 

\bigskip
\begin{tikzpicture}
\draw (4cm,14cm) node{
$\displaystyle{ \lim_{x\to 0} \frac{e^x-1}{\tan x} = \mbox{}}$ };
\draw (5.5cm,13.4cm) rectangle (8.5cm,14.6cm);
\end{tikzpicture}

\vspace{4cm}
\begin{tikzpicture}
\draw (4cm,14cm) node{
$\displaystyle{ \lim_{x\to 0} \frac{\tan(5x^2)}{x^2} = \mbox{}}$ };
\draw (5.5cm,13.4cm) rectangle (8.5cm,14.6cm);
\end{tikzpicture}
\vspace{5cm}

\hrule
{\problem[10 pts] \em Find an equation of the normal line to the
  curve $y=\ln(x e^{x^2})$ at the point $(1,1)$.} 
\newpage

%%%%%%%%%%%%%%%%%%%%%%%%%%%%%%%%%%%%% Page 3
\noindent{\large\bf MATH 141}\hfill{\large\bf Exam\#2(A).}\hfill{\large\bf
  Fall 2009}\hfill{\large\bf Page 3/5}\hrule

\bigskip
{\problem[30 pts] \em Sketch the graph of the rational function $f(x)
  = \displaystyle{\frac{2x^2-8}{x^2-16}}$.}
\begin{quotation}
Indicate clearly:
\begin{itemize}
\item Domain
\item $x$- and $y$-intercepts.
\item Vertical and horizontal asymptotes (any holes?).
\item Intervals of increase, decrease and different concavity.
\item Location of relative extrema and inflection points. 
\end{itemize}
\end{quotation}
\newpage

%%%%%%%%%%%%%%%%%%%%%%%%%%%%%%%%%%%%% Page 4
\noindent{\large\bf MATH 141}\hfill{\large\bf Exam\#2(A).}\hfill{\large\bf
  Fall 2009}\hfill{\large\bf Page 4/5}\hrule

\bigskip
{\problem[10 pts] \em Find the absolute extrema of $f(x) = 6x^{4/3} - 3x^{1/3}$ on the interval $[-1,1]$.}
\vspace{6cm}
\begin{flushright}
  \begin{tikzpicture}
    \draw (-2.25cm,2.5cm) node {Absolute maxima at };
    \draw (0cm,1.8cm) rectangle (5cm,3.2cm);
    \draw (-2.25cm,0.5cm) node {Absolute minima at };
    \draw (0cm,-0.2cm) rectangle (5cm,1.2cm);
  \end{tikzpicture}
\end{flushright}
\hrule
{\problem[10 pts] \em Use logarithmic differentiation to find the
  derivative of the function 
\begin{equation*}
y=\frac{\sin^2 x \tan^4 x}{(x^2+1)^2}
\end{equation*}
\vspace{7cm}
\begin{flushright}
  \begin{tikzpicture}
    \draw (-0.7cm,0.5cm) node {$\displaystyle{\frac{dy}{dx}}=$ };
    \draw (0cm,-0.2cm) rectangle (5cm,1.2cm);
  \end{tikzpicture}
\end{flushright}
\newpage

%%%%%%%%%%%%%%%%%%%%%%%%%%%%%%%%%%%%% Page 5
\noindent{\large\bf MATH 141}\hfill{\large\bf Exam\#2(A).}\hfill{\large\bf
  Fall 2009}\hfill{\large\bf Page 5/5}\hrule

\bigskip
{\problem[10 pts] \em An aircraft is climbing at $30^o$ angle to the horizontal.  How fast is the aircraft gaining altitude if its speed is 500 mi/h?}
\vspace{6cm}
\begin{flushright}
  \begin{tikzpicture}
    \draw (-4.25cm,0.5cm) node {The aircraft is gaining altitude at a speed of};
    \draw (0cm,-0.2cm) rectangle (5cm,1.2cm);
  \end{tikzpicture}
\end{flushright}
\hrule
{\problem[20 pts] \em A container with square base, vertical sides, and open top is to be made from 300 ft$^2$ of material.  Find the dimensions of the container with greatest volume.}
\vspace{11cm}
\begin{flushright}
  \begin{tikzpicture}
    \draw (-2.7cm,0.5cm) node {Dimensions of container: };
    \draw (0cm,-0.2cm) rectangle (5cm,1.2cm);
  \end{tikzpicture}
\end{flushright}
\end{document}
