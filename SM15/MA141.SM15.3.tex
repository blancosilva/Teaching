\documentclass[12pt]{article}

\usepackage{amsmath,amsthm,amsfonts,amssymb,amsxtra}
\usepackage{pgf,tikz}
\usetikzlibrary{arrows}
\renewcommand{\theenumi}{(\alph{enumi})} 
\renewcommand{\labelenumi}{\theenumi}

\pagestyle{empty}
\setlength{\textwidth}{7in}
\setlength{\oddsidemargin}{-0.5in}
\setlength{\topmargin}{-1.0in}
\setlength{\textheight}{9.5in}

\newtheorem{problem}{Problem}

\begin{document}

\noindent{\large\bf MATH 141}\hfill{\large\bf Exam\#3.}\hfill{\large\bf
  Summer E 2015}\hfill{\large\bf Page 1/5}\hrule

\bigskip
\begin{center}
  \begin{tabular}{|ll|}
    \hline & \cr
    {\bf Name: } & \makebox[12cm]{\hrulefill}\cr & \cr
    {\bf 4-digit code:} & \makebox[12cm]{\hrulefill}\cr & \cr
    \hline
  \end{tabular}
\end{center}
\begin{itemize}
\item Write your name and the last 4 digits of your SSN in the space provided above.
\item The test has six (6) pages, including this one.
\item Enter your answer in the box(es) provided.
\item You must show sufficient work to justify all answers unless
  otherwise stated in the problem.  Correct answers with inconsistent
  work may not be given credit.
\item Credit for each problem is given in parentheses at the right of
  the problem number.
\item No books, notes or calculators may be used on this test.
\end{itemize}
\hrule

\begin{center}
  \begin{tabular}{|c|c|c|}
    \hline
    &&\cr
    {\large\bf Page} & {\large\bf Max.~points} & {\large\bf Your points} \cr
    &&\cr
    \hline
    &&\cr
    {\Large 2} & \Large 30 & \cr
    &&\cr
    \hline
    &&\cr
    {\Large 3} & \Large 20 & \cr
    &&\cr
    \hline
    &&\cr
    {\Large 4} & \Large 20 & \cr
    &&\cr
    \hline
    &&\cr
    {\Large 5} & \Large 30 & \cr
    &&\cr
   \hline\hline
    &&\cr
    {\large\bf Total} & \Large 100 & \cr
    &&\cr
    \hline
  \end{tabular}
\end{center}
\newpage

%%%%%%%%%%%%%%%%%%%%%%%%%%%%%%%%%%%%% Page 1
\noindent{\large\bf MATH 141}\hfill{\large\bf Exam\#3.}\hfill{\large\bf
  Summer E 2015}\hfill{\large\bf Page 2/6}\hrule

\bigskip
{\problem[30 pts] \em Sketch on the next page the graph of the function $f(x) = \dfrac{x}{x-2}$.}

Make sure to indicate clearly:
\begin{itemize}
\item Zeros of the function.
\item Domain and Range
\item Vertical and horizontal asymptotes.
\item Extreme values and inflection points.
\item Intervals of increase/decrease.
\item Intervals of concavity.
\end{itemize}

Keep all the computations on this page
\newpage

%%%%%%%%%%%%%%%%%%%%%%%%%%%%%%%%%%%%% Page 2
\noindent{\large\bf MATH 141}\hfill{\large\bf Exam\#3.}\hfill{\large\bf
  Summer E 2015}\hfill{\large\bf Page 3/6}\hrule

\newpage 

%%%%%%%%%%%%%%%%%%%%%%%%%%%%%%%%%%%%% Page 3
\noindent{\large\bf MATH 141}\hfill{\large\bf Exam\#3.}\hfill{\large\bf
  Summer E 2015}\hfill{\large\bf Page 4/6}\hrule

\bigskip
{\problem[10 pts] \em Use the first or second derivative test to compute the local maxima and minima of the function $f(x) = x^{1/5} - x^{-4/5}$}

\vspace{10cm}
\hrule
{\problem[10] \em Find at least one critical point of the function $f(x) = 4x - \tan x$.  You \textbf{do not} have to indicate whether they are local maxima, local minima, or neither.}
\vspace{9cm}
\begin{flushright}
  \begin{tikzpicture}
    \draw (0cm,0cm) rectangle (5cm,1.2cm);
  \end{tikzpicture}
\end{flushright}
\newpage

%%%%%%%%%%%%%%%%%%%%%%%%%%%%%%%%%%%%% Page 4
\noindent{\large\bf MATH 141}\hfill{\large\bf Exam\#3.}\hfill{\large\bf
  Summer E 2015}\hfill{\large\bf Page 5/6}\hrule

\bigskip
{\problem[10 pts] \em Assume the function $f(x)$ satisfies the conditions of the Mean Value Theorem.  We know that $f(9)= 10$ and $4 \leq f'(x) \leq 5$ for $9 \leq x \leq 12$.  Estimate the maximum and minimum possible values for $f(12)$.}

\vspace{10cm}
\hrule
{\problem[10pts] \em Find the global maximum and global minimum \textbf{values} of the function $f(x)=2x^3-6x^2-48x+5$ on the interval $[-3,5]$.}
\newpage


%%%%%%%%%%%%%%%%%%%%%%%%%%%%%%%%%%%%% Page 5
\noindent{\large\bf MATH 141}\hfill{\large\bf Exam\#3.}\hfill{\large\bf
  Summer E 2015}\hfill{\large\bf Page 6/6}\hrule

\bigskip
{\problem \em Compute the following limits:}
\begin{itemize}
  \item[] (5pts) $\displaystyle{\lim_{x\to \infty} \frac{1}{5x+7}}$
  \vspace{2cm}
  \item[] (5pts) $\displaystyle{\lim_{x\to\infty} \frac{1-x-x^2}{5x^2-9}}$
  \vspace{2cm}
  \item[] (5pts) $\displaystyle{\lim_{x\to 0+} \ln x + \frac{3}{x}}$
  \vspace{2cm}
  \item[] (5pts) $\displaystyle{\lim_{x\to\infty} \frac{12 x^3}{5 e^x}}$
  \vspace{2cm}
  \item[] (5pts) $\displaystyle{\lim_{x\to \pi/2^{+}} \frac{1-\sin x}{\cos x}}$
  \vspace{2cm}
  \item[] (5pts) $\displaystyle{\lim_{x\to\infty} \Big(1-\frac{3}{x} \Big)^{x}}$
\end{itemize}

\end{document}
