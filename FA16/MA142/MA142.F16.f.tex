\documentclass[12pt]{article}

\usepackage{amsmath,amsthm,amsfonts,amssymb,amsxtra}
\usepackage{pgf,tikz}
\usetikzlibrary{arrows}
\renewcommand{\theenumi}{(\alph{enumi})} 
\renewcommand{\labelenumi}{\theenumi}

\pagestyle{empty}
\setlength{\textwidth}{7in}
\setlength{\oddsidemargin}{-0.5in}
\setlength{\topmargin}{-1.0in}
\setlength{\textheight}{9.5in}

\theoremstyle{definition}
\newtheorem{problem}{Problem}

\begin{document}

\noindent{\large\bf MATH 142}\hfill{\large\bf Final Exam.}\hfill{\large\bf
  Fall 2016}\hfill{\large\bf Page 1/8}\hrule

\bigskip
\begin{center}
  \begin{tabular}{|ll|}
    \hline & \cr
    {\bf Name: } & \makebox[12cm]{\hrulefill}\cr & \cr
    {\bf VIP ID:} & \makebox[12cm]{\hrulefill}\cr & \cr
    \hline
  \end{tabular}
\end{center}
\begin{itemize}
\item Write your name and VIP ID in the space provided above.
\item The test has eight (8) pages, including this one, and two pages of scratch paper (pages 7, 8). 
\item It also has a formula sheet attached.  \textbf{You may detach this formula sheet only when prompted to do so.  Failure to comply will result in losing the privilege to use it.}
\item Enter your answers in the boxes provided.
\item You must show sufficient work to justify all answers unless
  otherwise stated in the problem.  Correct answers with inconsistent
  work may not be given credit.
\item Credit for each problem is given in parentheses at the right of
  the problem number.
\item No books or notes may be used on this test.  A graphing calculator is allowed.
\end{itemize}
\hrule

\begin{center}
  \begin{tabular}{|c|c|c|}
    \hline
    &&\cr
    {\large\bf Page} & {\large\bf Max.~points} & {\large\bf Your points} \cr
    &&\cr
    \hline
    &&\cr
    {\Large 2} & \Large 20 & \cr
    &&\cr
    \hline
    &&\cr
    {\Large 3} & \Large 20 & \cr
    &&\cr
    \hline
    &&\cr
    {\Large 4} & \Large 20 & \cr
    &&\cr
    \hline
    &&\cr
    {\Large 5} & \Large 20 & \cr
    &&\cr
	\hline
    &&\cr
    {\Large 6} & \Large 20 & \cr
    &&\cr
  \hline\hline
    &&\cr
    {\large\bf Total} & \Large 100 & \cr
    &&\cr
    \hline
  \end{tabular}
\end{center}
\newpage

%%%%%%%%%%%%%%%%%%%%%%%%%%%%%%%%%%%%% Page 2
\noindent{\large\bf MATH 142}\hfill{\large\bf Final Exam.}\hfill{\large\bf
  Fall 2016}\hfill{\large\bf Page 2/8}\hrule

\bigskip
{\problem[5 pts] Evaluate the integral $\displaystyle{\int x e^x\, dx}$.}
\vspace{2cm}
\begin{flushright}
  \begin{tikzpicture}
    \draw (0cm,-0.2cm) rectangle (5cm,1.2cm);
  \end{tikzpicture}
\end{flushright}
\hrule

{\problem[5 pts] Evaluate the integral $\displaystyle{\int (x-3) \sqrt{x^2-6x+5}\, dx}$.}
\vspace{2cm}
\begin{flushright}
  \begin{tikzpicture}
    \draw (0cm,-0.2cm) rectangle (5cm,1.2cm);
  \end{tikzpicture}
\end{flushright}
\hrule

{\problem[5 pts] Evaluate the integral $\displaystyle{\int
    \sin^2 x \cos^3 x\, dx}$.
\vspace{3.5cm}
\begin{flushright}
  \begin{tikzpicture}
    \draw (0cm,-0.2cm) rectangle (5cm,1.2cm);
  \end{tikzpicture}
\end{flushright}
\hrule

{\problem[5 pts] Evaluate the integral $\displaystyle{\int \frac{x^3+1}{(x+1)^2(x^2+4)}\, dx}$.}
\vspace{4.4cm}
\begin{flushright}
  \begin{tikzpicture}
    \draw (0cm,-0.2cm) rectangle (5cm,1.2cm);
  \end{tikzpicture}
\end{flushright}
\newpage


%%%%%%%%%%%%%%%%%%%%%%%%%%%%%%%%%%%%% Page 3
\noindent{\large\bf MATH 142}\hfill{\large\bf Final Exam.}\hfill{\large\bf
  Fall 2016}\hfill{\large\bf Page 3/8}\hrule

\bigskip
{\problem[10 pts] Evaluate the following integral, or indicate if it is
divergent: $\displaystyle{\int_0^{\pi^2/4} \frac{\cos \sqrt{t}}{\sqrt{t}}\, dt}$
\vspace{10cm}
\begin{flushright}
  \begin{tikzpicture}
    \draw (0cm,-0.2cm) rectangle (5cm,1.2cm);
  \end{tikzpicture}
\end{flushright}
\hrule

\begin{problem}[10 pts]
Consider the region in the $xy$--plane enclosed by the cardioid with polar equation $r = 2(1+\cos \theta)$ (assume $0 \leq \theta \leq 2\pi$).  Express its area with an integral in terms of the parameter $\theta$, and compute its value.
\vspace{6cm}
\begin{flushright}
  \begin{tikzpicture}
    \draw (0cm,1.5cm) rectangle (5cm,2.9cm);
  \end{tikzpicture}
\end{flushright}
\end{problem}
\newpage

%%%%%%%%%%%%%%%%%%%%%%%%%%%%%%%%%%%%% Page 4
\noindent{\large\bf MATH 142}\hfill{\large\bf Final Exam.}\hfill{\large\bf
  Fall 2016}\hfill{\large\bf Page 4/8}\hrule

\bigskip
{\problem[10 pts] Find the general term of the sequence $\big\{
  3,2,\frac{5}{3}, \frac{3}{2}, \frac{7}{5}, \frac{4}{3}, \dotsc
  \big\}$, and compute its limit.}
\vspace{3cm}
\begin{flushright}
  \begin{tikzpicture}
    \draw (0cm,-0.2cm) rectangle (5cm,1.2cm);
    \draw (0cm,1.5cm) rectangle (5cm,2.9cm);
  \end{tikzpicture}
\end{flushright}
\hrule

{\problem[5 pts---all or nothing] Compute the limit of the sequence $\bigg\{
\dfrac{n^2+5n+2}{n^2+2n} \bigg\}_{n=1}^\infty$}
\vspace{4cm}
\begin{flushright}
  \begin{tikzpicture}
    \draw (0cm,-0.2cm) rectangle (5cm,1.2cm);
  \end{tikzpicture}
\end{flushright}
\hrule

{\problem[5 pts] Find a power series representation of the function $f(x) = x^2 \cos (2x)$}
\vspace{5cm}
\begin{flushright}
  \begin{tikzpicture}
    \draw (0cm,-0.2cm) rectangle (5cm,1.2cm);
  \end{tikzpicture}
\end{flushright}

\newpage

%%%%%%%%%%%%%%%%%%%%%%%%%%%%%%%%%%%%% Page 5
\noindent{\large\bf MATH 142}\hfill{\large\bf Final Exam.}\hfill{\large\bf
  Fall 2016}\hfill{\large\bf Page 5/8}\hrule
  
\bigskip

{\problem[10 pts] Study the convergence of the series
  $\displaystyle{\sum_{n=2}^\infty \frac{3^n+4^n}{5^n}}$.  If
  convergent, evaluate the sum.
\vspace{6.5cm}
\begin{flushright}
  \begin{tikzpicture}
    \draw (0cm,-0.2cm) rectangle (5cm,1.2cm);
    \draw (0cm,1.5cm) rectangle (5cm,2.9cm);
  \end{tikzpicture}
\end{flushright}
\hrule
{\problem[10 pts] Classify the series $\displaystyle{\sum_{n=1}^\infty
    \frac{(-1)^n n}{e^n}}$ as absolutely
  convergent, conditionally convergent, or divergent.}
\vspace{6.5cm}
\begin{flushright}
  \begin{tikzpicture}
    \draw (0cm,-0.2cm) rectangle (5cm,1.2cm);
  \end{tikzpicture}
\end{flushright}
\newpage

%%%%%%%%%%%%%%%%%%%%%%%%%%%%%%%%%%%%% Page 6
\noindent{\large\bf MATH 142}\hfill{\large\bf Final Exam.}\hfill{\large\bf
  Fall 2016}\hfill{\large\bf Page 6/8}\hrule

\bigskip
{\problem[10 pts] Find the interval of convergence of the power series
  $\displaystyle{\sum_{n=0}^\infty \frac{(-3)^n x^n}{\sqrt{n+1}}}$.}
\vspace{7.5cm}
\begin{flushright}
  \begin{tikzpicture}
    \draw (0cm,-0.2cm) rectangle (5cm,1.2cm);
  \end{tikzpicture}
\end{flushright}
\hrule
{\problem[10 pts] Express the function $f(x) = \dfrac{2x}{x^3+8}$ as a
  power series.}
\vspace{9cm}
\begin{flushright}
  \begin{tikzpicture}
    \draw (0cm,-0.2cm) rectangle (5cm,1.2cm);
  \end{tikzpicture}
\end{flushright}
\newpage

%%%%%%%%%%%%%%%%%%%%%%%%%%%%%%%%%%%%% Page 7
\noindent{\large\bf MATH 142}\hfill{\large\bf Final Exam.}\hfill{\large\bf
  Fall 2016}\hfill{\large\bf Page 7/8}\hrule

\bigskip
\noindent{\large Scratch paper}

\newpage
%%%%%%%%%%%%%%%%%%%%%%%%%%%%%%%%%%%%% Page 8
\noindent{\large\bf MATH 142}\hfill{\large\bf Final Exam.}\hfill{\large\bf
  Fall 2016}\hfill{\large\bf Page 8/8}\hrule

\bigskip
\noindent{\large Scratch paper}


\end{document}
