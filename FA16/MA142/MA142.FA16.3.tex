\documentclass[12pt]{article}

\usepackage{amsmath,amsthm,amsfonts,amssymb,amsxtra}
\usepackage{pgf,tikz}
\usetikzlibrary{arrows}
\renewcommand{\theenumi}{(\alph{enumi})} 
\renewcommand{\labelenumi}{\theenumi}

\pagestyle{empty}
\setlength{\textwidth}{7in}
\setlength{\oddsidemargin}{-0.5in}
\setlength{\topmargin}{-1.0in}
\setlength{\textheight}{9.5in}

\theoremstyle{definition}
\newtheorem{problem}{Problem}

\begin{document}

\noindent{\large\bf MATH 142}\hfill{\large\bf Exam\#3.}\hfill{\large\bf
Fall 2016}\hfill{\large\bf Page 1/5}\hrule

\bigskip
\begin{center}
\begin{tabular}{|ll|}
\hline & \cr
{\bf Name: } & \makebox[12cm]{\hrulefill}\cr & \cr
{\bf VIP ID:} & \makebox[12cm]{\hrulefill}\cr & \cr
\hline
\end{tabular}
\end{center}
\begin{itemize}
\item Write your name and VIP ID in the space provided above.
\item The test has five (5) pages, including this one.
\item You must show sufficient work to justify all answers unless otherwise stated in the problem.  Correct answers with inconsistent work may not be given credit.
\item Credit for each problem is given in parentheses at the right of the problem number.
\item No books, notes or calculators may be used on this test.
\end{itemize}
\hrule

\begin{center}
\begin{tikzpicture}
\draw (0,0) node{
\begin{tabular}{|c|c|c|}
\hline
&&\cr
{\large\bf Page} & {\large\bf Max.~points} & {\large\bf Your points} \cr
&&\cr
\hline
&&\cr
{\Large 2} & \Large 25 & \cr
&&\cr
\hline
&&\cr
{\Large 3} & \Large 25 & \cr
&&\cr
\hline
&&\cr
{\Large 4} & \Large 20 & \cr
&&\cr
\hline
&&\cr
{\Large 5} & \Large 30 & \cr
&&\cr
\hline\hline
&&\cr
{\large\bf Total} & \Large 100 & \cr
&&\cr
\hline
\end{tabular}};
\draw (0.5\linewidth, 0) node{
	\begin{tabular}{|c|c|}
	\hline & \\
	Function/Power series & convergence \\ & \\
	\hline \hline & \\
	$\displaystyle{\frac{1}{1-x} = \sum_{n=0}^\infty x^n}$ & $(-1, 1)$ \\ & \\
	\hline & \\
	$\displaystyle{e^x = \sum_{n=0}^\infty \frac{x^n}{n!}}$ & $(-\infty, \infty)$ \\ & \\
	\hline & \\
	$\displaystyle{\cos x = \sum_{n=0}^\infty \frac{(-1)^n}{(2n)!} x^{2n}}$ & $(-\infty, \infty)$ \\ & \\
	\hline & \\
	$\displaystyle{\tan^{-1} x = \sum_{n=0}^\infty \frac{(-1)^n}{2n+1} x^{2n+1}}$ & $[-1, 1]$ \\ & \\
	\hline & \\
	$\displaystyle{\ln (1-x) = \dotsb }$ (see problem 4) & ??? \\ & \\
	\hline
	\end{tabular}	};
\end{tikzpicture}
\end{center}

\newpage

%%%%%%%%%%%%%%%%%%%%%%%%%%%%%%%%%%%%% Page 2
\noindent{\large\bf MATH 142}\hfill{\large\bf Exam\#3.}\hfill{\large\bf
Fall 2016}\hfill{\large\bf Page 2/5}\hrule

\bigskip

\begin{problem}[15 pts]
For what values of $x$ is the following power series convergent?
\begin{equation*}
\sum_{n=1}^\infty \frac{(x-\pi)^n}{n}
\end{equation*}
\vspace{8cm}
\end{problem}
\hrule

\begin{problem}[10 pts]
Find the radius and interval of convergence of the following power series.
\begin{equation*}
\sum_{n=1}^\infty \frac{(-\pi)^n x^n}{\sqrt{n}}
\end{equation*}
\end{problem}


\newpage


%%%%%%%%%%%%%%%%%%%%%%%%%%%%%%%%%%%%% Page 3
\noindent{\large\bf MATH 142}\hfill{\large\bf Exam\#3.}\hfill{\large\bf
Fall 2016}\hfill{\large\bf Page 3/5}\hrule

\bigskip
\begin{problem}[10 pts]
Assume known that $\displaystyle{ \frac{1}{1-x} = \sum_{n=0}^\infty x^n \text{ for } \lvert x \rvert <1}$.

\noindent Express the function $f(x)=1/(1-x)^2$ as a power series by differentiating the previous equation.  What is the radius of convergence?
\vspace{8cm}
\end{problem}
\hrule

\begin{problem}[15 pts]
Find a power series representation for $f(x) = \ln(1-x)$ and its radius of convergence.
\end{problem}
\newpage

%%%%%%%%%%%%%%%%%%%%%%%%%%%%%%%%%%%%% Page 4
\noindent{\large\bf MATH 142}\hfill{\large\bf Exam\#3.}\hfill{\large\bf
Fall 2016}\hfill{\large\bf Page 4/5}\hrule

\bigskip

\begin{problem}[10 pts]
Find the Taylor series for $f(x) = e^x$ at $b=\pi$.

\textbf{Hint: } The Taylor series of $f(x)$ is $\displaystyle{ \sum_{n=0}^\infty \frac{f^{(n)}(b)}{n!} (x-b)^n }.$
\vspace{10cm}
\end{problem}
\hrule

\begin{problem}[10 pts]
Find the Maclaurin series for the function $f(x) = x^2 \sin x$

\textbf{Hint: } The MacLaurin series is just a Taylor series centered at $b=0$... But you may not need this at all, if you know a faster way to compute the power series of $\sin x$
\end{problem}
\newpage

%%%%%%%%%%%%%%%%%%%%%%%%%%%%%%%%%%%%% Page 5
\noindent{\large\bf MATH 142}\hfill{\large\bf Exam\#3.}\hfill{\large\bf
Fall 2016}\hfill{\large\bf Page 5/5}\hrule

\bigskip
\begin{problem}[15 pts]
Find the first three nonzero terms in the Maclaurin series for $f(x)=e^x\cos x$.

\textbf{Hint:} A relatively fast way to do this is to compute a Taylor or MacLaurin polynomial of $f(x)$ until you find those three non-zero coefficients.
\vspace{7cm}
\end{problem}
\hrule

\begin{problem}[15 pts]
Evaluate $\int e^{-x^2}\, dx$ as an infinite series.  For this, proceed as indicated:
\begin{enumerate}
	\item Find a power series expansion of the function $f(x) = e^{-x^2}$.  You may do this by modification of the power series of $e^x$.  Simplify as much as possible, to help you with easier expressions in the next step.
	\item In the interval of convergence of this power series, you are allowed to use the \emph{integration trick} to interchange integral with summation.  Apply this technique, amd simplify as much as you can. Once you have produced a power-series representation of this integral, you are done!
	\end{enumerate}
\end{problem}

\end{document}
