\documentclass[12pt]{article}

\usepackage{amsmath,amsthm,amsfonts,amssymb,amsxtra}
\usepackage{pgf,tikz}
\usetikzlibrary{arrows}
\renewcommand{\theenumi}{(\alph{enumi})} 
\renewcommand{\labelenumi}{\theenumi}

\pagestyle{empty}
\setlength{\textwidth}{7in}
\setlength{\oddsidemargin}{-0.5in}
\setlength{\topmargin}{-1.0in}
\setlength{\textheight}{9.5in}

\theoremstyle{definition}
\newtheorem{problem}{Problem}

\begin{document}

\noindent{\large\bf MATH 142}\hfill{\large\bf Exam \#1.}\hfill{\large\bf Fall 2016}\hfill{\large\bf Page 1/7}\hrule

\bigskip
\begin{center}
  \begin{tabular}{|ll|}
    \hline & \cr
    {\bf Name: } & \makebox[12cm]{\hrulefill}\cr & \cr
    {\bf 4-digit code:} & \makebox[12cm]{\hrulefill}\cr & \cr
    \hline
  \end{tabular}
\end{center}
\begin{itemize}
\item Write your name and the last 4 digits of your SSN in the space provided above.
\item The test has seven (7) pages, including this one.
\item Enter your answer in the box(es) provided.
\item You must show sufficient work to justify all answers unless
  otherwise stated in the problem.  Correct answers with inconsistent
  work may not be given credit.
\item Credit for each problem is given in parentheses at the right of
  the problem number.
\item No books, notes or calculators may be used on this test.
\end{itemize}
\hrule

\begin{center}
  \begin{tabular}{|c|c|c|}
    \hline
    &&\cr
    {\large\bf Page} & {\large\bf Max.~points} & {\large\bf Your points} \cr
    &&\cr
    \hline
    &&\cr
    {\Large 2} & \Large 20 & \cr
    &&\cr
    \hline
    &&\cr
    {\Large 3} & \Large 20 & \cr
    &&\cr
    \hline
    &&\cr
    {\Large 4} & \Large 10 & \cr
    &&\cr
    \hline
    &&\cr
    {\Large 5} & \Large 10 & \cr
    &&\cr
	\hline
    &&\cr
    {\Large 6} & \Large 20 & \cr
    &&\cr
  \hline
    &&\cr
    {\Large 7} & \Large 20 & \cr
    &&\cr
    \hline\hline
    &&\cr
    {\large\bf Total} & \Large 100 & \cr
    &&\cr
    \hline
  \end{tabular}
\end{center}
\newpage

%%%%%%%%%%%%%%%%%%%%%%%%%%%%%%%%%%%%% Page 2
\noindent{\large\bf MATH 142}\hfill{\large\bf Exam \#1.}\hfill{\large\bf Fall 2016}\hfill{\large\bf Page 2/7}\hrule

\bigskip
\begin{problem}[20 pts | 5 points each]
Evaluate each integral

\bigskip
\begin{tikzpicture}
\draw (4cm,14cm) node[left, text width=6.5cm]{
(a) $\displaystyle{\int \csc^2 x\, dx} = \mbox{}$ };
\draw (1cm,13.4cm) rectangle (9cm,14.6cm);
\draw (4cm, 11cm) node[left, text width=6.5cm]{
(b) $\displaystyle{\int \frac{1}{\csc x}\, dx} = \mbox{}$}; 
\draw (1cm,10.4cm) rectangle (9cm,11.6cm);
\draw (4cm, 8cm) node[left, text width=6.5cm]{
(c) $\displaystyle{\int \frac{x+1}{x}\, dx} = \mbox{}$}; 
\draw (1cm,7.4cm) rectangle (9cm,8.6cm);
\draw (4cm, 0cm) node[left, text width=6.5cm]{
(d) $\displaystyle{\int \frac{x}{x+1}\, dx} = \mbox{}$}; 
\draw (1cm, -0.6cm) rectangle (9cm,0.6cm);
\end{tikzpicture}
\end{problem}
\newpage

%%%%%%%%%%%%%%%%%%%%%%%%%%%%%%%%%%%%% Page 3
\noindent{\large\bf MATH 142}\hfill{\large\bf Exam \#1.}\hfill{\large\bf Fall 2016}\hfill{\large\bf Page 3/7}\hrule

\bigskip
\begin{problem}[10 pts]
Use \textbf{integration by parts} to evaluate the integral $\displaystyle{\int xe^{2x}\, dx}$.
\vspace{6cm}
\begin{flushright}
  \begin{tikzpicture}
    \draw (-1.25cm,0.5cm) node {$\displaystyle{\int xe^{2x}\, dx} = $};
    \draw (0cm,-0.2cm) rectangle (5cm,1.2cm);
  \end{tikzpicture}
\end{flushright}
\end{problem}
\hrule

\begin{problem}[10 pts]
Evaluate the integral $\displaystyle{\int \cos^2 x\, dx}$.
\vspace{10cm}
\begin{flushright}
  \begin{tikzpicture}
    \draw (-1.25cm,0.5cm) node {$\displaystyle{\int \cos^2 x\, dx} = $};
    \draw (0cm,-0.2cm) rectangle (5cm,1.2cm);
  \end{tikzpicture}
\end{flushright}
\end{problem}
\newpage

%%%%%%%%%%%%%%%%%%%%%%%%%%%%%%%%%%%%% Page 4
\noindent{\large\bf MATH 142}\hfill{\large\bf Exam \#1.}\hfill{\large\bf Fall 2016}\hfill{\large\bf Page 4/7}\hrule

\bigskip
\begin{problem}[10 pts]
Use the trigonometric substitution $x=a\sec \theta$ to evaluate the integral $\displaystyle{\int \frac{dx}{\sqrt{x^2-9}}}$.
\vspace{18cm}
\begin{flushright}
  \begin{tikzpicture}
    \draw (-4.5cm,0.5cm) node {$\displaystyle{\int \frac{dx}{\sqrt{x^2-9}}} = $};
    \draw (-3cm,-0.2cm) rectangle (5cm,1.2cm);
  \end{tikzpicture}
\end{flushright}
\end{problem}
\newpage

%%%%%%%%%%%%%%%%%%%%%%%%%%%%%%%%%%%%% Page 5
\noindent{\large\bf MATH 142}\hfill{\large\bf Exam \#1.}\hfill{\large\bf Fall 2016}\hfill{\large\bf Page 5/7}\hrule

\bigskip
\begin{problem}[10 pts]
Evaluate the integral $\displaystyle{\int \sin 2x \cos 5x\, dx}$. \newline You may find useful one of the following formulas.
\begin{align*}
\sin A \sin B &= \frac{1}{2} \big[ \cos (A-B) - \cos (A+B) \big] \\
\sin A \cos B &= \frac{1}{2} \big[ \sin (A-B) + \sin (A+B) \big] \\
\cos A \cos B &= \frac{1}{2} \big[ \cos (A-B) + \cos (A+B) \big]
\end{align*}
\vspace{16cm}
\begin{flushright}
  \begin{tikzpicture}
    \draw (-5.25cm,0.5cm) node {$\displaystyle{\int \sin 2x \cos 5x\, dx} = $};
    \draw (-3.25cm,-0.2cm) rectangle (5cm,1.2cm);
  \end{tikzpicture}
\end{flushright}
\end{problem}
\newpage

%%%%%%%%%%%%%%%%%%%%%%%%%%%%%%%%%%%%% Page 6
\noindent{\large\bf MATH 142}\hfill{\large\bf Exam \#1.}\hfill{\large\bf Fall 2016}\hfill{\large\bf Page 6/7}\hrule
  
\bigskip
\begin{problem}[20 pts]
Use \textbf{partial fractions} to evaluate the integral $\displaystyle{\int \frac{dx}{x^2+x-2}}$.
\vspace{18cm}
\begin{flushright}
  \begin{tikzpicture}
    \draw (-5.23cm,0.5cm) node {$\displaystyle{\int \frac{dx}{x^2+x-2}} = $};
    \draw (-3cm,-0.2cm) rectangle (5cm,1.2cm);
  \end{tikzpicture}
\end{flushright}
\end{problem}
\newpage

%%%%%%%%%%%%%%%%%%%%%%%%%%%%%%%%%%%%% Page 7
\noindent{\large\bf MATH 142}\hfill{\large\bf Exam \#1.}\hfill{\large\bf Fall 2016}\hfill{\large\bf Page 7/7}\hrule
 

\begin{problem}[20 pts | 10 points each]
Evaluate the integrals below

\begin{enumerate}
\item $\displaystyle{\int_0^{\infty} \frac{ \sin ( \tfrac{\pi}{2} e^{-x}) }{e^x} \, dx}$
\vspace{8cm}
\begin{flushright}
  \begin{tikzpicture}
    \draw (0cm,-0.2cm) rectangle (5cm,1.2cm);
  \end{tikzpicture}
\end{flushright}
\item $\displaystyle{\int_0^{\pi^2/4} \frac{\cos \sqrt{t}}{\sqrt{t}}\, dt}$
\vspace{8cm}
\begin{flushright}
  \begin{tikzpicture}
    \draw (0cm,-0.2cm) rectangle (5cm,1.2cm);
  \end{tikzpicture}
\end{flushright}
\end{enumerate}
\end{problem}

\end{document}
