\documentclass[12pt]{article}

\usepackage{amsmath,amsthm,amsfonts,amssymb,amsxtra}
\usepackage{pgf,tikz}
\usetikzlibrary{arrows}
\renewcommand{\theenumi}{(\alph{enumi})} 
\renewcommand{\labelenumi}{\theenumi}

\pagestyle{empty}
\setlength{\textwidth}{7in}
\setlength{\oddsidemargin}{-0.5in}
\setlength{\topmargin}{-1.0in}
\setlength{\textheight}{9.5in}

\theoremstyle{definition}
\newtheorem{problem}{Problem}

\begin{document}

\noindent{\large\bf MATH 142}\hfill{\large\bf Test\#4}\hfill{\large\bf
  Fall 2016}\hfill{\large\bf Page 1/6}\hrule

\bigskip
\begin{center}
  \begin{tabular}{|ll|}
    \hline & \cr
    {\bf Name: } & \makebox[12cm]{\hrulefill}\cr & \cr
    {\bf VIP ID:} & \makebox[12cm]{\hrulefill}\cr & \cr
    \hline
  \end{tabular}
\end{center}
\begin{itemize}
\item Write your name and VIP ID in the space provided above.
\item The test has six (6) pages including this one.
\item Enter your answers in the boxes provided.
\item You must show sufficient work to justify all answers unless otherwise stated in the problem.  Correct answers with inconsistent work may not be given credit.
\item Credit for each problem is given in parentheses at the right of the problem number.
\item No books or notes may be used on this test.  An approved calculator is allowed (and recommended).
\end{itemize}
\hrule

\begin{center}
  \begin{tabular}{|c|c|c|}
    \hline
    &&\cr
    {\large\bf Page} & {\large\bf Max.~points} & {\large\bf Your points} \cr
    &&\cr
    \hline
    &&\cr
    {\Large 2} & \Large 10 & \cr
    &&\cr
    \hline
    &&\cr
    {\Large 3} & \Large 30 & \cr
    &&\cr
    \hline
    &&\cr
    {\Large 4} & \Large 30 & \cr
    &&\cr
    \hline
    &&\cr
    {\Large 5} & \Large 20 & \cr
    &&\cr
	\hline
    &&\cr
    {\Large 6} & \Large 10 & \cr
    &&\cr
  \hline\hline
    &&\cr
    {\large\bf Total} & \Large 100 & \cr
    &&\cr
    \hline
  \end{tabular}
\end{center}
\newpage

%%%%%%%%%%%%%%%%%%%%%%%%%%%%%%%%%%%%% Page 2
\noindent{\large\bf MATH 142}\hfill{\large\bf Test\#4}\hfill{\large\bf
  Fall 2016}\hfill{\large\bf Page 2/6}\hrule

\bigskip
\begin{problem}[10 pts]
The expression below gives parametric equations and parameter interval for the motion of a particle in the $xy$--plane.  
\begin{equation*}
\begin{cases}
x(t) = t^2 \\
y(t) = t+1
\end{cases}
\qquad (-1 \leq t \leq 1)
\end{equation*}
\begin{enumerate}
\item Identify the particle's path by finding a Cartesian equation for it (one in terms of $x$ and $y$, without the $t$).
\vspace{2cm}
\begin{flushright}
  \begin{tikzpicture}
    \draw (0cm,-0.2cm) rectangle (5cm,1.2cm);
  \end{tikzpicture}
\end{flushright}
\item Graph the Cartesian equation.
\item Indicate the portion of the graph traced by the particle, and the direction of motion.
\end{enumerate}
\end{problem}
\newpage


%%%%%%%%%%%%%%%%%%%%%%%%%%%%%%%%%%%%% Page 3
\noindent{\large\bf MATH 142}\hfill{\large\bf Test\#4}\hfill{\large\bf
  Fall 2016}\hfill{\large\bf Page 3/6}\hrule


\bigskip
\begin{problem}[10 pts]
Find the tangent to the curve $x(t) = \sec t$, $y(t) = \tan t$, $-\tfrac{\pi}{2} \leq t \leq \tfrac{\pi}{2}$ at the point $(\sqrt{2}, 1)$ (where $t={\pi}/{4}$).
\vspace{4.5cm}
\begin{flushright}
  \begin{tikzpicture}
    \draw (0cm,-0.2cm) rectangle (5cm,1.2cm);
  \end{tikzpicture}
\end{flushright}
\end{problem}
\hrule

\begin{problem}[10 pts]
Consider the region enclosed by the astroid $x(t) = \cos^3 t$, $y(t) = \sin^3 t$, $0 \leq t \leq 2\pi$.  Express its area as an integral on the parameter variable $t$, and compute its value.  
\vspace{4.5cm}
\begin{flushright}
  \begin{tikzpicture}
    \draw (0cm,-0.2cm) rectangle (5cm,1.2cm);
  \end{tikzpicture}
\end{flushright}
\end{problem}
\hrule

\begin{problem}[10 pts]
Consider the astroid $x(t) = 2\cos^3 t$, $y(t) = 2\sin^3 t$, $0 \leq t \leq 2\pi$.  Express its length as an integral on the parameter variable $t$, and compute its value.
\end{problem}
\vspace{4.5cm}
\begin{flushright}
  \begin{tikzpicture}
    \draw (0cm,-0.2cm) rectangle (5cm,1.2cm);
  \end{tikzpicture}
\end{flushright}
\newpage

%%%%%%%%%%%%%%%%%%%%%%%%%%%%%%%%%%%%% Page 4
\noindent{\large\bf MATH 142}\hfill{\large\bf Test\#4}\hfill{\large\bf
  Fall 2016}\hfill{\large\bf Page 4/6}\hrule

\bigskip
\begin{problem}[10 pts]
Find Cartesian coordinates of the point given in polar coordinates by $P(\sqrt{2}, \pi/4)$.
\vspace{3.5cm}
\begin{flushright}
  \begin{tikzpicture}
    \draw (0cm,1.5cm) rectangle (5cm,2.9cm);
  \end{tikzpicture}
\end{flushright}
\end{problem}
\hrule

\begin{problem}[10 pts]
Find a polar equation for the circle $x^2 + (y-3)^2 = 9$.
\vspace{5.5cm}
\begin{flushright}
  \begin{tikzpicture}
    \draw (0cm,-0.2cm) rectangle (5cm,1.2cm);
  \end{tikzpicture}
\end{flushright}
\end{problem}
\hrule

\begin{problem}[10 pts]
Sketch the graph and find a Cartesian equation for the curve given in polar coordinates by $r \cos \theta = -4$.
\vspace{6cm}
\begin{flushright}
  \begin{tikzpicture}
    \draw (0cm,-0.2cm) rectangle (5cm,1.2cm);
  \end{tikzpicture}
\end{flushright}
\end{problem}
\newpage

%%%%%%%%%%%%%%%%%%%%%%%%%%%%%%%%%%%%% Page 5
\noindent{\large\bf MATH 142}\hfill{\large\bf Test\#4}\hfill{\large\bf
  Fall 2016}\hfill{\large\bf Page 5/6}\hrule
  
\bigskip
\begin{problem}[10 pts]
Consider the region in the $xy$--plane enclosed by the cardioid with polar equation $r = 2(1+\cos \theta)$ (assume $0 \leq \theta \leq 2\pi$).  Express its area with an integral in terms of the parameter $\theta$, and compute its value.
\vspace{8cm}
\begin{flushright}
  \begin{tikzpicture}
    \draw (0cm,1.5cm) rectangle (5cm,2.9cm);
  \end{tikzpicture}
\end{flushright}
\end{problem}
\hrule

\begin{problem}[10 pts]
Consider the cardioid $r = 1 - \cos \theta$ (assume again $0 \leq \theta \leq 2\pi$).  Express its length as an integral of the parameter $\theta$, and compute its value.
\vspace{8cm}
\begin{flushright}
  \begin{tikzpicture}
    \draw (0cm,-0.2cm) rectangle (5cm,1.2cm);
  \end{tikzpicture}
\end{flushright}
\end{problem}
\newpage

%%%%%%%%%%%%%%%%%%%%%%%%%%%%%%%%%%%%% Page 6
\noindent{\large\bf MATH 142}\hfill{\large\bf Test\#4}\hfill{\large\bf
  Fall 2016}\hfill{\large\bf Page 6/6}\hrule

\bigskip
\begin{problem}[10 pts]
Find the area and length of one of the leaves of the eight-leaved rose $r = \cos 4\theta$.  Sketch the graph, and make sure to label all intersections with the $x$ and $y$--axis.
\end{problem}
\vspace{18cm}
\begin{flushright}
  \begin{tikzpicture}
    \draw (0cm, 1.8cm) rectangle (5cm,3.2cm);
    \draw (0cm,-0.2cm) rectangle (5cm,1.2cm);
  \end{tikzpicture}
\end{flushright}

\end{document}
