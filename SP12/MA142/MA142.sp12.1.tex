\documentclass[12pt]{article}

\usepackage{amsmath,amsthm,amsfonts,amssymb,amsxtra}
\usepackage{pgf,tikz}
\usetikzlibrary{arrows}
\renewcommand{\theenumi}{(\alph{enumi})} 
\renewcommand{\labelenumi}{\theenumi}

\pagestyle{empty}
\setlength{\textwidth}{7in}
\setlength{\oddsidemargin}{-0.5in}
\setlength{\topmargin}{-1.0in}
\setlength{\textheight}{9.5in}

\newtheorem{problem}{Problem}

\begin{document}

\noindent{\large\bf MATH 142}\hfill{\large\bf First Midterm.}\hfill{\large\bf
  Spring 2012}\hfill{\large\bf Page 1/4}\hrule

\bigskip
\begin{center}
  \begin{tabular}{|ll|}
    \hline & \cr
    {\bf Name: } & \makebox[12cm]{\hrulefill}\cr & \cr
    {\bf 4-digit code:} & \makebox[12cm]{\hrulefill}\cr & \cr
    \hline
  \end{tabular}
\end{center}
\begin{itemize}
\item Write your name and the last 4 digits of your SSN in the space provided above.
\item The test has four (4) pages, including this one.
\item Show sufficient work to justify all answers unless otherwise
  stated in the problem.  Correct answers with inconsistent work may
  not be given credit. 
\item Credit for each problem is given at the right of each problem
  number. 
\item No books, notes or calculators may be used on this test.
\end{itemize}
\hrule

\begin{center}
  \begin{tabular}{|c|c|c|}
    \hline
    &&\cr
    {\large\bf Page} & {\large\bf Max} & {\large\bf Points} \cr
    &&\cr
    \hline
    &&\cr
    {\Large 2} & \Large 30 & \cr
    &&\cr
    \hline
    &&\cr
    {\Large 3} & \Large 35 & \cr
    &&\cr
    \hline
    &&\cr
    {\Large 4} & \Large 35 & \cr
    &&\cr
    \hline\hline
    &&\cr
    {\large\bf Total} & \Large 100 & \cr
    &&\cr
    \hline
  \end{tabular}
\end{center}
\newpage

%%%%%%%%%%%%%%%%%%%%%%%%%%%%%%%%%%%%% Page 2
\noindent{\large\bf MATH 142}\hfill{\large\bf First Midterm.}\hfill{\large\bf
  Spring 2012}\hfill{\large\bf Page 2/4}\hrule

\bigskip
{\problem[10 pts] \em Use the fundamental Theorem of Calculus to find
  the derivative of the function}
\begin{equation*}
h(x) = \int_2^{1/x} \arctan t\, dt
\end{equation*}
\vspace{1.5cm}

\begin{flushright}
  \begin{tikzpicture}
    \draw (0cm,-0.2cm) rectangle (5cm,1.2cm);
  \end{tikzpicture}
\end{flushright}
\hrule
{\problem[20 pts] \em Evaluate the indefinite integrals below}

\noindent
$\text{[5 pts]}\qquad\displaystyle{\int \cos \theta  \sin \theta\, d\theta}$
\vspace{1.5cm}
\begin{flushright}
  \begin{tikzpicture}
    \draw (0cm,-0.2cm) rectangle (5cm,1.2cm);
  \end{tikzpicture}
\end{flushright}
\noindent
$\text{[5 pts]}\qquad\displaystyle{\int e^x \sin ( e^x) \, dx}$
\vspace{1.5cm}
\begin{flushright}
  \begin{tikzpicture}
    \draw (0cm,-0.2cm) rectangle (5cm,1.2cm);
  \end{tikzpicture}
\end{flushright}
\noindent
$\text{[10 pts]}\qquad\displaystyle{\int \frac{\cos \sqrt{t}}{\sqrt{t}}\, dt}$
\vspace{4cm}
\begin{flushright}
  \begin{tikzpicture}
    \draw (0cm,-0.2cm) rectangle (5cm,1.2cm);
  \end{tikzpicture}
\end{flushright}
\newpage

%%%%%%%%%%%%%%%%%%%%%%%%%%%%%%%%%%%%% Page 3
\noindent{\large\bf MATH 142}\hfill{\large\bf First Midterm.}\hfill{\large\bf
  Spring 2012}\hfill{\large\bf Page 3/4}\hrule

\bigskip
{\problem[15 pts] \em Find the area of the region bounded by the
  graphs of $y=\dfrac{1}{x}$, $y=\dfrac{1}{x^2}$ and $x=2$.}
\vspace{8.5cm}
\begin{flushright}
  \begin{tikzpicture}
    \draw (0cm,-0.2cm) rectangle (5cm,1.2cm);
  \end{tikzpicture}
\end{flushright}
\hrule
{\problem[20 pts] \em Find the volume of the solid obtained by
  rotating the region bounded by the curves $y=\ln x$, $y=1$, $y=2$
  and $x=0$ about the $y$--axis.}
\vspace{8.5cm}
\begin{flushright}
  \begin{tikzpicture}
    \draw (0cm,-0.2cm) rectangle (5cm,1.2cm);
  \end{tikzpicture}
\end{flushright}
\newpage

%%%%%%%%%%%%%%%%%%%%%%%%%%%%%%%%%%%%% Page 4
\noindent{\large\bf MATH 142}\hfill{\large\bf First Midterm.}\hfill{\large\bf
  Spring 2012}\hfill{\large\bf Page 4/4}\hrule

\bigskip
{\problem[15 pts] \em Find the volume of the solid obtained by
  rotating the region bounded by the curves $x=4y^2-y^3$ and $x=0$
  about the $y$--axis.}
\vspace{8.5cm}
\begin{flushright}
  \begin{tikzpicture}
    \draw (0cm,-0.2cm) rectangle (5cm,1.2cm);
  \end{tikzpicture}
\end{flushright}
\hrule
{\problem[20 pts] \em  Find all numbers $b$ such that the average
  value of $f(x)=2+6x-3x^2$ on the interval $[0,b]$ is 3.}
\vspace{8.5cm}
\begin{flushright}
  \begin{tikzpicture}
    \draw (0cm,-0.2cm) rectangle (5cm,1.2cm);
  \end{tikzpicture}
\end{flushright}
\end{document}
