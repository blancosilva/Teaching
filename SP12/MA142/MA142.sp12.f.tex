\documentclass[12pt]{article}

\usepackage{amsmath,amsthm,amsfonts,amssymb,amsxtra}
\usepackage{pgf,tikz}
\usetikzlibrary{arrows}
\renewcommand{\theenumi}{(\alph{enumi})} 
\renewcommand{\labelenumi}{\theenumi}

\pagestyle{empty}
\setlength{\textwidth}{7in}
\setlength{\oddsidemargin}{-0.5in}
\setlength{\topmargin}{-1.0in}
\setlength{\textheight}{9.5in}

\newtheorem{problem}{Problem}

\begin{document}

\noindent{\large\bf MATH 142}\hfill{\large\bf Final Exam.}\hfill{\large\bf
  Spring 2012}\hfill{\large\bf Page 1/8}\hrule

\bigskip
\begin{center}
  \begin{tabular}{|ll|}
    \hline & \cr
    {\bf Name: } & \makebox[12cm]{\hrulefill}\cr & \cr
    {\bf 4-digit code:} & \makebox[12cm]{\hrulefill}\cr & \cr
    \hline
  \end{tabular}
\end{center}
\begin{itemize}
\item Write your name and the last 4 digits of your SSN in the space provided above.
\item The test has eight (8) pages, including this one, and the
  scratch paper in page 8.
\item Enter your answers in the boxes provided.
\item You must show sufficient work to justify all answers unless
  otherwise stated in the problem.  Correct answers with inconsistent
  work may not be given credit.
\item Credit for each problem is given in parentheses at the right of
  the problem number.
\item No books, notes or calculators may be used on this test.
\end{itemize}
\hrule

\begin{center}
  \begin{tabular}{|c|c|c|}
    \hline
    &&\cr
    {\large\bf Page} & {\large\bf Max.~points} & {\large\bf Your points} \cr
    &&\cr
    \hline
    &&\cr
    {\Large 2} & \Large 20 & \cr
    &&\cr
    \hline
    &&\cr
    {\Large 3} & \Large 15 & \cr
    &&\cr
    \hline
    &&\cr
    {\Large 4} & \Large 20 & \cr
    &&\cr
    \hline
    &&\cr
    {\Large 5} & \Large 20 & \cr
    &&\cr
	\hline
    &&\cr
    {\Large 6} & \Large 15 & \cr
    &&\cr
	\hline
    &&\cr
    {\Large 7} & \Large 10 & \cr
    &&\cr
    \hline\hline
    &&\cr
    {\large\bf Total} & \Large 100 & \cr
    &&\cr
    \hline
  \end{tabular}
\end{center}
\newpage

%%%%%%%%%%%%%%%%%%%%%%%%%%%%%%%%%%%%% Page 2
\noindent{\large\bf MATH 142}\hfill{\large\bf Final Exam.}\hfill{\large\bf
  Spring 2012}\hfill{\large\bf Page 2/8}\hrule

\bigskip
{\problem[5 pts] \em Evaluate the integral $\displaystyle{\int t^2
    e^t\, dt}$.}
\vspace{2cm}
\begin{flushright}
  \begin{tikzpicture}
    \draw (0cm,-0.2cm) rectangle (5cm,1.2cm);
  \end{tikzpicture}
\end{flushright}
\hrule

{\problem[5 pts] \em Evaluate the integral $\displaystyle{\int (x-3) \sqrt{x^2-6x+5}\, dx}$.}
\vspace{2cm}
\begin{flushright}
  \begin{tikzpicture}
    \draw (0cm,-0.2cm) rectangle (5cm,1.2cm);
  \end{tikzpicture}
\end{flushright}
\hrule

{\problem[5 pts] \em Evaluate the integral $\displaystyle{\int
    \frac{1}{x^3 e^{1/x}}\, dx}$.}
\vspace{2cm}
\begin{flushright}
  \begin{tikzpicture}
    \draw (0cm,-0.2cm) rectangle (5cm,1.2cm);
  \end{tikzpicture}
\end{flushright}
\hrule

{\problem[5 pts] \em Evaluate the integral $\displaystyle{\int \frac{x^3+1}{(x+1)^2(x^2+4)}\, dx}$.}
\vspace{5.5cm}
\begin{flushright}
  \begin{tikzpicture}
    \draw (0cm,-0.2cm) rectangle (5cm,1.2cm);
  \end{tikzpicture}
\end{flushright}
\newpage


%%%%%%%%%%%%%%%%%%%%%%%%%%%%%%%%%%%%% Page 3
\noindent{\large\bf MATH 142}\hfill{\large\bf Final Exam.}\hfill{\large\bf
  Spring 2012}\hfill{\large\bf Page 3/8}\hrule

\bigskip
{\problem[5 pts] \em Evaluate the following integral, or indicate if it is
divergent: $\displaystyle{\int_0^\infty \frac{x \tan^{-1}x}{(1+x^2)^{3/2}}\,
dx}$.}
\vspace{5.5cm}
\begin{flushright}
  \begin{tikzpicture}
    \draw (0cm,-0.2cm) rectangle (5cm,1.2cm);
  \end{tikzpicture}
\end{flushright}
\hrule

{\problem[5 pts] \em Find the volume of the solid obtained by rotating the region bounded by $y=x^2$ and $y=2-x$ around the line $x=1$.}
\vspace{4.5cm}
\begin{flushright}
  \begin{tikzpicture}
    \draw (0cm,-0.2cm) rectangle (5cm,1.2cm);
  \end{tikzpicture}
\end{flushright}
\hrule
 
{\problem [5 pts] \em Find the volume of the solid obtained by rotating the region bounded by $y=e^{-x}$, $y=1/e$, and $x=0$ around the line $y=0$.}
\vspace{4.5cm}
\begin{flushright}
  \begin{tikzpicture}
    \draw (0cm,-0.2cm) rectangle (5cm,1.2cm);
  \end{tikzpicture}
\end{flushright}

\newpage

%%%%%%%%%%%%%%%%%%%%%%%%%%%%%%%%%%%%% Page 4
\noindent{\large\bf MATH 142}\hfill{\large\bf Final Exam.}\hfill{\large\bf
  Spring 2012}\hfill{\large\bf Page 4/8}\hrule

\bigskip
{\problem[5 pts] \em Find the general term of the sequence $\bigg\{
  3,2,\dfrac{5}{3}, \dfrac{3}{2}, \dfrac{7}{5}, \dfrac{4}{3}, \dotsc
  \bigg\}$, and compute its limit.}
\vspace{2cm}
\begin{flushright}
  \begin{tikzpicture}
    \draw (0cm,-0.2cm) rectangle (5cm,1.2cm);
    \draw (0cm,1.5cm) rectangle (5cm,2.9cm);
  \end{tikzpicture}
\end{flushright}
\hrule

{\problem[5 pts---all or nothing] \em Compute the limit of the sequence $\bigg\{
\dfrac{n^2+5n+2}{\sqrt{n^4+1}} \bigg\}_{n=1}^\infty$}
\vspace{2cm}
\begin{flushright}
  \begin{tikzpicture}
    \draw (0cm,-0.2cm) rectangle (5cm,1.2cm);
  \end{tikzpicture}
\end{flushright}
\hrule

{\problem[5 pts---all or nothing] \em Compute the limit of the sequence $\big\{ \tan (\pi - 1/n)
\big\}_{n=1}^\infty$}
\vspace{2cm}
\begin{flushright}
  \begin{tikzpicture}
    \draw (0cm,-0.2cm) rectangle (5cm,1.2cm);
  \end{tikzpicture}
\end{flushright}
\hrule

{\problem[5 pts---all or nothing] \em Compute $\displaystyle{\lim_{n\to \infty}} \bigg(
  1 - \dfrac{2}{n} \bigg)^n$}
\vspace{2cm}
\begin{flushright}
  \begin{tikzpicture}
    \draw (0cm,-0.2cm) rectangle (5cm,1.2cm);
  \end{tikzpicture}
\end{flushright}
\newpage

%%%%%%%%%%%%%%%%%%%%%%%%%%%%%%%%%%%%% Page 5
\noindent{\large\bf MATH 142}\hfill{\large\bf Final Exam.}\hfill{\large\bf
  Spring 2012}\hfill{\large\bf Page 5/8}\hrule
  
\bigskip

{\problem[8 pts] \em Study the convergence of the series
  $\displaystyle{\sum_{n=2}^\infty \frac{3^n+4^n}{5^n}}$.  If
  convergent, evaluate the sum.
\vspace{2cm}
\begin{flushright}
  \begin{tikzpicture}
    \draw (0cm,-0.2cm) rectangle (5cm,1.2cm);
    \draw (0cm,1.5cm) rectangle (5cm,2.9cm);
  \end{tikzpicture}
\end{flushright}
\hrule
{\problem[5 pts] \em Classify the series $\displaystyle{\sum_{n=1}^\infty
    \frac{\cos (\pi n)}{n^{2/3}}}$ as absolutely convergent,
  conditionally convergent, or divergent.} 
\vspace{3.5cm}
\begin{flushright}
  \begin{tikzpicture}
    \draw (0cm,-0.2cm) rectangle (5cm,1.2cm);
  \end{tikzpicture}
\end{flushright}
\hrule
{\problem[7 pts] \em Classify the series $\displaystyle{\sum_{n=1}^\infty
    \frac{(-1)^n n}{e^n}}$ as absolutely
  convergent, conditionally convergent, or divergent.}
\vspace{4cm}
\begin{flushright}
  \begin{tikzpicture}
    \draw (0cm,-0.2cm) rectangle (5cm,1.2cm);
  \end{tikzpicture}
\end{flushright}
\newpage

%%%%%%%%%%%%%%%%%%%%%%%%%%%%%%%%%%%%% Page 6
\noindent{\large\bf MATH 142}\hfill{\large\bf Final Exam.}\hfill{\large\bf
  Spring 2012}\hfill{\large\bf Page 6/8}\hrule

\bigskip
{\problem[8 pts] \em Find the interval of convergence of the series
  $\displaystyle{\sum_{n=0}^\infty \frac{(-3)^n x^n}{\sqrt{n+1}}}$.}
\vspace{7.5cm}
\begin{flushright}
  \begin{tikzpicture}
    \draw (0cm,-0.2cm) rectangle (5cm,1.2cm);
  \end{tikzpicture}
\end{flushright}
\hrule
{\problem[7 pts] \em Express the function $f(x) = \dfrac{2x}{x^3+8}$ as a
  power series.}
\vspace{9cm}
\begin{flushright}
  \begin{tikzpicture}
    \draw (0cm,-0.2cm) rectangle (5cm,1.2cm);
  \end{tikzpicture}
\end{flushright}
\newpage

%%%%%%%%%%%%%%%%%%%%%%%%%%%%%%%%%%%%% Page 7
\noindent{\large\bf MATH 142}\hfill{\large\bf Final Exam.}\hfill{\large\bf
  Spring 2012}\hfill{\large\bf Page 7/8}\hrule

\bigskip
{\problem[10 pts] \em Express the function $f(x) =
  \dfrac{1}{\sqrt{2x-x^2}}$ as a Taylor series expanded about
  $a=1$. [{\small Hint: complete the square first, and then use the
    Taylor expression for $(1+x)^r$.}]
\vspace{7cm}
\begin{flushright}
  \begin{tikzpicture}
    \draw (0cm,-0.2cm) rectangle (5cm,1.2cm);
  \end{tikzpicture}
\end{flushright}
\hrule

\bigskip
\noindent{\large Scratch paper}
\newpage
%%%%%%%%%%%%%%%%%%%%%%%%%%%%%%%%%%%%% Page 8
\noindent{\large\bf MATH 142}\hfill{\large\bf Final Exam.}\hfill{\large\bf
  Spring 2012}\hfill{\large\bf Page 8/8}\hrule

\bigskip
\noindent{\large Scratch paper}


\end{document}
