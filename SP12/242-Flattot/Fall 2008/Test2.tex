\documentclass[a4paper,11pt]{article}

\usepackage{amsfonts,amsmath,amssymb,amscd,amstext}
\usepackage{theorem}

\renewcommand{\thepage}{} % astuce pour supprimer les num�ros de pages

\addtolength{\textheight}{3 cm}
\addtolength{\topmargin}{-2cm}
\addtolength{\textwidth}{3cm}
\addtolength{\oddsidemargin}{-2.1cm}
\addtolength{\evensidemargin}{-2.1cm}
%\theoremstyle{definition}

\theorembodyfont{\rmfamily}
\newtheorem{exo}{Exercise}

\def\ch{\mathrm{ch}}
\def\sh{\mathrm{sh}}
\def\th{\mathrm{th}}

\newcommand{\exoskip}{\addvspace{15pt plus 2pt minus 3pt}}
\newcommand{\R}{\mathbb{R}}
\newcommand{\C}{\mathbb{C}}
\newcommand{\N}{\mathbb{N}}
\newcommand{\Q}{\mathbb{Q}}
\newcommand{\Z}{\mathbb{Z}}
\newcommand{\K}{\mathbb{K}}
\newcommand{\eps}{\varepsilon}
\newcommand{\inter}{\stackrel{\circ}}
\newcommand{\fr}{\mathrm{Fr}}
\newcommand{\ds}{\displaystyle}
\newcommand{\E}{\mathrm{E}}

\begin{document}
{\large
\begin{center}{\bf Math 242 Test 2, Monday 20 October}
\end{center}}
\vspace*{0.5cm}
\large{\textbf{Name:} \hfill \textbf{Last 4 digits of SSN:}}\hspace{2cm}
\vspace{1cm}

\noindent Show all work \textbf{\LARGE{clearly}}, \textbf{\LARGE{make sentences}}. No work means no credit. The points are:\\
ex1: 25, ex2: 20, ex3: 20, ex4: 15, ex5: 20.

\vspace*{1cm}


\begin{exo}
We give the differential equation:
\[ \frac{dx}{dt}=2x^2-10x+12 .\]
\begin{enumerate}
\item What are the critical points ? Use a phase diagram to determine whether each critical point is stable or unstable.
\vspace*{4cm}
\item Solve this differential equation with $x_0=1$.
\end{enumerate}
\end{exo}


\newpage

\begin{exo}
We give an initial value problem and its exact solution $y(x)$:
\[ y'=2xy^2, \quad y(0)=1, \quad y(x)=\frac{1}{1-x^2}. \]
Apply Euler's method to approximate the solution on the interval $[0,0.4]$ with step size $h=0.1$. Write the formula you use for the computation. Then compare the four-decimal-place values of the approximate solution with the values of the exact solution using the following array. Does this step size look good ?
\vspace*{6cm}
{\renewcommand{\arraystretch}{1.5}
\[ \begin{array}{|r|c|c|c|c|c|}
      \hline \mbox{x} & 0 & 0.1 & 0.2 & 0.3 & 0.4  \\
      \hline \mbox{approx solution} & \hspace*{2cm} & \hspace*{2cm} & \hspace*{2cm} & \hspace*{2cm} & \hspace*{2cm}   \\
      \hline \mbox{exact solution} & \hspace*{2cm} & \hspace*{2cm} & \hspace*{2cm} & \hspace*{2cm} & \hspace*{2cm}  \\
      \hline
    \end{array} \]}
\end{exo}
%2.4.7

\vspace*{0.5cm}

\begin{exo}
Solve the differential equation:
\[ y^{(3)}-9y''+24y'-20y=0,\]
using the fact that the function $x\mapsto e^{5x}$ is solution of this differential equation.\\
Then find the unique solution satisfying the initial conditions:
\[ y(0)=1, \ y'(0)=0, \ y''(0)=-1.\]
\end{exo}

\newpage
\vspace*{3cm}

\begin{exo}
Find a linear homogeneous constant-coefficient equation with the general solution:
\[ y(x)= Ae^{-3x}+B\cos(2x)+ C\sin(2x)+x(D\cos(2x)+E\sin(2x)) .\]
\end{exo}

\vspace*{5cm}

\begin{exo}
Solve the initial value problem, where $y_p$ is a particular solution of the differential equation:
\[ y''+2y'-8y=14e^{3x}, \quad y(0)=3,\ y'(0)=2 \quad y_p(x)=2e^{3x}.\]
\end{exo}

\end{document}