\documentclass[a4paper,11pt]{article}

\usepackage{amsfonts,amsmath,amssymb,amscd,amstext}
\usepackage{theorem}

\renewcommand{\thepage}{} % astuce pour supprimer les num�ros de pages

\addtolength{\textheight}{3 cm}
\addtolength{\topmargin}{-2cm}
\addtolength{\textwidth}{3cm}
\addtolength{\oddsidemargin}{-2.1cm}
\addtolength{\evensidemargin}{-2.1cm}
%\theoremstyle{definition}

\theorembodyfont{\rmfamily}
\newtheorem{exo}{Exercise}

\def\ch{\mathrm{ch}}
\def\sh{\mathrm{sh}}
\def\th{\mathrm{th}}

\newcommand{\exoskip}{\addvspace{15pt plus 2pt minus 3pt}}
\newcommand{\R}{\mathbb{R}}
\newcommand{\C}{\mathbb{C}}
\newcommand{\N}{\mathbb{N}}
\newcommand{\Q}{\mathbb{Q}}
\newcommand{\Z}{\mathbb{Z}}
\newcommand{\K}{\mathbb{K}}
\newcommand{\eps}{\varepsilon}
\newcommand{\inter}{\stackrel{\circ}}
\newcommand{\fr}{\mathrm{Fr}}
\newcommand{\ds}{\displaystyle}
\newcommand{\E}{\mathrm{E}}

\begin{document}
{\large
\begin{center}{\bf Math 242 Test 1, Friday 19 September}
\end{center}}
\vspace*{0.5cm}
\large{\textbf{Name:} \hfill \textbf{Last 4 digits of SSN:}}\hspace{2cm}
\vspace{1cm}

\noindent Show all work clearly. No work means no credit. The points are:\\
ex1: 15, ex2: 15, ex3: 15, ex4: 20, ex5: 20, ex6: 15.
\vspace*{1cm}

\begin{exo}
The skid marks made by an automobile indicated that its brakes were fully applied for a distance of 75 m before it came to a stop. The car in question is known to have a \emph{\textbf{constant deceleration}} of 20 $m/s^2$ under these conditions.
\begin{enumerate}
\item Find the expression of the motion of the automobile when the brakes started (take $v_0$ for initial velocity).
\vspace*{4cm}
\item How fast - in m/s - was the car traveling when the brakes were first applied ? \\
(You will use that $\sqrt{40*75}=10\sqrt{30}$)
\end{enumerate}
\end{exo}
%1.2.31

\newpage

\begin{exo}
Solve the differential equation :
\[ xy^2+3y^2-x^2y'=0 .\]
\end{exo}
%1.review.2

\vspace*{7cm}

\begin{exo}
We are considering the following differential equation:
\[ 2x^2y+x^3y'=1. \]
\begin{enumerate}
\item On which intervals does there exist a unique solution?
\vspace*{2cm}
\item Solve the equation with the initial value $y(1)=3$.
\end{enumerate}
\end{exo}
%1.review.7

\newpage


\begin{exo}
We considere the following differential equation:
\[ 3y+x^3y^4+3xy'=0. \]
\begin{enumerate}
\item What kind of equation is it?
\vspace*{1cm}
\item What substitution do we have to do?
\vspace*{2cm}
\item What kind of differential equation do we obtain after the substitution?
\vspace*{4cm}
\item Solve this last differential equation and then find the expression of $y$.
\end{enumerate}
\end{exo}
%1.review.27

\newpage

\begin{exo}
We consider the following differential equation:
\[ 2xy^3+e^x+\left(3x^2y^2+\sin y\right) y'=0 .\]
\begin{enumerate}
\item Show that this equation is exact.
\vspace*{3cm}
\item Then solve this differential equation.
\end{enumerate}
\end{exo}
%1.review.4

\vspace*{7cm}


\begin{exo}
We consider the following differential equation:
\[ x^2y'=xy+3y^2 .\]
\begin{enumerate}
\item Write this differential equation as a homogeneous one.
\vspace*{1.5cm}
\item Then solve this differential equation.
\end{enumerate}
\end{exo}
%1.review.11

\end{document}