\documentclass[11pt]{amsart}
\overfullrule = 0pt

\usepackage{amssymb,amsmath,amsthm,epsfig}
\usepackage{showkeys}

\pagestyle{empty}

\topmargin=-0.5in
\textwidth=6.5in
\textheight=9.0in
\oddsidemargin=0.0in
\evensidemargin=0.0in

\begin{document}
\thispagestyle{empty}
\begin{center}
\Large{\bf Elementary Differential Equations \\ Math 242}
\end{center}
\vspace*{0.8cm}
\large{{\bf Instructor:} Antoine Flattot, 777-5882, flattot@math.sc.edu}\\
\large{{\bf Lectures:} 	MWF 1:25PM- 2:15PM, LC 101}\\
\large{{\bf Office Hours:} M 10:15-11:45, W 2:30-4:00 or by appointment, LC 400D}\\[0.8cm]
\noindent {\bf Overview:} Many of the principles, or laws, underlying the behavior of the natural world
are statements or relations involving rates at which things happen. When expressed in mathematical
terms, the relations are equations and the rates are derivatives. Equations containing derivatives are
called differential equations. Therefore, to understand and to investigate problems involving the 
motion of particules, the flow of current in electric circuits, the dissipation of heat in solid objects,
the propagation and detection of seismic waves, the change of populations, it is necessary to 
be able to solve or study differential equations.\\
We will interesting mainly in the resolution of some particuliar kind of differential equations, and in the case where we are not able to solve them, we will see a numerical approach to obtain solutions.\\[0.3cm] 
\noindent {\bf Learning Outcomes:} Students will master concepts and solve problems based upon ordinary differential equations of first order, higher order linear equations, Laplace transform methods, series methods; numerical solution of differential equations. Applications to physical sciences and engineering.\\[0.3cm]
\noindent {\bf Instructional Objectives:} We will cover as much as possible of the following topics:\\
\hspace*{0.3cm} (1) First order differential equations.\\
\hspace*{0.3cm} (2) Numerical solutions of differential equations.\\
\hspace*{0.3cm} (3) Higher order linear differential equations.\\
\hspace*{0.3cm} (4) The Laplace transform.\\[0.3cm]
\noindent {\bf Textbook:} {\em Differential Equations, computing and modeling}, by Edwards and Penney,
Fourth Edition.\\[0.3cm]
{\bf Prerequisite:} Completion of Math 142 with a grade of C or better, or qualification by placement.\\[0.3cm]
{\bf Grading:} Your grade in this course will be based on your performance on  quizzes, homeworks, three midterms 
exams and a final exam. The weights assigned to each of these components will be:
\[
\begin{array} {ll}
\mbox{ Quizzes \& Homeworks } & 25 \% \\
\mbox{ Three midterm exams } & 15 \% \mbox{ each} \\
\mbox{ Final exam } & 30 \%
\end{array}
\]
Course grades will be determined according to the scale:\\
A: 90-100, B+: 85-89, B: 80-84, C+: 75-79, C: 70-74, D+: 65-69, D: 60-64, F: 0-59.\\[0.3cm]
{\bf Exams:} {\em \bf Tentative} dates and topics for these exams are:
\[
\begin{array}{lll}
\mbox{ Exam 1 } & \mbox{ Monday, September 22} & \mbox{ Chapter } 1 \\
\mbox{ Exam 2 } & \mbox{ Monday, October 20} & \mbox{ Chapters 2 \& part 3}\\
\mbox{ Exam 3 } & \mbox{ Monday, Novembver 24} & \mbox{ end of Chapter 3 and Chapter } 7\\
\mbox{ Final} & \mbox{ Friday, December 12 2:00 PM}  & \mbox{ Comprehensive}
\end{array}
\]
The dates of the tests will be confirmed during the class, generally the previous week.\\
Thursday October 2 is the last day to drop a course or withdraw without a grade of ``WF'' being recorded.\\
There will be no make-up exams. In case that you miss an exam because of  documented reasons of illness, family
emergency, or participation in a University sponsored event, and if you have contact me immediately and provide me 
with the documentation, then the grade of the missed exam will be substituted by the grade of the final 
exam.\\[0.3cm]
{\bf Quizzes \& Homeworks:} A homework assignement (of approximately 3 or 6
problems) will be assigned each Monday and Friday and it will be collected at the next class. A quiz assignement, consisting on a course question and/or a short exercise, will be assigned each Wednesday. \textbf{Clear work} is desired; and in the case where the argument and/or the result cannot be read, no credits will be given.\\
There will be  \textbf{no make-up homeworks}. In case that you are unable to attend the class
during the day that you turn in homework or the day of the quizz, because of a documented
reason of illness, family emergency or of participation in a University
sponsored event, and if you have contact me immediately and provide me with the
documentation, then the grade of your missed homework or quiz will be dropped.\\[0.3cm]
{\bf Attendance:} Regular class attendance is important. Consistent
with the USC Undergraduate Bulletin, a grade penalty may be applied to
any student missing more than 4 classes.\\[0.3cm]
{\bf Math Labs:} The Math Labs is a free tutoring service. 
No appointment is necessary. See ``http://www.math.sc.edu/mathlab.html''.\\[0.3cm]
{\bf Academic Honesty:} Cheating and plagiarism will not be tolerated
in this course. You may discuss homework problems with others, but do
not copy solutions from another student or from a book. Violations of
this policy will be dealt with a matter consistent with University
regulations.\\[0.3cm]
{\bf Class Policy:} Turn off your cell phone during the class. No newspapers or headphones are allowed.


\end{document}
 