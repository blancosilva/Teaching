\documentclass[a4paper,11pt]{article}

\usepackage{amsfonts,amsmath,amssymb,amscd,amstext}
\usepackage{theorem}

\renewcommand{\thepage}{} % astuce pour supprimer les num�ros de pages

\addtolength{\textheight}{3 cm}
\addtolength{\topmargin}{-2cm}
\addtolength{\textwidth}{3cm}
\addtolength{\oddsidemargin}{-2.1cm}
\addtolength{\evensidemargin}{-2.1cm}
%\theoremstyle{definition}

\theorembodyfont{\rmfamily}
\newtheorem{exo}{Exercise}

\def\ch{\mathrm{ch}}
\def\sh{\mathrm{sh}}
\def\th{\mathrm{th}}

\newcommand{\exoskip}{\addvspace{15pt plus 2pt minus 3pt}}
\newcommand{\R}{\mathbb{R}}
\newcommand{\C}{\mathbb{C}}
\newcommand{\N}{\mathbb{N}}
\newcommand{\Q}{\mathbb{Q}}
\newcommand{\Z}{\mathbb{Z}}
\newcommand{\K}{\mathbb{K}}
\newcommand{\eps}{\varepsilon}
\newcommand{\inter}{\stackrel{\circ}}
\newcommand{\fr}{\mathrm{Fr}}
\newcommand{\ds}{\displaystyle}
\newcommand{\E}{\mathrm{E}}

\begin{document}
{\large
\begin{center}{\bf Math 242 Test 2, Tuesday 1 April}
\end{center}}
\vspace*{0.5cm}
\large{\textbf{Name:} \hfill \textbf{Last 4 digits of SSN:}}\hspace{2cm}
\vspace{1cm}

\noindent Show all work \textbf{\LARGE{clearly}}, \textbf{\LARGE{make sentences}}. No work means no credit. The points are:\\
ex1: 15, ex2: 15, ex3: 15, ex4: 15, ex5: 10, ex6: 15, ex7: 20.
\vspace*{1cm}


\begin{exo}
We give the differential equation:
\[ \frac{dx}{dt}=x^2-5x+4 .\]
\begin{enumerate}
\item What are the critical points ? Use a phase diagram to determine whether each critical point is stable or unstable.
\vspace*{4cm}
\item Solve this differential equation with $x_0=2$.
\end{enumerate}
\end{exo}
%2.2.9

\newpage

\begin{exo}
We give an initial value problem and its exact solution $y(x)$:
\[ y'=-3x^2y, \quad y(0)=3, \quad y(x)=3e^{-x^3}. \]
Apply Euler's method to approximate the solution on the interval $[0,1]$ with step size $h=0.25$. Write the formula you use for the computation. Then compare the four-decimal-place values of the approximate solution with the values of the exact solution using the following array. Does this step size look good ?
\vspace*{6cm}
{\renewcommand{\arraystretch}{1.5}
\[ \begin{array}{|r|c|c|c|c|c|}
      \hline \mbox{x} & 0 & 0.25 & 0.5 & 0.75 & 1 \\
      \hline \mbox{approx solution} & \hspace*{2cm} & \hspace*{2cm} & \hspace*{2cm} & \hspace*{2cm} & \hspace*{2cm} \\
      \hline \mbox{exact solution} & \hspace*{2cm} & \hspace*{2cm} & \hspace*{2cm} & \hspace*{2cm} & \hspace*{2cm} \\
      \hline
    \end{array} \]}
\end{exo}
%2.4.7

\vspace*{2cm}

\begin{exo}
Solve the differential equation:
\[ y^{(3)}-14y''+49y'-36y=0,\]
using the fact that the function $x\mapsto e^{9x}$ is solution of this differential equation.\\
Then find the unique solution satisfying the initial conditions:
\[ y(0)=1, \ y'(0)=0, \ y''(0)=6.\]
\end{exo}

\newpage
\vspace*{4cm}

\begin{exo}
Give the form of a particular solution in each case, but do not determine the values of the coefficients:
\begin{enumerate}
\item $y^{(13)}+2y'=(x^5+x^2+1)$,
\item $y^{(3)}-y''-y'+y=(x^3-2x)e^x$,
\item $y^{(3)}-y''-y'+y=5e^x(x^2+5x-8)\cos(27x)$.
\end{enumerate}
\end{exo}

\vspace*{7cm}

\begin{exo}
Find a linear homogeneous constant-coefficient equation with the general solution:
\[ y(x)= Ae^{-x}+B\cos(3x)+ C\sin(3x)+x(D\cos(3x)+E\sin(3x)) .\]
\end{exo}

\newpage

\begin{exo}
Solve the differential equation :
\[ y''+9y=2\sec(3x).\]
To find a particular solution you will use the method of variation of parameters.\\
We recall that $\ds{\sec x=\frac{1}{\cos x}}$ and that $\ds{\int \tan(3x)dx=-\frac{1}{3}\ln |\cos (3x)|}$.
\end{exo}

\newpage

\begin{exo}
Solve the initial value problem:
\[ y''+2y'-8y=8x-6+12e^{2x}, \quad y(0)=-1,\ y'(0)=7.\]
\end{exo}

\end{document}