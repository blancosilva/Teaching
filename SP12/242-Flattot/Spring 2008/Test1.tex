\documentclass[a4paper,11pt]{article}

\usepackage{amsfonts,amsmath,amssymb,amscd,amstext}
\usepackage{theorem}

\renewcommand{\thepage}{} % astuce pour supprimer les num�ros de pages

\addtolength{\textheight}{3 cm}
\addtolength{\topmargin}{-2cm}
\addtolength{\textwidth}{3cm}
\addtolength{\oddsidemargin}{-2.1cm}
\addtolength{\evensidemargin}{-2.1cm}
%\theoremstyle{definition}

\theorembodyfont{\rmfamily}
\newtheorem{exo}{Exercise}

\def\ch{\mathrm{ch}}
\def\sh{\mathrm{sh}}
\def\th{\mathrm{th}}

\newcommand{\exoskip}{\addvspace{15pt plus 2pt minus 3pt}}
\newcommand{\R}{\mathbb{R}}
\newcommand{\C}{\mathbb{C}}
\newcommand{\N}{\mathbb{N}}
\newcommand{\Q}{\mathbb{Q}}
\newcommand{\Z}{\mathbb{Z}}
\newcommand{\K}{\mathbb{K}}
\newcommand{\eps}{\varepsilon}
\newcommand{\inter}{\stackrel{\circ}}
\newcommand{\fr}{\mathrm{Fr}}
\newcommand{\ds}{\displaystyle}
\newcommand{\E}{\mathrm{E}}

\begin{document}
{\large
\begin{center}{\bf Math 242 Test 1, Tuesday 12 February}
\end{center}}
\vspace*{0.5cm}
\large{\textbf{Name:} \hfill \textbf{Last 4 digits of SSN:}}\hspace{2cm}
\vspace{1cm}

\noindent Show all work clearly. No work means no credit. The points are:\\
ex1: 10, ex2: 10, ex3: 15, ex4: 15, ex5: 20, ex6: 15, ex7: 15.
\vspace*{1cm}

\begin{exo}
The skid marks made by an automobile indicated that its brakes were fully applied for a distance of 75 m before it came to a stop. The car in question is known to have a \emph{\textbf{constant deceleration}} of 20 $m/s^2$ under these conditions.
\begin{enumerate}
\item Find the expression of the motion of the automobile when the brakes started (take $v_0$ for initial velocity).
\vspace*{4cm}
\item How fast - in m/s - was the car traveling when the brakes were first applied ? \\
(You will use that $\sqrt{40*75}=10\sqrt{30}$)
\end{enumerate}
\end{exo}
%1.2.31

\vspace*{5cm}

\begin{exo}
Solve the differential equation :
\[ y'=e^x+2+2y^2+y^2 e^x .\]
\textit{Hint:} Factor the right-hand side.
\end{exo}


\newpage

\begin{exo}
We are considering the following differential equation:
\[ xy'=2y+x^3 \cos x. \]
\begin{enumerate}
\item On which intervals does there exist a unique solution?
\vspace*{2cm}
\item Solve the equation with the initial value $y(\pi)=2\pi^2$.
\end{enumerate}
\end{exo}
%1.5.18

\vspace*{8cm}

\begin{exo}
Just before midday, the body of an apparent homicide victim is found in a room that is kept at a constant temperature of $70^\circ$F. At noon, the temperature of the body is $80^\circ$F and at 1pm it is $75^\circ$F.\\
Assume that the temperature of the body at the time of death was $100^\circ$F (in fact the natural temperature of a body is $98.6^\circ$F, but to simplify the computations we take $100^\circ$F instead of).
\begin{enumerate}
\item Using the Newton's law of cooling, determine the temperature of the body at a time $t$ (we take $t=0$ at noon). You will be carefull about the absolute value.
\newpage
\item What was the time of death (you can use that $\frac{\ln(3)}{\ln(2)}\approx 1.59$) ?
\end{enumerate}
\end{exo}
%1.4.65

\vspace*{5cm}

\begin{exo}
We considere the following differential equation:
\[ x^2y'+2xy=5y^4. \]
\begin{enumerate}
\item What kind of equation is it?
\vspace*{1cm}
\item What substitution do we have to do?
\vspace*{1cm}
\item What kind of differential equation do we obtain after the substitution?
\vspace*{1cm}
\item Solve this last differential equation and then find the expression of $y$.
\end{enumerate}
\end{exo}
%1.6.22

\newpage

\begin{exo}
We consider the following differential equation:
\[ e^y+y\cos x+\left(xe^y+\sin x\right) y'=0 .\]
\begin{enumerate}
\item Show that this equation is exact.
\vspace*{3cm}
\item Then solve this differential equation.
\end{enumerate}
\end{exo}
%1.6.23 review

\vspace*{7cm}


\begin{exo}
We consider the following differential equation:
\[ x^3y'=x^2 y-y^3 .\]
\begin{enumerate}
\item Write this differential equation as a homogeneous one.
\vspace*{1.5cm}
\item Then solve this differential equation.
\end{enumerate}
\end{exo}
%1.6.14

\end{document}