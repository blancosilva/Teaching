\documentclass[a4paper,11pt]{article}

\usepackage{amsfonts,amsmath,amssymb,amscd,amstext}
\usepackage{theorem}

\renewcommand{\thepage}{} % astuce pour supprimer les num�ros de pages

\addtolength{\textheight}{3 cm}
\addtolength{\topmargin}{-2cm}
\addtolength{\textwidth}{3cm}
\addtolength{\oddsidemargin}{-2.1cm}
\addtolength{\evensidemargin}{-2.1cm}
%\theoremstyle{definition}

\theorembodyfont{\rmfamily}
\newtheorem{exo}{Exercise}

\def\ch{\mathrm{ch}}
\def\sh{\mathrm{sh}}
\def\th{\mathrm{th}}

\newcommand{\exoskip}{\addvspace{15pt plus 2pt minus 3pt}}
\newcommand{\R}{\mathbb{R}}
\newcommand{\C}{\mathbb{C}}
\newcommand{\N}{\mathbb{N}}
\newcommand{\Q}{\mathbb{Q}}
\newcommand{\Z}{\mathbb{Z}}
\newcommand{\K}{\mathbb{K}}
\newcommand{\eps}{\varepsilon}
\newcommand{\inter}{\stackrel{\circ}}
\newcommand{\fr}{\mathrm{Fr}}
\newcommand{\ds}{\displaystyle}
\newcommand{\E}{\mathrm{E}}

\begin{document}
{\large
\begin{center}{\bf Math 242 Test 3, Wednesday 28 November}
\end{center}}
\vspace*{0.5cm}
\large{\textbf{Name:} \hfill \textbf{Last 4 digits of SSN:}}\hspace{2cm}
\vspace{1cm}

\noindent Show all work clearly, \textbf{MAKE SENTENCES}. No work means no credit. The points are:\\
ex1: 25, ex2: 15, ex3: 25, ex4: 20 and the course questions are over 15 points.
\vspace*{1cm}

\textbf{\large Course Questions}
\begin{enumerate}
\item Method of variation of parameters in the case $n=2$:\\
We consider the second-order linear differential equation 
\[ y''+P(x)y'+Q(x)y=f(x) ,\]
where $P$, $Q$ and $f$ are continuous. A general solution is given by:
\[ y_c(x)=c_1 y_1(x)+c_2y_2(x) ,\]
where $c_1$ and $c_2$ are constants.\\
Of what form can we search a particular solution ? To find this, we need to impose a condition, what is this condition ?
\vspace*{4cm}
\item Give the definition of the Laplace transform.
\vspace*{4cm}
\item State the theorem about the Laplace transform of integrals.
\end{enumerate}

\newpage

\begin{exo}
Find a particular solution of the following differential equation:
\[ y''-4y=13x\cos(3x).\]
You will use the fact that the family $(x^k\cos(3x),x^k\sin(3x))_k$ is linearly independent. This means, for example, that if $\alpha_1x\cos(3x)+\alpha_0\cos(3x)+\beta_1x\sin(3x)+\beta_0\sin(3x)=0$ then $\alpha_1=\alpha_0=\beta_1=\beta_0=0$.
\end{exo}

\newpage

\begin{exo}
Find the inverse Laplace transform of the functions:
\[ F(s)=\frac{3s-7}{s^2+9}, \quad \mbox{and} \quad H(s)=\frac{4}{s^2(s^2+16)}. \]
\end{exo}

\vspace*{10cm}

\begin{exo}
Solve the initial value problem using the Laplace transform:
\[ y''+y=\sin(2t), \quad y(0)=1, y'(0)=0. \]
\end{exo}

\newpage
\vspace*{5cm}

\begin{exo}
\begin{enumerate}
\item Use the partial fraction to find the inverse Laplace transform of 
\[ F(s)=\frac{3s+2}{s^2-s-12}. \]
\item Write the partial fractions of the rational function (we do not ask the value of the coefficients):
\[ G(s)=\frac{9s^4-5s^2+s}{(s-7)(s+3)^2(s^2+5s+10)^2}. \]
\end{enumerate}
\end{exo}


%\begin{exo}
%Solve the initial value problem using the Laplace transform:
%\[ x^{(3)}+x''-6x'=0, \qquad x(0)=0, x'(0)=x''(0)=1.\]
%\end{exo}


\end{document}