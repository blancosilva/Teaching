\documentclass[a4paper,11pt]{article}

\usepackage{amsfonts,amsmath,amssymb,amscd,amstext}
\usepackage{theorem}

\renewcommand{\thepage}{} % astuce pour supprimer les num�ros de pages

\addtolength{\textheight}{3 cm}
\addtolength{\topmargin}{-2cm}
\addtolength{\textwidth}{3cm}
\addtolength{\oddsidemargin}{-2.1cm}
\addtolength{\evensidemargin}{-2.1cm}
%\theoremstyle{definition}

\theorembodyfont{\rmfamily}
\newtheorem{exo}{Exercise}

\def\ch{\mathrm{ch}}
\def\sh{\mathrm{sh}}
\def\th{\mathrm{th}}

\newcommand{\exoskip}{\addvspace{15pt plus 2pt minus 3pt}}
\newcommand{\R}{\mathbb{R}}
\newcommand{\C}{\mathbb{C}}
\newcommand{\N}{\mathbb{N}}
\newcommand{\Q}{\mathbb{Q}}
\newcommand{\Z}{\mathbb{Z}}
\newcommand{\K}{\mathbb{K}}
\newcommand{\eps}{\varepsilon}
\newcommand{\inter}{\stackrel{\circ}}
\newcommand{\fr}{\mathrm{Fr}}
\newcommand{\ds}{\displaystyle}
\newcommand{\E}{\mathrm{E}}

\begin{document}
{\large
\begin{center}{\bf Math 242 Test 1, Friday 21 September}
\end{center}}
\vspace*{0.5cm}
\large{\textbf{Name:} \hfill \textbf{Last 4 digits of SSN:}}\hspace{2cm}
\vspace{1cm}

\noindent Show all work clearly. No work means no credit. The points are:\\
ex1: 7, ex2: 9, ex3: 9, ex4: 10, ex5: 10 plus 5 point for the writing.
\vspace*{1cm}

\begin{exo}
The skid marks made by an automobile indicated that its brakes were fully applied for a distance of 75 m before it came to a stop. The car in question is known to have a \emph{\textbf{constant deceleration}} of 20 $m/s^2$ under these conditions.
\begin{enumerate}
\item Find the expression of the motion of the automobile when the brakes start (take $v_0$ for initial velocity).
\vspace*{4cm}
\item How fast - in km/h - was the car traveling when the brakes were first applied ?
\vspace*{5cm}
\end{enumerate}
\end{exo}
%1.2.31

\begin{exo}
We are considering the following differential equation:
\[ 3xy'+y=12x. \]
\begin{enumerate}
\item On which intervals does there exist a unique solution?
\vspace*{2cm}
\item Solve the equation with the initial value $y(1)=0$.
\end{enumerate}
\end{exo}
%1.5.8

\newpage
\vspace*{6cm}

\begin{exo}
A cake is removed from an oven at $210^\circ$F and left to cool at room temperature, wich is $70^\circ$F. After 30 min the temperature of the cake is $140^\circ$F.
\begin{enumerate}
\item Using the Newton's law of cooling, determine the temperature of the cake at a time $t$ (we take $t=0$ when the cake is removed from the oven).
\vspace*{7cm}
\item When will the temperature of cake be $100^\circ$F (you can use that $\frac{\ln(14/3)}{\ln(2)}\approx 2.22$) ?
\end{enumerate}
\vspace*{3cm}
\end{exo}
%1.4.49

\newpage

\begin{exo}
We consider the following differential equation:
\[ (\cos x+\ln y)dx+\left(\frac{x}{y}+e^{y}\right)dy=0 .\]
\begin{enumerate}
\item Show that this equation is exact.
\vspace*{4cm}
\item Then solve this differential equation.
\vspace*{7cm}
\end{enumerate}
\end{exo}
%1.6.37

\begin{exo}
We consider the following differential equation:
\[ x(x+y)y'+y(3x+y)=0 .\]
\begin{enumerate}
\item Write this differential equation as a homogeneous one.
\vspace*{3cm}
\newpage
\item Then solve this differential equation (you can use a formula of type $\int \frac{f'}{f} =\hdots$)
\end{enumerate}
\end{exo}
%1.6.15

\end{document}