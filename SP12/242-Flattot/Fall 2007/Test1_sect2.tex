\documentclass[a4paper,11pt]{article}

\usepackage{amsfonts,amsmath,amssymb,amscd,amstext}
\usepackage{theorem}

\renewcommand{\thepage}{} % astuce pour supprimer les num�ros de pages

\addtolength{\textheight}{3 cm}
\addtolength{\topmargin}{-2cm}
\addtolength{\textwidth}{3cm}
\addtolength{\oddsidemargin}{-2.1cm}
\addtolength{\evensidemargin}{-2.1cm}
%\theoremstyle{definition}

\theorembodyfont{\rmfamily}
\newtheorem{exo}{Exercise}

\def\ch{\mathrm{ch}}
\def\sh{\mathrm{sh}}
\def\th{\mathrm{th}}

\newcommand{\exoskip}{\addvspace{15pt plus 2pt minus 3pt}}
\newcommand{\R}{\mathbb{R}}
\newcommand{\C}{\mathbb{C}}
\newcommand{\N}{\mathbb{N}}
\newcommand{\Q}{\mathbb{Q}}
\newcommand{\Z}{\mathbb{Z}}
\newcommand{\K}{\mathbb{K}}
\newcommand{\eps}{\varepsilon}
\newcommand{\inter}{\stackrel{\circ}}
\newcommand{\fr}{\mathrm{Fr}}
\newcommand{\ds}{\displaystyle}
\newcommand{\E}{\mathrm{E}}

\begin{document}
{\large
\begin{center}{\bf Math 242 Test 1, Friday 21 September}
\end{center}}
\vspace*{0.5cm}
\large{\textbf{Name:} \hfill \textbf{Last 4 digits of SSN:}}\hspace{2cm}
\vspace{1cm}

\noindent Show all work clearly. No work means no credit. The points are:\\
ex1: 7, ex2: 15, ex3: 9, ex4: 9, ex5: 5 plus 5 point for the writing.
\vspace*{1cm}

\begin{exo}
A spacecraft is in free fall towards the surface of the moon at a speed of 1000 $mi/h$. Its retrorockets, when fired, provide a \emph{\textbf{constant deceleration}} of 20,000 $mi/h^2$.
\begin{enumerate}
\item Find the expression of the motion of the spacecraft when the retrorockets start (ignore the moon's gravitational field). We take $y_0$ for the initial height.
\vspace*{4cm}
\item At what height above the lunar surface should the astronauts fire the retrorockets to insure a soft touchdown ($v=0$ at impact) ?
\vspace*{5cm}
\end{enumerate}
\end{exo}
%1.2.42

\begin{exo}
We considere the following differential equation:
\[ y^2(xy'+y)(1+x^4)^{1/2}=x. \]
\begin{enumerate}
\item Write this differential equation as a Bernouilli's equation.
\vspace*{2cm}
\item What substitution do we have to do?
\vspace*{2cm}
\item What differential equation do we obtain after the substitution?
\vspace*{4cm}
\item Solve this last differential equation and then find the expression of $y$.
\vspace*{6cm}
\end{enumerate}
\end{exo}
%1.6.25

\newpage

\begin{exo}
A pitcher of buttermilk initially at 25 \textsuperscript{o}C is to be cooled by setting it on the front porch, where the temperature is 5 \textsuperscript{o}C. Suppose that the temperature of the buttermilk has dropped to 15 \textsuperscript{o}C after 20 min. 
\begin{enumerate}
\item Using the Newton's law of cooling, determine the temperature of the buttermilk at a time $t$ ($t$ in minutes).
\vspace*{7cm}
\item When will the temperature of buttermilk be 10 \textsuperscript{o}C (you can use that $\ln 4=2\ln 2$)?
\end{enumerate}
\vspace*{3cm}
\end{exo}
%1.4.43


\begin{exo}
We consider the following differential equation:
\[ (1+ye^{xy})dx+(2y+xe^{xy})dy=0 .\]
\begin{enumerate}
\item Show that this equation is exact.
\vspace*{3cm}
\item Then solve this differential equation.
\end{enumerate}
\end{exo}
%1.6.36

\newpage
\vspace*{5cm}

\begin{exo}
Solve the differential equation:
\[ 2y\frac{dy}{dx}=\frac{x}{\sqrt{x^2-16}}, \quad y(5)=2.\]
(you can use a formula of type $\int f'f^n =\hdots$)
\end{exo}
%1.4.21

\end{document}