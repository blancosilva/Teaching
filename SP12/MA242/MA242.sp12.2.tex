\documentclass[12pt]{article}

\usepackage{amsmath,amsthm,amsfonts,amssymb,amsxtra}
\usepackage{pgf,tikz}
\usetikzlibrary{arrows}
\renewcommand{\theenumi}{(\alph{enumi})} 
\renewcommand{\labelenumi}{\theenumi}

\pagestyle{empty}
\setlength{\textwidth}{7in}
\setlength{\oddsidemargin}{-0.5in}
\setlength{\topmargin}{-1.0in}
\setlength{\textheight}{9.5in}

\newtheorem{problem}{Problem}

\begin{document}

\noindent{\large\bf MATH 242}\hfill{\large\bf Second Midterm.}\hfill{\large\bf
  Spring 2012}\hfill{\large\bf Page 1/4}\hrule

\bigskip
\begin{center}
  \begin{tabular}{|ll|}
    \hline & \cr
    {\bf Name: } & \makebox[12cm]{\hrulefill}\cr & \cr
    {\bf 4-digit code:} & \makebox[12cm]{\hrulefill}\cr & \cr
    \hline
  \end{tabular}
\end{center}
\begin{itemize}
\item Write your name and the last 4 digits of your SSN in the space provided above.
\item The test has four (4) pages, including this one.
\item Show sufficient work to justify all answers unless otherwise
  stated in the problem.  Correct answers with inconsistent work may
  not be given credit. 
\item Credit for each problem is given at the right of each problem
  number. 
\item No books, notes or calculators may be used on this test.
\end{itemize}
\hrule

\begin{center}
  \begin{tabular}{|c|c|c|}
    \hline
    &&\cr
    {\large\bf Page} & {\large\bf Max} & {\large\bf Points} \cr
    &&\cr
    \hline
    &&\cr
    {\Large 2} & \Large 30 & \cr
    &&\cr
    \hline
    &&\cr
    {\Large 3} & \Large 40 & \cr
    &&\cr
    \hline
    &&\cr
    {\Large 4} & \Large 30 & \cr
    &&\cr
    \hline\hline
    &&\cr
    {\large\bf Total} & \Large 100 & \cr
    &&\cr
    \hline
  \end{tabular}
\end{center}
\newpage

%%%%%%%%%%%%%%%%%%%%%%%%%%%%%%%%%%%%% Page 2
\noindent{\large\bf MATH 242}\hfill{\large\bf Second Midterm.}\hfill{\large\bf
  Spring 2012}\hfill{\large\bf Page 2/4}\hrule

\bigskip
{\problem[15 pts] \em Find the equilibria of the autonomous equation below, and
use a slope field to determine stability.}
\begin{equation*}
\frac{dy}{dx} = y^2-5y+4.
\end{equation*}
\begin{tikzpicture}[scale=0.8]
\draw[thick,->] (-1,0) -- (6,0);
\draw[thick,->] (0,-1) -- (0,6);
\foreach \x in {-0.5,0,...,5.5}
{
	\draw (-0.125,\x) -- (0.125,\x);
	\draw (\x,-0.125) -- (\x,0.125);
	\foreach \y in {-0.5,0,...,5.5}
		{\filldraw (\x,\y) circle (0.7pt);}
}
\end{tikzpicture}

\vspace{1.5cm}
\hrule
{\problem[15 pts] \em 
Apply Euler method to solve numerically the differential equation $y'=10y$
with initial condition $y(0)=1$ in the interval $[0,0.5]$ with a time-step
$h=0.1$.  Fill the table below with the corresponding values.} 
\begin{center}
\begin{tabular}{|l|r|l|l|l|l|l|}
\hline
$n$ & $n=0$ & $n=1$ & $n=2$ & $n=3$ & $n=4$ & $n=5$ \\
\hline
$x_n$ & $x_0=0$ &&&&& \\
\hline
$y_n$ & $y_0=1$ &&&&& \\
\hline
\end{tabular}
\end{center}
\newpage

%%%%%%%%%%%%%%%%%%%%%%%%%%%%%%%%%%%%% Page 3
\noindent{\large\bf MATH 242}\hfill{\large\bf Second Midterm.}\hfill{\large\bf
  Spring 2012}\hfill{\large\bf Page 3/4}\hrule

\bigskip
{\problem[15 pts] \em The skid marks made by an automobile indicated that its
brakes were fully applied for a distance of $10\,\text{m}$ before it came to a
stop.
The car was known to have at that moment a constant deceleration of $20\,
\text{m}/\text{s}^2$.  How fast (in $\text{km}/\text{h}$) was the car traveling
when the brakes were first applied?}
\vspace{4.5cm}
\begin{flushright}
  \begin{tikzpicture}
    \draw (0cm,-0.2cm) rectangle (5cm,1.2cm);
  \end{tikzpicture}
\end{flushright}
\hrule

{\problem[25 pts] \em Consider a prolific breed of rabbits whose birth
and death rates, $\beta$ and $\delta$, are each proportional to the
rabbit population $P=P(t)$, with $\beta > \delta$.}
\begin{itemize}
\item[{[15 pts]}] Show that $P(t)$ is, for some constant $k$, 
\begin{equation*}
P(t) = \frac{P_0}{1-kP_0 t}.
\end{equation*}
\vspace{6cm}
\item[{[10 pts]}] Suppose that $P_0 = 6$ and that there are nine
  rabbits after ten months.  When does doomsday occur?
\vspace{2cm}
\begin{flushright}
  \begin{tikzpicture}
    \draw (0cm,-0.2cm) rectangle (5cm,1.2cm);
  \end{tikzpicture}
\end{flushright}
\end{itemize}
\newpage

%%%%%%%%%%%%%%%%%%%%%%%%%%%%%%%%%%%%% Page 4
\noindent{\large\bf MATH 242}\hfill{\large\bf Second Midterm.}\hfill{\large\bf
  Spring 2012}\hfill{\large\bf Page 4/4}\hrule

\bigskip
{\problem[30 pts] \em Suppose that a car starts from rest, its engine
  providing an acceleration of 10 ft/$\text{s}^2$, while air
  resistance provides 0.1 ft/$\text{s}^2$ of deceleration for each
  foot per second of the car's velocity.}
\begin{itemize}
\item[{[10 pts]}] Find the car's maximum possible (limiting) velocity.
\vspace{4cm}
\begin{flushright}
  \begin{tikzpicture}
    \draw (0cm,-0.2cm) rectangle (5cm,1.2cm);
  \end{tikzpicture}
\end{flushright}
\item[{[20 pts]}] Find how long it takes the car to attain 90\% of its
  limiting velocity, and how far it travels while doing so.

\small{Hint: Since you are not allowed to use calculators, the answer
  of this problem has integer numbers.  Write everything as fractions and
  simplify as much as possible in order to get simple integrals, and
  you will not have trouble finding those.  You may also want to use
  here the approximation $\ln 0.1 \approx -2.3$ and its corresponding
  inverse, $e^{-2.3} \approx 0.1$.} 
\vspace{8cm}
\begin{flushright}
  \begin{tikzpicture}
    \draw (0cm,-0.2cm) rectangle (5cm,1.2cm);
  \end{tikzpicture}
\end{flushright}
\begin{flushright}
  \begin{tikzpicture}
    \draw (0cm,-0.2cm) rectangle (5cm,1.2cm);
  \end{tikzpicture}
\end{flushright}
\end{itemize}

\end{document}
