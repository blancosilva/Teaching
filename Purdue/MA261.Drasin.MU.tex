\documentclass[12pt]{article}

\usepackage{amsmath,pgf,tikz,pgflibraryarrows}
\renewcommand{\theenumi}{(\Alph{enumi})} 
\renewcommand{\labelenumi}{\theenumi}

\pagestyle{empty}
\setlength{\textwidth}{7in}
\setlength{\oddsidemargin}{-0.5in}
\setlength{\topmargin}{-1.0in}
\setlength{\textheight}{9.5in}

\newtheorem{problem}{Problem}

\begin{document}

\noindent{\bf MA261}\hfill{\bf Make-up Exam\#1.}\hfill{\bf Spring 2007}\hfill{\bf Page 1/6}

\bigskip
\begin{center}
  \begin{tabular}{|ll|}
    \hline & \cr
    {\bf Name: } & \makebox[12cm]{\hrulefill}\cr & \cr
    {\bf 10-Digit PUID:} &
    \makebox[12cm]{\hrulefill}\cr & \cr
    {\bf Recitation Instructor:} & \hrulefill \cr
    & \cr
    {\bf Recitation Time:} & \hrulefill \cr &
    \cr
    \hline
  \end{tabular}
\end{center}
\begin{itemize}
\item The examination has 10 problems, each worth 10 points.
\item Fill in the items at the top, and write your name in each page.
\item Do any necessary work at the space provided or on the back of
  the pages of this booklet.  Circle the correct answer.
\item No notes, books or calculators may be used.
\item Your recitation instructor will return your graded examination.
\end{itemize}
\hrule

\begin{center}
  \begin{tabular}{|c|c|c|}
    \hline &&\cr
    {\large\bf
      Page} &
    {\large\bf
      Max.~points}
    & {\large\bf
      Your points}
    \cr
    &&\cr
    \hline
    &&\cr
    {\Large
      2}
    &
    \Large
    30
    &
    \cr
    &&\cr
    \hline
    &&\cr
    {\Large
      3}
    &
    \Large
    40
    &
    \cr
    &&\cr
    \hline
    &&\cr
    {\Large
      4}
    &
    \Large
    30
    &
    \cr
    &&\cr
    \hline\hline
    &&\cr
    {\large\bf
      Total}
    &
    \Large
    100
    &
    \cr
    &&\cr
    \hline
  \end{tabular}
\end{center}
\newpage

%%%%%%%%%%%%%%%%%%%%%%%%%%%%%%%%%%%%% Page 1
\noindent{\bf MA261}\hfill{\bf Make-up Exam\#1.}\hfill Name: \makebox[4cm]{\hrulefill} \hfill{\bf Spring 2007}\hfill{\bf Page 2/6}
\bigskip
{\problem[10 pts] \em A vector equation of the line containing the
  point $(2,1,-3)$ and per\-pen\-di\-cular to the plane $3x-4y+2z=6$ is}
\begin{enumerate}
\item $(2+3t,1-4t,-3+2t)$
\item $(3+2t, 4-t,2-3t)$
\item $(2-3t, 1+4t, -3-2t)$
\item $(3-2t, 4+t,-2-3t)$
\end{enumerate}

{\problem[10 pts] \em Let ${\boldsymbol u}$ be a unit vector.  Find
  ${\boldsymbol u} \cdot {\boldsymbol v}$}
\begin{center}
  \begin{tikzpicture}
    \draw[white] (0cm,0cm) rectangle (\linewidth,4cm);
    \draw (0.25\linewidth, 2cm) node[text width=0.5\linewidth]{
      \begin{enumerate}
      \item $-1/\sqrt{2}$
      \item $-1$
      \item $1/\sqrt{2}$
      \item $1$
      \item $\sqrt{2}$
      \end{enumerate}
    };
    \begin{scope}[>=angle 45, xshift=0.75\linewidth, yshift=1cm]
      \draw[gray, thin] (0cm,0cm) rectangle (3cm, 3cm);
      \draw[gray, thin] (0cm,0cm) -- (3cm,3cm);
      \draw[gray, thin] (0cm,3cm) -- (3cm,0cm);
      \draw[very thick, ->,blue] (0cm,0cm) -- (3cm,0cm) node[midway,above]{${\boldsymbol u}$};
      \draw[very thick, ->,red] (1.5cm,1.5cm) -- (3cm,0cm) node[midway,above]{${\boldsymbol v}$};
    \end{scope}
  \end{tikzpicture}
\end{center}

{\problem[10 pts] \em The graph of the equation $16x^2 + z^2 = y^2-1$
  looks most like}
\begin{center}
  \begin{tikzpicture}
    \draw (0cm,0cm) rectangle (\linewidth, 12.5cm);
    \foreach \x in {0cm,2.5cm,5cm,7.5cm,10cm,12.5cm}
    \draw[very thin, gray] (0cm,\x) -- (\linewidth,\x);
    \foreach \y/\ytext in {1.25cm/(E), 3.75cm/(D), 6.25cm/(C),
      8.75cm/(B), 11.25cm/(A)}
    \draw (1cm,\y) node {\ytext};
  \end{tikzpicture}
\end{center}
\newpage

%%%%%%%%%%%%%%%%%%%%%%%%%%%%%%%%%%%%% Page 1 (2nd version)
\noindent{\bf MA261}\hfill{\bf Make-up Exam\#1.}\hfill Name: \makebox[4cm]{\hrulefill} \hfill{\bf Spring 2007}\hfill{\bf Page 2/6}
\bigskip
{\problem[10 pts] \em A vector equation of the line containing the
  point $(2,1,-3)$ and per\-pen\-di\-cular to the plane $3x-4y+2z=6$ is}
\begin{enumerate}
\item $(2+3t,1-4t,-3+2t)$
\item $(3+2t, 4-t,2-3t)$
\item $(2-3t, 1+4t, -3-2t)$
\item $(3-2t, 4+t,-2-3t)$
\end{enumerate}

{\problem[10 pts] \em Let ${\boldsymbol u}$ be a unit vector.  Find
  ${\boldsymbol u} \cdot {\boldsymbol v}$}
\begin{center}
  \begin{tikzpicture}
    \draw[white] (0cm,0cm) rectangle (\linewidth,4cm);
    \draw (0.25\linewidth, 2cm) node[text width=0.5\linewidth]{
      \begin{enumerate}
      \item $-1/\sqrt{2}$
      \item $-1$
      \item $1/\sqrt{2}$
      \item $1$
      \item $\sqrt{2}$
      \end{enumerate}
    };
    \begin{scope}[>=angle 45, xshift=0.75\linewidth, yshift=1cm]
      \draw[gray, thin] (0cm,0cm) rectangle (3cm, 3cm);
      \draw[gray, thin] (0cm,0cm) -- (3cm,3cm);
      \draw[gray, thin] (0cm,3cm) -- (3cm,0cm);
      \draw[very thick, ->,blue] (0cm,0cm) -- (3cm,0cm)
      node[midway,above]{${\boldsymbol u}$};
      \draw[very thick, ->,red] (1.5cm,1.5cm) -- (3cm,0cm)
      node[midway,above]{${\boldsymbol v}$};
    \end{scope}
  \end{tikzpicture}
\end{center}

{\problem[10 pts] \em The graph of the equation $16x^2 + z^2 = y^2-1$
  looks most like}
\begin{center}
  \begin{tikzpicture}
    \draw (0cm,0cm) rectangle (\linewidth, 12.5cm);
    \draw (0.1\linewidth,0.5cm) node{(A)};
    \draw (0.3\linewidth,0.5cm) node{(B)};
    \draw (0.5\linewidth,0.5cm) node{(C)};
    \draw (0.7\linewidth,0.5cm) node{(D)};
    \draw (0.9\linewidth,0.5cm) node{(E)};
  \end{tikzpicture}
\end{center}
\newpage

%%%%%%%%%%%%%%%%%%%%%%%%%%%%%%%%%%%%% Page 2
\noindent{\bf MA261}\hfill{\bf Make-up Exam\#1.}\hfill Name: \makebox[4cm]{\hrulefill}\hfill{\bf Spring 2007}\hfill{\bf Page 3/6}

\bigskip
{\problem[10 pts] \em The solid described by the equation $\rho \leq
  2$, $\pi/2 \leq \phi \leq \pi$, $0\leq \theta \leq \pi/2$ is}
\begin{enumerate}
\item The portion of the sphere $x^2+y^2+z^2 = 4$ to the right of the vertical plane $\{ y=0\}$.
\item The portion of the ball $x^2+y^2+z^2 \leq 4$ in front of the vertical plane $\{ x=0\}$.
\item The portion of the sphere $x^2+y^2+z^2 = 4$ below the first
  quadrant of the $x$--$y$ plane.
\item The portion of the sphere $x^2+y^2+z^2 = 4$ above the $x$--$y$
  plane.
\item The portion of the ball $x^2+y^2+z^2 \leq 4$ below the first
  quadrant of the $x$--$y$ plane.
\end{enumerate}

{\problem[10 pts] \em The length of the curve ${\boldsymbol r}(t) =
  (2t, 3\sin 2t, 3 \cos 2t)$ is}
\begin{enumerate}
\item $2\pi$
\item $\sqrt{2}\pi$
\item $2\pi\sqrt{10}$
\item $2\pi\sqrt{3}$
\item $\sqrt{13}\pi$
\end{enumerate}

{\problem[10 pts] \em A particle moves when $t \geq 0$ on a curve in
  the $x$--$y$ plane with acceleration ${\boldsymbol a}(t) = (t,2)$ and
  at $t=0$ having velocity $(-1,2)$ and position $(-4/3, -8)$.  When
  the $y$--coordinate is $0$, the $x$--coordinate is}
\begin{enumerate}
\item $-2$  
\item $0$  
\item $8/3$  
\item $-1/6$  
\item $-1$  
\end{enumerate}


{\problem[10 pts] \em The equation of the tangent plane to the surface
  $z=e^{x^2-y^2}$ when $x=-1$, $y=1$ is}
\begin{enumerate}
\item $z=1+2x+2y$  
\item $z=1-2x-2y$  
\item $z=-2x-2y$  
\item $z=2x+2y$  
\item $z=-1-2x+2y$  
\end{enumerate}
\newpage


%%%%%%%%%%%%%%%%%%%%%%%%%%%%%%%%%%%%% Page 3
\noindent{\bf MA261}\hfill{\bf Make-up Exam\#1.}\hfill Name: \makebox[4cm]{\hrulefill}\hfill{\bf Spring 2007}\hfill{\bf Page 4/6}

\bigskip

{\problem[10 pts] \em Let $h(x,y) = e^{x/y^3}$, then
  $\displaystyle{\frac{\partial^2 h}{\partial x \partial y} = }$}
\begin{enumerate}
\item $e^{x/y^3} \big( -3x/y^4 \big)$
\item $e^{x/y^3} \big( -3x/y^7 + 3/y^4 \big)$
\item $e^{x/y^3} y^{-3}$
\item $e^{x/y^3} \big( -3/y^4 \big)$
\item $-y^{-4} e^{x/y^3} \big( 3x/y^3 + 3 \big)$
\end{enumerate}
\vspace{2cm}

{\problem[10 pts] \em The level curves $f(x,y)=c$ of the function
  $f(x,y) = \displaystyle{ \frac{x}{x-y^2} }$ are}
\begin{enumerate}
\item parabolas.
\item ellipses with main axis on the $y$--axis.
\item circles with centers in the $y$--axis.
\item circles with centers in the $x$--axis.
\item hyperbolas.
\end{enumerate}
\vspace{2cm}

{\problem[10 pts] \em The line $x=1+2t$, $y=4t$, $z=2-t$ intersects
  the plane $4x-y+2z=2$ at the point}
\begin{enumerate}
\item $(5,-12,5)$
\item $(7,-12,-1)$
\item $(4,3,5)$
\item $(3,4,1)$
\item $(-1,-4,3)$
\end{enumerate}

\end{document}
