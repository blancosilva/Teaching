\documentclass[12pt]{article}

\usepackage{amssymb,amsmath}

\renewcommand{\theenumi}{\bf \Alph{enumi}.}
\renewcommand{\labelenumi}{\theenumi}

\pagestyle{empty}
\setlength{\textwidth}{7in}
\setlength{\oddsidemargin}{-0.5in}
\setlength{\topmargin}{-1.0in}
\setlength{\textheight}{9.5in}

\newtheorem{problem}{Problem}

\begin{document}

%%%%%%%%%%%%%%%%%%%%%%%%%%%%%%%%%%%%% Page 1
\noindent{\large\bf MA224}\hfill{\large\bf Exam\#2}\hfill{\large\bf
  Spring 2007}\hrule 
\bigskip

{\problem $\int \big( x - \frac{4}{\sqrt{x}} + 2x^{3/4}
    \big)\, dx = $} 
\begin{enumerate}
\item $\frac{1}{2}x^2 - 8\sqrt{x} + \frac{7}{2} x^{7/4} + C$.
\item $\frac{1}{2}x^2 - \frac{8}{3}x^{3/2} + \frac{8}{7} x^{7/4} + C$.
\item $\frac{1}{2}x^2 - 8\sqrt{x} + \frac{8}{7} x^{7/4} + C$.
\item $\frac{1}{2}x^2 - \frac{8}{3}x^{3/2} + \frac{7}{2} x^{7/4} + C$.
\item $\frac{1}{2}x^2 - 2\sqrt{x} + \frac{8}{7} x^{7/4} + C$.
\end{enumerate}
\vspace{5cm}

{\problem $\displaystyle{\int (3x-2)\root{3}\of{6x^2-8x}\, dx} = $}
\begin{enumerate}
\item $\displaystyle{\frac{3}{4} \Big( \frac{3x^2}{2}-x\Big)\, \big(
    6x^2-8x \big)^{4/3} + C}$. 
\item $\displaystyle{\frac{16}{3} \big( 6x^2-8x \big)^{4/3} + C}$. 
\item $\displaystyle{3 \, \big( 6x^2-8x \big)^{4/3} + C}$. 
\item $\displaystyle{\frac{3}{16} \big( 6x^2-8x \big)^{4/3} + C}$. 
\item $\displaystyle{\frac{4}{3} \Big( \frac{3x^2}{2}-x\Big)\, \big(
    6x^2-8x \big)^{4/3} + C}$. 
\end{enumerate}
\newpage

%%%%%%%%%%%%%%%%%%%%%%%%%%%%%%%%%%%%% Page 2
\noindent{\large\bf MA224}\hfill{\large\bf Exam\#2}\hfill{\large\bf
  Spring 2007}\hrule
\bigskip

{\problem $\displaystyle{\int_1^2 \frac{e^{-x/2}}{9}\, dx} =$}
\begin{enumerate}
\item $\displaystyle{-\frac{2}{9} \Big( \frac{1-e}{e^2} \Big)}$.
\item $\displaystyle{-\frac{1}{18} \Big( \frac{1-e^{1/2}}{e} \Big)}$.
\item $\displaystyle{\frac{1}{9} \Big( \frac{1-e}{e^2} \Big)}$.
\item $\displaystyle{-\frac{1}{18} \Big( \frac{1-e}{e^2} \Big)}$.
\item $\displaystyle{-\frac{2}{9} \Big( \frac{1-e^{1/2}}{e} \Big)}$.
\end{enumerate}
\vspace{4cm}

{\problem \em Find the integral below, rounded to 4 decmal places.
  \begin{equation*}
    \int_1^4 \frac{2}{5(3x+1)}\, dx.
  \end{equation*}}
\begin{enumerate}
\item $0.1572$.
\item $0.5545$.
\item $0.1848$.
\item $0.4715$.
\item $0.1414$.
\end{enumerate}
\newpage

%%%%%%%%%%%%%%%%%%%%%%%%%%%%%%%%%%%%% Page 3
\noindent{\large\bf MA224}\hfill{\large\bf Exam\#2}\hfill{\large\bf
  Spring 2007}\hrule
\bigskip

{\problem $\displaystyle{\int (x+1) e^{3x}\, dx} = $}
\begin{enumerate}
\item $\displaystyle{\frac{x+1}{3} e^{3x} - \frac{1}{9} e^{3x} + C}$.
\item $\displaystyle{3 (x+1) e^{3x} - 3 e^{3x} + C}$.
\item $\displaystyle{(x+1) e^{3x} - e^{3x} + C}$.
\item $\displaystyle{\frac{x+1}{3} e^{3x} - \frac{1}{3} e^{3x} + C}$.
\item $\displaystyle{3 (x+1) e^{3x} - 9 e^{3x} + C}$.
\end{enumerate}
\vspace{4cm}

{\problem \em Find $f$ if $f'(x) = \displaystyle{\frac{x^2+1}{x^4}}$
  and $f(1)=2$.}
\begin{enumerate}
\item $\displaystyle{f(x) = \frac{3}{x^3} + \frac{5}{x^5} -6}$.
\item $\displaystyle{f(x) = -\frac{1}{x} - \frac{1}{3x^3} +
    \frac{10}{3}}$.
\item $\displaystyle{f(x) = \frac{5}{3x^2} + \frac{5}{x^4} -
    \frac{14}{3}}$.
\item $\displaystyle{f(x) = -\frac{1}{x} - \frac{1}{3x^3} +2}$.
\item $\displaystyle{f(x) = \frac{3}{x^3} + \frac{5}{x^5} +2}$.
\end{enumerate}
\newpage

%%%%%%%%%%%%%%%%%%%%%%%%%%%%%%%%%%%%% Page 4
\noindent{\large\bf MA224}\hfill{\large\bf Exam\#2}\hfill{\large\bf
  Spring 2007}\hrule 
\bigskip

{\problem \em If $\int (4x^2-x+5)^2 f(x)\, dx = (4x^2-x+5)^3 + C$,
  then $f(x) =$}
\begin{enumerate}
\item $\displaystyle{\frac{8x-1}{3}}$.
\item $\displaystyle{\frac{1}{3}}$.
\item $8x-1$.
\item $3$.
\item $24x-3$.
\end{enumerate}
\vspace{5cm}
{\problem \em Find the average value of the function $f$ over the
  interval $[-1,2]$.
\begin{equation*}
  f(x) = x^2 \Big( 3x - \frac{7}{x} \Big)
\end{equation*}}
\begin{enumerate}
\item $\displaystyle{-\frac{19}{4}}$.
\item $\displaystyle{\frac{11}{6}}$.
\item $\displaystyle{\frac{1}{4}}$.
\item $\displaystyle{-\frac{19}{12}}$.
\item $\displaystyle{\frac{3}{4}}$.
\end{enumerate}
\newpage

%%%%%%%%%%%%%%%%%%%%%%%%%%%%%%%%%%%%% Page 5
\noindent{\large\bf MA224}\hfill{\large\bf Exam\#2}\hfill{\large\bf
  Spring 2007}\hrule
\bigskip

{\problem \em Use geometry to find the area under the graph of $f$ on
  the interval $(0,5)$.
  \begin{equation*}
    f(x) = \left\{ \begin{array}{cl}
        -3x+7 &\text{ if } x\leq 2, \\
        x/2 &\text{ if } x>2.
      \end{array} \right.
  \end{equation*}}
\begin{enumerate}
\item $10.75$ sq.~units.
\item $14.75$ sq.~units.
\item $8.25$ sq.~units.
\item $13.25$ sq.~units.
\item $15.75$ sq.~units.
\end{enumerate}
\vspace{5cm}

{\problem \em Find the area bounded by the curves $y=2x^2-7$ and
  $y=5-x^2$.}
\begin{enumerate}
\item $8$ sq.~units.
\item $32$ sq.~units.
\item $16$ sq.~units.
\item $48$ sq.~units.
\item $12$ sq.~units.
\end{enumerate}
\newpage

%%%%%%%%%%%%%%%%%%%%%%%%%%%%%%%%%%%%% Page 6
\noindent{\large\bf MA224}\hfill{\large\bf Exam\#2}\hfill{\large\bf
  Spring 2007}\hrule
\bigskip

{\problem \em Find the function $f$ given that the slope of the
  tangent to the graph of $f$ at any point is $x\ln x$ and the graph
  passes through the point $(1,1)$.}
\begin{enumerate}
\item $\displaystyle{f(x) = \frac{x^2\ln x}{2} - \frac{x^2}{4} +
    \frac{5}{4}}$.
\item $\displaystyle{f(x) = \frac{x\ln x^2}{2} - \frac{\ln x^3}{3} +
    1}$.
\item $\displaystyle{f(x) = \frac{x^2\ln x^2}{4} + 1}$.
\item $\displaystyle{f(x) = \frac{x\ln x^2}{2} - \frac{\ln x^3}{3} +
    \frac{1}{6}}$.
\item $\displaystyle{f(x) = \frac{x^2\ln x}{2} - \frac{x^2}{4} + 1}$.
\end{enumerate}
\vspace{4cm}

{\problem \em Let $f(x)$ represent the number of items a person has
  memorized $x$ minutes after being presented a lengthly list of
  items to learn.  If the learning rate is $0.2 (10 + 12x - 0.6x^2)$,
  how many items are memorized during the first 5 minutes?}
\begin{enumerate}
\item 11.
\item 90.
\item 57.
\item 175.
\item 35.
\end{enumerate}
\newpage

%%%%%%%%%%%%%%%%%%%%%%%%%%%%%%%%%%%%% Page 7
\noindent{\large\bf MA224}\hfill{\large\bf Exam\#2}\hfill{\large\bf
  Spring 2007}\hrule
\bigskip

{\problem \em The demand function for an item is $p = -0.3x^2+70$,
  where $p$ is the price in dollars and $x$ is the quantity demanded
  each week.  The supply function for the same item is $p =0.1 x^2
  +x+20$, where $x$ is the quantity available each week.  Which
  expression represents the producers' surplus if the market price is
  set at the equilibrium price?}
\begin{enumerate}
\item $\displaystyle{400 - \int_0^{40} (0.1x^2+x+20)\, dx}$.
\item $\displaystyle{400 - \int_0^{40} (-0.3x^2+70)\, dx}$.
\item $\displaystyle{400 - \int_0^{10} (0.1x^2+x+20)\, dx}$.
\item $\displaystyle{400 - \int_0^{10} (-0.3x^2+70)\, dx}$.
\item $\displaystyle{400 - \int_0^{40} (0.1x^2+x+20) - (-0.3x^2 + 70)\, dx}$.
\end{enumerate}
\end{document}
