\documentclass[12pt]{article}

\usepackage{amsmath,amsthm,amsfonts,amssymb,amsxtra}
\usepackage{pgf,tikz}
\usetikzlibrary{arrows}
\renewcommand{\theenumi}{(\alph{enumi})} 
\renewcommand{\labelenumi}{\theenumi}

\pagestyle{empty}
\setlength{\textwidth}{7in}
\setlength{\oddsidemargin}{-0.5in}
\setlength{\topmargin}{-1.0in}
\setlength{\textheight}{9.5in}

\newtheorem{problem}{Problem}

\begin{document}

\noindent{\large\bf MATH 142}\hfill{\large\bf Exam\#3.}\hfill{\large\bf
  Spring 2008}\hfill{\large\bf Page 1/6}\hrule

\bigskip
\begin{center}
  \begin{tabular}{|ll|}
    \hline & \cr
    {\bf Name: } & \makebox[12cm]{\hrulefill}\cr & \cr
    {\bf 4-digit code:} & \makebox[12cm]{\hrulefill}\cr & \cr
    \hline
  \end{tabular}
\end{center}
\begin{itemize}
\item Write your name and the last 4 digits of your SSN in the space provided above.
\item The test has six (6) pages, including this one.
\item Enter your answer in the box(es) provided.
\item You must show sufficient work to justify all answers unless
  otherwise stated in the problem.  Correct answers with inconsistent
  work may not be given credit.
\item Credit for each problem is given in parentheses at the right of
  the problem number.
\item No books, notes or calculators may be used on this test.
\end{itemize}
\hrule

\begin{center}
  \begin{tabular}{|c|c|c|}
    \hline
    &&\cr
    {\large\bf Page} & {\large\bf Max.~points} & {\large\bf Your points} \cr
    &&\cr
    \hline
    &&\cr
    {\Large 2} & \Large 20 & \cr
    &&\cr
    \hline
    &&\cr
    {\Large 3} & \Large 15 & \cr
    &&\cr
    \hline
    &&\cr
    {\Large 4} & \Large 20 & \cr
    &&\cr
    \hline
    &&\cr
    {\Large 5} & \Large 25 & \cr
    &&\cr
	\hline
    &&\cr
    {\Large 6} & \Large 20 & \cr
    &&\cr
    \hline\hline
    &&\cr
    {\large\bf Total} & \Large 100 & \cr
    &&\cr
    \hline
  \end{tabular}
\end{center}
\newpage

%%%%%%%%%%%%%%%%%%%%%%%%%%%%%%%%%%%%% Page 2
\noindent{\large\bf MATH 142}\hfill{\large\bf Exam\#3.}\hfill{\large\bf
  Spring 2008}\hfill{\large\bf Page 2/6}\hrule

\bigskip
{\problem[10 pts] \em  Find a formula for the general term of the following sequences:} 
\begin{enumerate}
\item $\displaystyle{\frac{1}{2}, \frac{3}{4}, \frac{5}{6}, \frac{7}{8}, \dotsc}$
\bigskip
\begin{flushright}
  \begin{tikzpicture}
    \draw (-0.5cm,0.5cm) node {$x_n = $};
    \draw (0cm,-0.2cm) rectangle (5cm,1.2cm);
  \end{tikzpicture}
\end{flushright}
\item $\displaystyle{1-\frac{1}{2}, \frac{1}{3} - \frac{1}{2}, \frac{1}{3} - \frac{1}{4}, \frac{1}{5}-\frac{1}{4}, \dotsc}$
\vspace{3cm}
\begin{flushright}
  \begin{tikzpicture}
    \draw (-0.5cm,0.5cm) node {$x_n = $};
    \draw (0cm,-0.2cm) rectangle (5cm,1.2cm);
  \end{tikzpicture}
\end{flushright}
\end{enumerate}
\hrule
{\problem[10pts] \em Write out the first five terms of the sequence $\left\{ \displaystyle{\frac{\ln n}{n}} \right\}_{n=1}^\infty$ \newline Determine whether the sequence converges, and if so find its limit.
\vspace{7cm}
\begin{flushright}
  \begin{tikzpicture}
    \draw (-3.75cm,1.9cm) node {First five terms:};
    \draw (-2cm, 1.3cm) rectangle (5cm, 2.7cm);
    \draw (-1.25cm,0.5cm) node {$\displaystyle{\lim_{n \to \infty} x_n} = $};
    \draw (0cm,-0.2cm) rectangle (5cm,1.2cm);
  \end{tikzpicture}
\end{flushright}
\newpage


%%%%%%%%%%%%%%%%%%%%%%%%%%%%%%%%%%%%% Page 3
\noindent{\large\bf MATH 142}\hfill{\large\bf Exam\#3.}\hfill{\large\bf
  Spring 2008}\hfill{\large\bf Page 3/6}\hrule

\bigskip
{\problem[5 pts] \em Use $x_{n+1}-x_n$ to show that the sequence $\big\{ n-n^2 \big\}_{n=1}^\infty$ is strictly increasing or strictly decreasing.}
\vspace{6.5cm}
\hrule
{\problem[5 pts] \em Use $x_{n+1}/x_n$ to show that the sequence $\left\{ \displaystyle{ne^{-n}} \right\}_{n=1}^\infty$ is strictly increasing or strictly decreasing.}
\vspace{6.5cm}
\hrule
{\problem[5 pts] \em Use \textbf{differentiation} to show that the sequence $\left\{ \displaystyle{3 - \frac{1}{n}} \right\}_{n=1}^\infty$ is strictly increasing or strictly decreasing.}
\newpage

%%%%%%%%%%%%%%%%%%%%%%%%%%%%%%%%%%%%% Page 4
\noindent{\large\bf MATH 142}\hfill{\large\bf Exam\#3.}\hfill{\large\bf
  Spring 2008}\hfill{\large\bf Page 4/6}\hrule

\bigskip
{\problem[20 pts] \em Determine whether the series converge, and if so find their sum:}
\begin{enumerate}
\item $\displaystyle{\sum_{k=1}^\infty \Big( -\frac{3}{2} \Big)^{k+1}}$
\vspace{7cm}
\begin{flushright}
  \begin{tikzpicture}
    \draw (-1.75cm,0.5cm) node {$\displaystyle{\sum_{k=1}^\infty \Big( -\frac{3}{2} \Big)^{k+1} }= $};
    \draw (0cm,-0.2cm) rectangle (5cm,1.2cm);
  \end{tikzpicture}
\end{flushright}
\item $\displaystyle{\sum_{k=1}^\infty \left( \frac{1}{2^k} -\frac{1}{2^{k+1}} \right)}$
\vspace{7cm}
\begin{flushright}
  \begin{tikzpicture}
    \draw (-2cm,0.5cm) node {$\displaystyle{\sum_{k=1}^\infty \left( \frac{1}{2^k} - \frac{1}{2^{k+1}} \right) }= $};
    \draw (0cm,-0.2cm) rectangle (5cm,1.2cm);
  \end{tikzpicture}
\end{flushright}
\end{enumerate}
\newpage

%%%%%%%%%%%%%%%%%%%%%%%%%%%%%%%%%%%%% Page 5
\noindent{\large\bf MATH 142}\hfill{\large\bf Exam\#3.}\hfill{\large\bf
  Spring 2008}\hfill{\large\bf Page 5/6}\hrule

\bigskip
{\problem[5 pts] \em Apply the \textbf{divergence test} and state what it tells you about the series.}
\begin{equation*}
\sum_{k=1}^\infty \Big( 1 + \frac{1}{k} \Big)^k.
\end{equation*}
\vspace{2cm}
\hrule
{\problem[10 pts] \em Use the \textbf{integral test} to determine whether the series $\displaystyle{\sum_{k=1}^\infty \frac{1}{1+9k^2}}$ converges.}
\vspace{6.5cm}
\hrule
{\problem[10 pts] \em Use the \textbf{ratio test} to determine whether the series $\displaystyle{\sum_{k=1}^\infty \frac{3^k}{k!}}$ converges.  If the test is inconclusive, then say so.}
\newpage

%%%%%%%%%%%%%%%%%%%%%%%%%%%%%%%%%%%%% Page 6
\noindent{\large\bf MATH 142}\hfill{\large\bf Exam\#3.}\hfill{\large\bf
  Spring 2008}\hfill{\large\bf Page 6/6}\hrule
  
\bigskip

{\problem[10 pts] \em Use the \textbf{root test} to determine whether the series $\displaystyle{\sum_{k=1}^\infty \Big( \frac{k}{100} \Big)^k}$ converges.  If the test is inconclusive, then say so.}
\vspace{9cm}
\hrule{\problem[10 pts] \em Classify the series $\displaystyle{\sum_{k=1}^\infty \frac{k \cos k\pi}{k^2+1}}$ as absolutely convergent, convergent or divergent.}

\end{document}