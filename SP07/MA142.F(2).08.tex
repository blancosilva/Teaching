\documentclass[12pt]{article}

\usepackage{amsmath,amsthm,amsfonts,amssymb,amsxtra}
\usepackage{pgf,tikz}
\usetikzlibrary{arrows}
\renewcommand{\theenumi}{(\alph{enumi})} 
\renewcommand{\labelenumi}{\theenumi}

\pagestyle{empty}
\setlength{\textwidth}{7in}
\setlength{\oddsidemargin}{-0.5in}
\setlength{\topmargin}{-1.0in}
\setlength{\textheight}{9.5in}

\newtheorem{problem}{Problem}

\begin{document}

\noindent{\large\bf MATH 141}\hfill{\large\bf Final Exam.}\hfill{\large\bf
  Spring 2008}\hfill{\large\bf Page 1/17}\hrule

\bigskip
\begin{center}
  \begin{tabular}{|ll|}
    \hline & \cr
    {\bf Name: } & \makebox[12cm]{\hrulefill}\cr & \cr
    {\bf 4-digit code:} & \makebox[12cm]{\hrulefill}\cr & \cr
    \hline
  \end{tabular}
\end{center}
\begin{itemize}
\item Write your name and the last 4 digits of your SSN in the space provided above.
\item The test has eleven (17) pages, including this one, and your
  help sheet.
\item For multi-choice questions, you should circle the answer you
  select.  On the other problems, you should enter your answer in the
  box(es) provided.
\item You must show sufficient work to justify all answers unless
  otherwise stated in the problem.  Correct answers with inconsistent
  work may not be given credit.
\item Credit for each problem is given in parentheses at the right of
  the problem number.
\item No books, notes or calculators may be used on this test.
\item \textbf{A:} 243--270 pts. \textbf{B+:} 230--242 pts. \textbf{B:} 216--229 pts. \textbf{C+:} 203--215 pts. \textbf{C:} 189--202 pts.\\ \textbf{D+:} 175--188 pts. \textbf{D:} 160--174 pts. \textbf{F}: less than 160 pts.
\end{itemize}
\hrule

\begin{center}
  \begin{tabular}{|c|c|c|}
    \hline
    &&\cr
    {\large\bf Page} & {\large\bf Max} & {\large\bf Points} \cr
    &&\cr
    \hline
    &&\cr
    {\Large 2} & \Large 20 & \cr
    &&\cr
    \hline
    &&\cr
    {\Large 3} & \Large 20 & \cr
    &&\cr
    \hline
    &&\cr
    {\Large 4} & \Large 20 & \cr
    &&\cr
    \hline
    &&\cr
    {\Large 5} & \Large 25 & \cr
    &&\cr
    \hline
    &&\cr
    {\Large 6} & \Large 15 & \cr
    &&\cr
    \hline\hline
    &&\cr
    {\large\bf Total} & \Large 100 & \cr
    &&\cr
    \hline
  \end{tabular}
  \begin{tabular}{|c|c|c|}
    \hline
    &&\cr
    {\large\bf Page} & {\large\bf Max} & {\large\bf Points} \cr
    &&\cr
    \hline
    &&\cr
    {\Large 7} & \Large 40 & \cr
    &&\cr
    \hline
    &&\cr
    {\Large 8} & \Large 20 & \cr
    &&\cr
    \hline
    &&\cr
    {\Large 9} & \Large 10 & \cr
    &&\cr
    \hline
    &&\cr
    {\Large 10} & \Large 10 & \cr
    &&\cr
    \hline
    &&\cr
    {\Large 11} & \Large 20 & \cr
    &&\cr
    \hline\hline
    &&\cr
    {\large\bf Total} & \Large 100 & \cr
    &&\cr
    \hline
  \end{tabular}
  \begin{tabular}{|c|c|c|}
    \hline
    &&\cr
    {\large\bf Page} & {\large\bf Max} & {\large\bf Points} \cr
    &&\cr
    \hline
    &&\cr
    {\Large 7} & \Large 20 & \cr
    &&\cr
    \hline
    &&\cr
    {\Large 8} & \Large 15 & \cr
    &&\cr
    \hline
    &&\cr
    {\Large 9} & \Large 20 & \cr
    &&\cr
    \hline
    &&\cr
    {\Large 10} & \Large 25 & \cr
    &&\cr
    \hline
    &&\cr
    {\Large 11} & \Large 20 & \cr
    &&\cr
    \hline\hline
    &&\cr
    {\large\bf Total} & \Large 100 & \cr
    &&\cr
    \hline
  \end{tabular}
\end{center}
\newpage
%%%%%%%%%%%%%%%%%%%%%%%%%%%%%%%%%%%%% Page 2
\noindent{\large\bf MATH 142}\hfill{\large\bf Final Exam.}\hfill{\large\bf
  Spring 2008}\hfill{\large\bf Page 2/17}\hrule

\bigskip
{\problem[10 pts] \em  Find the area of the region that is enclosed between the curves $y=x^2$ and $y=x+6$.} 
\vspace{8.5cm}
\begin{flushright}
  \begin{tikzpicture}
    \draw (-1cm,0.5cm) node {$A = $};
    \draw (0cm,-0.2cm) rectangle (5cm,1.2cm);
  \end{tikzpicture}
\end{flushright}
\hrule
{\problem[10 pts] \em Find the volume of the solid that is obtained when the region under the curve $y=\sqrt{x}$ over the interval $[1,4]$ is revolved about the $x$-axis.}
\vspace{8.5cm}
\begin{flushright}
  \begin{tikzpicture}
    \draw (-1cm,0.5cm) node {$V = $};
    \draw (0cm,-0.2cm) rectangle (5cm,1.2cm);
  \end{tikzpicture}
\end{flushright}
\newpage

%%%%%%%%%%%%%%%%%%%%%%%%%%%%%%%%%%%%% Page 3
\noindent{\large\bf MATH 142}\hfill{\large\bf Final Exam.}\hfill{\large\bf
  Spring 2008}\hfill{\large\bf Page 3/17}\hrule

\bigskip
{\problem[10 pts] \em Find the volume of the solid generated when the region enclosed by $y=\sqrt{x}$, $y=2$ and $x=0$ is revolved about the $y$-axis.}
\vspace{8.5cm}
\begin{flushright}
  \begin{tikzpicture}
    \draw (-1cm,0.5cm) node {$V = $};
    \draw (0cm,-0.2cm) rectangle (5cm,1.2cm);
  \end{tikzpicture}
\end{flushright}
\hrule
{\problem[10 pts] \em Find the arclength of the curve $y=x^{3/2}$ from $x=1$ to $x=2$.}
\vspace{8.5cm}
\begin{flushright}
  \begin{tikzpicture}
    \draw (-1cm,0.5cm) node {$L = $};
    \draw (0cm,-0.2cm) rectangle (5cm,1.2cm);
  \end{tikzpicture}
\end{flushright}
\newpage

%%%%%%%%%%%%%%%%%%%%%%%%%%%%%%%%%%%%% Page 4
\noindent{\large\bf MATH 142}\hfill{\large\bf Final Exam.}\hfill{\large\bf
  Spring 2008}\hfill{\large\bf Page 4/17}\hrule

\bigskip
{\problem[10 pts] \em Find the area of the surface that is generated by revolving the portion of the curve $y=x^3$ between $x=0$ and $x=1$ about the $x$-axis.}
\vspace{8.5cm}
\begin{flushright}
  \begin{tikzpicture}
    \draw (-1cm,0.5cm) node {$A = $};
    \draw (0cm,-0.2cm) rectangle (5cm,1.2cm);
  \end{tikzpicture}
\end{flushright}
\hrule
{\problem[10 pts] \em Find the average value of the function $f(x) = 1/x$ over the interval $[1,e]$.}
\vspace{8.5cm}
\begin{flushright}
  \begin{tikzpicture}
    \draw (-1cm,0.5cm) node {$f_{ave} = $};
    \draw (0cm,-0.2cm) rectangle (5cm,1.2cm);
  \end{tikzpicture}
\end{flushright}
\newpage

%%%%%%%%%%%%%%%%%%%%%%%%%%%%%%%%%%%%% Page 5
\noindent{\large\bf MATH 142}\hfill{\large\bf Final Exam.}\hfill{\large\bf
  Spring 2008}\hfill{\large\bf Page 5/17}\hrule

\bigskip
{\problem[15 pts] \em Find a positive value of $k$ such that the average value of $f(x) = \displaystyle{\frac{1}{\sqrt{k^2-x^2}}}$ over the interval $[-k,k]$ is $\pi$.}
\begin{center}
\begin{tikzpicture}
\draw (0.5\linewidth, 0cm) node[text justified, text width=10.25cm, draw, rounded corners, shade =gray] {
You may find the following table useful:

\begin{tabular}{|c||c|c|c|c|c|c|c|c|c|}
\hline
&&&&&&&&&\\
\textbf{angle} $\theta$ & $-\frac{\pi}{2}$ & $-\frac{\pi}{3}$ & $-\frac{\pi}{4}$ & $-\frac{\pi}{6}$ & $0$ & $\frac{\pi}{6}$ & $\frac{\pi}{4}$ & $\frac{\pi}{3}$ & $\frac{\pi}{2}$ \\
&&&&&&&&&\\
\hline
&&&&&&&&&\\
$\sin(\theta)$ & $-1$ & $-\frac{\sqrt{3}}{2}$ & $-\frac{\sqrt{2}}{2}$ & $-\frac{1}{2}$ & $0$ & $\frac{1}{2}$ & $\frac{\sqrt{2}}{2}$ & $\frac{\sqrt{3}}{2}$ & $1$ \\
&&&&&&&&&\\
\hline
\end{tabular}
};
\end{tikzpicture}
\end{center}
\vspace{6cm}
\begin{flushright}
  \begin{tikzpicture}
    \draw (-1cm,0.5cm) node {$k = $};
    \draw (0cm,-0.2cm) rectangle (5cm,1.2cm);
  \end{tikzpicture}
\end{flushright}
\hrule
{\problem[10 pts] \em Evaluate the integral $\displaystyle{\int x^2 \sqrt{x-1}\, dx}$.}
\newpage


%%%%%%%%%%%%%%%%%%%%%%%%%%%%%%%%%%%%% Page 6
\noindent{\large\bf MATH 142}\hfill{\large\bf Final Exam.}\hfill{\large\bf
  Spring 2008}\hfill{\large\bf Page 6/17}\hrule
  
\bigskip
{\problem[15 pts] \em A spring exerts a force of $4 N$ when stretched $2~m$ beyond its natural length.
\begin{enumerate}
\item How much work was performed in stretching the spring to this length?
\vspace{8.5cm}
\begin{flushright}
  \begin{tikzpicture}
    \draw (-1cm,0.5cm) node {$W = $};
    \draw (0cm,-0.2cm) rectangle (5cm,1.2cm);
  \end{tikzpicture}
\end{flushright}
\item How far beyond its natural length can the spring be stretched with $36 J$ of work?
\vspace{8.5cm}
\begin{flushright}
  \begin{tikzpicture}
    \draw (-1cm,0.5cm) node {$b = $};
    \draw (0cm,-0.2cm) rectangle (5cm,1.2cm);
  \end{tikzpicture}
\end{flushright}
\end{enumerate}
\newpage

%%%%%%%%%%%%%%%%%%%%%%%%%%%%%%%%%%%%% Page 2
\noindent{\large\bf MATH 142}\hfill{\large\bf Final Exam.}\hfill{\large\bf
  Spring 2008}\hfill{\large\bf Page 7/17}\hrule

\bigskip
{\problem[40 pts] \em  Evaluate each integral:} 

\bigskip
\begin{tikzpicture}
\draw (4cm,14cm) node[left, text width=6.5cm]{
(a) $\displaystyle{\int \csc^2 x\, dx} = \mbox{}$ };
\draw (1cm,13.4cm) rectangle (9cm,14.6cm);
\draw (4cm, 11cm) node[left, text width=6.5cm]{
(b) $\displaystyle{\int \frac{1}{\csc x}\, dx} = \mbox{}$}; 
\draw (1cm,10.4cm) rectangle (9cm,11.6cm);
\draw (4cm, 8cm) node[left, text width=6.5cm]{
(c) $\displaystyle{\int \frac{x+1}{x}\, dx} = \mbox{}$}; 
\draw (1cm,7.4cm) rectangle (9cm,8.6cm);
\draw (4cm, 0cm) node[left, text width=6.5cm]{
(d) $\displaystyle{\int \frac{x}{x+1}\, dx} = \mbox{}$}; 
\draw (1cm, -0.6cm) rectangle (9cm,0.6cm);
\end{tikzpicture}
\newpage

%%%%%%%%%%%%%%%%%%%%%%%%%%%%%%%%%%%%% Page 3
\noindent{\large\bf MATH 142}\hfill{\large\bf Final Exam.}\hfill{\large\bf
  Spring 2008}\hfill{\large\bf Page 8/17}\hrule

\bigskip
{\problem[10 pts] \em Use \textbf{integration by parts} to evaluate the integral $\displaystyle{\int xe^{2x}\, dx}$.}
\vspace{6cm}
\begin{flushright}
  \begin{tikzpicture}
    \draw (-1.25cm,0.5cm) node {$\displaystyle{\int xe^{2x}\, dx} = $};
    \draw (0cm,-0.2cm) rectangle (5cm,1.2cm);
  \end{tikzpicture}
\end{flushright}
\hrule
{\problem[10 pts] \em Evaluate the improper integral $\displaystyle{\int_1^\infty \frac{dx}{x^3}}$.}
\vspace{10cm}
\begin{flushright}
  \begin{tikzpicture}
    \draw (-1.25cm,0.5cm) node {$\displaystyle{\int_1^\infty \frac{dx}{x^3}} = $};
    \draw (0cm,-0.2cm) rectangle (5cm,1.2cm);
  \end{tikzpicture}
\end{flushright}
\newpage

%%%%%%%%%%%%%%%%%%%%%%%%%%%%%%%%%%%%% Page 4
\noindent{\large\bf MATH 142}\hfill{\large\bf Final Exam.}\hfill{\large\bf
  Spring 2008}\hfill{\large\bf Page 9/17}\hrule

\bigskip
{\problem[10 pts] \em Use a \textbf{trigonometric substitution} to evaluate the integral $\displaystyle{\int \frac{dx}{\sqrt{x^2-9}}}$.}
\vspace{18cm}
\begin{flushright}
  \begin{tikzpicture}
    \draw (-4.5cm,0.5cm) node {$\displaystyle{\int \frac{dx}{\sqrt{x^2-9}}} = $};
    \draw (-3cm,-0.2cm) rectangle (5cm,1.2cm);
  \end{tikzpicture}
\end{flushright}
\newpage

%%%%%%%%%%%%%%%%%%%%%%%%%%%%%%%%%%%%% Page 5
\noindent{\large\bf MATH 142}\hfill{\large\bf Final Exam.}\hfill{\large\bf
  Spring 2008}\hfill{\large\bf Page 10/17}\hrule

\bigskip
{\problem[10 pts] \em Evaluate the integral $\displaystyle{\int \sin^2 x \cos^2 x\, dx}$. \newline Use trigonometric simplification and one of the following reduction formulas.}
\begin{center}
\begin{tikzpicture}
\draw (0.5\linewidth, 0cm) node[text justified, text width=10.25cm, draw, rounded corners, shade =gray] {
\begin{align*}
\int \sin^n x\, dx &= -\frac{1}{n} \sin^{n-1} x \cos x + \frac{n-1}{n} \int \sin^{n-2} x\, dx \\
\int \cos^n x\, dx &= \frac{1}{n} \cos^{n-1} x \sin x + \frac{n-1}{n} \int \cos^{n-2} x\, dx
\end{align*}
};
\end{tikzpicture}
\end{center}
\vspace{16cm}
\begin{flushright}
  \begin{tikzpicture}
    \draw (-5.25cm,0.5cm) node {$\displaystyle{\int \sin^2 x \cos^2 x\, dx} = $};
    \draw (-3.25cm,-0.2cm) rectangle (5cm,1.2cm);
  \end{tikzpicture}
\end{flushright}
\newpage

%%%%%%%%%%%%%%%%%%%%%%%%%%%%%%%%%%%%% Page 6
\noindent{\large\bf MATH 142}\hfill{\large\bf Final Exam.}\hfill{\large\bf
  Spring 2008}\hfill{\large\bf Page 11/17}\hrule
  
\bigskip
{\problem[20 pts] \em Use \textbf{partial fractions} to evaluate the integral $\displaystyle{\int \frac{dx}{x^2+x-2}}$.}
\vspace{18cm}
\begin{flushright}
  \begin{tikzpicture}
    \draw (-5.23cm,0.5cm) node {$\displaystyle{\int \frac{dx}{x^2+x-2}} = $};
    \draw (-3cm,-0.2cm) rectangle (5cm,1.2cm);
  \end{tikzpicture}
\end{flushright}
\newpage

%%%%%%%%%%%%%%%%%%%%%%%%%%%%%%%%%%%%% Page 2
\noindent{\large\bf MATH 142}\hfill{\large\bf Final Exam.}\hfill{\large\bf
  Spring 2008}\hfill{\large\bf Page 12/17}\hrule

\bigskip
{\problem[10 pts] \em  Find a formula for the general term of the following sequences:} 
\begin{enumerate}
\item $\displaystyle{\frac{1}{2}, \frac{3}{4}, \frac{5}{6}, \frac{7}{8}, \dotsc}$
\bigskip
\begin{flushright}
  \begin{tikzpicture}
    \draw (-0.5cm,0.5cm) node {$x_n = $};
    \draw (0cm,-0.2cm) rectangle (5cm,1.2cm);
  \end{tikzpicture}
\end{flushright}
\item $\displaystyle{1-\frac{1}{2}, \frac{1}{3} - \frac{1}{2}, \frac{1}{3} - \frac{1}{4}, \frac{1}{5}-\frac{1}{4}, \dotsc}$
\vspace{3cm}
\begin{flushright}
  \begin{tikzpicture}
    \draw (-0.5cm,0.5cm) node {$x_n = $};
    \draw (0cm,-0.2cm) rectangle (5cm,1.2cm);
  \end{tikzpicture}
\end{flushright}
\end{enumerate}
\hrule
{\problem[10pts] \em Write out the first five terms of the sequence $\left\{ \displaystyle{\frac{\ln n}{n}} \right\}_{n=1}^\infty$ \newline Determine whether the sequence converges, and if so find its limit.
\vspace{7cm}
\begin{flushright}
  \begin{tikzpicture}
    \draw (-3.75cm,1.9cm) node {First five terms:};
    \draw (-2cm, 1.3cm) rectangle (5cm, 2.7cm);
    \draw (-1.25cm,0.5cm) node {$\displaystyle{\lim_{n \to \infty} x_n} = $};
    \draw (0cm,-0.2cm) rectangle (5cm,1.2cm);
  \end{tikzpicture}
\end{flushright}
\newpage


%%%%%%%%%%%%%%%%%%%%%%%%%%%%%%%%%%%%% Page 3
\noindent{\large\bf MATH 142}\hfill{\large\bf Final Exam.}\hfill{\large\bf
  Spring 2008}\hfill{\large\bf Page 13/17}\hrule

\bigskip
{\problem[5 pts] \em Use $x_{n+1}-x_n$ to show that the sequence $\big\{ n-n^2 \big\}_{n=1}^\infty$ is strictly increasing or strictly decreasing.}
\vspace{6.5cm}
\hrule
{\problem[5 pts] \em Use $x_{n+1}/x_n$ to show that the sequence $\left\{ \displaystyle{ne^{-n}} \right\}_{n=1}^\infty$ is strictly increasing or strictly decreasing.}
\vspace{6.5cm}
\hrule
{\problem[5 pts] \em Use \textbf{differentiation} to show that the sequence $\left\{ \displaystyle{3 - \frac{1}{n}} \right\}_{n=1}^\infty$ is strictly increasing or strictly decreasing.}
\newpage

%%%%%%%%%%%%%%%%%%%%%%%%%%%%%%%%%%%%% Page 4
\noindent{\large\bf MATH 142}\hfill{\large\bf Final Exam.}\hfill{\large\bf
  Spring 2008}\hfill{\large\bf Page 14/17}\hrule

\bigskip
{\problem[20 pts] \em Determine whether the series converge, and if so find their sum:}
\begin{enumerate}
\item $\displaystyle{\sum_{k=1}^\infty \Big( -\frac{3}{2} \Big)^{k+1}}$
\vspace{7cm}
\begin{flushright}
  \begin{tikzpicture}
    \draw (-1.75cm,0.5cm) node {$\displaystyle{\sum_{k=1}^\infty \Big( -\frac{3}{2} \Big)^{k+1} }= $};
    \draw (0cm,-0.2cm) rectangle (5cm,1.2cm);
  \end{tikzpicture}
\end{flushright}
\item $\displaystyle{\sum_{k=1}^\infty \left( \frac{1}{2^k} -\frac{1}{2^{k+1}} \right)}$
\vspace{7cm}
\begin{flushright}
  \begin{tikzpicture}
    \draw (-2cm,0.5cm) node {$\displaystyle{\sum_{k=1}^\infty \left( \frac{1}{2^k} - \frac{1}{2^{k+1}} \right) }= $};
    \draw (0cm,-0.2cm) rectangle (5cm,1.2cm);
  \end{tikzpicture}
\end{flushright}
\end{enumerate}
\newpage

%%%%%%%%%%%%%%%%%%%%%%%%%%%%%%%%%%%%% Page 5
\noindent{\large\bf MATH 142}\hfill{\large\bf Final Exam.}\hfill{\large\bf
  Spring 2008}\hfill{\large\bf Page 15/17}\hrule

\bigskip
{\problem[5 pts] \em Apply the \textbf{divergence test} and state what it tells you about the series.}
\begin{equation*}
\sum_{k=1}^\infty \Big( 1 + \frac{1}{k} \Big)^k.
\end{equation*}
\vspace{2cm}
\hrule
{\problem[10 pts] \em Use the \textbf{integral test} to determine whether the series $\displaystyle{\sum_{k=1}^\infty \frac{1}{1+9k^2}}$ converges.}
\vspace{6.5cm}
\hrule
{\problem[10 pts] \em Use the \textbf{ratio test} to determine whether the series $\displaystyle{\sum_{k=1}^\infty \frac{3^k}{k!}}$ converges.  If the test is inconclusive, then say so.}
\newpage

%%%%%%%%%%%%%%%%%%%%%%%%%%%%%%%%%%%%% Page 6
\noindent{\large\bf MATH 142}\hfill{\large\bf Final Exam.}\hfill{\large\bf
  Spring 2008}\hfill{\large\bf Page 16/17}\hrule
  
\bigskip

{\problem[10 pts] \em Use the \textbf{root test} to determine whether the series $\displaystyle{\sum_{k=1}^\infty \Big( \frac{k}{100} \Big)^k}$ converges.  If the test is inconclusive, then say so.}
\vspace{9cm}
\hrule{\problem[10 pts] \em Classify the series $\displaystyle{\sum_{k=1}^\infty \frac{k \cos k\pi}{k^2+1}}$ as absolutely convergent, convergent or divergent.}

\end{document}
