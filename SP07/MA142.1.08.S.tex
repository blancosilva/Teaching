\documentclass[12pt]{article}

\usepackage{amsmath,amsthm,amsfonts,amssymb,amsxtra}
\usepackage{pgf,tikz}
\usetikzlibrary{arrows}
\renewcommand{\theenumi}{(\alph{enumi})} 
\renewcommand{\labelenumi}{\theenumi}

\pagestyle{empty}
\setlength{\textwidth}{7in}
\setlength{\oddsidemargin}{-0.5in}
\setlength{\topmargin}{-1.0in}
\setlength{\textheight}{9.5in}

\newtheorem{problem}{Problem}

\begin{document}

\noindent{\large\bf MATH 142}\hfill{\large\bf Exam\#1.}\hfill{\large\bf
  Spring 2008}\hfill{\large\bf Page 1/6}\hrule

\bigskip
\begin{center}
  \begin{tabular}{|ll|}
    \hline & \cr
    {\bf Name: } & \makebox[12cm]{\hrulefill}\cr & \cr
    {\bf 4-digit code:} & \makebox[12cm]{\hrulefill}\cr & \cr
    \hline
  \end{tabular}
\end{center}
\begin{itemize}
\item Write your name and the last 4 digits of your SSN in the space provided above.
\item The test has six (6) pages, including this one.
\item Enter your answer in the box(es) provided.
\item You must show sufficient work to justify all answers unless
  otherwise stated in the problem.  Correct answers with inconsistent
  work may not be given credit.
\item Credit for each problem is given in parentheses at the right of
  the problem number.
\item No books, notes or calculators may be used on this test.
\end{itemize}
\hrule

\begin{center}
  \begin{tabular}{|c|c|c|}
    \hline
    &&\cr
    {\large\bf Page} & {\large\bf Max.~points} & {\large\bf Your points} \cr
    &&\cr
    \hline
    &&\cr
    {\Large 2} & \Large 20 & \cr
    &&\cr
    \hline
    &&\cr
    {\Large 3} & \Large 20 & \cr
    &&\cr
    \hline
    &&\cr
    {\Large 4} & \Large 20 & \cr
    &&\cr
    \hline
    &&\cr
    {\Large 5} & \Large 25 & \cr
    &&\cr
	\hline
    &&\cr
    {\Large 6} & \Large 15 & \cr
    &&\cr
    \hline\hline
    &&\cr
    {\large\bf Total} & \Large 100 & \cr
    &&\cr
    \hline
  \end{tabular}
\end{center}
\newpage

%%%%%%%%%%%%%%%%%%%%%%%%%%%%%%%%%%%%% Page 2
\noindent{\large\bf MATH 142}\hfill{\large\bf Exam\#1.}\hfill{\large\bf
  Spring 2008}\hfill{\large\bf Page 2/6}\hrule

\bigskip
{\problem[10 pts] \em  Find the area of the region that is enclosed between the curves $y=x^2$ and $y=x+6$.} 
\vspace{8.5cm}
\begin{flushright}
  \begin{tikzpicture}
    \draw (-1cm,0.5cm) node {$A = $};
    \draw (0cm,-0.2cm) rectangle (5cm,1.2cm);
  \end{tikzpicture}
\end{flushright}
\hrule
{\problem[10 pts] \em Find the volume of the solid that is obtained when the region under the curve $y=\sqrt{x}$ over the interval $[1,4]$ is revolved about the $x$-axis.}
\vspace{8.5cm}
\begin{flushright}
  \begin{tikzpicture}
    \draw (-1cm,0.5cm) node {$V = $};
    \draw (0cm,-0.2cm) rectangle (5cm,1.2cm);
  \end{tikzpicture}
\end{flushright}
\newpage

%%%%%%%%%%%%%%%%%%%%%%%%%%%%%%%%%%%%% Page 3
\noindent{\large\bf MATH 142}\hfill{\large\bf Exam\#1.}\hfill{\large\bf
  Spring 2008}\hfill{\large\bf Page 3/6}\hrule

\bigskip
{\problem[10 pts] \em Find the volume of the solid generated when the region enclosed by $y=\sqrt{x}$, $y=2$ and $x=0$ is revolved about the $y$-axis.}
\vspace{8.5cm}
\begin{flushright}
  \begin{tikzpicture}
    \draw (-1cm,0.5cm) node {$V = $};
    \draw (0cm,-0.2cm) rectangle (5cm,1.2cm);
  \end{tikzpicture}
\end{flushright}
\hrule
{\problem[10 pts] \em Find the arclength of the curve $y=x^{3/2}$ from $x=1$ to $x=2$.}
\vspace{8.5cm}
\begin{flushright}
  \begin{tikzpicture}
    \draw (-1cm,0.5cm) node {$L = $};
    \draw (0cm,-0.2cm) rectangle (5cm,1.2cm);
  \end{tikzpicture}
\end{flushright}
\newpage

%%%%%%%%%%%%%%%%%%%%%%%%%%%%%%%%%%%%% Page 4
\noindent{\large\bf MATH 142}\hfill{\large\bf Exam\#1.}\hfill{\large\bf
  Spring 2008}\hfill{\large\bf Page 4/6}\hrule

\bigskip
{\problem[10 pts] \em Find the area of the surface that is generated by revolving the portion of the curve $y=x^3$ between $x=0$ and $x=1$ about the $x$-axis.}
\vspace{8.5cm}
\begin{flushright}
  \begin{tikzpicture}
    \draw (-1cm,0.5cm) node {$A = $};
    \draw (0cm,-0.2cm) rectangle (5cm,1.2cm);
  \end{tikzpicture}
\end{flushright}
\hrule
{\problem[10 pts] \em Find the average value of the function $f(x) = 1/x$ over the interval $[1,e]$.}
\vspace{8.5cm}
\begin{flushright}
  \begin{tikzpicture}
    \draw (-1cm,0.5cm) node {$f_{ave} = $};
    \draw (0cm,-0.2cm) rectangle (5cm,1.2cm);
  \end{tikzpicture}
\end{flushright}
\newpage

%%%%%%%%%%%%%%%%%%%%%%%%%%%%%%%%%%%%% Page 5
\noindent{\large\bf MATH 142}\hfill{\large\bf Exam\#1.}\hfill{\large\bf
  Spring 2008}\hfill{\large\bf Page 5/6}\hrule

\bigskip
{\problem[15 pts] \em Find a positive value of $k$ such that the average value of $f(x) = \displaystyle{\frac{1}{\sqrt{k^2-x^2}}}$ over the interval $[-k,k]$ is $\pi$.}
\begin{center}
\begin{tikzpicture}
\draw (0.5\linewidth, 0cm) node[text justified, text width=10.25cm, draw, rounded corners, shade =gray] {
You may find the following table useful:

\begin{tabular}{|c||c|c|c|c|c|c|c|c|c|}
\hline
&&&&&&&&&\\
\textbf{angle} $\theta$ & $-\frac{\pi}{2}$ & $-\frac{\pi}{3}$ & $-\frac{\pi}{4}$ & $-\frac{\pi}{6}$ & $0$ & $\frac{\pi}{6}$ & $\frac{\pi}{4}$ & $\frac{\pi}{3}$ & $\frac{\pi}{2}$ \\
&&&&&&&&&\\
\hline
&&&&&&&&&\\
$\sin(\theta)$ & $-1$ & $-\frac{\sqrt{3}}{2}$ & $-\frac{\sqrt{2}}{2}$ & $-\frac{1}{2}$ & $0$ & $\frac{1}{2}$ & $\frac{\sqrt{2}}{2}$ & $\frac{\sqrt{3}}{2}$ & $1$ \\
&&&&&&&&&\\
\hline
\end{tabular}
};
\end{tikzpicture}
\end{center}
\vspace{6cm}
\begin{flushright}
  \begin{tikzpicture}
    \draw (-1cm,0.5cm) node {$k = $};
    \draw (0cm,-0.2cm) rectangle (5cm,1.2cm);
  \end{tikzpicture}
\end{flushright}
\hrule
{\problem[10 pts] \em Evaluate the integral $\displaystyle{\int x^2 \sqrt{x-1}\, dx}$.}
\newpage


%%%%%%%%%%%%%%%%%%%%%%%%%%%%%%%%%%%%% Page 6
\noindent{\large\bf MATH 142}\hfill{\large\bf Exam\#1.}\hfill{\large\bf
  Spring 2008}\hfill{\large\bf Page 6/6}\hrule
  
\bigskip
{\problem[15 pts] \em A spring exerts a force of $4 N$ when stretched $2~m$ beyond its natural length.
\begin{enumerate}
\item How much work was performed in stretching the spring to this length?
\vspace{8.5cm}
\begin{flushright}
  \begin{tikzpicture}
    \draw (-1cm,0.5cm) node {$W = $};
    \draw (0cm,-0.2cm) rectangle (5cm,1.2cm);
  \end{tikzpicture}
\end{flushright}
\item How far beyond its natural length can the spring be stretched with $36 J$ of work?
\vspace{8.5cm}
\begin{flushright}
  \begin{tikzpicture}
    \draw (-1cm,0.5cm) node {$b = $};
    \draw (0cm,-0.2cm) rectangle (5cm,1.2cm);
  \end{tikzpicture}
\end{flushright}
\end{enumerate}

\end{document}